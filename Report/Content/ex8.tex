\section{Exercise 8}
\subsection{Kinetic Energy}
Since the system contains rigid bodies, there will be both rotational and linear kinetic energies.
The linear kinetic energies are given by
\begin{align*}
	K_{\text{Lin},1} &= \frac{1}{2}m_1\|\bm{v}_{c1}\|^2 \\
	K_{\text{Lin},2} &= \frac{1}{2}m_2\|\bm{v}_{c2}\|^2
\end{align*}
where $\bm{v}_{c1}, \bm{v}_{c2}$ are the velocities of the center of masses of links 1 and 2 respectively.
\begin{align*}
	\|\bm{v}_{c1}\|^2 &= \dot{x}_{c1}^2 + \dot{y}_{c1}^2 \\
	\|\bm{v}_{c2}\|^2 &= \dot{x}_{c2}^2 + \dot{y}_{c2}^2
\end{align*}
where 
\begin{align*}
	x_{c1} &= \frac{l_1}{2}\cos\theta_1\\
	y_{c1} &= \frac{l_1}{2}\sin\theta_1\\
	x_{c2} &= l_1\cos\theta_1 + \frac{l_2}{2}\cos\left(\theta_1 + \theta_2\right)\\
	y_{c2} &= l_1\sin\theta_1 + \frac{l_2}{2}\sin\left(\theta_1 + \theta_2\right)
\end{align*}
The rotational kinetic energies are given by
\begin{align*}
	K_{\text{Rot},1} &= \frac{1}{2}I_{c1}\| \\
	K_{\text{Rot},2} &= \frac{1}{2}I_{c2}\|\bm{\omega}_{2}\|^2
\end{align*}
where 
\begin{align*}
	I_{c1} &= \frac{1}{12}m_1 l_1^2\\
	I_{c2} &= \frac{1}{12}m_2 l_2^2 \\
	\|\bm{\omega}_{1}\|^2 &= \dot{\theta}_1^2\\
	\|\bm{\omega}_{2}\|^2 &= \left(\dot{\theta}_1 + \dot{\theta}_{2}\right)^2
\end{align*}
The kinetic energy of the system is given by
\begin{align*}
	K_{\text{sys}} &= K_{\text{Rot},1} + K_{\text{Rot},2} + K_{\text{Lin},1} + K_{\text{Lin},2}\\
	&= \frac{1}{2} l_1^2 \dot{\theta}_1^2\left(\frac{1}{3} m_1+m_2\right)+\frac{1}{6} m_2 l_2{ }^2\left(\dot{\theta}_1+\dot{\theta}_2\right)^2+\frac{1}{2} m_2 l_1 l_2 \cos \left(\theta_2\right) \dot{\theta}_1\left(\dot{\theta}+\dot{\theta}_2\right)
\end{align*}
\subsection{Potential Energy}
Considering the base joint to be the datum, the potential energy of the system is given by
\begin{equation*}
	P_{\text{sys}}=P_1+P_2=m_1 g \frac{l_1}{2} \sin \theta_1+m_2 g\left[l_1 \sin \theta_1+\frac{l_2}{2} \sin \left(\theta_1+\theta_2\right)\right]
\end{equation*}

\subsection{Lagrangian 2nd Kind}
The equations of motion can be derived from the system lagrangian
\begin{align*}
	&L_{\text{sys}} = K_{\text{sys}} - P_{\text{sys}}\\
	&\frac{d}{dt}\left(\frac{\partial L}{\partial \dot{\bm{q}}}\right) - \frac{\partial L}{\partial \bm{q}} = 0
\end{align*}
where $\bm{q} = \left(\theta_1 \text{ } \theta_2\right)\trans$
The individual terms are
\begin{align*}
	\frac{\partial L}{\partial \theta}&=-m_1 g \frac{l_1}{2} \cos \theta_1-m_2 g\left[l_1 \cos \theta_1+\frac{l_2}{2} \cos \left(\theta_1+\theta_2\right)\right] \\
	\frac{\partial L}{\partial \theta_2}&=-\frac{1}{2} m_2 l_1 l_2 \sin \left(\theta_2\right) \dot{\theta}_1\left(\dot{\theta}_1+\dot{\theta}_2\right)-\frac{1}{2} m_2 g l_1 \cos \left(\theta_1+\theta_2\right)
\end{align*}
\begin{align*}
	\frac{d}{dt}\left(\frac{\partial L}{\partial \dot{\theta}_1}\right)&=
	 \left(\frac{1}{3} m_1 l_1{ }^2+m_2 l_1{ }^2+\frac{1}{3} m_2 l_2{ }^2+m_2 l_1 l_2 \cos \theta_1\right) \ddot{\theta}_1 \\
		& +\left(\frac{1}{3} m_2 l_2{ }^2+\frac{1}{2} m_2 l_1 l_2 \cos \theta_1\right) \ddot{\theta}_2+\left(-m_2 l_1 l_2 \sin \theta_1 \dot{\theta}_2\right) \dot{\theta}_1 \\
		& +\left(-\frac{1}{2} m_2 l_1 l_2 \sin \theta_1 \dot{\theta}_2\right)\dot{\theta}_2
	\\
	\frac{d}{dt}\left(\frac{\partial L}{\partial \dot{\theta}_2}\right)&=\left(\frac{1}{3} m_2 l_2^2+\frac{1}{2} m_2 l_1 l_2 \cos \theta_2\right) \ddot{\theta}_1+\frac{1}{3} m_2 l_2^2 \ddot{\theta}_2-\frac{1}{2} m_2 l_1 l_2 \sin \theta_2 \dot{\theta}_1 \dot{\theta}_2
\end{align*}
The final equations of motion are represented by
\begin{align*}
	&\left[\begin{array}{lc}
		\frac{1}{3} m_1 l_1{ }^2+m_2 l_1{ }^2+\frac{1}{3} m_2 l_2{ }^2+m_2 l_1 l_2 \cos \theta_1 & \frac{1}{3} m_2 l_2{ }^2+\frac{1}{2} m_2 l_1 l_2 \cos \theta_2 \\
		\frac{1}{3} m_2 l_2{ }^2+\frac{1}{2} m_2 l_1 l_2 \cos \theta_2 & \frac{1}{3} m_2 l_2{ }^2
	\end{array}\right]\begin{bmatrix}
	\ddot{\theta}_1\\
	\ddot{\theta}_2
\end{bmatrix} \\
& \left[\begin{array}{cc}
	-m_2 l_1 l_2 \sin \theta_1 \dot{\theta}_2 & -\frac{1}{2} m_2 l_1 l_2 \sin \theta_2 \dot{\theta}_2 \\
	\frac{1}{2} m_2 l_1 l_2 \sin \theta_2 \dot{\theta}_1 & 0
\end{array}\right]\begin{bmatrix}
\dot{\theta}_1\\
\dot{\theta}_2
\end{bmatrix} \\ 
& \begin{bmatrix}
	m_1g\frac{l_1}{2}\cos\theta_1 + m_2g\left(l_1\cos\theta_1 + \frac{l_2}{2}\cos\left(\theta_1 + \theta_2\right)\right)\\
	\frac{1}{2}m_2gl_1\cos\left(\theta_1 + \theta_2\right)
\end{bmatrix} = 0
\end{align*}
The matrices $\bm{M}, \bm{C}$ and the vector $\bm{G}$ can be trivially found from the above equations.

The equations of motion have been symbolically derived and verified in the Jupyter Notebook file \emph{2RRobotDynamics.ipynb}.
