\section{Exercise 7}
\subsection{Inverse Dynamics}
Inverse dynamics is done by balancing the moments about the hinge
\begin{align*}
	ml^2\ddot{\theta} &= \tau - mgl\sin\theta\\
	\implies \tau &= ml\left(l\ddot{\theta} + g\sin\theta\right)
\end{align*}
\subsection{Forward Dynamics}
The forward dynamics is given by
\begin{equation*}
	\ddot{\theta} = \frac{\tau}{ml^2} - \frac{g\sin\theta}{l}
\end{equation*}

\subsection{Euler Numerical Integration}
A forward Euler numerical scheme is used to approximate the solution of the unforced system, whose state-space can be written as
\begin{equation*}
	\begin{bmatrix}
		\dot{\theta} \\
		\ddot{\theta}
	\end{bmatrix} = \begin{bmatrix}
	\dot{\theta}\\
	-\frac{g\sin\theta}{l}
\end{bmatrix} = f\left(\bm{x}\right) = \dot{\bm{x}}
\end{equation*}
Then, the euler numerical integration scheme is given by
\begin{equation*}
	\bm{x}\left(t_{i+1}\right) = \bm{x}\left(t_{i}\right) + f\left(\bm{x}\left(t_i\right)\right)\Delta t
\end{equation*}
where $\Delta t = t_{i+1} - t_i$.

The euler-integrator is implemented for the unforced system in the python file \emph{eulerIntergrator.py} and the resultant visualization of the pendulum can be accessed from \emph{pendulum.gif}.