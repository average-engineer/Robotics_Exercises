\section{Exercise 11}
In order to make use of the convenience of inbuilt tools already available, the \emph{ode45} solver from MATLAB\textsuperscript{\textregistered} is made use of to emulate the fourth-order Runge-Kutta scheme. The implementation of the same along with the euler integrator can be found in \emph{Pendulum.m}.
\subsection{Trade-offs between Euler Integrator and Fourth-order Runge-Kutta}
\begin{enumerate}
	\item The euler integrator implemented is the forward euler method which is an explicit method and is highly sensitive to the time step-size. Very large step-sizes can lead to numerical instabilities. On the other hand, the fourth-order Runge-Kutta handles smaller step-sizes much better.
	\begin{figure}[h!]
		\centering
		\scalebox{0.8}{% This file was created by matlab2tikz.
%
%The latest updates can be retrieved from
%  http://www.mathworks.com/matlabcentral/fileexchange/22022-matlab2tikz-matlab2tikz
%where you can also make suggestions and rate matlab2tikz.
%
\definecolor{mycolor1}{rgb}{0.00000,0.44700,0.74100}%
\definecolor{mycolor2}{rgb}{0.85000,0.32500,0.09800}%
%
\begin{tikzpicture}

\begin{axis}[%
width=4.521in,
height=3.566in,
at={(0.758in,0.481in)},
scale only axis,
xmin=0,
xmax=5,
xlabel style={font=\color{white!15!black}},
xlabel={Time (s)},
ymin=-350,
ymax=150,
ylabel style={font=\color{white!15!black}},
ylabel={Angle (deg)},
axis background/.style={fill=white},
axis x line*=bottom,
axis y line*=left,
xmajorgrids,
ymajorgrids,
legend style={legend cell align=left, align=left, draw=white!15!black}
]
\addplot [color=mycolor1, line width=2.0pt]
  table[row sep=crcr]{%
0	30\\
0.0505050505050505	30\\
0.101010101010101	28.5948210074417\\
0.151515151515152	25.784463022325\\
0.202020202020202	21.6290325909809\\
0.252525252525253	16.2511331319197\\
0.303030303030303	9.83734798268533\\
0.353535353535354	2.63708976986872\\
0.404040404040404	-5.04332315332736\\
0.454545454545455	-12.8530396660208\\
0.505050505050505	-20.4157004969547\\
0.555555555555556	-27.3531940613703\\
0.606060606060606	-33.3103536163473\\
0.656565656565657	-37.9762257370757\\
0.707070707070707	-41.0987227619353\\
0.757575757575758	-42.491909713495\\
0.808080808080808	-42.0376841039149\\
0.858585858585859	-39.685100751321\\
0.909090909090909	-35.4506476300646\\
0.95959595959596	-29.4215906163708\\
1.01010101010101	-21.7625217761975\\
1.06060606060606	-12.7229151174843\\
1.11111111111111	-2.64133896175593\\
1.16161616161616	8.05918016809986\\
1.21212121212121	18.8892110892385\\
1.26262626262626	29.3252414464106\\
1.31313131313131	38.8514485816236\\
1.36363636363636	47.001236266472\\
1.41414141414141	53.3880769361556\\
1.46464646464646	57.7195105354174\\
1.51515151515152	59.7950883695103\\
1.56565656565657	59.4946666280332\\
1.61616161616162	56.7654444897645\\
1.66666666666667	51.6148687437277\\
1.71717171717172	44.1136142463982\\
1.76767676767677	34.409447647101\\
1.81818181818182	22.7490374686644\\
1.86868686868687	9.50048541896667\\
1.91919191919192	-4.83481933044424\\
1.96969696969697	-19.6339904201567\\
2.02020202020202	-34.1962949598582\\
2.07070707070707	-47.8142900413284\\
2.12121212121212	-59.8527799240411\\
2.17171717171717	-69.8088729321457\\
2.22222222222222	-77.3347431341527\\
2.27272727272727	-82.2229617275409\\
2.32323232323232	-84.3692048785893\\
2.37373737373737	-83.7309392957799\\
2.42424242424242	-80.2958762454482\\
2.47474747474747	-74.0672609645633\\
2.52525252525253	-65.068500155916\\
2.57575757575758	-53.3673423569041\\
2.62626262626263	-39.1177170833726\\
2.67676767676768	-22.6128427415182\\
2.72727272727273	-4.33486929084819\\
2.77777777777778	15.0236931251525\\
2.82828282828283	34.594678090017\\
2.87878787878788	53.4371663984373\\
2.92929292929293	70.6840253817909\\
2.97979797979798	85.6735934129014\\
3.03030303030303	98.0110019826711\\
3.08080808080808	107.546060763111\\
3.13131313131313	114.298186924844\\
3.18181818181818	118.370707283257\\
3.23232323232323	119.881821570276\\
3.28282828282828	118.920125409004\\
3.33333333333333	115.521694692139\\
3.38383838383838	109.663372475741\\
3.43434343434343	101.268920797518\\
3.48484848484848	90.2279947779899\\
3.53535353535354	76.4308922208086\\
3.58585858585859	59.8234539287971\\
3.63636363636364	40.4841013723993\\
3.68686868686869	18.715248746718\\
3.73737373737374	-4.87819233162436\\
3.78787878787879	-29.3733791231719\\
3.83838383838384	-53.6295795568994\\
3.88888888888889	-76.5073024465272\\
3.93939393939394	-97.1221249490965\\
3.98989898989899	-115.004156286956\\
4.04040404040404	-130.097513983091\\
4.09090909090909	-142.643908517098\\
4.14141414141414	-153.040521542196\\
4.19191919191919	-161.731902468136\\
4.24242424242424	-169.149178835116\\
4.29292929292929	-175.685509809151\\
4.34343434343434	-181.692783792774\\
4.39393939393939	-187.488631976281\\
4.44444444444444	-193.367499125906\\
4.49494949494949	-199.612638763682\\
4.54545454545455	-206.507522099823\\
4.5959595959596	-214.345728421647\\
4.64646464646465	-223.438240513762\\
4.6969696969697	-234.116314953929\\
4.74747474747475	-246.726713670458\\
4.7979797979798	-261.614088530786\\
4.84848484848485	-279.083144517717\\
4.8989898989899	-299.332510770568\\
4.94949494949495	-322.35699393874\\
5	-347.831523160428\\
};
\addlegendentry{Euler}

\addplot [color=mycolor2, line width=2.0pt]
  table[row sep=crcr]{%
0	30\\
0.0505050505050505	29.285738297339\\
0.101010101010101	27.1741255690426\\
0.151515151515152	23.7584814087315\\
0.202020202020202	19.1912258409719\\
0.252525252525253	13.6853078742338\\
0.303030303030303	7.5032153684629\\
0.353535353535354	0.950929940009744\\
0.404040404040404	-5.6496061409724\\
0.454545454545455	-11.970293834728\\
0.505050505050505	-17.7006549380713\\
0.555555555555556	-22.5588170504412\\
0.606060606060606	-26.3200000478961\\
0.656565656565657	-28.8201114808882\\
0.707070707070707	-29.9421922568338\\
0.757575757575758	-29.6319498374435\\
0.808080808080808	-27.9096400256326\\
0.858585858585859	-24.8542617032716\\
0.909090909090909	-20.5973137236791\\
0.95959595959596	-15.3338055555283\\
1.01010101010101	-9.31622907349762\\
1.06060606060606	-2.83715163858675\\
1.11111111111111	3.78114035302544\\
1.16161616161616	10.2128663740926\\
1.21212121212121	16.1398565913421\\
1.26262626262626	21.2723809090799\\
1.31313131313131	25.3654385783988\\
1.36363636363636	28.2373526053844\\
1.41414141414141	29.7616992018797\\
1.46464646464646	29.8624331460784\\
1.51515151515152	28.5362985595455\\
1.56565656565657	25.8501423515721\\
1.61616161616162	21.9195425355783\\
1.66666666666667	16.9211686250292\\
1.71717171717172	11.0919068217285\\
1.76767676767677	4.71270484104762\\
1.81818181818182	-1.89811394503788\\
1.86868686868687	-8.41487170264624\\
1.91919191919192	-14.5150847248266\\
1.96969696969697	-19.9011326722875\\
2.02020202020202	-24.3116544763747\\
2.07070707070707	-27.5458050822071\\
2.12121212121212	-29.4647236728904\\
2.17171717171717	-29.9756435683266\\
2.22222222222222	-29.0528480126386\\
2.27272727272727	-26.744696171575\\
2.32323232323232	-23.1542142104208\\
2.37373737373737	-18.4398193949089\\
2.42424242424242	-12.8212422870625\\
2.47474747474747	-6.56899533260801\\
2.52525252525253	0.00816688608127165\\
2.57575757575758	6.58368878703617\\
2.62626262626263	12.8333697936216\\
2.67676767676768	18.4508185136042\\
2.72727272727273	23.162270250303\\
2.77777777777778	26.7480803712726\\
2.82828282828283	29.0529695851303\\
2.87878787878788	29.9714109687645\\
2.92929292929293	29.4560609025985\\
2.97979797979798	27.5342277753588\\
3.03030303030303	24.2966342256506\\
3.08080808080808	19.8837043593252\\
3.13131313131313	14.4980804264148\\
3.18181818181818	8.39796602858261\\
3.23232323232323	1.88101807899627\\
3.28282828282828	-4.72706080966549\\
3.33333333333333	-11.1014085861011\\
3.38383838383838	-16.9273349945882\\
3.43434343434343	-21.9216561021036\\
3.48484848484848	-25.8464574909104\\
3.53535353535354	-28.5283053409861\\
3.58585858585859	-29.8499500030766\\
3.63636363636364	-29.744144661695\\
3.68686868686869	-28.2163449460526\\
3.73737373737374	-25.3424767351208\\
3.78787878787879	-21.2472036018303\\
3.83838383838384	-16.1157130319591\\
3.88888888888889	-10.1917525355587\\
3.93939393939394	-3.76166867343652\\
3.98989898989899	2.85266982166774\\
4.04040404040404	9.32541594629973\\
4.09090909090909	15.3368202732044\\
4.14141414141414	20.5944285770472\\
4.19191919191919	24.8440438586504\\
4.24242424242424	27.8932630958117\\
4.29292929292929	29.6119320475225\\
4.34343434343434	29.9159321742926\\
4.39393939393939	28.7887992566544\\
4.44444444444444	26.2880353469164\\
4.49494949494949	22.5252421856462\\
4.54545454545455	17.6676094556064\\
4.5959595959596	11.9425198755105\\
4.64646464646465	5.62624514904665\\
4.6969696969697	-0.968110006118956\\
4.74747474747475	-7.51297763392702\\
4.7979797979798	-13.6860278913718\\
4.84848484848485	-19.1856582406955\\
4.8989898989899	-23.7444755637295\\
4.94949494949495	-27.1508425236146\\
5	-29.2556579852903\\
};
\addlegendentry{Runge-Kutta}

\end{axis}
\end{tikzpicture}%}
		\caption{Solution of the open-loop pendulum from Euler-Integrator and Runge-Kutta for time step-size = $0.05$}
		\label{fig:eulerVsRK}
	\end{figure}

	\begin{figure}[h!]
		\centering
		\scalebox{0.8}{% This file was created by matlab2tikz.
%
%The latest updates can be retrieved from
%  http://www.mathworks.com/matlabcentral/fileexchange/22022-matlab2tikz-matlab2tikz
%where you can also make suggestions and rate matlab2tikz.
%
\definecolor{mycolor1}{rgb}{0.00000,0.44700,0.74100}%
\definecolor{mycolor2}{rgb}{0.85000,0.32500,0.09800}%
%
\begin{tikzpicture}

\begin{axis}[%
width=4.521in,
height=3.566in,
at={(0.758in,0.481in)},
scale only axis,
xmin=0,
xmax=5,
xlabel style={font=\color{white!15!black}},
xlabel={Time (s)},
ymin=-40,
ymax=40,
ylabel style={font=\color{white!15!black}},
ylabel={Angle (deg)},
axis background/.style={fill=white},
axis x line*=bottom,
axis y line*=left,
xmajorgrids,
ymajorgrids,
legend style={legend cell align=left, align=left, draw=white!15!black}
]
\addplot [color=mycolor1, line width=2.0pt]
  table[row sep=crcr]{%
0	30\\
0.005005005005005	30\\
0.01001001001001	29.9859482100744\\
0.015015015015015	29.9578446302233\\
0.02002002002002	29.9156952298651\\
0.025025025025025	29.8595119503712\\
0.03003003003003	29.7893127075905\\
0.035035035035035	29.7051213943464\\
0.04004004004004	29.6069678828755\\
0.045045045045045	29.4948880271685\\
0.0500500500500501	29.3689236651614\\
0.0550550550550551	29.2291226207171\\
0.0600600600600601	29.0755387053242\\
0.0650650650650651	28.9082317194338\\
0.0700700700700701	28.7272674533413\\
0.0750750750750751	28.5327176875137\\
0.0800800800800801	28.3246601922538\\
0.0850850850850851	28.1031787265806\\
0.0900900900900901	27.8683630362031\\
0.0950950950950951	27.6203088504495\\
0.1001001001001	27.359117878014\\
0.105105105105105	27.0848978013698\\
0.11011011011011	26.7977622696958\\
0.115115115115115	26.4978308901559\\
0.12012012012012	26.1852292173659\\
0.125125125125125	25.8600887408788\\
0.13013013013013	25.5225468705139\\
0.135135135135135	25.1727469193555\\
0.14014014014014	24.810838084241\\
0.145145145145145	24.4369754235589\\
0.15015015015015	24.0513198321769\\
0.155155155155155	23.6540380133189\\
0.16016016016016	23.2453024472108\\
0.165165165165165	22.82529135632\\
0.17017017017017	22.3941886670102\\
0.175175175175175	21.9521839674438\\
0.18018018018018	21.4994724615622\\
0.185185185185185	21.0362549189846\\
0.19019019019019	20.5627376206693\\
0.195195195195195	20.0791323001893\\
0.2002002002002	19.5856560804828\\
0.205205205205205	19.0825314059477\\
0.21021021021021	18.5699859697579\\
0.215215215215215	18.0482526362914\\
0.22022022022022	17.5175693585705\\
0.225225225225225	16.9781790906275\\
0.23023023023023	16.4303296947201\\
0.235235235235235	15.8742738433376\\
0.24024024024024	15.3102689159489\\
0.245245245245245	14.7385768904631\\
0.25025025025025	14.1594642293837\\
0.255255255255255	13.5732017606573\\
0.26026026026026	12.9800645532312\\
0.265265265265265	12.3803317873522\\
0.27027027027027	11.7742866196558\\
0.275275275275275	11.1622160431106\\
0.28028028028028	10.5444107419014\\
0.285285285285285	9.92116494135071\\
0.29029029029029	9.29277625299551\\
0.295295295295295	8.65954551495414\\
0.3003003003003	8.02177662773266\\
0.305305305305305	7.37977638563898\\
0.31031031031031	6.73385430398718\\
0.315315315315315	6.08432244229118\\
0.32032032032032	5.43149522366106\\
0.325325325325325	4.77568925063028\\
0.33033033033033	4.11722311765521\\
0.335335335335335	3.45641722054155\\
0.34034034034034	2.79359356306374\\
0.345345345345345	2.12907556105487\\
0.35035035035035	1.46318784425418\\
0.355355355355355	0.796256056208482\\
0.36036036036036	0.128606652531281\\
0.365365365365365	-0.539433302169716\\
0.37037037037037	-1.20753633838081\\
0.375375375375375	-1.87537478646608\\
0.38038038038038	-2.54262098182373\\
0.385385385385385	-3.20894747009033\\
0.39039039039039	-3.87402721206804\\
0.395395395395395	-4.53753378805091\\
0.4004004004004	-5.19914160122842\\
0.405405405405405	-5.85852607984837\\
0.41041041041041	-6.51536387782565\\
0.415415415415415	-7.16933307348975\\
0.42042042042042	-7.82011336617109\\
0.425425425425425	-8.46738627033457\\
0.43043043043043	-9.11083530697835\\
0.435435435435435	-9.7501461920266\\
0.44044044044044	-10.3850070214562\\
0.445445445445445	-11.0151084529101\\
0.45045045045045	-11.6401438835641\\
0.455455455455455	-12.2598096240259\\
0.46046046046046	-12.8738050680633\\
0.465465465465465	-13.4818328579708\\
0.47047047047047	-14.0835990454018\\
0.475475475475475	-14.6788132475105\\
0.48048048048048	-15.2671887982615\\
0.485485485485485	-15.8484428947866\\
0.49049049049049	-16.4222967386818\\
0.495495495495495	-16.9884756721576\\
0.500500500500501	-17.5467093089713\\
0.505505505505506	-18.0967316600886\\
0.510510510510511	-18.6382812540396\\
0.515515515515516	-19.1711012519482\\
0.520520520520521	-19.6949395572359\\
0.525525525525526	-20.2095489200118\\
0.530530530530531	-20.7146870361805\\
0.535535535535536	-21.2101166413137\\
0.540540540540541	-21.6956055993452\\
0.545545545545546	-22.1709269861648\\
0.550550550550551	-22.6358591681993\\
0.555555555555556	-23.0901858760815\\
0.560560560560561	-23.5336962735207\\
0.565565565565566	-23.9661850214988\\
0.570570570570571	-24.3874523379274\\
0.575575575575576	-24.7973040529094\\
0.580580580580581	-25.1955516597587\\
0.585585585585586	-25.5820123619387\\
0.590590590590591	-25.9565091160874\\
0.595595595595596	-26.3188706713014\\
0.600600600600601	-26.6689316048594\\
0.605605605605606	-27.0065323545659\\
0.610610610610611	-27.3315192479012\\
0.615615615615616	-27.6437445281665\\
0.620620620620621	-27.9430663778126\\
0.625625625625626	-28.2293489391411\\
0.630630630630631	-28.5024623325698\\
0.635635635635636	-28.762282672647\\
0.640640640640641	-29.0086920820033\\
0.645645645645646	-29.2415787034228\\
0.650650650650651	-29.4608367102123\\
0.655655655655656	-29.6663663150449\\
0.660660660660661	-29.8580737774462\\
0.665665665665666	-30.0358714100888\\
0.670670670670671	-30.1996775840502\\
0.675675675675676	-30.3494167331876\\
0.680680680680681	-30.4850193577691\\
0.685685685685686	-30.6064220274989\\
0.690690690690691	-30.7135673840606\\
0.695695695695696	-30.8064041432958\\
0.700700700700701	-30.8848870971259\\
0.705705705705706	-30.9489771153131\\
0.710710710710711	-30.9986411471495\\
0.715715715715716	-31.033852223148\\
0.720720720720721	-31.0545894568027\\
0.725725725725726	-31.0608380464723\\
0.730730730730731	-31.0525892774286\\
0.735735735735736	-31.0298405241025\\
0.740740740740741	-30.9925952525465\\
0.745745745745746	-30.9408630231229\\
0.750750750750751	-30.8746594934117\\
0.755755755755756	-30.7940064213251\\
0.760760760760761	-30.6989316683991\\
0.765765765765766	-30.5894692032253\\
0.770770770770771	-30.4656591049699\\
0.775775775775776	-30.3275475669199\\
0.780780780780781	-30.1751868999816\\
0.785785785785786	-30.0086355360486\\
0.790790790790791	-29.8279580311419\\
0.795795795795796	-29.633225068219\\
0.800800800800801	-29.4245134595346\\
0.805805805805806	-29.2019061484271\\
0.810810810810811	-28.9654922103978\\
0.815815815815816	-28.7153668533371\\
0.820820820820821	-28.4516314167464\\
0.825825825825826	-28.1743933697965\\
0.830830830830831	-27.8837663080533\\
0.835835835835836	-27.5798699486991\\
0.840840840840841	-27.2628301240679\\
0.845845845845846	-26.9327787733106\\
0.850850850850851	-26.5898539319995\\
0.855855855855856	-26.2341997194791\\
0.860860860860861	-25.8659663237663\\
0.865865865865866	-25.4853099838006\\
0.870870870870871	-25.0923929688453\\
0.875875875875876	-24.6873835548375\\
0.880880880880881	-24.2704559974872\\
0.885885885885886	-23.8417905019273\\
0.890890890890891	-23.4015731887159\\
0.895895895895896	-22.9499960559994\\
0.900900900900901	-22.4872569376453\\
0.905905905905906	-22.0135594571628\\
0.910910910910911	-21.5291129772302\\
0.915915915915916	-21.0341325446618\\
0.920920920920921	-20.5288388306488\\
0.925925925925926	-20.0134580661218\\
0.930930930930931	-19.4882219720922\\
0.935935935935936	-18.9533676848376\\
0.940940940940941	-18.4091376758134\\
0.945945945945946	-17.8557796661803\\
0.950950950950951	-17.2935465358544\\
0.955955955955956	-16.7226962270007\\
0.960960960960961	-16.1434916419036\\
0.965965965965966	-15.5562005351664\\
0.970970970970971	-14.9610954002077\\
0.975975975975976	-14.3584533500385\\
0.980980980980981	-13.7485559923223\\
0.985985985985986	-13.1316892987394\\
0.990990990990991	-12.5081434686919\\
0.995995995995996	-11.8782127874086\\
1.001001001001	-11.2421954785247\\
1.00600600600601	-10.6003935512308\\
1.01101101101101	-9.95311264210646\\
1.01601601601602	-9.30066185176875\\
1.02102102102102	-8.64335357648947\\
1.02602602602603	-7.9815033349487\\
1.03103103103103	-7.31542959031343\\
1.03603603603604	-6.645453567846\\
1.04104104104104	-5.9718990682646\\
1.04604604604605	-5.29509227709423\\
1.05105105105105	-4.61536157026198\\
1.05605605605606	-3.93303731620547\\
1.06106106106106	-3.24845167477665\\
1.06606606606607	-2.56193839323639\\
1.07107107107107	-1.8738325996465\\
1.07607607607608	-1.18447059397683\\
1.08108108108108	-0.494189637253894\\
1.08608608608609	0.196672260913901\\
1.09109109109109	0.887776556093234\\
1.0960960960961	1.57878438371803\\
1.1011011011011	2.26935677436608\\
1.10610610610611	2.95915486926678\\
1.11111111111111	3.64784013568559\\
1.11611611611612	4.33507458183107\\
1.12112112112112	5.02052097093233\\
1.12612612612613	5.7038430341378\\
1.13113113113113	6.38470568189095\\
1.13613613613614	7.06277521344471\\
1.14114114114114	7.73771952418361\\
1.14614614614615	8.40920831043159\\
1.15115115115115	9.07691327143306\\
1.15615615615616	9.7405083082066\\
1.16116116116116	10.3996697189823\\
1.16616616616617	11.0540763909476\\
1.17117117117117	11.7034099880412\\
1.17617617617618	12.3473551345485\\
1.18118118118118	12.9855995942709\\
1.18618618618619	13.6178344450542\\
1.19119119119119	14.2437542484833\\
1.1961961961962	14.8630572145648\\
1.2012012012012	15.4754453612399\\
1.20620620620621	16.0806246685873\\
1.21121121121121	16.6783052275962\\
1.21621621621622	17.2682013834084\\
1.22122122122122	17.8500318729467\\
1.22622622622623	18.4235199568674\\
1.23123123123123	18.9883935457948\\
1.23623623623624	19.5443853208102\\
1.24124124124124	20.0912328481923\\
1.24624624624625	20.6286786884203\\
1.25125125125125	21.1564704994686\\
1.25625625625626	21.6743611344425\\
1.26126126126126	22.1821087336166\\
1.26626626626627	22.6794768109574\\
1.27127127127127	23.1662343352235\\
1.27627627627628	23.6421558057539\\
1.28128128128128	24.1070213230664\\
1.28628628628629	24.5606166544017\\
1.29129129129129	25.0027332943606\\
1.2962962962963	25.4331685207929\\
1.3013013013013	25.8517254461038\\
1.30630630630631	26.2582130641569\\
1.31131131131131	26.6524462929568\\
1.31631631631632	27.0342460133026\\
1.32132132132132	27.4034391036099\\
1.32632632632633	27.7598584711029\\
1.33133133133133	28.1033430795818\\
1.33633633633634	28.433737973974\\
1.34134134134134	28.7508943018795\\
1.34634634634635	29.054669332321\\
1.35135135135135	29.3449264719086\\
1.35635635635636	29.6215352786304\\
1.36136136136136	29.8843714734744\\
1.36636636636637	30.1333169500876\\
1.37137137137137	30.3682597826734\\
1.37637637637638	30.5890942323214\\
1.38138138138138	30.7957207519636\\
1.38638638638639	30.9880459901392\\
1.39139139139139	31.1659827937469\\
1.3963963963964	31.3294502099555\\
1.4014014014014	31.4783734874349\\
1.40640640640641	31.6126840770607\\
1.41141141141141	31.7323196322367\\
1.41641641641642	31.8372240089713\\
1.42142142142142	31.9273472658296\\
1.42642642642643	32.002645663877\\
1.43143143143143	32.0630816667149\\
1.43643643643644	32.1086239407012\\
1.44144144144144	32.139247355433\\
1.44644644644645	32.1549329845597\\
1.45145145145145	32.1556681069807\\
1.45645645645646	32.141446208471\\
1.46146146146146	32.1122669837627\\
1.46646646646647	32.0681363390998\\
1.47147147147147	32.0090663952721\\
1.47647647647648	31.9350754911158\\
1.48148148148148	31.8461881874623\\
1.48648648648649	31.7424352714987\\
1.49149149149149	31.6238537614935\\
1.4964964964965	31.4904869118263\\
1.5015015015015	31.3423842182512\\
1.50650650650651	31.1796014233067\\
1.51151151151151	31.0022005217767\\
1.51651651651652	30.810249766095\\
1.52152152152152	30.6038236715707\\
1.52652652652653	30.3830030213069\\
1.53153153153153	30.1478748706688\\
1.53653653653654	29.898532551151\\
1.54154154154154	29.6350756734822\\
1.54654654654655	29.3576101297972\\
1.55155155155155	29.0662480946969\\
1.55655655655656	28.7611080250108\\
1.56156156156156	28.4423146580669\\
1.56656656656657	28.1099990082685\\
1.57157157157157	27.7642983617733\\
1.57657657657658	27.4053562690618\\
1.58158158158158	27.0333225351811\\
1.58658658658659	26.6483532074448\\
1.59159159159159	26.2506105603686\\
1.5965965965966	25.8402630776187\\
1.6016016016016	25.4174854307514\\
1.60660660660661	24.9824584545197\\
1.61161161161161	24.5353691185281\\
1.61661661661662	24.0764104950158\\
1.62162162162162	23.6057817225553\\
1.62662662662663	23.1236879654549\\
1.63163163163163	22.6303403686619\\
1.63663663663664	22.1259560079693\\
1.64164164164164	21.6107578353356\\
1.64664664664665	21.0849746191394\\
1.65165165165165	20.5488408791954\\
1.65665665665666	20.0025968163763\\
1.66166166166166	19.4464882366905\\
1.66666666666667	18.8807664696833\\
1.67167167167167	18.3056882810419\\
1.67667667667668	17.7215157792991\\
1.68168168168168	17.1285163165471\\
1.68668668668669	16.5269623830895\\
1.69169169169169	15.9171314959773\\
1.6966966966967	15.2993060813926\\
1.7017017017017	14.6737733508627\\
1.70670670670671	14.0408251713066\\
1.71171171171171	13.4007579289369\\
1.71671671671672	12.7538723870581\\
1.72172172172172	12.1004735378254\\
1.72672672672673	11.4408704480479\\
1.73173173173173	10.7753760991397\\
1.73673673673674	10.1043072213461\\
1.74174174174174	9.42798412238969\\
1.74674674674675	8.7467305107052\\
1.75175175175175	8.06087331344842\\
1.75675675675676	7.37074248948761\\
1.76176176176176	6.67667083760314\\
1.76676676676677	5.97899380014041\\
1.77177177177177	5.27804926237902\\
1.77677677677678	4.57417734789782\\
1.78178178178178	3.86772021023199\\
1.78678678678679	3.15902182113288\\
1.79179179179179	2.4484277557555\\
1.7967967967968	1.73628497511106\\
1.8018018018018	1.02294160613336\\
1.80680680680681	0.308746719717728\\
1.81181181181181	-0.405949892900137\\
1.81681681681682	-1.12079794505112\\
1.82182182182182	-1.83544688044557\\
1.82682682682683	-2.54954609950831\\
1.83183183183183	-3.26274518584959\\
1.83683683683684	-3.974694132485\\
1.84184184184184	-4.68504356741845\\
1.84684684684685	-5.39344497820442\\
1.85185185185185	-6.09955093511021\\
1.85685685685686	-6.80301531250415\\
1.86186186186186	-7.50349350810324\\
1.86686686686687	-8.20064265972224\\
1.87187187187187	-8.89412185917641\\
1.87687687687688	-9.58359236300154\\
1.88188188188188	-10.268717799668\\
1.88688688688689	-10.9491643729793\\
1.89189189189189	-11.6246010613613\\
1.8968968968969	-12.2946998127641\\
1.9019019019019	-12.9591357349161\\
1.90690690690691	-13.6175872806886\\
1.91191191191191	-14.2697364283465\\
1.91691691691692	-14.915268856483\\
1.92192192192192	-15.5538741134525\\
1.92692692692693	-16.185245781142\\
1.93193193193193	-16.8090816329357\\
1.93693693693694	-17.4250837857549\\
1.94194194194194	-18.0329588460723\\
1.94694694694695	-18.6324180498232\\
1.95195195195195	-19.2231773961567\\
1.95695695695696	-19.8049577749916\\
1.96196196196196	-20.3774850883611\\
1.96696696696697	-20.9404903655528\\
1.97197197197197	-21.4937098720686\\
1.97697697697698	-22.0368852124489\\
1.98198198198198	-22.5697634270251\\
1.98698698698699	-23.0920970826804\\
1.99199199199199	-23.6036443577189\\
1.996996996997	-24.1041691209558\\
2.002002002002	-24.5934410051597\\
2.00700700700701	-25.0712354749911\\
2.01201201201201	-25.537333889596\\
2.01701701701702	-25.9915235600225\\
2.02202202202202	-26.4335978016447\\
2.02702702702703	-26.8633559817839\\
2.03203203203203	-27.280603562729\\
2.03703703703704	-27.6851521403643\\
2.04204204204204	-28.0768194786221\\
2.04704704704705	-28.4554295399795\\
2.05205205205205	-28.820812512228\\
2.05705705705706	-29.1728048317447\\
2.06206206206206	-29.5112492034966\\
2.06706706706707	-29.8359946180142\\
2.07207207207207	-30.1468963655651\\
2.07707707707708	-30.4438160477638\\
2.08208208208208	-30.7266215868476\\
2.08708708708709	-30.9951872328476\\
2.09209209209209	-31.2493935688809\\
2.0970970970971	-31.4891275147831\\
2.1021021021021	-31.7142823292981\\
2.10710710710711	-31.9247576110327\\
2.11211211211211	-32.1204592983783\\
2.11711711711712	-32.3012996685948\\
2.12212212212212	-32.4671973362407\\
2.12712712712713	-32.618077251127\\
2.13213213213213	-32.7538706959609\\
2.13713713713714	-32.8745152838364\\
2.14214214214214	-32.9799549557164\\
2.14714714714715	-33.0701399780397\\
2.15215215215215	-33.1450269405769\\
2.15715715715716	-33.2045787546434\\
2.16216216216216	-33.2487646517665\\
2.16716716716717	-33.2775601828921\\
2.17217217217217	-33.2909472182015\\
2.17717717717718	-33.2889139475946\\
2.18218218218218	-33.2714548818846\\
2.18718718718719	-33.2385708547347\\
2.19219219219219	-33.1902690253503\\
2.1971971971972	-33.1265628819314\\
2.2022022022022	-33.0474722458716\\
2.20720720720721	-32.9530232766777\\
2.21221221221221	-32.8432484775705\\
2.21721721721722	-32.7181867017118\\
2.22222222222222	-32.5778831589909\\
2.22722722722723	-32.4223894232878\\
2.23223223223223	-32.2517634401197\\
2.23723723723724	-32.066069534563\\
2.24224224224224	-31.8653784193287\\
2.24724724724725	-31.6497672028597\\
2.25225225225225	-31.4193193973047\\
2.25725725725726	-31.1741249262112\\
2.26226226226226	-30.9142801317708\\
2.26726726726727	-30.6398877814386\\
2.27227227227227	-30.3510570737367\\
2.27727727727728	-30.0479036430473\\
2.28228228228228	-29.7305495631856\\
2.28728728728729	-29.3991233495408\\
2.29229229229229	-29.053759959563\\
2.2972972972973	-28.6946007913672\\
2.3023023023023	-28.3217936802226\\
2.30730730730731	-27.9354928926876\\
2.31231231231231	-27.5358591181503\\
2.31731731731732	-27.1230594575283\\
2.32232232232232	-26.6972674088831\\
2.32732732732733	-26.2586628497009\\
2.33233233233233	-25.8074320155945\\
2.33733733733734	-25.3437674751806\\
2.34234234234234	-24.8678681008893\\
2.34734734734735	-24.3799390354694\\
2.35235235235235	-23.8801916539538\\
2.35735735735736	-23.3688435208581\\
2.36236236236236	-22.8461183423933\\
2.36736736736737	-22.3122459134792\\
2.37237237237237	-21.7674620593583\\
2.37737737737738	-21.2120085716179\\
2.38238238238238	-20.6461331384428\\
2.38738738738739	-20.0700892689318\\
2.39239239239239	-19.4841362113274\\
2.3973973973974	-18.8885388650222\\
2.4024024024024	-18.2835676862242\\
2.40740740740741	-17.6694985871762\\
2.41241241241241	-17.0466128288489\\
2.41741741741742	-16.4151969070406\\
2.42242242242242	-15.7755424318417\\
2.42742742742743	-15.1279460004383\\
2.43243243243243	-14.4727090632549\\
2.43743743743744	-13.8101377834538\\
2.44244244244244	-13.1405428898356\\
2.44744744744745	-12.4642395232046\\
2.45245245245245	-11.7815470762868\\
2.45745745745746	-11.0927890273131\\
2.46246246246246	-10.3982927673996\\
2.46746746746747	-9.69838942188411\\
2.47247247247247	-8.99341366579805\\
2.47747747747748	-8.28370353367566\\
2.48248248248248	-7.56960022392549\\
2.48748748748749	-6.85144789800948\\
2.49249249249249	-6.12959347469581\\
2.4974974974975	-5.4043864196718\\
2.5025025025025	-4.67617853082147\\
2.50750750750751	-3.94532371949065\\
2.51251251251251	-3.21217778807876\\
2.51751751751752	-2.47709820431197\\
2.52252252252252	-1.74044387256626\\
2.52752752752753	-1.00257490262144\\
2.53253253253253	-0.263852376238323\\
2.53753753753754	0.47536188803958\\
2.54254254254254	1.2147055714506\\
2.54754754754755	1.95381609253047\\
2.55255255255255	2.69233084515949\\
2.55755755755756	3.42988743671924\\
2.56256256256256	4.16612392593628\\
2.56756756756757	4.90067905999131\\
2.57257257257257	5.63319251047511\\
2.57757757757758	6.36330510777742\\
2.58258258258258	7.0906590735013\\
2.58758758758759	7.81489825050362\\
2.59259259259259	8.53566833017229\\
2.5975975975976	9.25261707656216\\
2.6026026026026	9.96539454702462\\
2.60760760760761	10.6736533089803\\
2.61261261261261	11.3770486524998\\
2.61761761761762	12.0752387983756\\
2.62262262262262	12.7678851013843\\
2.62762762762763	13.4546522484604\\
2.63263263263263	14.1352084515216\\
2.63763763763764	14.8092256347056\\
2.64264264264264	15.4763796158039\\
2.64764764764765	16.1363502816955\\
2.65265265265265	16.7888217576119\\
2.65765765765766	17.433482570082\\
2.66266266266266	18.0700258034356\\
2.66766766766767	18.6981492497607\\
2.67267267267267	19.3175555522407\\
2.67767767767768	19.927952341814\\
2.68268268268268	20.5290523671274\\
2.68768768768769	21.1205736177739\\
2.69269269269269	21.7022394408296\\
2.6976976976977	22.273778650725\\
2.7027027027027	22.8349256325089\\
2.70770770770771	23.3854204385811\\
2.71271271271271	23.9250088789914\\
2.71771771771772	24.4534426054215\\
2.72272272272272	24.9704791889807\\
2.72772772772773	25.4758821919678\\
2.73273273273273	25.969421233764\\
2.73773773773774	26.4508720510355\\
2.74274274274274	26.9200165524411\\
2.74774774774775	27.3766428680486\\
2.75275275275275	27.8205453936774\\
2.75775775775776	28.2515248303922\\
2.76276276276276	28.669388219383\\
2.76776776776777	29.0739489724726\\
2.77277277277277	29.4650268984994\\
2.77777777777778	29.8424482258273\\
2.78278278278278	30.2060456212385\\
2.78778778778779	30.5556582054682\\
2.79279279279279	30.8911315656389\\
2.7977977977978	31.2123177648549\\
2.8028028028028	31.5190753492137\\
2.80780780780781	31.8112693524917\\
2.81281281281281	32.0887712987556\\
2.81781781781782	32.3514592031475\\
2.82282282282282	32.5992175710886\\
2.82782782782783	32.8319373961362\\
2.83283283283283	33.0495161567248\\
2.83783783783784	33.2518578120135\\
2.84284284284284	33.438872797052\\
2.84784784784785	33.6104780174695\\
2.85285285285285	33.7665968438798\\
2.85785785785786	33.9071591061854\\
2.86286286286286	34.0321010879517\\
2.86786786786787	34.1413655210108\\
2.87287287287287	34.2349015804419\\
2.87787787787788	34.3126648800606\\
2.88288288288288	34.3746174685398\\
2.88788788788789	34.4207278262667\\
2.89289289289289	34.450970863029\\
2.8978978978979	34.4653279166085\\
2.9029029029029	34.463786752345\\
2.90790790790791	34.446341563719\\
2.91291291291291	34.4129929739859\\
2.91791791791792	34.3637480388814\\
2.92292292292292	34.2986202503985\\
2.92792792792793	34.2176295416261\\
2.93293293293293	34.1208022926199\\
2.93793793793794	34.0081713372635\\
2.94294294294294	33.8797759710604\\
2.94794794794795	33.7356619597857\\
2.95295295295295	33.5758815489079\\
2.95795795795796	33.400493473679\\
2.96296296296296	33.2095629697765\\
2.96796796796797	33.003161784365\\
2.97297297297297	32.7813681874356\\
2.97797797797798	32.5442669832622\\
2.98298298298298	32.2919495218074\\
2.98798798798799	32.0245137098933\\
2.99299299299299	31.7420640219434\\
2.997997997998	31.444711510091\\
3.003003003003	31.132573813437\\
3.00800800800801	30.8057751662327\\
3.01301301301301	30.4644464047522\\
3.01801801801802	30.1087249726128\\
3.02302302302302	29.7387549242915\\
3.02802802802803	29.3546869265845\\
3.03303303303303	28.9566782577441\\
3.03803803803804	28.5448928040299\\
3.04304304304304	28.1195010534026\\
3.04804804804805	27.6806800860901\\
3.05305305305305	27.228613561751\\
3.05805805805806	26.7634917029642\\
3.06306306306306	26.2855112747707\\
3.06806806806807	25.7948755600002\\
3.07307307307307	25.2917943301146\\
3.07807807807808	24.7764838113082\\
3.08308308308308	24.2491666456109\\
3.08808808808809	23.7100718467458\\
3.09309309309309	23.1594347505033\\
3.0980980980981	22.5974969594044\\
3.1031031031031	22.0245062814359\\
3.10810810810811	21.4407166626543\\
3.11311311311311	20.8463881134684\\
3.11811811811812	20.2417866284261\\
3.12312312312312	19.6271840993491\\
3.12812812812813	19.0028582216733\\
3.13313313313313	18.369092393876\\
3.13813813813814	17.7261756098873\\
3.14314314314314	17.0744023444056\\
3.14814814814815	16.41407243106\\
3.15315315315315	15.7454909333817\\
3.15815815815816	15.0689680085741\\
3.16316316316316	14.384818764092\\
3.16816816816817	13.6933631070668\\
3.17317317317317	12.9949255866388\\
3.17817817817818	12.2898352292828\\
3.18318318318318	11.57842536724\\
3.18818818818819	10.8610334601935\\
3.19319319319319	10.1380009103494\\
3.1981981981982	9.40967287111439\\
3.2032032032032	8.67639804958038\\
3.20820820820821	7.93852850305585\\
3.21321321321321	7.1964194299047\\
3.21821821821822	6.45042895497724\\
3.22322322322322	5.70091790994009\\
3.22822822822823	4.94824960883294\\
3.23323323323323	4.19278961919975\\
3.23823823823824	3.43490552916116\\
3.24324324324324	2.67496671081165\\
3.24824824824825	1.91334408034115\\
3.25325325325325	1.15040985529491\\
3.25825825825826	0.386537309397754\\
3.26326326326326	-0.377899474620128\\
3.26826826826827	-1.14252585375008\\
3.27327327327327	-1.90696687453164\\
3.27827827827828	-2.67084752352109\\
3.28328328328328	-3.43379297794091\\
3.28828828828829	-4.19542885604817\\
3.29329329329329	-4.95538146676077\\
3.2982982982983	-5.71327805808357\\
3.3033033033033	-6.46874706388111\\
3.30830830830831	-7.22141834855096\\
3.31331331331331	-7.97092344916032\\
3.31831831831832	-8.71689581461917\\
3.32332332332332	-9.45897104147601\\
3.32832832832833	-10.1967871059363\\
3.33333333333333	-10.9299845917195\\
3.33833833833834	-11.6582069133886\\
3.34334334334334	-12.3811005348034\\
3.34834834834835	-13.0983151823716\\
3.35335335335335	-13.8095040527893\\
3.35835835835836	-14.5143240149895\\
3.36336336336336	-15.2124358060363\\
3.36836836836837	-15.903504220729\\
3.37337337337337	-16.5871982947056\\
3.37837837837838	-17.2631914808576\\
3.38338338338338	-17.931161818898\\
3.38838838838839	-18.5907920979465\\
3.39339339339339	-19.2417700120232\\
3.3983983983984	-19.8837883083708\\
3.4034034034034	-20.5165449285458\\
3.40840840840841	-21.1397431422503\\
3.41341341341341	-21.7530916738975\\
3.41841841841842	-22.3563048219302\\
3.42342342342342	-22.9491025709354\\
3.42842842842843	-23.5312106966212\\
3.43343343343343	-24.1023608637443\\
3.43843843843844	-24.6622907170996\\
3.44344344344344	-25.2107439657003\\
3.44844844844845	-25.7474704603011\\
3.45345345345345	-26.272226264431\\
3.45845845845846	-26.784773719122\\
3.46346346346346	-27.2848815015346\\
3.46846846846847	-27.7723246776956\\
3.47347347347347	-28.2468847495779\\
3.47847847847848	-28.7083496967622\\
3.48348348348348	-29.156514012932\\
3.48848848848849	-29.5911787374633\\
3.49349349349349	-30.0121514823751\\
3.4984984984985	-30.4192464549182\\
3.5035035035035	-30.8122844760787\\
3.50850850850851	-31.1910929952823\\
3.51351351351351	-31.5555061015832\\
3.51851851851852	-31.9053645316258\\
3.52352352352352	-32.2405156746651\\
3.52852852852853	-32.5608135749317\\
3.53353353353353	-32.8661189316238\\
3.53853853853854	-33.1562990968063\\
3.54354354354354	-33.4312280714901\\
3.54854854854855	-33.6907865001615\\
3.55355355355355	-33.934861664023\\
3.55855855855856	-34.1633474731995\\
3.56356356356356	-34.3761444581551\\
3.56856856856857	-34.5731597605572\\
3.57357357357357	-34.7543071238124\\
3.57857857857858	-34.9195068834896\\
3.58358358358358	-35.0686859578319\\
3.58858858858859	-35.2017778385486\\
3.59359359359359	-35.3187225820639\\
3.5985985985986	-35.4194668013859\\
3.6036036036036	-35.5039636587462\\
3.60860860860861	-35.5721728591429\\
3.61361361361361	-35.6240606449102\\
3.61861861861862	-35.6595997914168\\
3.62362362362362	-35.6787696039822\\
3.62862862862863	-35.681555916086\\
3.63363363363363	-35.667951088925\\
3.63863863863864	-35.6379540123598\\
3.64364364364364	-35.5915701072753\\
3.64864864864865	-35.5288113293623\\
3.65365365365365	-35.4496961743111\\
3.65865865865866	-35.3542496843925\\
3.66366366366366	-35.2425034563835\\
3.66866866866867	-35.1144956507789\\
3.67367367367367	-34.9702710022148\\
3.67867867867868	-34.8098808310123\\
3.68368368368368	-34.6333830557346\\
3.68868868868869	-34.4408422066352\\
3.69369369369369	-34.2323294398582\\
3.6986986986987	-34.0079225522394\\
3.7037037037037	-33.7677059965386\\
3.70870870870871	-33.5117708969234\\
3.71371371371371	-33.2402150645071\\
3.71871871871872	-32.9531430127346\\
3.72372372372372	-32.6506659723937\\
3.72872872872873	-32.3329019060197\\
3.73373373373373	-31.9999755214503\\
3.73873873873874	-31.6520182842773\\
3.74374374374374	-31.2891684289294\\
3.74874874874875	-30.9115709681184\\
3.75375375375375	-30.5193777003654\\
3.75875875875876	-30.1127472153238\\
3.76376376376376	-29.6918448966071\\
3.76876876876877	-29.2568429218241\\
3.77377377377377	-28.8079202595249\\
3.77877877877878	-28.3452626627543\\
3.78378378378378	-27.8690626589117\\
3.78878878878879	-27.3795195356154\\
3.79379379379379	-26.8768393222722\\
3.7987987987988	-26.3612347670542\\
3.8038038038038	-25.8329253089929\\
3.80880880880881	-25.2921370449014\\
3.81381381381381	-24.7391026908496\\
3.81881881881882	-24.1740615379192\\
3.82382382382382	-23.5972594019807\\
3.82882882882883	-23.008948567244\\
3.83383383383383	-22.4093877233482\\
3.83883883883884	-21.7988418957701\\
3.84384384384384	-21.177582369348\\
3.84884884884885	-20.5458866047341\\
3.85385385385385	-19.9040381476069\\
3.85885885885886	-19.2523265304982\\
3.86386386386386	-18.5910471671046\\
3.86886886886887	-17.9205012389837\\
3.87387387387387	-17.2409955745507\\
3.87887887887888	-16.5528425203218\\
3.88388388388388	-15.8563598043728\\
3.88888888888889	-15.1518703920088\\
3.89389389389389	-14.4397023336666\\
3.8988988988989	-13.7201886051004\\
3.9039039039039	-12.9936669399283\\
3.90890890890891	-12.2604796546441\\
3.91391391391391	-11.5209734662302\\
3.91891891891892	-10.7754993025316\\
3.92392392392392	-10.0244121055832\\
3.92892892892893	-9.26807062810623\\
3.93393393393393	-8.50683722342182\\
3.93893893893894	-7.74107762905078\\
3.94394394394394	-6.97116074429985\\
3.94894894894895	-6.19745840215615\\
3.95395395395395	-5.42034513583728\\
3.95895895895896	-4.64019794036645\\
3.96396396396396	-3.85739602956388\\
3.96896896896897	-3.07232058886516\\
3.97397397397397	-2.28535452439595\\
3.97897897897898	-1.49688220874848\\
3.98398398398398	-0.707289223920407\\
3.98898898898899	0.0830378981110439\\
3.99399399399399	0.873711936695724\\
3.998998998999	1.66434524520564\\
4.004004004004	2.45455001461956\\
4.00900900900901	3.24393853749402\\
4.01401401401401	4.03212347181558\\
4.01901901901902	4.81871810422933\\
4.02402402402402	5.60333661214129\\
4.02902902902903	6.38559432419678\\
4.03403403403403	7.16510797864406\\
4.03903903903904	7.94149597910129\\
4.04404404404404	8.71437864725638\\
4.04904904904905	9.48337847204213\\
4.05405405405405	10.2481203548446\\
4.05905905905906	11.0082318503196\\
4.06406406406406	11.76334340241\\
4.06906906906907	12.5130885751796\\
4.07407407407407	13.2571042780975\\
4.07907907907908	13.9950309854331\\
4.08408408408408	14.7265129494456\\
4.08908908908909	15.4511984070763\\
4.09409409409409	16.1687397798806\\
4.0990990990991	16.8787938669618\\
4.1041041041041	17.5810220306986\\
4.10910910910911	18.275090375084\\
4.11411411411411	18.9606699165264\\
4.11911911911912	19.6374367469862\\
4.12412412412412	20.3050721893572\\
4.12912912912913	20.9632629450237\\
4.13413413413413	21.611701233559\\
4.13913913913914	22.2500849245536\\
4.14414414414414	22.878117661592\\
4.14914914914915	23.4955089784233\\
4.15415415415415	24.1019744073944\\
4.15915915915916	24.6972355802431\\
4.16416416416416	25.2810203213682\\
4.16916916916917	25.853062733721\\
4.17417417417417	26.4131032774797\\
4.17917917917918	26.9608888416917\\
4.18418418418418	27.496172809087\\
4.18918918918919	28.0187151142822\\
4.19419419419419	28.5282822956129\\
4.1991991991992	29.0246475408448\\
4.2042042042042	29.5075907270298\\
4.20920920920921	29.9768984547818\\
4.21421421421421	30.43236407726\\
4.21921921921922	30.873787724155\\
4.22422422422422	31.3009763209801\\
4.22922922922923	31.7137436039756\\
4.23423423423423	32.1119101309404\\
4.23923923923924	32.4953032883047\\
4.24424424424424	32.8637572947615\\
4.24924924924925	33.2171132017741\\
4.25425425425425	33.5552188912741\\
4.25925925925926	33.8779290708647\\
4.26426426426426	34.1851052668369\\
4.26926926926927	34.4766158153044\\
4.27427427427427	34.7523358517535\\
4.27927927927928	35.0121472993005\\
4.28428428428428	35.2559388559384\\
4.28928928928929	35.483605981046\\
4.29429429429429	35.6950508814227\\
4.2992992992993	35.8901824971014\\
4.3043043043043	36.0689164871789\\
4.30930930930931	36.2311752158918\\
4.31431431431431	36.3768877391518\\
4.31931931931932	36.5059897917409\\
4.32432432432432	36.6184237753518\\
4.32932932932933	36.7141387476436\\
4.33433433433433	36.7930904124683\\
4.33933933933934	36.8552411114066\\
4.34434434434434	36.9005598167346\\
4.34934934934935	36.9290221259277\\
4.35435435435435	36.9406102577893\\
4.35935935935936	36.9353130502762\\
4.36436436436436	36.9131259600725\\
4.36936936936937	36.874051063949\\
4.37437437437437	36.8180970619251\\
4.37937937937938	36.7452792822336\\
4.38438438438438	36.6556196880691\\
4.38938938938939	36.5491468860855\\
4.39439439439439	36.4258961365871\\
4.3993993993994	36.2859093653425\\
4.4044044044044	36.1292351769315\\
4.40940940940941	35.9559288695184\\
4.41441441441441	35.7660524509272\\
4.41941941941942	35.5596746558788\\
4.42442442442442	35.3368709642315\\
4.42942942942943	35.0977236200524\\
4.43443443443443	34.8423216513294\\
4.43943943943944	34.5707608901203\\
4.44444444444444	34.2831439929192\\
4.44944944944945	33.979580461007\\
4.45445445445445	33.66018666054\\
4.45945945945946	33.3250858421171\\
4.46446446446446	32.9744081595544\\
4.46946946946947	32.6082906875852\\
4.47447447447447	32.226877438194\\
4.47947947947948	31.8303193752814\\
4.48448448448448	31.4187744273527\\
4.48948948948949	30.9924074979115\\
4.49449449449449	30.551390473238\\
4.4994994994995	30.0959022272227\\
4.5045045045045	29.6261286229273\\
4.50950950950951	29.1422625105375\\
4.51451451451451	28.6445037213773\\
4.51951951951952	28.133059057649\\
4.52452452452452	27.6081422775707\\
4.52952952952953	27.069974075582\\
4.53453453453453	26.5187820572987\\
4.53953953953954	25.9548007088978\\
4.54454454454454	25.3782713606292\\
4.54954954954955	24.7894421441545\\
4.55455455455455	24.1885679434289\\
4.55955955955956	23.5759103388528\\
4.56456456456456	22.9517375444356\\
4.56956956956957	22.3163243377303\\
4.57457457457457	21.6699519823152\\
4.57957957957958	21.0129081426184\\
4.58458458458458	20.3454867909015\\
4.58958958958959	19.6679881062419\\
4.59459459459459	18.9807183653759\\
4.5995995995996	18.2839898252892\\
4.6046046046046	17.5781205974707\\
4.60960960960961	16.8634345137671\\
4.61461461461461	16.1402609838099\\
4.61961961961962	15.4089348440112\\
4.62462462462462	14.6697961981563\\
4.62962962962963	13.9231902496518\\
4.63463463463463	13.1694671255167\\
4.63963963963964	12.4089816922379\\
4.64464464464464	11.6420933636401\\
4.64964964964965	10.8691659009533\\
4.65465465465465	10.0905672052908\\
4.65965965965966	9.30666910278249\\
4.66466466466466	8.51784712263631\\
4.66966966966967	7.72448026843287\\
4.67467467467467	6.92695078298429\\
4.67967967967968	6.12564390711713\\
4.68468468468468	5.32094763276495\\
4.68968968968969	4.513252450781\\
4.69469469469469	3.70295109390463\\
4.6996996996997	2.89043827533669\\
4.7047047047047	2.0761104233988\\
4.70970970970971	1.26036541276903\\
4.71471471471471	0.443602292802315\\
4.71971971971972	-0.373778986542907\\
4.72472472472472	-1.19137785063893\\
4.72972972972973	-2.00879337744249\\
4.73473473473473	-2.82562457551996\\
4.73973973973974	-3.64147066229248\\
4.74474474474474	-4.45593134194774\\
4.74974974974975	-5.26860708246669\\
4.75475475475475	-6.07909939121675\\
4.75975975975976	-6.88701108856948\\
4.76476476476476	-7.69194657900921\\
4.76976976976977	-8.49351211921003\\
4.77477477477477	-9.29131608257198\\
4.77977977977978	-10.0849692197228\\
4.78478478478478	-10.874084914509\\
4.78978978978979	-11.6582794350207\\
4.79479479479479	-12.4371721792143\\
4.7997997997998	-13.2103859147232\\
4.8048048048048	-13.9775470124689\\
4.80980980980981	-14.7382856737133\\
4.81481481481481	-15.4922361502204\\
4.81981981981982	-16.2390369572224\\
4.82482482482482	-16.9783310789189\\
4.82982982982983	-17.7097661662645\\
4.83483483483483	-18.4329947268338\\
4.83983983983984	-19.1476743065835\\
4.84484484484484	-19.8534676633629\\
4.84984984984985	-20.5500429320565\\
4.85485485485485	-21.2370737812739\\
4.85985985985986	-21.9142395615346\\
4.86486486486486	-22.5812254449235\\
4.86986986986987	-23.2377225562281\\
4.87487487487487	-23.883428095594\\
4.87987987987988	-24.5180454527658\\
4.88488488488488	-25.1412843130093\\
4.88988988988989	-25.752860754837\\
4.89489489489489	-26.3524973396832\\
4.8998998998999	-26.9399231937022\\
4.9049049049049	-27.5148740818837\\
4.90990990990991	-28.0770924747024\\
4.91491491491491	-28.6263276075373\\
4.91991991991992	-29.1623355331171\\
4.92492492492492	-29.6848791672627\\
4.92992992992993	-30.1937283282126\\
4.93493493493493	-30.6886597698349\\
4.93993993993994	-31.1694572090348\\
4.94494494494494	-31.6359113476823\\
4.94994994994995	-32.087819889391\\
4.95495495495495	-32.5249875514862\\
4.95995995995996	-32.9472260725045\\
4.96496496496496	-33.354354215574\\
4.96996996996997	-33.7461977680232\\
4.97497497497497	-34.1225895375679\\
4.97997997997998	-34.4833693454268\\
4.98498498498498	-34.8283840167114\\
4.98998998998999	-35.1574873684331\\
4.99499499499499	-35.4705401954668\\
5	-35.7674102548017\\
};
\addlegendentry{Euler}

\addplot [color=mycolor2, line width=2.0pt]
  table[row sep=crcr]{%
0	30\\
0.005005005005005	29.9929602819251\\
0.01001001001001	29.9718441241212\\
0.015015015015015	29.9366605152566\\
0.02002002002002	29.8874247268826\\
0.025025025025025	29.8241580736137\\
0.03003003003003	29.7468871396061\\
0.035035035035035	29.6556445617156\\
0.04004004004004	29.5504690498857\\
0.045045045045045	29.4314053871474\\
0.0500500500500501	29.2985044296192\\
0.0550550550550551	29.151823106507\\
0.0600600600600601	28.9914244201044\\
0.0650650650650651	28.8173774457923\\
0.0700700700700701	28.6297573320391\\
0.0750750750750751	28.4286453004009\\
0.0800800800800801	28.2141286471041\\
0.0850850850850851	27.9863089039131\\
0.0900900900900901	27.7452910914274\\
0.0950950950950951	27.4911774197375\\
0.1001001001001	27.2240760190272\\
0.105105105105105	26.9441009395741\\
0.11011011011011	26.6513721517495\\
0.115115115115115	26.3460155460181\\
0.12012012012012	26.0281629329386\\
0.125125125125125	25.6979520431631\\
0.13013013013013	25.3555265274375\\
0.135135135135135	25.0010359566011\\
0.14014014014014	24.6346358215872\\
0.145145145145145	24.2564875334225\\
0.15015015015015	23.8667584232274\\
0.155155155155155	23.465621742216\\
0.16016016016016	23.0532566616959\\
0.165165165165165	22.6298482730687\\
0.17017017017017	22.1955875878292\\
0.175175175175175	21.7506715375662\\
0.18018018018018	21.2953029739619\\
0.185185185185185	20.8296906687924\\
0.19019019019019	20.3540493139271\\
0.195195195195195	19.8685995213295\\
0.2002002002002	19.3735678230564\\
0.205205205205205	18.8691866712584\\
0.21021021021021	18.3556944381796\\
0.215215215215215	17.833335416158\\
0.22022022022022	17.302359817625\\
0.225225225225225	16.7630237751057\\
0.23023023023023	16.2155893412191\\
0.235235235235235	15.6603244886775\\
0.24024024024024	15.0975031102871\\
0.245245245245245	14.5273330204672\\
0.25025025025025	13.950010690021\\
0.255255255255255	13.3658350930906\\
0.26026026026026	12.7751065699675\\
0.265265265265265	12.1781263209965\\
0.27027027027027	11.5751964065763\\
0.275275275275275	10.9666197471587\\
0.28028028028028	10.3527001232494\\
0.285285285285285	9.73374217540739\\
0.29029029029029	9.11005140424526\\
0.295295295295295	8.48193417042912\\
0.3003003003003	7.84969769467861\\
0.305305305305305	7.21365005776689\\
0.31031031031031	6.57410020052063\\
0.315315315315315	5.93135792382005\\
0.32032032032032	5.28573388859886\\
0.325325325325325	4.63753961584434\\
0.33033033033033	3.98708748659724\\
0.335335335335335	3.33469074195188\\
0.34034034034034	2.68066348305609\\
0.345345345345345	2.0253206711112\\
0.35035035035035	1.3689781273721\\
0.355355355355355	0.711952533147186\\
0.36036036036036	0.054561429798367\\
0.365365365365365	-0.602876781258884\\
0.37037037037037	-1.26004283855562\\
0.375375375375375	-1.91661662056933\\
0.38038038038038	-2.57227714572402\\
0.385385385385385	-3.22670257239012\\
0.39039039039039	-3.87957019888458\\
0.395395395395395	-4.53055646347081\\
0.4004004004004	-5.17933694435868\\
0.405405405405405	-5.82558635970454\\
0.41041041041041	-6.46897856761128\\
0.415415415415415	-7.10918656612816\\
0.42042042042042	-7.74588249325099\\
0.425425425425425	-8.37873762692201\\
0.43043043043043	-9.00742238503002\\
0.435435435435435	-9.63170678648987\\
0.44044044044044	-10.2513984907315\\
0.445445445445445	-10.8661712242719\\
0.45045045045045	-11.4757037095625\\
0.455455455455455	-12.0796805424817\\
0.46046046046046	-12.6777921923344\\
0.465465465465465	-13.2697350018528\\
0.47047047047047	-13.8552111871959\\
0.475475475475475	-14.4339288379492\\
0.48048048048048	-15.0056019171257\\
0.485485485485485	-15.5699502611648\\
0.49049049049049	-16.1266995799331\\
0.495495495495495	-16.6755814567238\\
0.500500500500501	-17.2163333482572\\
0.505505505505506	-17.7486985846804\\
0.510510510510511	-18.2724263695675\\
0.515515515515516	-18.7872717799192\\
0.520520520520521	-19.2929957661634\\
0.525525525525526	-19.7893651521548\\
0.530530530530531	-20.2761526351748\\
0.535535535535536	-20.7531367859319\\
0.540540540540541	-21.2201020485615\\
0.545545545545546	-21.6768387406257\\
0.550550550550551	-22.1231430531136\\
0.555555555555556	-22.5588170504412\\
0.560560560560561	-22.9836686704514\\
0.565565565565566	-23.3975117244138\\
0.570570570570571	-23.8001658970251\\
0.575575575575576	-24.1914567464089\\
0.580580580580581	-24.5712157041155\\
0.585585585585586	-24.9392800751222\\
0.590590590590591	-25.2954930378331\\
0.595595595595596	-25.6397036440794\\
0.600600600600601	-25.971766819119\\
0.605605605605606	-26.2915433616367\\
0.610610610610611	-26.5989009736032\\
0.615615615615616	-26.8937566348966\\
0.620620620620621	-27.1759907228618\\
0.625625625625626	-27.4454585322147\\
0.630630630630631	-27.7020225738436\\
0.635635635635636	-27.9455525748091\\
0.640640640640641	-28.1759254783442\\
0.645645645645646	-28.3930254438541\\
0.650650650650651	-28.5967438469165\\
0.655655655655656	-28.7869792792813\\
0.660660660660661	-28.9636375488709\\
0.665665665665666	-29.1266316797798\\
0.670670670670671	-29.2758819122751\\
0.675675675675676	-29.411315702796\\
0.680680680680681	-29.5328677239543\\
0.685685685685686	-29.6404798645338\\
0.690690690690691	-29.7341012294909\\
0.695695695695696	-29.8136881399544\\
0.700700700700701	-29.8792041332251\\
0.705705705705706	-29.9306199627764\\
0.710710710710711	-29.967913598254\\
0.715715715715716	-29.9910702254758\\
0.720720720720721	-30.0000822464324\\
0.725725725725726	-29.9949492792862\\
0.730730730730731	-29.9756781583724\\
0.735735735735736	-29.9422829341982\\
0.740740740740741	-29.8947848734435\\
0.745745745745746	-29.8332124589602\\
0.750750750750751	-29.7576013897727\\
0.755755755755756	-29.6679945810776\\
0.760760760760761	-29.5644421642442\\
0.765765765765766	-29.4470014868136\\
0.770770770770771	-29.3157371124997\\
0.775775775775776	-29.1707208211885\\
0.780780780780781	-29.0120316089384\\
0.785785785785786	-28.8397556879801\\
0.790790790790791	-28.6539864867167\\
0.795795795795796	-28.4548246497236\\
0.800800800800801	-28.2423623914311\\
0.805805805805806	-28.0166195197252\\
0.810810810810811	-27.7776777435492\\
0.815815815815816	-27.5256362729774\\
0.820820820820821	-27.2606003201222\\
0.825825825825826	-26.982681099134\\
0.830830830830831	-26.6919958262008\\
0.835835835835836	-26.3886677195485\\
0.840840840840841	-26.0728259994409\\
0.845845845845846	-25.7446058881797\\
0.850850850850851	-25.4041486101044\\
0.855855855855856	-25.0516013915923\\
0.860860860860861	-24.6871174610586\\
0.865865865865866	-24.3108560489565\\
0.870870870870871	-23.9229823877767\\
0.875875875875876	-23.523667712048\\
0.880880880880881	-23.1130892583372\\
0.885885885885886	-22.6914302652485\\
0.890890890890891	-22.2588799734244\\
0.895895895895896	-21.8156336255451\\
0.900900900900901	-21.3618924663284\\
0.905905905905906	-20.8978637425304\\
0.910910910910911	-20.4237607029447\\
0.915915915915916	-19.9398025984029\\
0.920920920920921	-19.4462146817745\\
0.925925925925926	-18.9432282079666\\
0.930930930930931	-18.4310804339246\\
0.935935935935936	-17.9100146186313\\
0.940940940940941	-17.3802800231076\\
0.945945945945946	-16.8421319104121\\
0.950950950950951	-16.2958315456415\\
0.955955955955956	-15.7416461959301\\
0.960960960960961	-15.1798491304502\\
0.965965965965966	-14.6107196204118\\
0.970970970970971	-14.034542939063\\
0.975975975975976	-13.4516103616895\\
0.980980980980981	-12.8621581606901\\
0.985985985985986	-12.2663400810497\\
0.990990990990991	-11.6644617720435\\
0.995995995995996	-11.056836577758\\
1.001001001001	-10.443778033014\\
1.00600600600601	-9.82559986336624\\
1.01101101101101	-9.20261598510381\\
1.01601601601602	-8.57514050524988\\
1.02102102102102	-7.94348772156183\\
1.02602602602603	-7.30797212253119\\
1.03103103103103	-6.66890838738368\\
1.03603603603604	-6.02661138607921\\
1.04104104104104	-5.38139617931182\\
1.04604604604605	-4.73357801850977\\
1.05105105105105	-4.08347234583546\\
1.05605605605606	-3.43139479418549\\
1.06106106106106	-2.77766118719062\\
1.06606606606607	-2.12258753921578\\
1.07107107107107	-1.4664900553601\\
1.07607607607608	-0.809685131456835\\
1.08108108108108	-0.15248935407346\\
1.08608608608609	0.504780499488386\\
1.09109109109109	1.1618074611929\\
1.0960960960961	1.81827437227012\\
1.1011011011011	2.47386388321585\\
1.10610610610611	3.12825845379179\\
1.11111111111111	3.78114035302544\\
1.11611611611612	4.43219165921013\\
1.12112112112112	5.08109425990496\\
1.12612612612613	5.72752985193495\\
1.13113113113113	6.37117994139089\\
1.13613613613614	7.01172584362939\\
1.14114114114114	7.64884868327292\\
1.14614614614615	8.28222939420977\\
1.15115115115115	8.91154871959403\\
1.15615615615616	9.5364872118456\\
1.16116116116116	10.1567252326503\\
1.16616616616617	10.7719429529596\\
1.17117117117117	11.3818575436004\\
1.17617617617618	11.9863098319142\\
1.18118118118118	12.5850022495247\\
1.18618618618619	13.1776247340236\\
1.19119119119119	13.7638733145762\\
1.1961961961962	14.3434501119214\\
1.2012012012012	14.9160633383715\\
1.20620620620621	15.4814272978124\\
1.21121121121121	16.0392623857035\\
1.21621621621622	16.5892950890776\\
1.22122122122122	17.1312579865411\\
1.22622622622623	17.664889748274\\
1.23123123123123	18.1899351360297\\
1.23623623623624	18.706145003135\\
1.24124124124124	19.2132762944904\\
1.24624624624625	19.7110920465699\\
1.25125125125125	20.1993613874209\\
1.25625625625626	20.6778595366644\\
1.26126126126126	21.1463678054949\\
1.26626626626627	21.6046735966803\\
1.27127127127127	22.0525704045622\\
1.27627627627628	22.4898578150555\\
1.28128128128128	22.9163415056488\\
1.28628628628629	23.3318332454042\\
1.29129129129129	23.7361508949571\\
1.2962962962963	24.1291184065167\\
1.3013013013013	24.5105658238654\\
1.30630630630631	24.8803292823594\\
1.31131131131131	25.2382510089283\\
1.31631631631632	25.5841793220751\\
1.32132132132132	25.9179686318765\\
1.32632632632633	26.2394794399826\\
1.33133133133133	26.548578339617\\
1.33633633633634	26.8451380155768\\
1.34134134134134	27.1290391441431\\
1.34634634634635	27.4002116547022\\
1.35135135135135	27.6585419510763\\
1.35635635635636	27.9038961877476\\
1.36136136136136	28.1361477425412\\
1.36636636636637	28.3551772166259\\
1.37137137137137	28.5608724345133\\
1.37637637637638	28.7531284440586\\
1.38138138138138	28.9318475164599\\
1.38638638638639	29.0969391462589\\
1.39139139139139	29.2483200513402\\
1.3963963963964	29.3859141729318\\
1.4014014014014	29.5096526756049\\
1.40640640640641	29.619473947274\\
1.41141141141141	29.7153235991967\\
1.41641641641642	29.797154465974\\
1.42142142142142	29.86492660555\\
1.42642642642643	29.9186072992122\\
1.43143143143143	29.958171051591\\
1.43643643643644	29.9835995906604\\
1.44144144144144	29.9948818677375\\
1.44644644644645	29.9920140574827\\
1.45145145145145	29.9749995578994\\
1.45645645645646	29.9438489903345\\
1.46146146146146	29.8985801994779\\
1.46646646646647	29.8392182533631\\
1.47147147147147	29.7657954433665\\
1.47647647647648	29.6783512842078\\
1.48148148148148	29.5769325139501\\
1.48648648648649	29.4615930939994\\
1.49149149149149	29.3323942091054\\
1.4964964964965	29.1894042673606\\
1.5015015015015	29.032698900201\\
1.50650650650651	28.8623609624058\\
1.51151151151151	28.6784805320973\\
1.51651651651652	28.4811549107411\\
1.52152152152152	28.2704886231462\\
1.52652652652653	28.0465934174646\\
1.53153153153153	27.8095835622186\\
1.53653653653654	27.5594904469815\\
1.54154154154154	27.2963895159506\\
1.54654654654655	27.0203927837376\\
1.55155155155155	26.7316180883036\\
1.55655655655656	26.4301890909586\\
1.56156156156156	26.1162352763621\\
1.56656656656657	25.7898919525226\\
1.57157157157157	25.4513002507978\\
1.57657657657658	25.1006071258947\\
1.58158158158158	24.7379653558695\\
1.58658658658659	24.3635335421275\\
1.59159159159159	23.9774761094232\\
1.5965965965966	23.5799633058604\\
1.6016016016016	23.1711712028921\\
1.60660660660661	22.7512816953204\\
1.61161161161161	22.3204825012965\\
1.61661661661662	21.8789671623212\\
1.62162162162162	21.4269350432441\\
1.62662662662663	20.9645913322641\\
1.63163163163163	20.4921470409293\\
1.63663663663664	20.0098190041372\\
1.64164164164164	19.5178298801343\\
1.64664664664665	19.0164081505162\\
1.65165165165165	18.5057881202279\\
1.65665665665666	17.9862099175636\\
1.66166166166166	17.4579194941665\\
1.66666666666667	16.9211686250292\\
1.67167167167167	16.3762149084935\\
1.67667667667668	15.8233217662501\\
1.68168168168168	15.2627584433393\\
1.68668668668669	14.6948000081504\\
1.69169169169169	14.1197273524219\\
1.6966966966967	13.5378271912415\\
1.7017017017017	12.949392063046\\
1.70670670670671	12.3547106042475\\
1.71171171171171	11.7539355339566\\
1.71671671671672	11.1473307756337\\
1.72172172172172	10.5352148398278\\
1.72672672672673	9.91790613816699\\
1.73173173173173	9.2957229833582\\
1.73673673673674	8.66898358918727\\
1.74174174174174	8.03800607051899\\
1.74674674674675	7.40310844329707\\
1.75175175175175	6.76460862454415\\
1.75675675675676	6.12282443236178\\
1.76176176176176	5.47807358593045\\
1.76676676676677	4.83067370550957\\
1.77177177177177	4.18094231243747\\
1.77677677677678	3.52919682913142\\
1.78178178178178	2.8757545790876\\
1.78678678678679	2.22093278688113\\
1.79179179179179	1.56504857816603\\
1.7967967967968	0.908418979675272\\
1.8018018018018	0.251360919220738\\
1.80680680680681	-0.405808774306756\\
1.81181181181181	-1.06277337093748\\
1.81681681681682	-1.71921623962276\\
1.82182182182182	-2.37482084823503\\
1.82682682682683	-3.02927076356775\\
1.83183183183183	-3.6822496513355\\
1.83683683683684	-4.33344127617392\\
1.84184184184184	-4.98252950163973\\
1.84684684684685	-5.62919829021068\\
1.85185185185185	-6.27313170328566\\
1.85685685685686	-6.91401390118462\\
1.86186186186186	-7.55152914314853\\
1.86686686686687	-8.18536178733952\\
1.87187187187187	-8.81519629084073\\
1.87687687687688	-9.44071720965639\\
1.88188188188188	-10.0616091987118\\
1.88688688688689	-10.6775570118534\\
1.89189189189189	-11.2882455018486\\
1.8968968968969	-11.8933598609957\\
1.9019019019019	-12.4926977659531\\
1.90690690690691	-13.086039803606\\
1.91191191191191	-13.673078804211\\
1.91691691691692	-14.2535137707103\\
1.92192192192192	-14.8270498787318\\
1.92692692692693	-15.3933984765891\\
1.93193193193193	-15.9522770852812\\
1.93693693693694	-16.5034093984928\\
1.94194194194194	-17.0465252825945\\
1.94694694694695	-17.581360776642\\
1.95195195195195	-18.1076580923769\\
1.95695695695696	-18.6251656142266\\
1.96196196196196	-19.1336378993036\\
1.96696696696697	-19.6328356774066\\
1.97197197197197	-20.1225258510195\\
1.97697697697698	-20.6024814953119\\
1.98198198198198	-21.0724818581392\\
1.98698698698699	-21.5323123600422\\
1.99199199199199	-21.9817645942474\\
1.996996996997	-22.4206363266669\\
2.002002002002	-22.8487314958984\\
2.00700700700701	-23.2658602132254\\
2.01201201201201	-23.6718387626166\\
2.01701701701702	-24.0664896007268\\
2.02202202202202	-24.449641356896\\
2.02702702702703	-24.8211288331501\\
2.03203203203203	-25.1807930042006\\
2.03703703703704	-25.5284810174443\\
2.04204204204204	-25.864046192964\\
2.04704704704705	-26.187348023528\\
2.05205205205205	-26.49825217459\\
2.05705705705706	-26.7966304842897\\
2.06206206206206	-27.082360963452\\
2.06706706706707	-27.3553307631563\\
2.07207207207207	-27.615476615972\\
2.07707707707708	-27.8626862800916\\
2.08208208208208	-28.0968309365836\\
2.08708708708709	-28.3177889924712\\
2.09209209209209	-28.5254460807315\\
2.0970970970971	-28.7196950602963\\
2.1021021021021	-28.9004360160512\\
2.10710710710711	-29.0675762588367\\
2.11211211211211	-29.2210303254471\\
2.11711711711712	-29.3607199786314\\
2.12212212212212	-29.4865742070925\\
2.12712712712713	-29.598529225488\\
2.13213213213213	-29.6965284744296\\
2.13713713713714	-29.7805226204832\\
2.14214214214214	-29.8504695561693\\
2.14714714714715	-29.9063343999625\\
2.15215215215215	-29.9480894962917\\
2.15715715715716	-29.9757144155401\\
2.16216216216216	-29.9891959540454\\
2.16716716716717	-29.9885281340994\\
2.17217217217217	-29.9737122039483\\
2.17717717717718	-29.9447566377924\\
2.18218218218218	-29.9016771357867\\
2.18718718718719	-29.8444966240402\\
2.19219219219219	-29.7732452546162\\
2.1971971971972	-29.6879604055326\\
2.2022022022022	-29.5886866807611\\
2.20720720720721	-29.4754759102283\\
2.21221221221221	-29.3483871498146\\
2.21721721721722	-29.2074866813549\\
2.22222222222222	-29.0528480126386\\
2.22722722722723	-28.8845518774091\\
2.23223223223223	-28.7026862353642\\
2.23723723723724	-28.507346272156\\
2.24224224224224	-28.2986343993911\\
2.24724724724725	-28.0766602546301\\
2.25225225225225	-27.8415407013881\\
2.25725725725726	-27.5933977715716\\
2.26226226226226	-27.3322721021768\\
2.26726726726727	-27.0582342408277\\
2.27227227227227	-26.7714020314421\\
2.27727727727728	-26.4718990571474\\
2.28228228228228	-26.1598546402802\\
2.28728728728729	-25.8354038423868\\
2.29229229229229	-25.4986874642226\\
2.2972972972973	-25.1498520457527\\
2.3023023023023	-24.7890498661514\\
2.30730730730731	-24.4164389438024\\
2.31231231231231	-24.032183036299\\
2.31731731731732	-23.6364516404439\\
2.32232232232232	-23.229419992249\\
2.32732732732733	-22.8112690669357\\
2.33233233233233	-22.382185578935\\
2.33733733733734	-21.9423619818871\\
2.34234234234234	-21.4919964686417\\
2.34734734734735	-21.0312929712579\\
2.35235235235235	-20.5604611610042\\
2.35735735735736	-20.0797164483585\\
2.36236236236236	-19.5892799830082\\
2.36736736736737	-19.0893786538501\\
2.37237237237237	-18.5802450889902\\
2.37737737737738	-18.0621176557442\\
2.38238238238238	-17.535240460637\\
2.38738738738739	-16.9998633494032\\
2.39239239239239	-16.4562419069864\\
2.3973973973974	-15.90463745754\\
2.4024024024024	-15.3453170644266\\
2.40740740740741	-14.7785535302183\\
2.41241241241241	-14.2046253966964\\
2.41741741741742	-13.623816944852\\
2.42242242242242	-13.0364181948854\\
2.42742742742743	-12.4427249062061\\
2.43243243243243	-11.8430042640574\\
2.43743743743744	-11.2374081097251\\
2.44244244244244	-10.6262366867684\\
2.44744744744745	-10.0098106853368\\
2.45245245245245	-9.38845056583904\\
2.45745745745746	-8.76247655894379\\
2.46246246246246	-8.13220866557916\\
2.46746746746747	-7.49796665693295\\
2.47247247247247	-6.86007007445253\\
2.47747747747748	-6.2188382298449\\
2.48248248248248	-5.5745902050767\\
2.48748748748749	-4.92764485237413\\
2.49249249249249	-4.27832079422306\\
2.4974974974975	-3.62693642336894\\
2.5025025025025	-2.97380990281683\\
2.50750750750751	-2.31925916583143\\
2.51251251251251	-1.66360191593702\\
2.51751751751752	-1.00715562691753\\
2.52252252252252	-0.35023754281649\\
2.52752752752753	0.306835322062974\\
2.53253253253253	0.963746183158108\\
2.53753753753754	1.62017848564654\\
2.54254254254254	2.27581590444629\\
2.54754754754755	2.93034234421575\\
2.55255255255255	3.58344193935373\\
2.55755755755756	4.23479905399941\\
2.56256256256256	4.88409828203234\\
2.56756756756757	5.53102444707246\\
2.57257257257257	6.17526260248014\\
2.57757757757758	6.81649803135608\\
2.58258258258258	7.45441624654137\\
2.58758758758759	8.08870299061756\\
2.59259259259259	8.71904423590649\\
2.5975975975976	9.34512618447041\\
2.6026026026026	9.96663526811203\\
2.60760760760761	10.5832581483744\\
2.61261261261261	11.1946817165408\\
2.61761761761762	11.8005930936352\\
2.62262262262262	12.4006870298493\\
2.62762762762763	12.9947920834175\\
2.63263263263263	13.5826529840585\\
2.63763763763764	14.1639668042701\\
2.64264264264264	14.738436824302\\
2.64764764764765	15.3057725321561\\
2.65265265265265	15.865689623586\\
2.65765765765766	16.4179100020971\\
2.66266266266266	16.962161778947\\
2.66766766766767	17.498179273145\\
2.67267267267267	18.0257030114525\\
2.67767767767768	18.5444797283826\\
2.68268268268268	19.0542623662005\\
2.68768768768769	19.5548100749232\\
2.69269269269269	20.0458882123196\\
2.6976976976977	20.5272683439106\\
2.7027027027027	20.998728242969\\
2.70770770770771	21.4600518905194\\
2.71271271271271	21.9110294753385\\
2.71771771771772	22.3514573939548\\
2.72272272272272	22.7811382506487\\
2.72772772772773	23.1998808574525\\
2.73273273273273	23.6075002341506\\
2.73773773773774	24.003817608279\\
2.74274274274274	24.3886604151259\\
2.74774774774775	24.7618622977313\\
2.75275275275275	25.1232631068871\\
2.75775775775776	25.4727089011371\\
2.76276276276276	25.810051946777\\
2.76776776776777	26.1351507178545\\
2.77277277277277	26.4478698961692\\
2.77777777777778	26.7480803712726\\
2.78278278278278	27.035659240468\\
2.78778778778779	27.3104898088108\\
2.79279279279279	27.5724816550469\\
2.7977977977978	27.8215655749323\\
2.8028028028028	28.0576141341465\\
2.80780780780781	28.2805041219984\\
2.81281281281281	28.490119554394\\
2.81781781781782	28.6863516738359\\
2.82282282282282	28.8690989494239\\
2.82782782782783	29.0382670768543\\
2.83283283283283	29.1937689784206\\
2.83783783783784	29.3355248030131\\
2.84284284284284	29.463461926119\\
2.84784784784785	29.5775149498222\\
2.85285285285285	29.6776257028037\\
2.85785785785786	29.7637432403412\\
2.86286286286286	29.8358238443096\\
2.86786786786787	29.8938310231802\\
2.87287287287287	29.9377355120216\\
2.87787787787788	29.967515272499\\
2.88288288288288	29.9831554928748\\
2.88788788788789	29.9846485880078\\
2.89289289289289	29.9719941993543\\
2.8978978978979	29.9451991949668\\
2.9029029029029	29.9042776694953\\
2.90790790790791	29.8492509441863\\
2.91291291291291	29.7801475668832\\
2.91791791791792	29.6970033120265\\
2.92292292292292	29.5998611806534\\
2.92792792792793	29.4887714003981\\
2.93293293293293	29.3637914254914\\
2.93793793793794	29.2249859367615\\
2.94294294294294	29.0724268416329\\
2.94794794794795	28.9061932741274\\
2.95295295295295	28.7263715948635\\
2.95795795795796	28.5330553910566\\
2.96296296296296	28.326345476519\\
2.96796796796797	28.1063498916599\\
2.97297297297297	27.8731839034854\\
2.97797797797798	27.6269700055983\\
2.98298298298298	27.3678159410683\\
2.98798798798799	27.0957454738004\\
2.99299299299299	26.8108638485137\\
2.997997997998	26.5132942202346\\
3.003003003003	26.2031654409068\\
3.00800800800801	25.8806120593909\\
3.01301301301301	25.5457743214652\\
3.01801801801802	25.198798169825\\
3.02302302302302	24.8398352440829\\
3.02802802802803	24.4690428807689\\
3.03303303303303	24.0865841133302\\
3.03803803803804	23.6926276721311\\
3.04304304304304	23.2873479844535\\
3.04804804804805	22.8709251744963\\
3.05305305305305	22.4435450633758\\
3.05805805805806	22.0053991691254\\
3.06306306306306	21.5566847066959\\
3.06806806806807	21.0976045879555\\
3.07307307307307	20.6283674216894\\
3.07807807807808	20.1491875136001\\
3.08308308308308	19.6602848663077\\
3.08808808808809	19.1618851793491\\
3.09309309309309	18.6542198491787\\
3.0980980980981	18.1375259691683\\
3.1031031031031	17.6120463296067\\
3.10810810810811	17.0780294177001\\
3.11311311311311	16.535729417572\\
3.11811811811812	15.9854062102631\\
3.12312312312312	15.4273253737314\\
3.12812812812813	14.861758182852\\
3.13313313313313	14.2889816094177\\
3.13813813813814	13.7092783221381\\
3.14314314314314	13.1229366866403\\
3.14814814814815	12.5302507654687\\
3.15315315315315	11.9315184136707\\
3.15815815815816	11.326915326132\\
3.16316316316316	10.7166806987945\\
3.16816816816817	10.1011363699051\\
3.17317317317317	9.48060388451181\\
3.17817817817818	8.85540449446442\\
3.18318318318318	8.22585915841416\\
3.18818818818819	7.59228854181379\\
3.19319319319319	6.95501301691759\\
3.1981981981982	6.31435266278136\\
3.2032032032032	5.6706272652624\\
3.20820820820821	5.02415631701957\\
3.21321321321321	4.3752590175132\\
3.21821821821822	3.7242542730052\\
3.22322322322322	3.07146069655894\\
3.22822822822823	2.41719660803934\\
3.23323323323323	1.76178003411284\\
3.23823823823824	1.10552870824738\\
3.24324324324324	0.448760070712455\\
3.24824824824825	-0.208208731420947\\
3.25325325325325	-0.865060844280313\\
3.25825825825826	-1.52147970719163\\
3.26326326326326	-2.17714905267931\\
3.26826826826827	-2.8317529064663\\
3.27327327327327	-3.48497558747403\\
3.27827827827828	-4.13650170782235\\
3.28328328328328	-4.78601617282965\\
3.28828828828829	-5.43320418101279\\
3.29329329329329	-6.07775122408705\\
3.2982982982983	-6.71934308696631\\
3.3033033033033	-7.35766584776282\\
3.30830830830831	-7.99240587778735\\
3.31331331331331	-8.62324984154919\\
3.31831831831832	-9.24988469675599\\
3.32332332332332	-9.87199769431404\\
3.32832832832833	-10.489276378328\\
3.33333333333333	-11.1014085861011\\
3.33833833833834	-11.7080824481349\\
3.34334334334334	-12.3089863881296\\
3.34834834834835	-12.9038762559361\\
3.35335335335335	-13.492573233566\\
3.35835835835836	-14.0747761646017\\
3.36336336336336	-14.65018672424\\
3.36836836836837	-15.2185128216352\\
3.37337337337337	-15.7794685998988\\
3.37837837837838	-16.3327744361\\
3.38338338338338	-16.8781569412654\\
3.38838838838839	-17.415348960379\\
3.39339339339339	-17.9440895723821\\
3.3983983983984	-18.4641240901739\\
3.4034034034034	-18.9752040606105\\
3.40840840840841	-19.4770872645059\\
3.41341341341341	-19.9695377166313\\
3.41841841841842	-20.4523256657154\\
3.42342342342342	-20.9252275944444\\
3.42842842842843	-21.388026219462\\
3.43343343343343	-21.8405104913691\\
3.43843843843844	-22.2824755947245\\
3.44344344344344	-22.7137229480439\\
3.44844844844845	-23.1340602038009\\
3.45345345345345	-23.5433012484264\\
3.45845845845846	-23.9412662023087\\
3.46346346346346	-24.3277814197936\\
3.46846846846847	-24.7026794891842\\
3.47347347347347	-25.0657992327415\\
3.47847847847848	-25.4169857066833\\
3.48348348348348	-25.7560902011855\\
3.48848848848849	-26.082970240381\\
3.49349349349349	-26.3974895823603\\
3.4984984984985	-26.6995182191714\\
3.5035035035035	-26.9889323768196\\
3.50850850850851	-27.2656145152679\\
3.51351351351351	-27.5294552270123\\
3.51851851851852	-27.7803942598627\\
3.52352352352352	-28.0183272276669\\
3.52852852852853	-28.243129375275\\
3.53353353353353	-28.4546831733514\\
3.53853853853854	-28.6528783183742\\
3.54354354354354	-28.8376117326361\\
3.54854854854855	-29.0087875642437\\
3.55355355355355	-29.1663171871175\\
3.55855855855856	-29.3101192009925\\
3.56356356356356	-29.4401194314175\\
3.56856856856857	-29.5562509297556\\
3.57357357357357	-29.6584539731839\\
3.57857857857858	-29.7466760646938\\
3.58358358358358	-29.8208719330905\\
3.58858858858859	-29.8810035329936\\
3.59359359359359	-29.9270400448366\\
3.5985985985986	-29.9589578748673\\
3.6036036036036	-29.9767406551475\\
3.60860860860861	-29.9803792435531\\
3.61361361361361	-29.9698717237741\\
3.61861861861862	-29.9452234053148\\
3.62362362362362	-29.9064468234934\\
3.62862862862863	-29.8535617394422\\
3.63363363363363	-29.7865951401079\\
3.63863863863864	-29.7055812382508\\
3.64364364364364	-29.6105614724459\\
3.64864864864865	-29.5015845070819\\
3.65365365365365	-29.3787062323617\\
3.65865865865866	-29.2419897643025\\
3.66366366366366	-29.0915054447353\\
3.66866866866867	-28.9273308413055\\
3.67367367367367	-28.7495507474724\\
3.67867867867868	-28.5582571825096\\
3.68368368368368	-28.3535493915046\\
3.68868868868869	-28.1355338453592\\
3.69369369369369	-27.9043242407892\\
3.6986986986987	-27.6600415003247\\
3.7037037037037	-27.4028137720193\\
3.70870870870871	-27.1327043704329\\
3.71371371371371	-26.8497670937971\\
3.71871871871872	-26.5541246402041\\
3.72372372372372	-26.2459053661989\\
3.72872872872873	-25.9252432867792\\
3.73373373373373	-25.5922780753956\\
3.73873873873874	-25.247155063951\\
3.74374374374374	-24.8900252428014\\
3.74874874874875	-24.5210452607552\\
3.75375375375375	-24.1403774250739\\
3.75875875875876	-23.7481897014713\\
3.76376376376376	-23.3446557141142\\
3.76876876876877	-22.9299547456219\\
3.77377377377377	-22.5042717370666\\
3.77877877877878	-22.0677972879731\\
3.78378378378378	-21.620727656319\\
3.78878878878879	-21.1632647585345\\
3.79379379379379	-20.6956161695025\\
3.7987987987988	-20.2179951225588\\
3.8038038038038	-19.7306205094917\\
3.80880880880881	-19.2337168805424\\
3.81381381381381	-18.7275144444046\\
3.81881881881882	-18.2122490682248\\
3.82382382382382	-17.6881622776024\\
3.82882882882883	-17.1555012565892\\
3.83383383383383	-16.6145188476899\\
3.83883883883884	-16.0654735518619\\
3.84384384384384	-15.5086295285153\\
3.84884884884885	-14.9442565955127\\
3.85385385385385	-14.3726302291699\\
3.85885885885886	-13.7940315642549\\
3.86386386386386	-13.2087473939888\\
3.86886886886887	-12.6170701700451\\
3.87387387387387	-12.0192980025501\\
3.87887887887888	-11.4156895424868\\
3.88388388388388	-10.8064044130972\\
3.88888888888889	-10.1917525355587\\
3.89389389389389	-9.5720567797898\\
3.8988988988989	-8.94763964794501\\
3.9039039039039	-8.31882327441507\\
3.90890890890891	-7.68592942582683\\
3.91391391391391	-7.04927950104324\\
3.91891891891892	-6.40919453116341\\
3.92392392392392	-5.76599517952256\\
3.92892892892893	-5.12000174169205\\
3.93393393393393	-4.47153414547938\\
3.93893893893894	-3.82091195092817\\
3.94394394394394	-3.16845435031816\\
3.94894894894895	-2.51448016816524\\
3.95395395395395	-1.85930786122142\\
3.95895895895896	-1.20325551847485\\
3.96396396396396	-0.546640861149781\\
3.96896896896897	0.110218757293367\\
3.97397397397397	0.767006351158056\\
3.97897897897898	1.42340530251164\\
3.98398398398398	2.0790993611853\\
3.98898898898899	2.73377264477413\\
3.99399399399399	3.38710963863706\\
3.998998998999	4.03879519589692\\
4.004004004004	4.68851453744043\\
4.00900900900901	5.33595325191805\\
4.01401401401401	5.98079729574424\\
4.01901901901902	6.62273299309732\\
4.02402402402402	7.26144703591941\\
4.02902902902903	7.89662648391659\\
4.03403403403403	8.52795876455868\\
4.03903903903904	9.15513167307955\\
4.04404404404404	9.77783337247679\\
4.04904904904905	10.3957523935119\\
4.05405405405405	11.0085776347103\\
4.05905905905906	11.6159983623612\\
4.06406406406406	12.2177042105177\\
4.06906906906907	12.8133949537089\\
4.07407407407407	13.4029047434101\\
4.07907907907908	13.9859750394208\\
4.08408408408408	14.562305787374\\
4.08908908908909	15.1316031968962\\
4.09409409409409	15.6935797416078\\
4.0990990990991	16.2479541591229\\
4.1041041041041	16.7944514510493\\
4.10910910910911	17.3328028829885\\
4.11411411411411	17.8627459845358\\
4.11911911911912	18.3840245492803\\
4.12412412412412	18.8963886348046\\
4.12912912912913	19.3995945626853\\
4.13413413413413	19.8934049184926\\
4.13913913913914	20.3775885517906\\
4.14414414414414	20.8519205761368\\
4.14914914914915	21.3161823690827\\
4.15415415415415	21.7701615721735\\
4.15915915915916	22.2136520909482\\
4.16416416416416	22.6464540949394\\
4.16916916916917	23.0683740176734\\
4.17417417417417	23.4792245566704\\
4.17917917917918	23.8788246734442\\
4.18418418418418	24.2669995935026\\
4.18918918918919	24.6435808063467\\
4.19419419419419	25.0084060654717\\
4.1991991991992	25.3613193883663\\
4.2042042042042	25.7021710565132\\
4.20920920920921	26.0308176153886\\
4.21421421421421	26.3471218744626\\
4.21921921921922	26.6509529071988\\
4.22422422422422	26.9421860510548\\
4.22922922922923	27.2207029074819\\
4.23423423423423	27.4863913419249\\
4.23923923923924	27.7391607171089\\
4.24424424424424	27.9789497261777\\
4.24924924924925	28.205637344739\\
4.25425425425425	28.4191044310875\\
4.25925925925926	28.6192390683788\\
4.26426426426426	28.8059365646292\\
4.26926926926927	28.9790994527158\\
4.27427427427427	29.1386374903765\\
4.27927927927928	29.2844676602101\\
4.28428428428428	29.4165141696763\\
4.28928928928929	29.5347084510952\\
4.29429429429429	29.6389891616483\\
4.2992992992993	29.7293021833773\\
4.3043043043043	29.8056006231853\\
4.30930930930931	29.8678448128358\\
4.31431431431431	29.9160023089532\\
4.31931931931932	29.9500478930229\\
4.32432432432432	29.9699635713908\\
4.32932932932933	29.9757385752639\\
4.33433433433433	29.9673693607099\\
4.33933933933934	29.9448596086573\\
4.34434434434434	29.9082202248954\\
4.34934934934935	29.8574693400743\\
4.35435435435435	29.7926323097049\\
4.35935935935936	29.7137417141592\\
4.36436436436436	29.6208373586695\\
4.36936936936937	29.5139662733294\\
4.37437437437437	29.3931827130929\\
4.37937937937938	29.2585481577752\\
4.38438438438438	29.110131312052\\
4.38938938938939	28.94800810546\\
4.39439439439439	28.7722616923966\\
4.3993993993994	28.5829824521201\\
4.4044044044044	28.3802679887496\\
4.40940940940941	28.1642231312649\\
4.41441441441441	27.9349599335067\\
4.41941941941942	27.6925976741766\\
4.42442442442442	27.4372628568369\\
4.42942942942943	27.1690794494334\\
4.43443443443443	26.8880775727304\\
4.43943943943944	26.5943535666648\\
4.44444444444444	26.2880353469164\\
4.44944944944945	25.9692564419761\\
4.45445445445445	25.6381559931452\\
4.45945945945946	25.2948787545354\\
4.46446446446446	24.9395750930693\\
4.46946946946947	24.57240098848\\
4.47447447447447	24.1935180333111\\
4.47947947947948	23.803093432917\\
4.48448448448448	23.4013000054625\\
4.48948948948949	22.9883161819231\\
4.49449449449449	22.564326006085\\
4.4994994994995	22.1295191345447\\
4.5045045045045	21.6840908367095\\
4.50950950950951	21.2282419947974\\
4.51451451451451	20.7621791038368\\
4.51951951951952	20.2861142716669\\
4.52452452452452	19.8002652189372\\
4.52952952952953	19.3048552791081\\
4.53453453453453	18.8001133984504\\
4.53953953953954	18.2862741360456\\
4.54454454454454	17.7635776637858\\
4.54954954954955	17.2322697663738\\
4.55455455455455	16.6926018413226\\
4.55955955955956	16.1448308989563\\
4.56456456456456	15.5892195624093\\
4.56956956956957	15.0260360676267\\
4.57457457457457	14.4555542633641\\
4.57957957957958	13.8780536111879\\
4.58458458458458	13.2938191854749\\
4.58958958958959	12.7031416734126\\
4.59459459459459	12.106317374999\\
4.5995995995996	11.5036482030428\\
4.6046046046046	10.8953518717392\\
4.60960960960961	10.2816235163862\\
4.61461461461461	9.66278762093956\\
4.61961961961962	9.03916899335776\\
4.62462462462462	8.41109194269356\\
4.62962962962963	7.77888027909405\\
4.63463463463463	7.14285731380059\\
4.63963963963964	6.50334585914887\\
4.64464464464464	5.86066822856889\\
4.64964964964965	5.21514623658495\\
4.65465465465465	4.56710119881565\\
4.65965965965966	3.91685393197391\\
4.66466466466466	3.26472475386695\\
4.66966966966967	2.6110334833963\\
4.67467467467467	1.9560994405578\\
4.67967967967968	1.30024144644158\\
4.68468468468468	0.643777823232104\\
4.68968968968969	-0.0129736057918767\\
4.69469469469469	-0.669695516257301\\
4.6996996996997	-1.32607108269681\\
4.7047047047047	-1.9817839785487\\
4.70970970970971	-2.636518376157\\
4.71471471471471	-3.28995894677143\\
4.71971971971972	-3.94179086054737\\
4.72472472472472	-4.59169978654592\\
4.72972972972973	-5.23937189273388\\
4.73473473473473	-5.88449384598373\\
4.73973973973974	-6.52675281207362\\
4.74474474474474	-7.16583645568745\\
4.74974974974975	-7.80143294041474\\
4.75475475475475	-8.43323092875078\\
4.75975975975976	-9.0609195820965\\
4.76476476476476	-9.68418856075853\\
4.76976976976977	-10.3027280239492\\
4.77477477477477	-10.9162286297866\\
4.77977977977978	-11.5243815352943\\
4.78478478478478	-12.1268783964019\\
4.78978978978979	-12.7234113679443\\
4.79479479479479	-13.3137026777863\\
4.7997997997998	-13.8975953539004\\
4.8048048048048	-14.4748078272472\\
4.80980980980981	-15.0450439922824\\
4.81481481481481	-15.6080140587757\\
4.81981981981982	-16.1634345518118\\
4.82482482482482	-16.7110283117891\\
4.82982982982983	-17.2505244944208\\
4.83483483483483	-17.7816585707343\\
4.83983983983984	-18.3041723270713\\
4.84484484484484	-18.8178138650879\\
4.84984984984985	-19.3223376017546\\
4.85485485485485	-19.8175042693562\\
4.85985985985986	-20.3030809154918\\
4.86486486486486	-20.778840903075\\
4.86986986986987	-21.2445639103336\\
4.87487487487487	-21.7000359308098\\
4.87987987987988	-22.1450492733602\\
4.88488488488488	-22.5794025621558\\
4.88988988988989	-23.0029007366817\\
4.89489489489489	-23.4153550517377\\
4.8998998998999	-23.8165830774377\\
4.9049049049049	-24.20640869921\\
4.90990990990991	-24.5846621177973\\
4.91491491491491	-24.9511798492566\\
4.91991991991992	-25.3058047249594\\
4.92492492492492	-25.6483858915914\\
4.92992992992993	-25.9787788111526\\
4.93493493493493	-26.2968452609575\\
4.93993993993994	-26.6024533336349\\
4.94494494494494	-26.895477437128\\
4.94994994994995	-27.1757982946942\\
4.95495495495495	-27.4433029449054\\
4.95995995995996	-27.6978847416479\\
4.96496496496496	-27.9394230986447\\
4.96996996996997	-28.1677701577789\\
4.97497497497497	-28.3828246730048\\
4.97997997997998	-28.5844933933722\\
4.98498498498498	-28.7726891444461\\
4.98998998998999	-28.9473308283072\\
4.99499499499499	-29.1083434235515\\
5	-29.2556579852903\\
};
\addlegendentry{Runge-Kutta}

\end{axis}
\end{tikzpicture}%}
		\caption{Solution of the open-loop pendulum from Euler-Integrator and Runge-Kutta for time step-size = $0.005$}
		\label{fig:eulerVsRK_1}
	\end{figure}
	\figref{fig:eulerVsRK} demonstrates the diverging behavior of the euler integrator for step-size of $0.05$ seconds. On the other hand, the two solutions are a lot closer for $0.005$ seconds as demonstrated by \figref{fig:eulerVsRK_1}.
	\item Euler integrator is easier to implement compared to the fourth-order Runge-Kutta since the fourth-order Runge-Kutta requires four approximations of the derivative and Euler integrator requires only one.
	\item Error accumulation over time is faster in Euler as already demonstrated in \figref{fig:eulerVsRK}.
\end{enumerate}

\subsection{Linearization of the System about the upper equilibrium point}
The non-linear state-space is given by
\begin{equation*}
	\begin{bmatrix}
		\dot{\theta}\\
		\ddot{\theta}
	\end{bmatrix} = \begin{bmatrix}
	\dot{\theta}\\
	\frac{u}{ml^2} - \frac{b\dot{\theta}}{ml} - \frac{g\sin\theta}{l}
\end{bmatrix} = \bm{f}\left(\bm{x},u\right)
\end{equation*}
The linear states will be given by
\begin{equation*}
	\begin{bmatrix}
		z\\
		\dot{z}
	\end{bmatrix} = \begin{bmatrix}
	\pi - \theta\\
	-\dot{\theta}
\end{bmatrix} = \begin{bmatrix}
\delta \theta\\
\delta\dot{\theta}
\end{bmatrix}
\end{equation*}
The first-order taylor series approximation around $\bm{x}_E = \left(\pi,0\right)$ is given by
\begin{align*}
	\begin{bmatrix}
		\delta \dot{\theta}\\
		\delta\ddot{\theta}
	\end{bmatrix} &\approx \left.\frac{\partial \bm{f}}{\partial \bm{x}}\right\vert_{\bm{x} = \left(\pi,0\right)}\begin{bmatrix}
	\delta \theta\\
	\delta\dot{\theta}
\end{bmatrix}\\
\implies\begin{bmatrix}
	\dot{z}\\
	\ddot{z}
\end{bmatrix} &= \begin{bmatrix}
0 & 1 \\
-\frac{g}{l} & -\frac{b}{ml}
\end{bmatrix}\begin{bmatrix}
\pi - \theta\\
-\dot{\theta}
\end{bmatrix} + \begin{bmatrix}
0\\
\frac{1}{ml^2}
\end{bmatrix}u\\
\implies \begin{bmatrix}
	\dot{z}\\
	\ddot{z}
\end{bmatrix} &= \begin{bmatrix}
	0 & 1 \\
	-\frac{g}{l} & -\frac{b}{ml}
\end{bmatrix}\begin{bmatrix}
z\\
\dot{z}
\end{bmatrix} + \begin{bmatrix}
	0\\
	\frac{1}{ml^2}
\end{bmatrix}u
\end{align*}
Hence, for the linear system around the upper equilibrium point, 
\begin{align*}
	\bm{A} &= \begin{bmatrix}
		0 & 1 \\
		-\frac{g}{l} & -\frac{b}{ml}
	\end{bmatrix}\\
\bm{b} &= \begin{bmatrix}
	0\\
	\frac{1}{ml^2}
\end{bmatrix}
\end{align*}

\subsection{Linear Quadratic Regulator Design}
A LQR is designed assuming full-state feedback. Assuming an infinite-time horizon, the algebraic Riccati equation is solved in order to obtain the LQR gain matrix $\bm{K}_{\text{LQR}} \in \mathbb{R}^{1\times2}$. Positive definite and symmetric matrices $\bm{Q} \in \mathbb{R}^{2\times2}$ and $\bm{R} \in \mathbb{R}$ are used to prioritize the minimization of the transient energy (states reach reference quicker) and the minimization of the actuator energy (actuator limits are prioritized) respectively.

The closed-loop linear system with LQR is formulated as
\begin{align*}
	\dot{\bm{z}} &= \bm{A}\bm{z} + \bm{b}u & u = -\bm{K}_{\text{LQR}}\bm{z}\\
	\dot{\bm{z}} &= \left(\bm{A} - \bm{b} \bm{K}_{\text{LQR}}\right)\bm{z}  & \bm{A}_{\text{CL}} = \bm{A} - \bm{b} \bm{K}_{\text{LQR}}
\end{align*}

\subsection{Strategy for Swing-up and Balancing}
\begin{enumerate}
	\item Perform swing-up using energy error till a close value to $180$ degrees is reached.
	\item Define a threshold $\left(\epsilon\right)$ for linear behaviour of the pendulum about its upright equilibrium point. For this implementation, $\epsilon = 10$ degrees.
	\item Switch to the LQR Controller once the linear threshold is reached in order to balance the pendulum at its upright position.
\end{enumerate}

For this implementation, $\bm{Q} = 0.1\bm{I}_{2\times2}$ and $R = 1$. The values of $\bm{Q}$ and $R$ can naturally be fine-tuned in order to ensure a balance between faster response time and the actuator limit of $1 Nm$.

This strategy implementation can be found in \emph{swingUp\textunderscore Balance\textunderscore Pend.m}.

\begin{figure}[h!]
	\centering
	\scalebox{0.8}{% This file was created by matlab2tikz.
%
%The latest updates can be retrieved from
%  http://www.mathworks.com/matlabcentral/fileexchange/22022-matlab2tikz-matlab2tikz
%where you can also make suggestions and rate matlab2tikz.
%
\definecolor{mycolor1}{rgb}{0.00000,0.44700,0.74100}%
\definecolor{mycolor2}{rgb}{0.85000,0.32500,0.09800}%
%
\begin{tikzpicture}

\begin{axis}[%
width=4.521in,
height=3.566in,
at={(0.758in,0.481in)},
scale only axis,
xmin=0,
xmax=10,
xlabel style={font=\color{white!15!black}},
xlabel={Time (s)},
ymin=-200,
ymax=200,
ylabel style={font=\color{white!15!black}},
ylabel={Angle (deg)},
axis background/.style={fill=white},
axis x line*=bottom,
axis y line*=left,
xmajorgrids,
ymajorgrids,
legend style={at={(0.97,0.03)}, anchor=south east, legend cell align=left, align=left, draw=white!15!black}
]
\addplot [color=mycolor1, line width=2.0pt]
  table[row sep=crcr]{%
0	0\\
0.01001001001001	0.0229527964285707\\
0.02002002002002	0.0917354675841067\\
0.03003003003003	0.206165207888467\\
0.04004004004004	0.365971012420373\\
0.0500500500500501	0.570795798577409\\
0.0600600600600601	0.820194750916577\\
0.0700700700700701	1.1136353211543\\
0.0800800800800801	1.4504972281664\\
0.0900900900900901	1.83007245798815\\
0.1001001001001	2.25156618018758\\
0.11011011011011	2.7141211616656\\
0.12012012012012	3.2167747896164\\
0.13013013013013	3.75848377004713\\
0.14014014014014	4.33814243626362\\
0.15015015015015	4.95458274887042\\
0.16016016016016	5.60657429577076\\
0.17017017017017	6.2928242921666\\
0.18018018018018	7.01197758055856\\
0.19019019019019	7.76261663074601\\
0.2002002002002	8.54326153982699\\
0.21021021021021	9.35237003219824\\
0.22022022022022	10.1883374595552\\
0.23023023023023	11.049496800892\\
0.24024024024024	11.9341186625016\\
0.25025025025025	12.8404112779754\\
0.26026026026026	13.7665615115009\\
0.27027027027027	14.7110592633673\\
0.28028028028028	15.6720597518029\\
0.29029029029029	16.647624511023\\
0.3003003003003	17.6358295576027\\
0.31031031031031	18.6347653904763\\
0.32032032032032	19.6425369909379\\
0.33033033033033	20.6572638226409\\
0.34034034034034	21.6770798315981\\
0.35035035035035	22.7001334461819\\
0.36036036036036	23.7245875771244\\
0.37037037037037	24.7486196175167\\
0.38038038038038	25.7704214428099\\
0.39039039039039	26.7881994108142\\
0.4004004004004	27.8001743616996\\
0.41041041041041	28.8045816179952\\
0.42042042042042	29.7996709845901\\
0.43043043043043	30.7837067487324\\
0.44044044044044	31.75496768003\\
0.45045045045045	32.7119601606338\\
0.46046046046046	33.6533255292228\\
0.47047047047047	34.5774333778422\\
0.48048048048048	35.4827132482347\\
0.49049049049049	36.3676582845688\\
0.500500500500501	37.2308252334397\\
0.510510510510511	38.0708344438683\\
0.520520520520521	38.8863698673021\\
0.530530530530531	39.6761790576147\\
0.540540540540541	40.4390731711059\\
0.550550550550551	41.1739269665017\\
0.560560560560561	41.8796788049543\\
0.570570570570571	42.5553306500422\\
0.580580580580581	43.1999480677701\\
0.590590590590591	43.8126602265688\\
0.600600600600601	44.3926598972954\\
0.610610610610611	44.9392034532332\\
0.620620620620621	45.4516108700918\\
0.630630630630631	45.929265726007\\
0.640640640640641	46.3716152015405\\
0.650650650650651	46.7780482201538\\
0.660660660660661	47.1479498890227\\
0.670670670670671	47.4808550670263\\
0.680680680680681	47.7763530050871\\
0.690690690690691	48.0340873398986\\
0.700700700700701	48.2537560939258\\
0.710710710710711	48.4351116754052\\
0.720720720720721	48.5779608783445\\
0.730730730730731	48.6821640814574\\
0.740740740740741	48.7476197117965\\
0.750750750750751	48.7742780866202\\
0.760760760760761	48.7260405286235\\
0.770770770770771	48.5477381201039\\
0.780780780780781	48.2393409027903\\
0.790790790790791	47.80138153309\\
0.800800800800801	47.2345753355244\\
0.810810810810811	46.5400185262456\\
0.820820820820821	45.718721141524\\
0.830830830830831	44.7717809684676\\
0.840840840840841	43.7005326043546\\
0.850850850850851	42.5065474566329\\
0.860860860860861	41.1916337429207\\
0.870870870870871	39.757836491006\\
0.880880880880881	38.2074375388469\\
0.890890890890891	36.5429555345717\\
0.900900900900901	34.7671459364784\\
0.910910910910911	32.8830010130353\\
0.920920920920921	30.8937498428805\\
0.930930930930931	28.8028583148223\\
0.940940940940941	26.6140291278388\\
0.950950950950951	24.3312017910784\\
0.960960960960961	21.9584811496885\\
0.970970970970971	19.4994102915786\\
0.980980980980981	16.9588260707284\\
0.990990990990991	14.3420296606986\\
1.001001001001	11.6543693960581\\
1.01101101101101	8.90124077238413\\
1.02102102102102	6.08808644626185\\
1.03103103103103	3.22039623528472\\
1.04104104104104	0.303707118054355\\
1.05105105105105	-2.65639676581946\\
1.06106106106106	-5.65428411571876\\
1.07107107107107	-8.68427647001741\\
1.08108108108108	-11.7406482060811\\
1.09109109109109	-14.8176265402673\\
1.1011011011011	-17.9093915279255\\
1.11111111111111	-21.0100760633966\\
1.12112112112112	-24.1137658800139\\
1.13113113113113	-27.214499550102\\
1.14114114114114	-30.3062684849776\\
1.15115115115115	-33.3830169349491\\
1.16116116116116	-36.4389313665938\\
1.17117117117117	-39.470185845111\\
1.18118118118118	-42.4724993825067\\
1.19119119119119	-45.4415980504507\\
1.2012012012012	-48.3734503432864\\
1.21121121121121	-51.2642671780303\\
1.22122122122122	-54.110501894372\\
1.23123123123123	-56.9088502546747\\
1.24124124124124	-59.6562504439747\\
1.25125125125125	-62.3498830699814\\
1.26126126126126	-64.9871711630776\\
1.27127127127127	-67.5657801763195\\
1.28128128128128	-70.0836179854364\\
1.29129129129129	-72.5388348888308\\
1.3013013013013	-74.9298236075785\\
1.31131131131131	-77.2552192854287\\
1.32132132132132	-79.5138994888037\\
1.33133133133133	-81.704984206799\\
1.34134134134134	-83.8278319319326\\
1.35135135135135	-85.8820369860402\\
1.36136136136136	-87.8675816607711\\
1.37137137137137	-89.7845653189511\\
1.38138138138138	-91.6331367073867\\
1.39139139139139	-93.4134939568656\\
1.4014014014014	-95.1258845821563\\
1.41141141141141	-96.7706054820084\\
1.42142142142142	-98.3480029391523\\
1.43143143143143	-99.8584726202991\\
1.44144144144144	-101.302459576141\\
1.45145145145145	-102.680458241352\\
1.46146146146146	-103.993012434585\\
1.47147147147147	-105.240715358476\\
1.48148148148148	-106.424209599641\\
1.49149149149149	-107.544187128676\\
1.5015015015015	-108.60138930016\\
1.51151151151151	-109.59660685265\\
1.52152152152152	-110.530679908688\\
1.53153153153153	-111.404497974793\\
1.54154154154154	-112.218742484467\\
1.55155155155155	-112.973858246497\\
1.56156156156156	-113.670473958859\\
1.57157157157157	-114.309185316365\\
1.58158158158158	-114.890554118494\\
1.59159159159159	-115.415108269394\\
1.6016016016016	-115.883341777878\\
1.61161161161161	-116.29571475743\\
1.62162162162162	-116.652653156254\\
1.63163163163163	-116.954523923055\\
1.64164164164164	-117.201641402215\\
1.65165165165165	-117.394281233386\\
1.66166166166166	-117.532671669821\\
1.67167167167167	-117.616992254548\\
1.68168168168168	-117.643715323163\\
1.69169169169169	-117.552853946502\\
1.7017017017017	-117.312605464729\\
1.71171171171171	-116.927634850716\\
1.72172172172172	-116.396781051036\\
1.73173173173173	-115.719740714583\\
1.74174174174174	-114.896009669889\\
1.75175175175175	-113.925016559296\\
1.76176176176176	-112.806128993998\\
1.77177177177177	-111.538653554044\\
1.78178178178178	-110.121835788332\\
1.79179179179179	-108.554860214614\\
1.8018018018018	-106.836850319492\\
1.81181181181181	-104.966916251494\\
1.82182182182182	-102.944448667506\\
1.83183183183183	-100.768405876758\\
1.84184184184184	-98.4378574055404\\
1.85185185185185	-95.9521346922693\\
1.86186186186186	-93.3108310874875\\
1.87187187187187	-90.5138018538637\\
1.88188188188188	-87.5611641661933\\
1.89189189189189	-84.4532971113975\\
1.9019019019019	-81.1908416885242\\
1.91191191191191	-77.7747008087474\\
1.92192192192192	-74.2060392953672\\
1.93193193193193	-70.4865448887531\\
1.94194194194194	-66.6187766950201\\
1.95195195195195	-62.6056544205321\\
1.96196196196196	-58.4507303484753\\
1.97197197197197	-54.1581967093097\\
1.98198198198198	-49.7328856807689\\
1.99199199199199	-45.1802693878601\\
2.002002002002	-40.506800432524\\
2.01201201201201	-35.7326025261893\\
2.02202202202202	-30.8952743598986\\
2.03203203203203	-25.995265613072\\
2.04204204204204	-21.0413684438095\\
2.05205205205205	-16.0515603223589\\
2.06206206206206	-11.0207798490254\\
2.07207207207207	-5.96805488772194\\
2.08208208208208	-0.8995305855319\\
2.09209209209209	4.16857373986216\\
2.1021021021021	9.22997087281148\\
2.11211211211211	14.272486672133\\
2.12212212212212	19.2795631061329\\
2.13213213213213	24.2556959570022\\
2.14214214214214	29.1798336167826\\
2.15215215215215	34.0502899567801\\
2.16216216216216	38.8632149147962\\
2.17217217217217	43.6037812292299\\
2.18218218218218	48.2718028661812\\
2.19219219219219	52.8569422496284\\
2.2022022022022	57.3523126989561\\
2.21221221221221	61.7545843688472\\
2.22222222222222	66.0543317847427\\
2.23223223223223	70.2499010028821\\
2.24224224224224	74.3432325800556\\
2.25225225225225	78.3275795142006\\
2.26226226226226	82.1979597299701\\
2.27227227227227	85.9609637734166\\
2.28228228228228	89.6166514505759\\
2.29229229229229	93.1594157902009\\
2.3023023023023	96.5880134107986\\
2.31231231231231	99.9074180061144\\
2.32232232232232	103.11840673213\\
2.33233233233233	106.217400519946\\
2.34234234234234	109.206242005908\\
2.35235235235235	112.089978864564\\
2.36236236236236	114.870938280889\\
2.37237237237237	117.549503379162\\
2.38238238238238	120.126949946962\\
2.39239239239239	122.607435409911\\
2.4024024024024	124.994197075806\\
2.41241241241241	127.289531934031\\
2.42242242242242	129.494835593051\\
2.43243243243243	131.612556670165\\
2.44244244244244	133.646359213077\\
2.45245245245245	135.599379967933\\
2.46246246246246	137.474343579566\\
2.47247247247247	139.273562591499\\
2.48248248248248	140.99898361706\\
2.49249249249249	142.653293797445\\
2.5025025025025	144.239481915011\\
2.51251251251251	145.76024879946\\
2.52252252252252	147.218125956425\\
2.53253253253253	148.615475567464\\
2.54254254254254	149.954490490064\\
2.55255255255255	151.237194257642\\
2.56256256256256	152.465644411251\\
2.57257257257257	153.642227914988\\
2.58258258258258	154.769043065266\\
2.59259259259259	155.848104018952\\
2.6026026026026	156.881351237399\\
2.61261261261261	157.870651486442\\
2.62262262262262	158.817797836402\\
2.63263263263263	159.724509662083\\
2.64264264264264	160.592432642775\\
2.65265265265265	161.423138762251\\
2.66266266266266	162.218204553366\\
2.67267267267267	162.979527186244\\
2.68268268268268	163.708434053652\\
2.69269269269269	164.406122894476\\
2.7027027027027	165.073769039103\\
2.71271271271271	165.712525409422\\
2.72272272272272	166.323522518825\\
2.73273273273273	166.907868472207\\
2.74274274274274	167.466648965962\\
2.75275275275275	168.00092728799\\
2.76276276276276	168.511744317691\\
2.77277277277277	169.000118525968\\
2.78278278278278	169.467045975227\\
2.79279279279279	169.913500319374\\
2.8028028028028	170.34043280382\\
2.81281281281281	170.748772265476\\
2.82282282282282	171.139425132757\\
2.83283283283283	171.513275425578\\
2.84284284284284	171.87114896781\\
2.85285285285285	172.213633241034\\
2.86286286286286	172.541281132488\\
2.87287287287287	172.854641762086\\
2.88288288288288	173.154253667203\\
2.89289289289289	173.440644802666\\
2.9029029029029	173.714332540763\\
2.91291291291291	173.975823671234\\
2.92292292292292	174.225614401278\\
2.93293293293293	174.46419035555\\
2.94294294294294	174.692026576162\\
2.95295295295295	174.909587522681\\
2.96296296296296	175.117327072132\\
2.97297297297297	175.315688518994\\
2.98298298298298	175.505104575205\\
2.99299299299299	175.685997370159\\
3.003003003003	175.858778450705\\
3.01301301301301	176.023848781151\\
3.02302302302302	176.181593752131\\
3.03303303303303	176.332256793037\\
3.04304304304304	176.476057270391\\
3.05305305305305	176.613239134726\\
3.06306306306306	176.744040604954\\
3.07307307307307	176.868694168364\\
3.08308308308308	176.987426580626\\
3.09309309309309	177.100458865787\\
3.1031031031031	177.208006316273\\
3.11311311311311	177.31027849289\\
3.12312312312312	177.407479224821\\
3.13313313313313	177.499806609629\\
3.14314314314314	177.587453013254\\
3.15315315315315	177.670605070018\\
3.16316316316316	177.749443682618\\
3.17317317317317	177.824144022132\\
3.18318318318318	177.894875528016\\
3.19319319319319	177.961801908105\\
3.2032032032032	178.025080617562\\
3.21321321321321	178.084798099588\\
3.22322322322322	178.141031750989\\
3.23323323323323	178.193889339523\\
3.24324324324324	178.243473389025\\
3.25325325325325	178.289881179411\\
3.26326326326326	178.333204746676\\
3.27327327327327	178.373530882894\\
3.28328328328328	178.410941136216\\
3.29329329329329	178.445511810874\\
3.3033033033033	178.477313967179\\
3.31331331331331	178.50641315833\\
3.32332332332332	178.532866537968\\
3.33333333333333	178.556725237152\\
3.34334334334334	178.57803613702\\
3.35335335335335	178.596841765533\\
3.36336336336336	178.613180297477\\
3.37337337337337	178.627085554463\\
3.38338338338338	178.638587004925\\
3.39339339339339	178.647709764122\\
3.4034034034034	178.654474594139\\
3.41341341341341	178.658897903883\\
3.42342342342342	178.660906142055\\
3.43343343343343	178.660292669433\\
3.44344344344344	178.656962220684\\
3.45345345345345	178.650830143186\\
3.46346346346346	178.641819608493\\
3.47347347347347	178.629861612338\\
3.48348348348348	178.614894974628\\
3.49349349349349	178.596866339451\\
3.5035035035035	178.575730175071\\
3.51351351351351	178.551448773927\\
3.52352352352352	178.523992252638\\
3.53353353353353	178.493338551998\\
3.54354354354354	178.459466043874\\
3.55355355355355	178.422309671223\\
3.56356356356356	178.381797453765\\
3.57357357357357	178.337853023538\\
3.58358358358358	178.290393537415\\
3.59359359359359	178.239329677108\\
3.6036036036036	178.184565649166\\
3.61361361361361	178.125999184973\\
3.62362362362362	178.063521540752\\
3.63363363363363	177.997017497561\\
3.64364364364364	177.926365361296\\
3.65365365365365	177.85143696269\\
3.66366366366366	177.772097838209\\
3.67367367367367	177.688205800792\\
3.68368368368368	177.599601688927\\
3.69369369369369	177.506114742675\\
3.7037037037037	177.407564023583\\
3.71371371371371	177.303758414692\\
3.72372372372372	177.194496620527\\
3.73373373373373	177.079567167106\\
3.74374374374374	176.958748401934\\
3.75375375375375	176.831808494007\\
3.76376376376376	176.698505433808\\
3.77377377377377	176.55858703331\\
3.78378378378378	176.411790925975\\
3.79379379379379	176.257844566756\\
3.8038038038038	176.096465232091\\
3.81381381381381	175.927360019912\\
3.82382382382382	175.750225849637\\
3.83383383383383	175.564742058338\\
3.84384384384384	175.370541674422\\
3.85385385385385	175.167247500119\\
3.86386386386386	174.95446356449\\
3.87387387387387	174.731773175357\\
3.88388388388388	174.498738919302\\
3.89389389389389	174.254902661668\\
3.9039039039039	173.999785546559\\
3.91391391391391	173.732887996839\\
3.92392392392392	173.453689714132\\
3.93393393393393	173.161649678824\\
3.94394394394394	172.856206150059\\
3.95395395395395	172.536776665744\\
3.96396396396396	172.202758042545\\
3.97397397397397	171.85352637589\\
3.98398398398398	171.488437039967\\
3.99399399399399	171.106824687722\\
4.004004004004	170.708003250866\\
4.01401401401401	170.291276373517\\
4.02402402402402	169.855946478489\\
4.03403403403403	169.401107800808\\
4.04404404404404	168.92579286621\\
4.05405405405405	168.429004206315\\
4.06406406406406	167.90971435862\\
4.07407407407407	167.366865866505\\
4.08408408408408	166.799371279229\\
4.09409409409409	166.206113151929\\
4.1041041041041	165.585944045625\\
4.11411411411411	164.937686527215\\
4.12412412412412	164.260133169478\\
4.13413413413413	163.552046551074\\
4.14414414414414	162.812159256541\\
4.15415415415415	162.039173876297\\
4.16416416416416	161.231763006643\\
4.17417417417417	160.388569249758\\
4.18418418418418	159.508205213699\\
4.19419419419419	158.589253512407\\
4.2042042042042	157.629850642178\\
4.21421421421421	156.627785310345\\
4.22422422422422	155.581129645818\\
4.23423423423423	154.487940572245\\
4.24424424424424	153.34625964377\\
4.25425425425425	152.154113045034\\
4.26426426426426	150.909511591169\\
4.27427427427427	149.610450727805\\
4.28428428428428	148.255023394642\\
4.29429429429429	146.841426624607\\
4.3043043043043	145.366503957427\\
4.31431431431431	143.82717754134\\
4.32432432432432	142.220667816882\\
4.33433433433433	140.544493516889\\
4.34434434434434	138.796471666498\\
4.35435435435435	136.974717583146\\
4.36436436436436	135.077644876571\\
4.37437437437437	133.102770342532\\
4.38438438438438	131.045724339711\\
4.39439439439439	128.903065582184\\
4.4044044044044	126.672142480945\\
4.41441441441441	124.351092074801\\
4.42442442442442	121.938976172864\\
4.43443443443443	119.432739621096\\
4.44444444444444	116.826819841299\\
4.45445445445445	114.11804762283\\
4.46446446446446	111.305876391075\\
4.47447447447447	108.392331315184\\
4.48448448448448	105.376239608899\\
4.49449449449449	102.249382968416\\
4.5045045045045	99.0076041767762\\
4.51451451451451	95.6520920318265\\
4.52452452452452	92.1887347404293\\
4.53453453453453	88.6167371217581\\
4.54454454454454	84.9308737185251\\
4.55455455455455	81.129893633619\\
4.56456456456456	77.216514721589\\
4.57457457457457	73.1962385367554\\
4.58458458458458	69.0693688716887\\
4.59459459459459	64.8370184204438\\
4.6046046046046	60.5051044720796\\
4.61461461461461	56.0799586333403\\
4.62462462462462	51.5584857839806\\
4.63463463463463	46.9478100281619\\
4.64464464464464	42.2572465287935\\
4.65465465465465	37.4889608721356\\
4.66466466466466	32.6548361649825\\
4.67467467467467	27.7605371170437\\
4.68468468468468	22.8133513165728\\
4.69469469469469	17.8387470095735\\
4.7047047047047	12.8294449116092\\
4.71471471471471	7.79177980711158\\
4.72472472472472	2.7538413473603\\
4.73473473473473	-2.30464996620196\\
4.74474474474474	-7.36115883819812\\
4.75475475475475	-12.4107039783849\\
4.76476476476476	-17.438672853173\\
4.77477477477477	-22.4346559486303\\
4.78478478478478	-27.385041352948\\
4.79479479479479	-32.2762351133043\\
4.8048048048048	-37.1132323047384\\
4.81481481481481	-41.8756655635394\\
4.82482482482482	-46.5599020773377\\
4.83483483483483	-51.1713100265758\\
4.84484484484484	-55.694039839295\\
4.85485485485485	-60.1282366293217\\
4.86486486486486	-64.4699786609676\\
4.87487487487487	-68.7089585854046\\
4.88488488488488	-72.845123696406\\
4.89489489489489	-76.8761433513461\\
4.9049049049049	-80.7936940679452\\
4.91491491491491	-84.600815177862\\
4.92492492492492	-88.2981528956639\\
4.93493493493493	-91.8811311820014\\
4.94494494494494	-95.3509526403762\\
4.95495495495495	-98.7109422090528\\
4.96496496496496	-101.960322644408\\
4.97497497497497	-105.097253907048\\
4.98498498498498	-108.125962193518\\
4.99499499499499	-111.049418204958\\
5.00500500500501	-113.868301802871\\
5.01501501501502	-116.582558479714\\
5.02502502502503	-119.196155664744\\
5.03503503503504	-121.712627084799\\
5.04504504504505	-124.134341619886\\
5.05505505505506	-126.462538294041\\
5.06506506506507	-128.699254559307\\
5.07507507507508	-130.848327496617\\
5.08508508508509	-132.912996248188\\
5.0950950950951	-134.895992132268\\
5.10510510510511	-136.799538643134\\
5.11511511511512	-138.625498027323\\
5.12512512512513	-140.376822390715\\
5.13513513513514	-142.056593916365\\
5.14514514514515	-143.667601991959\\
5.15515515515516	-145.212426593213\\
5.16516516516517	-146.693438283873\\
5.17517517517518	-148.112798215719\\
5.18518518518519	-149.472519944498\\
5.1951951951952	-150.775090545631\\
5.20520520520521	-152.022885575854\\
5.21521521521522	-153.218116250092\\
5.22522522522523	-154.362904303609\\
5.23523523523524	-155.459281991999\\
5.24524524524525	-156.509192091193\\
5.25525525525526	-157.514487897456\\
5.26526526526527	-158.47693322739\\
5.27527527527528	-159.398202417928\\
5.28528528528529	-160.280120817919\\
5.2952952952953	-161.124466218694\\
5.30530530530531	-161.932694302504\\
5.31531531531532	-162.706224851151\\
5.32532532532533	-163.446442071129\\
5.33533533533534	-164.15469459362\\
5.34534534534535	-164.832295474499\\
5.35535535535536	-165.48052219433\\
5.36536536536537	-166.100616658368\\
5.37537537537538	-166.693785196558\\
5.38538538538539	-167.261198563536\\
5.3953953953954	-167.803991938628\\
5.40540540540541	-168.32326492585\\
5.41541541541542	-168.820081553909\\
5.42542542542543	-169.295473386666\\
5.43543543543544	-169.750536535878\\
5.44544544544545	-170.186108304518\\
5.45545545545546	-170.60289436718\\
5.46546546546547	-171.001588210114\\
5.47547547547548	-171.382871131235\\
5.48548548548549	-171.747412240115\\
5.4954954954955	-172.095868457988\\
5.50550550550551	-172.42888451775\\
5.51551551551552	-172.747092963954\\
5.52552552552553	-173.051114152815\\
5.53553553553554	-173.341556252211\\
5.54554554554555	-173.619015241676\\
5.55555555555556	-173.884074912407\\
5.56556556556557	-174.137306867261\\
5.57557557557558	-174.379270520756\\
5.58558558558559	-174.610513099069\\
5.5955955955956	-174.83156964004\\
5.60560560560561	-175.042962993166\\
5.61561561561562	-175.245198560147\\
5.62562562562563	-175.438573836633\\
5.63563563563564	-175.62335076967\\
5.64564564564565	-175.799841618656\\
5.65565565565566	-175.968351817494\\
5.66566566566567	-176.129179974588\\
5.67567567567568	-176.28261787285\\
5.68568568568569	-176.428950469695\\
5.6956956956957	-176.568455897042\\
5.70570570570571	-176.701405461315\\
5.71571571571572	-176.828063643443\\
5.72572572572573	-176.94868809886\\
5.73573573573574	-177.063529657501\\
5.74574574574575	-177.172832323811\\
5.75575575575576	-177.276833276734\\
5.76576576576577	-177.375762869722\\
5.77577577577578	-177.469844630731\\
5.78578578578579	-177.55929526222\\
5.7957957957958	-177.64432419526\\
5.80580580580581	-177.725077796693\\
5.81581581581582	-177.801670229053\\
5.82582582582583	-177.874232192238\\
5.83583583583584	-177.942890249393\\
5.84584584584585	-178.007766826916\\
5.85585585585586	-178.068980214455\\
5.86586586586587	-178.12664456491\\
5.87587587587588	-178.18086989443\\
5.88588588588589	-178.231762082414\\
5.8958958958959	-178.279422871515\\
5.90590590590591	-178.323949867632\\
5.91591591591592	-178.365436539919\\
5.92592592592593	-178.403972220778\\
5.93593593593594	-178.439642105863\\
5.94594594594595	-178.472527254077\\
5.95595595595596	-178.502704587576\\
5.96596596596597	-178.530246891765\\
5.97597597597598	-178.5552228153\\
5.98598598598599	-178.577733594561\\
5.995995995996	-178.59788596531\\
6.00600600600601	-178.615677743175\\
6.01601601601602	-178.631098238993\\
6.02602602602603	-178.64413134152\\
6.03603603603604	-178.654755517426\\
6.04604604604605	-178.662943811303\\
6.05605605605606	-178.668663845659\\
6.06606606606607	-178.671877820919\\
6.07607607607608	-178.672542515429\\
6.08608608608609	-178.670609285449\\
6.0960960960961	-178.666024065159\\
6.10610610610611	-178.658727366656\\
6.11611611611612	-178.648654279957\\
6.12612612612613	-178.635734472995\\
6.13613613613614	-178.61989219162\\
6.14614614614615	-178.601046259602\\
6.15615615615616	-178.579110078628\\
6.16616616616617	-178.553991786793\\
6.17617617617618	-178.525710670029\\
6.18618618618619	-178.494292152519\\
6.1961961961962	-178.459668758285\\
6.20620620620621	-178.4217676753\\
6.21621621621622	-178.380510755491\\
6.22622622622623	-178.335814514734\\
6.23623623623624	-178.287590132859\\
6.24624624624625	-178.235743453647\\
6.25625625625626	-178.180174984829\\
6.26626626626627	-178.12077989809\\
6.27627627627628	-178.057448029066\\
6.28628628628629	-177.990063877344\\
6.2962962962963	-177.918506606463\\
6.30630630630631	-177.842650043914\\
6.31631631631632	-177.76236268114\\
6.32632632632633	-177.677507673534\\
6.33633633633634	-177.587942840443\\
6.34634634634635	-177.49351683427\\
6.35635635635636	-177.394027680284\\
6.36636636636637	-177.289276838349\\
6.37637637637638	-177.179063120483\\
6.38638638638639	-177.063173643336\\
6.3963963963964	-176.941383828188\\
6.40640640640641	-176.813457400951\\
6.41641641641642	-176.679146392171\\
6.42642642642643	-176.538191137024\\
6.43643643643644	-176.390320275317\\
6.44644644644645	-176.235250751489\\
6.45645645645646	-176.072687814612\\
6.46646646646647	-175.902325018389\\
6.47647647647648	-175.723844221153\\
6.48648648648649	-175.536915585872\\
6.4964964964965	-175.341197580142\\
6.50650650650651	-175.136336976194\\
6.51651651651652	-174.921963813673\\
6.52652652652653	-174.697659251179\\
6.53653653653654	-174.462985030924\\
6.54654654654655	-174.217481779756\\
6.55655655655656	-173.960666602781\\
6.56656656656657	-173.692033083356\\
6.57657657657658	-173.411051283095\\
6.58658658658659	-173.117167741865\\
6.5965965965966	-172.809805477788\\
6.60660660660661	-172.48836398724\\
6.61661661661662	-172.152219244852\\
6.62662662662663	-171.800723703509\\
6.63663663663664	-171.433206294349\\
6.64664664664665	-171.048972426766\\
6.65665665665666	-170.647303988409\\
6.66666666666667	-170.227459345178\\
6.67667667667668	-169.788673341232\\
6.68668668668669	-169.330157298981\\
6.6966966966967	-168.851098222755\\
6.70670670670671	-168.350552839831\\
6.71671671671672	-167.827487997337\\
6.72672672672673	-167.280867131464\\
6.73673673673674	-166.709612731334\\
6.74674674674675	-166.112606338993\\
6.75675675675676	-165.488688549418\\
6.76676676676677	-164.836659010513\\
6.77677677677678	-164.15527642311\\
6.78678678678679	-163.44325854097\\
6.7967967967968	-162.699282170782\\
6.80680680680681	-161.921983172161\\
6.81681681681682	-161.109956457653\\
6.82682682682683	-160.26175599273\\
6.83683683683684	-159.375894795792\\
6.84684684684685	-158.45083225719\\
6.85685685685686	-157.48480246144\\
6.86686686686687	-156.475878213127\\
6.87687687687688	-155.422115226504\\
6.88688688688689	-154.321555290068\\
6.8968968968969	-153.172226266561\\
6.90690690690691	-151.972142092972\\
6.91691691691692	-150.719302780533\\
6.92692692692693	-149.411694414722\\
6.93693693693694	-148.047289836298\\
6.94694694694695	-146.624056121933\\
6.95695695695696	-145.139246181594\\
6.96696696696697	-143.589967237269\\
6.97697697697698	-141.973530514873\\
6.98698698698699	-140.287451244254\\
6.996996996997	-138.529448659188\\
7.00700700700701	-136.697445997379\\
7.01701701701702	-134.789469196743\\
7.02702702702703	-132.802287594994\\
7.03703703703704	-130.732099346066\\
7.04704704704705	-128.575830471742\\
7.05705705705706	-126.331167777242\\
7.06706706706707	-123.996558851218\\
7.07707707707708	-121.571606990197\\
7.08708708708709	-119.051616890966\\
7.0970970970971	-116.43049166766\\
7.10710710710711	-113.705264641945\\
7.11711711711712	-110.876114617244\\
7.12712712712713	-107.946262026909\\
7.13713713713714	-104.91432032254\\
7.14714714714715	-101.771996490311\\
7.15715715715716	-98.5148953259127\\
7.16716716716717	-95.1429817817733\\
7.17717717717718	-91.6604431423508\\
7.18718718718719	-88.0693110087418\\
7.1971971971972	-84.3662648290595\\
7.20720720720721	-80.5505107796337\\
7.21721721721722	-76.6244257162653\\
7.22722722722723	-72.5935031195576\\
7.23723723723724	-68.4568109493647\\
7.24724724724725	-64.2125305571507\\
7.25725725725726	-59.8684905198118\\
7.26726726726727	-55.4297410010386\\
7.27727727727728	-50.8951377776441\\
7.28728728728729	-46.2814059427089\\
7.2972972972973	-41.6016315265728\\
7.30730730730731	-36.8350244318029\\
7.31731731731732	-31.9956194881082\\
7.32732732732733	-27.092385774686\\
7.33733733733734	-22.1355716896557\\
7.34734734734735	-17.1370178525357\\
7.35735735735736	-12.1024185048422\\
7.36736736736737	-7.04559443362164\\
7.37737737737738	-1.97457712307659\\
7.38738738738739	3.10114582709003\\
7.3973973973974	8.16869512937311\\
7.40740740740741	13.2216953271212\\
7.41741741741742	18.2472794516683\\
7.42742742742743	23.2374237505883\\
7.43743743743744	28.185825519904\\
7.44744744744745	33.0791442961354\\
7.45745745745746	37.9131798219802\\
7.46746746746747	42.6774248463481\\
7.47747747747748	47.3633018631824\\
7.48748748748749	51.9693408290501\\
7.4974974974975	56.48660558932\\
7.50750750750751	60.9083980878287\\
7.51751751751752	65.2350776625148\\
7.52752752752753	69.4585295021932\\
7.53753753753754	73.5742874540131\\
7.54754754754755	77.5846833065137\\
7.55755755755756	81.4846741259587\\
7.56756756756757	85.270691820614\\
7.57757757757758	88.9468485128684\\
7.58758758758759	92.5121312381477\\
7.5975975975976	95.9620913727698\\
7.60760760760761	99.3002811351486\\
7.61761761761762	102.529506635436\\
7.62762762762763	105.648993289255\\
7.63763763763764	108.658086063902\\
7.64764764764765	111.561143570616\\
7.65765765765766	114.361093284225\\
7.66766766766767	117.058954828535\\
7.67767767767768	119.655056457775\\
7.68768768768769	122.153247390287\\
7.6976976976977	124.557109404062\\
7.70770770770771	126.869268228012\\
7.71771771771772	129.091424779753\\
7.72772772772773	131.225246619838\\
7.73773773773774	133.274378761885\\
7.74774774774775	135.242105076966\\
7.75775775775776	137.131297876371\\
7.76776776776777	138.944429200196\\
7.77777777777778	140.683570817339\\
7.78778778778779	142.350928301901\\
7.7977977977978	143.949579331251\\
7.80780780780781	145.482276547235\\
7.81781781781782	146.951604176698\\
7.82782782782783	148.359981852826\\
7.83783783783784	149.709664615154\\
7.84784784784785	151.002742909557\\
7.85785785785786	152.241172374024\\
7.86786786786787	153.427253932044\\
7.87787787787788	154.563180493238\\
7.88788788788789	155.65097680142\\
7.8978978978979	156.692595827152\\
7.90790790790791	157.689918767741\\
7.91791791791792	158.644755047241\\
7.92792792792793	159.558842316452\\
7.93793793793794	160.433846452918\\
7.94794794794795	161.271361560932\\
7.95795795795796	162.072909971531\\
7.96796796796797	162.840224322354\\
7.97797797797798	163.575034257857\\
7.98798798798799	164.278516431552\\
7.997997997998	164.951820881689\\
8.00800800800801	165.596077491565\\
8.01801801801802	166.212395989522\\
8.02802802802803	166.801865948945\\
8.03803803803804	167.365556788266\\
8.04804804804805	167.904517770962\\
8.05805805805806	168.419778005556\\
8.06806806806807	168.912346445614\\
8.07807807807808	169.38321188975\\
8.08808808808809	169.83334298162\\
8.0980980980981	170.263688209929\\
8.10810810810811	170.675175908424\\
8.11811811811812	171.068714255899\\
8.12812812812813	171.445191276193\\
8.13813813813814	171.805474838189\\
8.14814814814815	172.150392021771\\
8.15815815815816	172.48049747488\\
8.16816816816817	172.796302403683\\
8.17817817817818	173.098334270986\\
8.18818818818819	173.387110298676\\
8.1981981981982	173.663137467726\\
8.20820820820821	173.926912518191\\
8.21821821821822	174.178921949208\\
8.22822822822823	174.419642018999\\
8.23823823823824	174.64953874487\\
8.24824824824825	174.869067903209\\
8.25825825825826	175.078675029487\\
8.26826826826827	175.27879541826\\
8.27827827827828	175.469854123164\\
8.28828828828829	175.652265956923\\
8.2982982982983	175.82643549134\\
8.30830830830831	175.992757057304\\
8.31831831831832	176.151614744787\\
8.32832832832833	176.303381851249\\
8.33833833833834	176.448320450063\\
8.34834834834835	176.58663024911\\
8.35835835835836	176.718545072137\\
8.36836836836837	176.844293179083\\
8.37837837837838	176.964097266077\\
8.38838838838839	177.07817446544\\
8.3983983983984	177.186736345684\\
8.40840840840841	177.289988911513\\
8.41841841841842	177.388132603821\\
8.42842842842843	177.481362299695\\
8.43843843843844	177.569867312412\\
8.44844844844845	177.65383139144\\
8.45845845845846	177.733432722439\\
8.46846846846847	177.808843927262\\
8.47847847847848	177.880232063949\\
8.48848848848849	177.947758626736\\
8.4984984984985	178.011579546047\\
8.50850850850851	178.071845188499\\
8.51851851851852	178.128656246277\\
8.52852852852853	178.182072903146\\
8.53853853853854	178.232197207645\\
8.54854854854855	178.279126355221\\
8.55855855855856	178.322952395516\\
8.56856856856857	178.363762232371\\
8.57857857857858	178.40163762382\\
8.58858858858859	178.436655182097\\
8.5985985985986	178.46888637363\\
8.60860860860861	178.498397519045\\
8.61861861861862	178.52524934476\\
8.62862862862863	178.549493344661\\
8.63863863863864	178.571177522218\\
8.64864864864865	178.590345862574\\
8.65865865865866	178.607037644076\\
8.66866866866867	178.621287438121\\
8.67867867867868	178.633125010769\\
8.68868868868869	178.642575252165\\
8.6986986986987	178.649658542763\\
8.70870870870871	178.654390814253\\
8.71871871871872	178.656783549559\\
8.72872872872873	178.656765673249\\
8.73873873873874	178.654123001882\\
8.74874874874875	178.648747304121\\
8.75875875875876	178.640544068626\\
8.76876876876877	178.629428957837\\
8.77877877877878	178.615327807975\\
8.78878878878879	178.598176629039\\
8.7987987987988	178.577921604811\\
8.80880880880881	178.554519092851\\
8.81881881881882	178.527935624502\\
8.82882882882883	178.498147904883\\
8.83883883883884	178.465141762912\\
8.84884884884885	178.428865584073\\
8.85885885885886	178.389243045739\\
8.86886886886887	178.346199919827\\
8.87887887887888	178.299655436791\\
8.88888888888889	178.249522285623\\
8.8988988988989	178.195706613853\\
8.90890890890891	178.13810802755\\
8.91891891891892	178.076619591318\\
8.92892892892893	178.011127828302\\
8.93893893893894	177.941512720185\\
8.94894894894895	177.867647707184\\
8.95895895895896	177.789400677412\\
8.96896896896897	177.706631272743\\
8.97897897897898	177.61918124593\\
8.98898898898899	177.526881753199\\
8.998998998999	177.429553997227\\
9.00900900900901	177.327009227137\\
9.01901901901902	177.219048738498\\
9.02902902902903	177.105463873331\\
9.03903903903904	176.986036020099\\
9.04904904904905	176.860536613718\\
9.05905905905906	176.728727135549\\
9.06906906906907	176.590359113401\\
9.07907907907908	176.445174121531\\
9.08908908908909	176.292903780643\\
9.0990990990991	176.13326975789\\
9.10910910910911	175.965983766871\\
9.11911911911912	175.790747567634\\
9.12912912912913	175.607233924687\\
9.13913913913914	175.415079330815\\
9.14914914914915	175.213911600866\\
9.15915915915916	175.003338331622\\
9.16916916916917	174.78294688279\\
9.17917917917918	174.55230437701\\
9.18918918918919	174.310957699847\\
9.1991991991992	174.058433499797\\
9.20920920920921	173.794238188285\\
9.21921921921922	173.517857939662\\
9.22922922922923	173.228758691211\\
9.23923923923924	172.926386143142\\
9.24924924924925	172.610165758595\\
9.25925925925926	172.279502763636\\
9.26926926926927	171.933782147264\\
9.27927927927928	171.572368661404\\
9.28928928928929	171.194606820909\\
9.2992992992993	170.799820903563\\
9.30930930930931	170.387339697969\\
9.31931931931932	169.956404100709\\
9.32932932932933	169.506100993007\\
9.33933933933934	169.035484074219\\
9.34934934934935	168.543576437243\\
9.35935935935936	168.029370568514\\
9.36936936936937	167.491828348011\\
9.37937937937938	166.929881049254\\
9.38938938938939	166.342429339301\\
9.3993993993994	165.728343278755\\
9.40940940940941	165.086462321756\\
9.41941941941942	164.415595315988\\
9.42942942942943	163.714520502674\\
9.43943943943944	162.981985516578\\
9.44944944944945	162.216707386007\\
9.45945945945946	161.417372532806\\
9.46946946946947	160.582636772363\\
9.47947947947948	159.711125313606\\
9.48948948948949	158.801385838261\\
9.4994994994995	157.851337757505\\
9.50950950950951	156.858994620506\\
9.51951951951952	155.822465531062\\
9.52952952952953	154.739840149204\\
9.53953953953954	153.6091886912\\
9.54954954954955	152.428561929554\\
9.55955955955956	151.195991193008\\
9.56956956956957	149.909488366535\\
9.57957957957958	148.56724827565\\
9.58958958958959	147.167340691794\\
9.5995995995996	145.706543429494\\
9.60960960960961	144.181822112177\\
9.61961961961962	142.590435609326\\
9.62962962962963	140.929936036477\\
9.63963963963964	139.198168755222\\
9.64964964964965	137.393272373207\\
9.65965965965966	135.513678744132\\
9.66966966966967	133.557028891693\\
9.67967967967968	131.518877973039\\
9.68968968968969	129.395742036167\\
9.6996996996997	127.184919320909\\
9.70970970970971	124.884482261491\\
9.71971971971972	122.493405823012\\
9.72972972972973	120.008982885873\\
9.73973973973974	117.425636652445\\
9.74974974974975	114.739997739193\\
9.75975975975976	111.95124747104\\
9.76976976976977	109.06111177215\\
9.77977977977978	106.069257062946\\
9.78978978978979	102.967489337802\\
9.7997997997998	99.7511071494332\\
9.80980980980981	96.4207871742273\\
9.81981981981982	92.9822532249166\\
9.82982982982983	89.4357929032181\\
9.83983983983984	85.7757310857689\\
9.84984984984985	82.0002935411933\\
9.85985985985986	78.1116939535215\\
9.86986986986987	74.1157154820262\\
9.87987987987988	70.0131753941058\\
9.88988988988989	65.803793438926\\
9.8998998998999	61.492136375064\\
9.90990990990991	57.0856855776916\\
9.91991991991992	52.5835917320974\\
9.92992992992993	47.988190735722\\
9.93993993993994	43.3105620263894\\
9.94994994994995	38.5552381061135\\
9.95995995995996	33.7282146753111\\
9.96996996996997	28.8424817375351\\
9.97997997997998	23.9023320431045\\
9.98998998998999	18.9209860977309\\
10	13.9061046086299\\
};
\addlegendentry{Energy-Based}

\addplot [color=mycolor2, line width=2.0pt]
  table[row sep=crcr]{%
0	0\\
0.01001001001001	0.0229527964285707\\
0.02002002002002	0.0917354675841067\\
0.03003003003003	0.206165207888467\\
0.04004004004004	0.365971012420373\\
0.0500500500500501	0.570795798577409\\
0.0600600600600601	0.820194750916577\\
0.0700700700700701	1.1136353211543\\
0.0800800800800801	1.4504972281664\\
0.0900900900900901	1.83007245798815\\
0.1001001001001	2.25156618018758\\
0.11011011011011	2.7141211616656\\
0.12012012012012	3.2167747896164\\
0.13013013013013	3.75848377004713\\
0.14014014014014	4.33814243626362\\
0.15015015015015	4.95458274887042\\
0.16016016016016	5.60657429577076\\
0.17017017017017	6.2928242921666\\
0.18018018018018	7.01197758055856\\
0.19019019019019	7.76261663074601\\
0.2002002002002	8.54326153982699\\
0.21021021021021	9.35237003219824\\
0.22022022022022	10.1883374595552\\
0.23023023023023	11.049496800892\\
0.24024024024024	11.9341186625016\\
0.25025025025025	12.8404112779754\\
0.26026026026026	13.7665615115009\\
0.27027027027027	14.7110592633673\\
0.28028028028028	15.6720597518029\\
0.29029029029029	16.647624511023\\
0.3003003003003	17.6358295576027\\
0.31031031031031	18.6347653904763\\
0.32032032032032	19.6425369909379\\
0.33033033033033	20.6572638226409\\
0.34034034034034	21.6770798315981\\
0.35035035035035	22.7001334461819\\
0.36036036036036	23.7245875771244\\
0.37037037037037	24.7486196175167\\
0.38038038038038	25.7704214428099\\
0.39039039039039	26.7881994108142\\
0.4004004004004	27.8001743616996\\
0.41041041041041	28.8045816179952\\
0.42042042042042	29.7996709845901\\
0.43043043043043	30.7837067487324\\
0.44044044044044	31.75496768003\\
0.45045045045045	32.7119601606338\\
0.46046046046046	33.6533255292228\\
0.47047047047047	34.5774333778422\\
0.48048048048048	35.4827132482347\\
0.49049049049049	36.3676582845688\\
0.500500500500501	37.2308252334397\\
0.510510510510511	38.0708344438683\\
0.520520520520521	38.8863698673021\\
0.530530530530531	39.6761790576147\\
0.540540540540541	40.4390731711059\\
0.550550550550551	41.1739269665017\\
0.560560560560561	41.8796788049543\\
0.570570570570571	42.5553306500422\\
0.580580580580581	43.1999480677701\\
0.590590590590591	43.8126602265688\\
0.600600600600601	44.3926598972954\\
0.610610610610611	44.9392034532332\\
0.620620620620621	45.4516108700918\\
0.630630630630631	45.929265726007\\
0.640640640640641	46.3716152015405\\
0.650650650650651	46.7780482201538\\
0.660660660660661	47.1479498890227\\
0.670670670670671	47.4808550670263\\
0.680680680680681	47.7763530050871\\
0.690690690690691	48.0340873398986\\
0.700700700700701	48.2537560939258\\
0.710710710710711	48.4351116754052\\
0.720720720720721	48.5779608783445\\
0.730730730730731	48.6821640814574\\
0.740740740740741	48.7476197117965\\
0.750750750750751	48.7742780866202\\
0.760760760760761	48.7260405286235\\
0.770770770770771	48.5477381201039\\
0.780780780780781	48.2393409027903\\
0.790790790790791	47.80138153309\\
0.800800800800801	47.2345753355244\\
0.810810810810811	46.5400185262456\\
0.820820820820821	45.718721141524\\
0.830830830830831	44.7717809684676\\
0.840840840840841	43.7005326043546\\
0.850850850850851	42.5065474566329\\
0.860860860860861	41.1916337429207\\
0.870870870870871	39.757836491006\\
0.880880880880881	38.2074375388469\\
0.890890890890891	36.5429555345717\\
0.900900900900901	34.7671459364784\\
0.910910910910911	32.8830010130353\\
0.920920920920921	30.8937498428805\\
0.930930930930931	28.8028583148223\\
0.940940940940941	26.6140291278388\\
0.950950950950951	24.3312017910784\\
0.960960960960961	21.9584811496885\\
0.970970970970971	19.4994102915786\\
0.980980980980981	16.9588260707284\\
0.990990990990991	14.3420296606986\\
1.001001001001	11.6543693960581\\
1.01101101101101	8.90124077238413\\
1.02102102102102	6.08808644626185\\
1.03103103103103	3.22039623528472\\
1.04104104104104	0.303707118054355\\
1.05105105105105	-2.65639676581946\\
1.06106106106106	-5.65428411571876\\
1.07107107107107	-8.68427647001741\\
1.08108108108108	-11.7406482060811\\
1.09109109109109	-14.8176265402673\\
1.1011011011011	-17.9093915279255\\
1.11111111111111	-21.0100760633966\\
1.12112112112112	-24.1137658800139\\
1.13113113113113	-27.214499550102\\
1.14114114114114	-30.3062684849776\\
1.15115115115115	-33.3830169349491\\
1.16116116116116	-36.4389313665938\\
1.17117117117117	-39.470185845111\\
1.18118118118118	-42.4724993825067\\
1.19119119119119	-45.4415980504507\\
1.2012012012012	-48.3734503432864\\
1.21121121121121	-51.2642671780303\\
1.22122122122122	-54.110501894372\\
1.23123123123123	-56.9088502546747\\
1.24124124124124	-59.6562504439747\\
1.25125125125125	-62.3498830699814\\
1.26126126126126	-64.9871711630776\\
1.27127127127127	-67.5657801763195\\
1.28128128128128	-70.0836179854364\\
1.29129129129129	-72.5388348888308\\
1.3013013013013	-74.9298236075785\\
1.31131131131131	-77.2552192854287\\
1.32132132132132	-79.5138994888037\\
1.33133133133133	-81.704984206799\\
1.34134134134134	-83.8278319319326\\
1.35135135135135	-85.8820369860402\\
1.36136136136136	-87.8675816607711\\
1.37137137137137	-89.7845653189511\\
1.38138138138138	-91.6331367073867\\
1.39139139139139	-93.4134939568656\\
1.4014014014014	-95.1258845821563\\
1.41141141141141	-96.7706054820084\\
1.42142142142142	-98.3480029391523\\
1.43143143143143	-99.8584726202991\\
1.44144144144144	-101.302459576141\\
1.45145145145145	-102.680458241352\\
1.46146146146146	-103.993012434585\\
1.47147147147147	-105.240715358476\\
1.48148148148148	-106.424209599641\\
1.49149149149149	-107.544187128676\\
1.5015015015015	-108.60138930016\\
1.51151151151151	-109.59660685265\\
1.52152152152152	-110.530679908688\\
1.53153153153153	-111.404497974793\\
1.54154154154154	-112.218742484467\\
1.55155155155155	-112.973858246497\\
1.56156156156156	-113.670473958859\\
1.57157157157157	-114.309185316365\\
1.58158158158158	-114.890554118494\\
1.59159159159159	-115.415108269394\\
1.6016016016016	-115.883341777878\\
1.61161161161161	-116.29571475743\\
1.62162162162162	-116.652653156254\\
1.63163163163163	-116.954523923055\\
1.64164164164164	-117.201641402215\\
1.65165165165165	-117.394281233386\\
1.66166166166166	-117.532671669821\\
1.67167167167167	-117.616992254548\\
1.68168168168168	-117.643715323163\\
1.69169169169169	-117.552853946502\\
1.7017017017017	-117.312605464729\\
1.71171171171171	-116.927634850716\\
1.72172172172172	-116.396781051036\\
1.73173173173173	-115.719740714583\\
1.74174174174174	-114.896009669889\\
1.75175175175175	-113.925016559296\\
1.76176176176176	-112.806128993998\\
1.77177177177177	-111.538653554044\\
1.78178178178178	-110.121835788332\\
1.79179179179179	-108.554860214614\\
1.8018018018018	-106.836850319492\\
1.81181181181181	-104.966916251494\\
1.82182182182182	-102.944448667506\\
1.83183183183183	-100.768405876758\\
1.84184184184184	-98.4378574055404\\
1.85185185185185	-95.9521346922693\\
1.86186186186186	-93.3108310874875\\
1.87187187187187	-90.5138018538637\\
1.88188188188188	-87.5611641661933\\
1.89189189189189	-84.4532971113975\\
1.9019019019019	-81.1908416885242\\
1.91191191191191	-77.7747008087474\\
1.92192192192192	-74.2060392953672\\
1.93193193193193	-70.4865448887531\\
1.94194194194194	-66.6187766950201\\
1.95195195195195	-62.6056544205321\\
1.96196196196196	-58.4507303484753\\
1.97197197197197	-54.1581967093097\\
1.98198198198198	-49.7328856807689\\
1.99199199199199	-45.1802693878601\\
2.002002002002	-40.506800432524\\
2.01201201201201	-35.7326025261893\\
2.02202202202202	-30.8952743598986\\
2.03203203203203	-25.995265613072\\
2.04204204204204	-21.0413684438095\\
2.05205205205205	-16.0515603223589\\
2.06206206206206	-11.0207798490254\\
2.07207207207207	-5.96805488772194\\
2.08208208208208	-0.8995305855319\\
2.09209209209209	4.16857373986216\\
2.1021021021021	9.22997087281148\\
2.11211211211211	14.272486672133\\
2.12212212212212	19.2795631061329\\
2.13213213213213	24.2556959570022\\
2.14214214214214	29.1798336167826\\
2.15215215215215	34.0502899567801\\
2.16216216216216	38.8632149147962\\
2.17217217217217	43.6037812292299\\
2.18218218218218	48.2718028661812\\
2.19219219219219	52.8569422496284\\
2.2022022022022	57.3523126989561\\
2.21221221221221	61.7545843688472\\
2.22222222222222	66.0543317847427\\
2.23223223223223	70.2499010028821\\
2.24224224224224	74.3432325800556\\
2.25225225225225	78.3275795142006\\
2.26226226226226	82.1979597299701\\
2.27227227227227	85.9609637734166\\
2.28228228228228	89.6166514505759\\
2.29229229229229	93.1594157902009\\
2.3023023023023	96.5880134107986\\
2.31231231231231	99.9074180061144\\
2.32232232232232	103.11840673213\\
2.33233233233233	106.217400519946\\
2.34234234234234	109.206242005908\\
2.35235235235235	112.089978864564\\
2.36236236236236	114.870938280889\\
2.37237237237237	117.549503379162\\
2.38238238238238	120.126949946962\\
2.39239239239239	122.607435409911\\
2.4024024024024	124.994197075806\\
2.41241241241241	127.289531934031\\
2.42242242242242	129.494835593051\\
2.43243243243243	131.612556670165\\
2.44244244244244	133.646359213077\\
2.45245245245245	135.599379967933\\
2.46246246246246	137.474343579566\\
2.47247247247247	139.273562591499\\
2.48248248248248	140.99898361706\\
2.49249249249249	142.653293797445\\
2.5025025025025	144.239481915011\\
2.51251251251251	145.76024879946\\
2.52252252252252	147.218125956425\\
2.53253253253253	148.615475567464\\
2.54254254254254	149.954490490064\\
2.55255255255255	151.237194257642\\
2.56256256256256	152.465644411251\\
2.57257257257257	153.642227914988\\
2.58258258258258	154.769043065266\\
2.59259259259259	155.848104018952\\
2.6026026026026	156.881351237399\\
2.61261261261261	157.870651486442\\
2.62262262262262	158.817797836402\\
2.63263263263263	159.724509662083\\
2.64264264264264	160.592432642775\\
2.65265265265265	161.423138762251\\
2.66266266266266	162.217121916256\\
2.67267267267267	162.970035655038\\
2.68268268268268	163.684620486452\\
2.69269269269269	164.364889731494\\
2.7027027027027	165.014728751864\\
2.71271271271271	165.63789494997\\
2.72272272272272	166.238017768926\\
2.73273273273273	166.818598692551\\
2.74274274274274	167.383011245372\\
2.75275275275275	167.93450099262\\
2.76276276276276	168.476185540234\\
2.77277277277277	169.011054534857\\
2.78278278278278	169.54196966384\\
2.79279279279279	170.071664655238\\
2.8028028028028	170.602745277815\\
2.81281281281281	171.137689341038\\
2.82282282282282	171.678846695082\\
2.83283283283283	172.228439230828\\
2.84284284284284	172.786355007093\\
2.85285285285285	173.344810918511\\
2.86286286286286	173.901902388064\\
2.87287287287287	174.456512181738\\
2.88288288288288	175.007570042761\\
2.89289289289289	175.554052691607\\
2.9029029029029	176.094983825992\\
2.91291291291291	176.629434120876\\
2.92292292292292	177.156521228462\\
2.93293293293293	177.675409778198\\
2.94294294294294	178.185311376773\\
2.95295295295295	178.685484608121\\
2.96296296296296	179.175235033421\\
2.97297297297297	179.653915191091\\
2.98298298298298	180.120924596798\\
2.99299299299299	180.575709743449\\
3.003003003003	181.017747957259\\
3.01301301301301	181.446479829071\\
3.02302302302302	181.861365034906\\
3.03303303303303	182.261908005621\\
3.04304304304304	182.647657907482\\
3.05305305305305	183.018208642161\\
3.06306306306306	183.373198846737\\
3.07307307307307	183.712311893694\\
3.08308308308308	184.035275890925\\
3.09309309309309	184.341863681729\\
3.1031031031031	184.631892844812\\
3.11311311311311	184.905225694286\\
3.12312312312312	185.161769279671\\
3.13313313313313	185.401475385892\\
3.14314314314314	185.624340533282\\
3.15315315315315	185.830405977581\\
3.16316316316316	186.019757709936\\
3.17317317317317	186.192526456899\\
3.18318318318318	186.34888768043\\
3.19319319319319	186.488930181533\\
3.2032032032032	186.612606722936\\
3.21321321321321	186.720120275193\\
3.22322322322322	186.811699478538\\
3.23323323323323	186.887591237367\\
3.24324324324324	186.948060720234\\
3.25325325325325	186.993391359849\\
3.26326326326326	187.023884853082\\
3.27327327327327	187.03986116096\\
3.28328328328328	187.041658508671\\
3.29329329329329	187.029633385557\\
3.3033033033033	187.004160545122\\
3.31331331331331	186.965633005026\\
3.32332332332332	186.914462047088\\
3.33333333333333	186.851077217286\\
3.34334334334334	186.775926325754\\
3.35335335335335	186.689475446787\\
3.36336336336336	186.592208918836\\
3.37337337337337	186.484577284639\\
3.38338338338338	186.366968223841\\
3.39339339339339	186.23990709386\\
3.4034034034034	186.103921799947\\
3.41341341341341	185.959535279424\\
3.42342342342342	185.80726550168\\
3.43343343343343	185.647625468174\\
3.44344344344344	185.481123212433\\
3.45345345345345	185.308261800053\\
3.46346346346346	185.129539328698\\
3.47347347347347	184.945448928102\\
3.48348348348348	184.756478760066\\
3.49349349349349	184.563112018461\\
3.5035035035035	184.365826929226\\
3.51351351351351	184.165096750368\\
3.52352352352352	183.961389771965\\
3.53353353353353	183.755154617554\\
3.54354354354354	183.546742016984\\
3.55355355355355	183.336645707672\\
3.56356356356356	183.125378447861\\
3.57357357357357	182.913433606521\\
3.58358358358358	182.701285163348\\
3.59359359359359	182.489387708766\\
3.6036036036036	182.278176443926\\
3.61361361361361	182.068067180705\\
3.62362362362362	181.859456341707\\
3.63363363363363	181.652720960266\\
3.64364364364364	181.448218680438\\
3.65365365365365	181.24628775701\\
3.66366366366366	181.047247055495\\
3.67367367367367	180.85139605213\\
3.68368368368368	180.659014833884\\
3.69369369369369	180.47036409845\\
3.7037037037037	180.285685154247\\
3.71371371371371	180.105199920424\\
3.72372372372372	179.929110926855\\
3.73373373373373	179.757601314141\\
3.74374374374374	179.590856882775\\
3.75375375375375	179.429145198656\\
3.76376376376376	179.272681753736\\
3.77377377377377	179.121657872485\\
3.78378378378378	178.976247717918\\
3.79379379379379	178.836608291597\\
3.8038038038038	178.702879433633\\
3.81381381381381	178.57518382268\\
3.82382382382382	178.453626975943\\
3.83383383383383	178.338297249171\\
3.84384384384384	178.229265836661\\
3.85385385385385	178.126586771257\\
3.86386386386386	178.03029692435\\
3.87387387387387	177.940416005877\\
3.88388388388388	177.856946564323\\
3.89389389389389	177.779873986719\\
3.9039039039039	177.709166498644\\
3.91391391391391	177.644775164221\\
3.92392392392392	177.586633886124\\
3.93393393393393	177.534659405571\\
3.94394394394394	177.488751302327\\
3.95395395395395	177.448862660338\\
3.96396396396396	177.415018701563\\
3.97397397397397	177.387119758398\\
3.98398398398398	177.365057142237\\
3.99399399399399	177.348716456751\\
4.004004004004	177.337977597889\\
4.01401401401401	177.332714753877\\
4.02402402402402	177.332796405216\\
4.03403403403403	177.338085324688\\
4.04404404404404	177.348438577348\\
4.05405405405405	177.363707520531\\
4.06406406406406	177.383737803849\\
4.07407407407407	177.408369369188\\
4.08408408408408	177.437436450715\\
4.09409409409409	177.470767574872\\
4.1041041041041	177.508185560379\\
4.11411411411411	177.549507518232\\
4.12412412412412	177.594544851705\\
4.13413413413413	177.643124732838\\
4.14414414414414	177.695087205133\\
4.15415415415415	177.750227858738\\
4.16416416416416	177.808343692329\\
4.17417417417417	177.869234375646\\
4.18418418418418	177.9327022495\\
4.19419419419419	177.998552325765\\
4.2042042042042	178.066592287381\\
4.21421421421421	178.136632488358\\
4.22422422422422	178.208485953768\\
4.23423423423423	178.281968379751\\
4.24424424424424	178.356898133515\\
4.25425425425425	178.433096253333\\
4.26426426426426	178.510386448543\\
4.27427427427427	178.588595099551\\
4.28428428428428	178.66755125783\\
4.29429429429429	178.747093234676\\
4.3043043043043	178.827079795584\\
4.31431431431431	178.907328289559\\
4.32432432432432	178.987656953358\\
4.33433433433433	179.067891341538\\
4.34434434434434	179.147864326459\\
4.35435435435435	179.227416098282\\
4.36436436436436	179.30639416497\\
4.37437437437437	179.384653352286\\
4.38438438438438	179.462055803798\\
4.39439439439439	179.538470980873\\
4.4044044044044	179.61377566268\\
4.41441441441441	179.687853946191\\
4.42442442442442	179.760597246179\\
4.43443443443443	179.831904295219\\
4.44444444444444	179.901681143686\\
4.45445445445445	179.96984115976\\
4.46446446446446	180.036305029418\\
4.47447447447447	180.101000756444\\
4.48448448448448	180.16386366242\\
4.49449449449449	180.224821454767\\
4.5045045045045	180.283781326587\\
4.51451451451451	180.340672743774\\
4.52452452452452	180.395431947366\\
4.53453453453453	180.448001440675\\
4.54454454454454	180.498329989281\\
4.55455455455455	180.546372621037\\
4.56456456456456	180.592090626068\\
4.57457457457457	180.635451556768\\
4.58458458458458	180.676429227806\\
4.59459459459459	180.715003716119\\
4.6046046046046	180.751161360917\\
4.61461461461461	180.784894763682\\
4.62462462462462	180.816202788166\\
4.63463463463463	180.845090560394\\
4.64464464464464	180.871569468659\\
4.65465465465465	180.895657163531\\
4.66466466466466	180.917377557845\\
4.67467467467467	180.936760826713\\
4.68468468468468	180.953842883801\\
4.69469469469469	180.968625678033\\
4.7047047047047	180.981123352572\\
4.71471471471471	180.991374933774\\
4.72472472472472	180.99942164896\\
4.73473473473473	181.005306926416\\
4.74474474474474	181.009076395395\\
4.75475475475475	181.010777886113\\
4.76476476476476	181.010461429753\\
4.77477477477477	181.008179258465\\
4.78478478478478	181.003985805361\\
4.79479479479479	180.997937704522\\
4.8048048048048	180.990093790992\\
4.81481481481481	180.980515100781\\
4.82482482482482	180.969264870866\\
4.83483483483483	180.956408539187\\
4.84484484484484	180.942013744651\\
4.85485485485485	180.926150327131\\
4.86486486486486	180.908890028766\\
4.87487487487487	180.890291699861\\
4.88488488488488	180.870424031902\\
4.89489489489489	180.849364428104\\
4.9049049049049	180.827189316416\\
4.91491491491491	180.803974149524\\
4.92492492492492	180.779793404848\\
4.93493493493493	180.754720584546\\
4.94494494494494	180.728828215508\\
4.95495495495495	180.702187849364\\
4.96496496496496	180.674870062475\\
4.97497497497497	180.646944455942\\
4.98498498498498	180.618479655597\\
4.99499499499499	180.589543312012\\
5.00500500500501	180.560202100491\\
5.01501501501502	180.530521721076\\
5.02502502502503	180.500566898543\\
5.03503503503504	180.470393631223\\
5.04504504504505	180.440062096784\\
5.05505505505506	180.409643268919\\
5.06506506506507	180.379205375793\\
5.07507507507508	180.348813877668\\
5.08508508508509	180.3185314669\\
5.0950950950951	180.288418067942\\
5.10510510510511	180.258530837341\\
5.11511511511512	180.228924163741\\
5.12512512512513	180.199649667881\\
5.13513513513514	180.170756202595\\
5.14514514514515	180.142289852813\\
5.15515515515516	180.114293935561\\
5.16516516516517	180.086808999959\\
5.17517517517518	180.059872827223\\
5.18518518518519	180.033520430665\\
5.1951951951952	180.007784055694\\
5.20520520520521	179.982693179811\\
5.21521521521522	179.958274512615\\
5.22522522522523	179.934552265142\\
5.23523523523524	179.911559380227\\
5.24524524524525	179.889328043496\\
5.25525525525526	179.867884489109\\
5.26526526526527	179.847252562811\\
5.27527527527528	179.827453721933\\
5.28528528528529	179.80850703539\\
5.2952952952953	179.790429183683\\
5.30530530530531	179.773234458901\\
5.31531531531532	179.756934764714\\
5.32532532532533	179.74153961638\\
5.33533533533534	179.727056140742\\
5.34534534534535	179.713489076229\\
5.35535535535536	179.700840772854\\
5.36536536536537	179.689111192216\\
5.37537537537538	179.678297907499\\
5.38538538538539	179.668396103474\\
5.3953953953954	179.659398576496\\
5.40540540540541	179.651295734505\\
5.41541541541542	179.644075597028\\
5.42542542542543	179.637726110987\\
5.43543543543544	179.632250317072\\
5.44544544544545	179.62763797014\\
5.45545545545546	179.623874263375\\
5.46546546546547	179.620943542986\\
5.47547547547548	179.618829308204\\
5.48548548548549	179.617514211281\\
5.4954954954955	179.616980057494\\
5.50550550550551	179.61720780514\\
5.51551551551552	179.61817756554\\
5.52552552552553	179.619868603037\\
5.53553553553554	179.622259334996\\
5.54554554554555	179.625327331806\\
5.55555555555556	179.629049316876\\
5.56556556556557	179.633401166639\\
5.57557557557558	179.638357910552\\
5.58558558558559	179.643893731091\\
5.5955955955956	179.649981963756\\
5.60560560560561	179.656596045067\\
5.61561561561562	179.663714428026\\
5.62562562562563	179.671309125719\\
5.63563563563564	179.679350970326\\
5.64564564564565	179.687811152696\\
5.65565565565566	179.696661222346\\
5.66566566566567	179.705873087461\\
5.67567567567568	179.715419014892\\
5.68568568568569	179.725271630162\\
5.6956956956957	179.735403917459\\
5.70570570570571	179.745789219638\\
5.71571571571572	179.756401238226\\
5.72572572572573	179.767214033415\\
5.73573573573574	179.778202024065\\
5.74574574574575	179.789339987705\\
5.75575575575576	179.800603060531\\
5.76576576576577	179.811967121178\\
5.77577577577578	179.823412230774\\
5.78578578578579	179.834912778388\\
5.7957957957958	179.846442088133\\
5.80580580580581	179.857974531386\\
5.81581581581582	179.869485526787\\
5.82582582582583	179.880951540241\\
5.83583583583584	179.892350084917\\
5.84584584584585	179.903659721246\\
5.85585585585586	179.914860056924\\
5.86586586586587	179.925931746911\\
5.87587587587588	179.93685649343\\
5.88588588588589	179.94761704597\\
5.8958958958959	179.95819720128\\
5.90590590590591	179.968581803376\\
5.91591591591592	179.978756743537\\
5.92592592592593	179.988708960306\\
5.93593593593594	179.998426439489\\
5.94594594594595	180.007898214156\\
5.95595595595596	180.017114364641\\
5.96596596596597	180.026065266877\\
5.97597597597598	180.034737287472\\
5.98598598598599	180.043119154776\\
5.995995995996	180.051201047315\\
6.00600600600601	180.058974053182\\
6.01601601601602	180.066430170038\\
6.02602602602603	180.073562305114\\
6.03603603603604	180.08036427521\\
6.04604604604605	180.086830806693\\
6.05605605605606	180.092957535501\\
6.06606606606607	180.098741007139\\
6.07607607607608	180.104178676682\\
6.08608608608609	180.109268908773\\
6.0960960960961	180.114010977625\\
6.10610610610611	180.118405067018\\
6.11611611611612	180.122452270301\\
6.12612612612613	180.126154590394\\
6.13613613613614	180.129514939782\\
6.14614614614615	180.132537140523\\
6.15615615615616	180.135225924241\\
6.16616616616617	180.137584586568\\
6.17617617617618	180.139612052926\\
6.18618618618619	180.141313226433\\
6.1961961961962	180.142693735692\\
6.20620620620621	180.143759533993\\
6.21621621621622	180.144516899311\\
6.22622622622623	180.144972434306\\
6.23623623623624	180.145133066324\\
6.24624624624625	180.145006047397\\
6.25625625625626	180.144598954243\\
6.26626626626627	180.143919688266\\
6.27627627627628	180.142976475554\\
6.28628628628629	180.141777866883\\
6.2962962962963	180.140332737712\\
6.30630630630631	180.13865028819\\
6.31631631631632	180.136740043147\\
6.32632632632633	180.134611852101\\
6.33633633633634	180.132275889256\\
6.34634634634635	180.129741768787\\
6.35635635635636	180.127017903988\\
6.36636636636637	180.124115264835\\
6.37637637637638	180.121044853524\\
6.38638638638639	180.117817539497\\
6.3963963963964	180.114444059443\\
6.40640640640641	180.110935017297\\
6.41641641641642	180.107300884243\\
6.42642642642643	180.103551998711\\
6.43643643643644	180.099698566379\\
6.44644644644645	180.09575066017\\
6.45645645645646	180.091718220257\\
6.46646646646647	180.087611054057\\
6.47647647647648	180.083438836236\\
6.48648648648649	180.079211108708\\
6.4964964964965	180.074937280632\\
6.50650650650651	180.070626080321\\
6.51651651651652	180.066285321028\\
6.52652652652653	180.061925201113\\
6.53653653653654	180.057555749873\\
6.54654654654655	180.053186600233\\
6.55655655655656	180.04882698875\\
6.56656656656657	180.044485755607\\
6.57657657657658	180.040171344618\\
6.58658658658659	180.035891803227\\
6.5965965965966	180.031654782506\\
6.60660660660661	180.027467537157\\
6.61661661661662	180.023336925509\\
6.62662662662663	180.019269409524\\
6.63663663663664	180.01527105479\\
6.64664664664665	180.011347530526\\
6.65665665665666	180.007504109579\\
6.66666666666667	180.003745668426\\
6.67667667667668	180.000076687173\\
6.68668668668669	179.996501249555\\
6.6966966966967	179.993023042938\\
6.70670670670671	179.989646067636\\
6.71671671671672	179.986375558116\\
6.72672672672673	179.983215578114\\
6.73673673673674	179.980169802413\\
6.74674674674675	179.97724155985\\
6.75675675675676	179.97443383331\\
6.76676676676677	179.971749259729\\
6.77677677677678	179.969190130097\\
6.78678678678679	179.966758389451\\
6.7967967967968	179.964455636884\\
6.80680680680681	179.962283125535\\
6.81681681681682	179.960241762599\\
6.82682682682683	179.958332109318\\
6.83683683683684	179.956554380988\\
6.84684684684685	179.954908446954\\
6.85685685685686	179.953393830614\\
6.86686686686687	179.952009709416\\
6.87687687687688	179.950754914858\\
6.88688688688689	179.949627932493\\
6.8968968968969	179.948626901921\\
6.90690690690691	179.947751231919\\
6.91691691691692	179.947000886711\\
6.92692692692693	179.946373854678\\
6.93693693693694	179.945867992923\\
6.94694694694695	179.945481034479\\
6.95695695695696	179.945210588314\\
6.96696696696697	179.945054139326\\
6.97697697697698	179.945009048345\\
6.98698698698699	179.945072552133\\
6.996996996997	179.945241763385\\
7.00700700700701	179.945513670727\\
7.01701701701702	179.945885138716\\
7.02702702702703	179.946352907844\\
7.03703703703704	179.946913594532\\
7.04704704704705	179.947563691134\\
7.05705705705706	179.948299565937\\
7.06706706706707	179.949117463158\\
7.07707707707708	179.950013502946\\
7.08708708708709	179.950984263589\\
7.0970970970971	179.952026332729\\
7.10710710710711	179.953135510986\\
7.11711711711712	179.954307644928\\
7.12712712712713	179.955538630546\\
7.13713713713714	179.956824413259\\
7.14714714714715	179.958160987912\\
7.15715715715716	179.959544398775\\
7.16716716716717	179.960970739543\\
7.17717717717718	179.96243615334\\
7.18718718718719	179.963936832712\\
7.1971971971972	179.965469019633\\
7.20720720720721	179.967029005503\\
7.21721721721722	179.968613131146\\
7.22722722722723	179.970217786814\\
7.23723723723724	179.971839412183\\
7.24724724724725	179.973474901847\\
7.25725725725726	179.975121043657\\
7.26726726726727	179.976773921356\\
7.27727727727728	179.978429762251\\
7.28728728728729	179.980084943711\\
7.2972972972973	179.981735993166\\
7.30730730730731	179.983379588111\\
7.31731731731732	179.985012556101\\
7.32732732732733	179.986631874757\\
7.33733733733734	179.988234671759\\
7.34734734734735	179.989818224854\\
7.35735735735736	179.991379961847\\
7.36736736736737	179.992917460609\\
7.37737737737738	179.994428449072\\
7.38738738738739	179.995910805232\\
7.3973973973974	179.997362557146\\
7.40740740740741	179.998781882935\\
7.41741741741742	180.000167110783\\
7.42742742742743	180.001516718935\\
7.43743743743744	180.002829335556\\
7.44744744744745	180.004103261371\\
7.45745745745746	180.00533659569\\
7.46746746746747	180.006527832787\\
7.47747747747748	180.007675598374\\
7.48748748748749	180.008778649606\\
7.4974974974975	180.009835875078\\
7.50750750750751	180.010846294825\\
7.51751751751752	180.011809060324\\
7.52752752752753	180.012723454492\\
7.53753753753754	180.013588891687\\
7.54754754754755	180.014404917707\\
7.55755755755756	180.015171209793\\
7.56756756756757	180.015887576625\\
7.57757757757758	180.016553958323\\
7.58758758758759	180.01717042645\\
7.5975975975976	180.017737184008\\
7.60760760760761	180.018254565442\\
7.61761761761762	180.018723036635\\
7.62762762762763	180.019143194912\\
7.63763763763764	180.019515755306\\
7.64764764764765	180.019840688543\\
7.65765765765766	180.020118247698\\
7.66766766766767	180.020349202239\\
7.67767767767768	180.020534368922\\
7.68768768768769	180.020674611795\\
7.6976976976977	180.020770842191\\
7.70770770770771	180.020824018735\\
7.71771771771772	180.020835147341\\
7.72772772772773	180.020805281211\\
7.73773773773774	180.020735520837\\
7.74774774774775	180.020627014001\\
7.75775775775776	180.020480955772\\
7.76776776776777	180.020298588509\\
7.77777777777778	180.020081201861\\
7.78778778778779	180.019830132766\\
7.7977977977978	180.01954676545\\
7.80780780780781	180.019232531429\\
7.81781781781782	180.01888890392\\
7.82782782782783	180.018517030897\\
7.83783783783784	180.01811829268\\
7.84784784784785	180.017694306808\\
7.85785785785786	180.017246668984\\
7.86786786786787	180.016776953076\\
7.87787787787788	180.016286711116\\
7.88788788788789	180.015777473298\\
7.8978978978979	180.015250747981\\
7.90790790790791	180.01470802169\\
7.91791791791792	180.014150759111\\
7.92792792792793	180.013580403096\\
7.93793793793794	180.01299837466\\
7.94794794794795	180.012406072983\\
7.95795795795796	180.011804875407\\
7.96796796796797	180.01119613744\\
7.97797797797798	180.010581192753\\
7.98798798798799	180.009961353182\\
7.997997997998	180.009337866966\\
8.00800800800801	180.008711843385\\
8.01801801801802	180.00808468583\\
8.02802802802803	180.007457789381\\
8.03803803803804	180.006832491445\\
8.04804804804805	180.006210071751\\
8.05805805805806	180.005591752357\\
8.06806806806807	180.004978697645\\
8.07807807807808	180.004372014324\\
8.08808808808809	180.003772751428\\
8.0980980980981	180.003181900316\\
8.10810810810811	180.002600394673\\
8.11811811811812	180.002029110511\\
8.12812812812813	180.001468866167\\
8.13813813813814	180.000920422302\\
8.14814814814815	180.000384481906\\
8.15815815815816	179.999861690292\\
8.16816816816817	179.999352635099\\
8.17817817817818	179.998857846293\\
8.18818818818819	179.998377796165\\
8.1981981981982	179.997913005415\\
8.20820820820821	179.997464177841\\
8.21821821821822	179.997031831545\\
8.22822822822823	179.996616429365\\
8.23823823823824	179.996218386228\\
8.24824824824825	179.995838069156\\
8.25825825825826	179.995475797264\\
8.26826826826827	179.995131841758\\
8.27827827827828	179.99480642594\\
8.28828828828829	179.994499725203\\
8.2982982982983	179.994211867032\\
8.30830830830831	179.993942931007\\
8.31831831831832	179.993692948799\\
8.32832832832833	179.993461904175\\
8.33833833833834	179.99324973299\\
8.34834834834835	179.993056323197\\
8.35835835835836	179.992881514839\\
8.36836836836837	179.992725100052\\
8.37837837837838	179.992586823066\\
8.38838838838839	179.992466380615\\
8.3983983983984	179.992363702919\\
8.40840840840841	179.992278695238\\
8.41841841841842	179.992211027773\\
8.42842842842843	179.992160355211\\
8.43843843843844	179.992126316726\\
8.44844844844845	179.992108535982\\
8.45845845845846	179.99210662113\\
8.46846846846847	179.992120164812\\
8.47847847847848	179.992148744153\\
8.48848848848849	179.992191920771\\
8.4984984984985	179.992249240769\\
8.50850850850851	179.992320234741\\
8.51851851851852	179.992404417765\\
8.52852852852853	179.992501289411\\
8.53853853853854	179.992610333735\\
8.54854854854855	179.992731019282\\
8.55855855855856	179.992862799086\\
8.56856856856857	179.993005110666\\
8.57857857857858	179.993157404718\\
8.58858858858859	179.993319357921\\
8.5985985985986	179.993490387832\\
8.60860860860861	179.993669844029\\
8.61861861861862	179.993857089409\\
8.62862862862863	179.99405150019\\
8.63863863863864	179.994252465905\\
8.64864864864865	179.994459389411\\
8.65865865865866	179.994671686881\\
8.66866866866867	179.994888787809\\
8.67867867867868	179.995110135007\\
8.68868868868869	179.995335184608\\
8.6986986986987	179.995563406061\\
8.70870870870871	179.995794282138\\
8.71871871871872	179.996027308928\\
8.72872872872873	179.99626199584\\
8.73873873873874	179.996497865601\\
8.74874874874875	179.996734454261\\
8.75875875875876	179.996971311184\\
8.76876876876877	179.997207999057\\
8.77877877877878	179.997444093884\\
8.78878878878879	179.997679184991\\
8.7987987987988	179.997912901622\\
8.80880880880881	179.998144880752\\
8.81881881881882	179.998374650981\\
8.82882882882883	179.998601758378\\
8.83883883883884	179.998825772897\\
8.84884884884885	179.999046288376\\
8.85885885885886	179.999262922539\\
8.86886886886887	179.999475316994\\
8.87887887887888	179.999683137235\\
8.88888888888889	179.999886072638\\
8.8988988988989	180.000083836468\\
8.90890890890891	180.000276165871\\
8.91891891891892	180.00046282188\\
8.92892892892893	180.000643589412\\
8.93893893893894	180.00081827727\\
8.94894894894895	180.00098671814\\
8.95895895895896	180.001148768594\\
8.96896896896897	180.001304309089\\
8.97897897897898	180.001453243966\\
8.98898898898899	180.001595501452\\
8.998998998999	180.001731033658\\
9.00900900900901	180.001859816579\\
9.01901901901902	180.001981850096\\
9.02902902902903	180.002097017042\\
9.03903903903904	180.002205130651\\
9.04904904904905	180.002306157568\\
9.05905905905906	180.002400077837\\
9.06906906906907	180.002486884498\\
9.07907907907908	180.002566583583\\
9.08908908908909	180.002639194116\\
9.0990990990991	180.002704748113\\
9.10910910910911	180.002763290584\\
9.11911911911912	180.002814879531\\
9.12912912912913	180.002859585947\\
9.13913913913914	180.00289749382\\
9.14914914914915	180.002928700129\\
9.15915915915916	180.002953314846\\
9.16916916916917	180.002971460936\\
9.17917917917918	180.002983274355\\
9.18918918918919	180.002988904053\\
9.1991991991992	180.002988511973\\
9.20920920920921	180.002982273048\\
9.21921921921922	180.002970375207\\
9.22922922922923	180.002953019369\\
9.23923923923924	180.002930419446\\
9.24924924924925	180.002902796657\\
9.25925925925926	180.002870178169\\
9.26926926926927	180.002832693989\\
9.27927927927928	180.002790577274\\
9.28928928928929	180.002744058825\\
9.2992992992993	180.002693367085\\
9.30930930930931	180.002638728143\\
9.31931931931932	180.002580365731\\
9.32932932932933	180.002518501224\\
9.33933933933934	180.00245335364\\
9.34934934934935	180.002385139642\\
9.35935935935936	180.002314073536\\
9.36936936936937	180.002240367272\\
9.37937937937938	180.002164230444\\
9.38938938938939	180.002085870288\\
9.3993993993994	180.002005491684\\
9.40940940940941	180.001923297158\\
9.41941941941942	180.001839486877\\
9.42942942942943	180.001754258653\\
9.43943943943944	180.00166780794\\
9.44944944944945	180.001580327838\\
9.45945945945946	180.001492009088\\
9.46946946946947	180.001403040078\\
9.47947947947948	180.001313606836\\
9.48948948948949	180.001223777763\\
9.4994994994995	180.001133489073\\
9.50950950950951	180.001042987583\\
9.51951951951952	180.000952523692\\
9.52952952952953	180.000862336361\\
9.53953953953954	180.000772653115\\
9.54954954954955	180.00068369004\\
9.55955955955956	180.000595651787\\
9.56956956956957	180.00050873157\\
9.57957957957958	180.000423111165\\
9.58958958958959	180.000338960912\\
9.5995995995996	180.000256439713\\
9.60960960960961	180.000175695035\\
9.61961961961962	180.000096862905\\
9.62962962962963	180.000020067916\\
9.63963963963964	179.999945423222\\
9.64964964964965	179.999873030542\\
9.65965965965966	179.999802980156\\
9.66966966966967	179.999735350907\\
9.67967967967968	179.999670210204\\
9.68968968968969	179.999607614015\\
9.6996996996997	179.999547606875\\
9.70970970970971	179.999490221879\\
9.71971971971972	179.999435480686\\
9.72972972972973	179.999383393518\\
9.73973973973974	179.999333959161\\
9.74974974974975	179.999287164963\\
9.75975975975976	179.999242986835\\
9.76976976976977	179.999201389251\\
9.77977977977978	179.99916232525\\
9.78978978978979	179.999125736431\\
9.7997997997998	179.999091552958\\
9.80980980980981	179.999059693557\\
9.81981981981982	179.999030221214\\
9.82982982982983	179.999003405639\\
9.83983983983984	179.998979236388\\
9.84984984984985	179.998957689881\\
9.85985985985986	179.998938738996\\
9.86986986986987	179.998922353073\\
9.87987987987988	179.998908497908\\
9.88988988988989	179.998897135756\\
9.8998998998999	179.998888225331\\
9.90990990990991	179.998881721806\\
9.91991991991992	179.998877576812\\
9.92992992992993	179.998875738439\\
9.93993993993994	179.998876151236\\
9.94994994994995	179.998878756211\\
9.95995995995996	179.998883490829\\
9.96996996996997	179.998890289015\\
9.97997997997998	179.998899081153\\
9.98998998998999	179.998909794085\\
10	179.998922351111\\
};
\addlegendentry{Energy-Based +  LQR}

\end{axis}
\end{tikzpicture}%}
	\caption{Swing-up and Balance Response}
	\label{fig:energyLQR}
\end{figure}
\figref{fig:energyLQR} clearly shows the decent performance of the implemented strategy where the upright position is maintained as compared to when only the energy based controller is used.