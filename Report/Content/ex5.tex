\section{Exercise 5}
\subsection{Forward Kinematics}
The frame of reference is assumed to originate at the input revolute joint $\left(q_1\right)$, with the Z-axis coming out of the plane and the Y axis pointing upwards.
It is clear from the given loop geometry that
\begin{align*}
	l_1 &= l_2\cos\left(q_1\right) + q_3\cos\left(\pi - q_2\right) \\
	&= l_2\cos\left(q_1\right) - q_3\cos\left(q_2\right) \\
	\implies\cos\left(q_1\right) &= \frac{q_3\cos\left(q_2\right) + l_1}{l_2} \\
	\implies q_1 &= \arccos\left(\frac{q_3\cos\left(q_2\right) + l_1}{l_2}\right) = f\left(q_3 = d\right)
\end{align*}
\subsection{Inverse Kinematics}
Now, we know
\begin{align*}
	l_1 &= l_2\cos\left(q_1\right) - q_3\cos\left(q_2\right) \\
	\implies q_3 &= \frac{l_2\cos\left(q_1\right) - l_1}{\cos\left(q_2\right)} & \text{given } q_2 > \frac{\pi}{2} \\
	&= f^{-1}\left(q_1\right) = d
\end{align*}
\subsection{Maximum Output Angular Velocity}
The forward kinematics is given by
\begin{align*}
	q_1 &= f\left(d\right) \\
	\dot{q}_1 &= \frac{\partial f}{\partial d}\dot{d}\\
	\implies \dot{q}_{1,\text{max}} &=  \frac{\partial f}{\partial d}\dot{d}_{\text{max}}
\end{align*}
Variable substitution is done in order to obtain $\frac{\partial f}{\partial d}$
\begin{align*}
	u &= \frac{d}{l_2}\cos\left(q_2\right) + \frac{l_1}{l_2}  \\
	\implies \frac{\partial f}{\partial d} &= \frac{\partial f}{\partial u}\frac{\partial u}{\partial d} &\text{where } f = \arccos\left(u\right)\\
	\implies \frac{\partial f}{\partial d} &= \frac{-1}{\sqrt{1-u^2}}\frac{1}{l_2}\cos\left(q_2\right)\\
	&= \frac{-\cos\left(q_2\right)}{\sqrt{l_2^2 - \left(l_1 + d\cos q_2\right)^2}}
\end{align*}
Hence, the maximum output angular velocity for a given maximum actuator velocity is given by
\begin{equation*}
	\dot{q}_{1,\text{max}} = \frac{-\cos\left(q_2\right)\dot{d}_{\text{max}}}{\sqrt{l_2^2 - \left(l_1 + d\cos q_2\right)^2}}
\end{equation*}
\subsection{Maximum Output Torque}
Let $f$ and $\tau$ be the input actuator force and output torque respectively. By the principle of virtual work we know,
\begin{equation}
	f\delta d = \tau\delta q_1
	\label{eq:virtualWork}
\end{equation}
$\delta d$ and $\delta q_1$ are related by the Jacobian (in this case scalar value) which is nothing but
\begin{equation*}
	J = \frac{-\cos\left(q_2\right)}{\sqrt{l_2^2 - \left(l_1 + d\cos q_2\right)^2}}
\end{equation*}
Hence, \eqref{eq:virtualWork} can be formulated as
\begin{align*}
	f\delta d &= \tau J \delta d\\
	\implies \tau &= \frac{1}{J} f\\
	\implies \tau_{\text{max}} &= \frac{1}{J} f_{\text{max}}\\
	\implies \tau_{\text{max}} &= -\frac{\sqrt{l_2^2 - \left(l_1 + d\cos q_2\right)^2}}{\cos\left(q_2\right)} f_{\text{max}}\\
\end{align*}
\subsection{Singularities}
The mechanism becomes singular whenever its Jacobian loses rank i.e. a degree of freedom is lost. In this case, it corresponds to the set $\left\{\left(q_1, q_2, d\right)\trans | J = 0\right\}$ which is given by
\begin{align*}
	\cos\left(q_2\right) &= 0\\
	\implies q_2 &= \frac{\pi}{2} & \text{assuming } q_2 \in \left[0,\pi\right]
\end{align*}