\section{Stability and LFT}
\subsection{Parametric Uncertainty Modelling}
The mass varies from $-50\%$ to $80\%$ of the nominal mass value. We can model mass uncertainty as,
\begin{align*}
    m_p = \frac{\Bar{m} + b_m\Delta_m}{1 + d_m\Delta_m}, \left|\Delta_m\right| \le 1\\
    b_m = 0.385\Bar{m}, d_m = -0.23
\end{align*}
In a similar way, we can model spring stiffness and output gain, which vary with non-uniform bounds as the mass.
\begin{align*}
    k_p = \frac{\Bar{k} + b_k\Delta_k}{1 + d_k\Delta_m}, \left|\Delta_k\right| \le 1\\
    b_k = -0.28\Bar{k}, d_k = -0.6\\
    h_p = \frac{\Bar{h} + b_h\Delta_h}{1 + d_h\Delta_h}, \left|\Delta_h\right| \le 1\\
    b_h = -0.28\Bar{h}, d_h = -0.6
\end{align*}
Since the damping varies with uniform bounds,
\begin{align*}
    cp = \Bar{c}\left(1 + r_c\Delta_c\right), \left|\Delta_c\right| \le 1\\
    r_c = \frac{c_{max} - c_{min}}{c_{max} + c_{min}}
\end{align*}
Since these are parametric uncertainties, $\{\Delta_m,\Delta_c,\Delta_k,\Delta_h\} \in \mathbb{R}$. 
For any function $f(\delta)$ of the form,
$$f(\delta) = \frac{\alpha+\beta\delta}{1+\gamma\delta}$$
we can use lower LFT to represent $f(\delta)$ \cite{Fran01} as,
\begin{align*}
    f(\delta) = \mathcal{F}_l(\bm{M},\delta), \bm{M} = 
    \begin{bmatrix}
        \alpha & \beta-\alpha\gamma\\
        1 & -\gamma
    \end{bmatrix}\\
    f^{-1}(\delta) = \mathcal{F}_l(\Tilde{\bm{M}},\delta), \Tilde{\bm{M}} = 
    \begin{bmatrix}
        \frac{1}{\alpha} & \frac{\gamma}{\alpha}^- \frac{\beta}{\alpha^2}\\
        1 & \frac{-\beta}{\alpha}
    \end{bmatrix}
\end{align*}
Thus, we can represent the parametric uncertainties in the form of lower LFTs with their respective $\Delta$s as,
\begin{align*}
    m_p = \mathcal{F}_l\left(\begin{bmatrix}
        \frac{1}{\Bar{m}}&\frac{d_m}{\Bar{m}} - \frac{b_m}{\Bar{m}^2}\\
        1&\frac{-b_m}{\Bar{m}}
    \end{bmatrix},\Delta_m\right)\\
    c_p = \mathcal{F}_l\left(\begin{bmatrix}
        \Bar{c}&1\\
        r_c\Bar{c}&0
    \end{bmatrix},\Delta_c\right)\\
    k_p = \mathcal{F}_l\left(\begin{bmatrix}
        \Bar{k}&b_k-\Bar{k}d_k\\
        1&-d_k
    \end{bmatrix},\Delta_k\right)\\
    h_p = \mathcal{F}_l\left(\begin{bmatrix}
        \Bar{h}&b_h-\Bar{h}d_h\\
        1&-d_h
    \end{bmatrix},\Delta_h\right)\\
\end{align*}
\subsection{Block Diagram}
\begin{figure}[h!]
  \centering
  \tikzstyle{block}     = [draw, rectangle, minimum height=0.5cm, minimum width=0.5cm]
    \tikzstyle{branch}    = [circle, inner sep=0pt, minimum size=1mm, fill=black, draw=black]
    \tikzstyle{connector} = [->, thin]
    \tikzstyle{dummy}     = [inner sep=0pt, minimum size=0pt]
    \tikzstyle{inout}     = []
    \tikzstyle{sum}       = [circle, inner sep=0pt, minimum size=2mm, draw=black, thick]
    \begin{tikzpicture}[auto, node distance=2cm, >=stealth']
      %%%%%%%%%%%%%%%%%%%%%%% BLOCKS %%%%%%%%%%%%%%%%%%%%%%%%%%%%%%%%%%
      % Mass
      \node[block] (m) {$\begin{matrix}
          \frac{1}{\Bar{m}}&\frac{d_m}{\Bar{m}} - \frac{b_m}{\Bar{m}^2}\\
        1&\frac{-b_m}{\Bar{m}}
      \end{matrix}$};
      % Differentiator 1
      \node[block, right of = m] (diff1) {$\frac{1}{s}$};
      % Differentiator 1
      \node[block, right of = diff1] (diff2) {$\frac{1}{s}$};
      % Output Gain
      \node[block, right of = diff2] (h) {$\begin{matrix}
        \Bar{h}&b_h-\Bar{h}d_h\\
        1&-d_h
    \end{matrix}$};
     % Damping
     \node[block, below of = m, node distance = 4cm] (c) {$\begin{matrix}
        \Bar{c}&1\\
        r_c\Bar{c}&0
     \end{matrix}$};
     % Stiffness
     \node[block, below of = c, node distance = 4cm] (k) {$\begin{matrix}
        \Bar{k}&b_k-\Bar{k}d_k\\
        1&-d_k
     \end{matrix}$};
     % Deltam
     \node[block,below of = m] (delm) {$\Delta_m$};
     % Deltah
     \node[block,below of = h] (delh) {$\Delta_h$};
     % Deltac
     \node[block,below of = c] (delc) {$\Delta_c$};
     % Deltak
     \node[block,below of = k] (delk) {$\Delta_k$};
     % W2 weight
     \node[block,left of = m, node distance = 3cm] (w2) {$W_2$};
     % W1 weight
     \node[block,right of = h, node distance = 3cm] (w1) {$W_1$};
     % Controller
     \node[block,left of = w2] (K) {$K$};

     %%%%%%%%%%%%%%%%%%%%%%% SUMMATIONS %%%%%%%%%%%%%%%%%%%%%%%%%%%%%%%%%%
     \node[sum,left of = K] (s1) {};
     \node[sum,left of = m] (s2) {};
     \node[sum,right of = h] (s3) {};
     \node[dummy] (d1) [below right=0.1cm and 0.05cm of s1] {$-$};
     \node[dummy] (d2) [below right=0.1cm and 0.05cm of s2] {$-$};

     %%%%%%%%%%%%%%%%%%%%%%% I/O SIGNALS %%%%%%%%%%%%%%%%%%%%%%%%%%%%%%%%%%
     \node[inout,right of = w1, node distance = 0.75cm] (z) {$z$};
     \node[inout,above of = s3] (d) {$d$};

     %%%%%%%%%%%%%%%%%%%%%%% BRANCHES %%%%%%%%%%%%%%%%%%%%%%%%%%%%%%%%%%
     \node[branch,right of = diff1,node distance = 1cm] (b1) {};
     \node[branch,right of = diff2,node distance = 0.5cm] (b2) {};
     \node[branch,right of = s3,node distance = 0.25cm] (b3) {};
     \node[branch,below of = s2,node distance = 4cm] (b4) {};
     %\node[branch,below of = b4,node distance = 4cm] (b5);
     \node[branch,below of = b3,node distance = 11cm] (b6) {};

     %%%%%%%%%%%%%%%%%%%%%%% CONNECTIONS %%%%%%%%%%%%%%%%%%%%%%%%%%%%%%%%%%
     % Interconnection between m and delm
     \draw[->] (delm.west) -| ++(-1,1) node [yshift = -0.3cm, xshift = -0.3cm] {$u_m$} |- (m.200);
     \draw[->] (m.-20) -| ++(0.25,-1) node [yshift = 0.3cm, xshift = 0.3cm] {$y_m$} |- (delm.east);

     % Interconnection between c and delc
     \draw[->] (delc.east) -| ++(1,1) node [yshift = -0.3cm, xshift = -0.3cm] {$u_c$} |- (c.-20);
     \draw[->] (c.200) -| ++(-0.75,-1) node [yshift = 0.3cm, xshift = 0.3cm] {$y_c$} |- (delc.west);

     % Interconnection between k and delk
     \draw[->] (delk.east) -| ++(1,1) node [yshift = -0.3cm, xshift = 0.3cm] {$u_k$} |- (k.-20);
     \draw[->] (k.200) -| ++(-0.25,-1) node [yshift = 0.3cm, xshift = 0.3cm] {$y_k$} |- (delk.west);

     % Interconnection between h and delh
     \draw[->] (delh.west) -| ++(-1,1) node [yshift = -0.3cm, xshift = 0.3cm] {$u_h$} |- (h.200);
     \draw[->] (h.-20) -| ++(0.25,-1) node [yshift = 0.3cm, xshift = 0.3cm] {$y_h$} |- (delh.east);

     \draw[connector] (s1) -- (K);
     \draw[connector] (K) -- (w2);
     \draw[connector] (w2) -- node[] {$F$} (s2);
     \draw[connector] (s2) -- (m.180);
     \draw[connector] (m.0) -- node[] {$\dot{x}_2$} (diff1);
     \draw[connector] (diff1) -- node[] {$x_2$} (diff2);
     \draw[connector] (diff2) -- node[] {$x_1$} (h.180);
     \draw[connector] (h.0) -- (s3);
     \draw[connector] (d) -- (s3);
     \draw[connector] (s3) -- (w1);
     \draw[connector] (w1) -- (z);
     \draw[thick] (b3) -- (b6);
     \draw[connector] (b6) -| (s1);
     \draw[connector] (c.180) -- (b4);
     \draw[connector] (k.180) -| (s2);
     \draw[connector] (b1) |- (c.0);
     \draw[connector] (b2) |- (k.0);
     
    
    \end{tikzpicture}
    
	  \caption{Block Diagram for the uncertain dynamic system.}
    \label{fig:blockDiagramA}
\end{figure}

\begin{figure}[htb]
  \centering
  \tikzstyle{block}     = [draw, rectangle, minimum height=1cm, minimum width=1.2cm]
    \tikzstyle{branch}    = [circle, inner sep=0pt, minimum size=1mm, fill=black, draw=black]
    \tikzstyle{connector} = [->, thin]
    \tikzstyle{dummy}     = [inner sep=0pt, minimum size=0pt]
    \tikzstyle{inout}     = []
    \tikzstyle{sum}       = [circle, inner sep=0pt, minimum size=2mm, draw=black, thick]
    \begin{tikzpicture}[auto, node di   stance=1.5cm, >=stealth']
    % Blocks
      \node[block] (P) {$\bm{P}$};
      \node[block, below of = P] (K) {$K$};
      \node[block, above of = P] (delta) {$\Delta$};
      % Signals
      \node[inout, left of = P,xshift = -1.5cm] (w) {$\begin{bmatrix}
      x_1\\
      x_2\\
      d
      \end{bmatrix}$};
      \node[inout, right of = P,xshift = 1.5cm] (z) {$\begin{bmatrix}
      \dot{x_1}\\
      \dot{x_2}\\
      W_1y
      \end{bmatrix}$};
      % Connections
      \draw[->] (K.west) -| ++(-1,1) node [yshift = -0.3cm, xshift = -0.3cm] {$\bm{u}$} |- (P.210);
      \draw[->] (P.-30) -| ++(1,-1) node [yshift = 0.3cm, xshift = 0.3cm] {$\bm{v}$} |- (K.east);
      \draw[->] (delta.west) -| ++(-1,-1) node [yshift = 0.2cm, xshift = -0.3cm] {$\bm{u}_\Delta$} |- (P.150);
      \draw[->] (P.30) -| ++(1,1) node [yshift = -0.3cm, xshift = 0.3cm] {$\bm{y}_\Delta$} |- (delta.east);
      \draw[connector] (P.0) -- (z);
	 \draw[connector] (w) -- (P.180);
    \end{tikzpicture}
	  \caption{Generalized Uncertain Closed Loop Configuration with Parametric Uncertainty.}
    \label{fig:blockDiagramB}
\end{figure}

The system can be represented in block diagram form as done if Figure \ref{fig:blockDiagramA}. Moreover, we can generalize the block diagram in form of a lower LFT and upper LFT as seen in Figure \ref{fig:blockDiagramB}. In the generalized configuration, the following is considered,
\begin{align*}
    \bm{\Delta} =
    \begin{bmatrix}
        \Delta_m & 0 & 0 & 0\\
        0 & \Delta_c & 0 & 0\\
        0 & 0 & \Delta_k & 0\\
        0 & 0 & 0 & \Delta_m
    \end{bmatrix},
    \bm{y}_\Delta = \begin{bmatrix}
        y_m\\
        y_c\\
        y_k\\
        y_h
    \end{bmatrix},
    \bm{u}_\Delta = \begin{bmatrix}
        u_m\\
        u_c\\
        u_k\\
        u_h
    \end{bmatrix}\\
    z = W_1y, v = -y,
    \begin{bmatrix}
        \bm{y}_\Delta\\
        \dot{\bm{x}}\\
        z\\
        v
    \end{bmatrix} = 
    \bm{P}\begin{bmatrix}
        \bm{u}_\Delta\\
        \bm{x}\\
        d\\
        u
    \end{bmatrix}
\end{align*}
Analyzing the relations between different signals, the generalized plant $\mathbf{P}$ can be derived as,
\begin{align*}
    \bm{P} = 
    \begin{bmatrix}
        \frac{-b_m}{\Bar{m}} & -1 & -(b_k-\Bar{k}d_k) & 0 & -\Bar{k} & -\Bar{c} & 0 & 1\\
        0 & 0 & 0 & 0 & 0 & r_c\Bar{c} & 0 & 0\\
        0 & 0 & -d_k & 0 & 1 & 0 & 0 & 0\\
        0 & 0 & 0 & -d_h & 1 & 0 & 0 & 0\\
        0 & 0 & 0 & 0 & 0 & 1 & 0 & 0\\
        \frac{d_m}{\Bar{m}} - \frac{b_m}{\Bar{m}^2} & \frac{-1}{\Bar{m}} & \frac{-(b_k-\Bar{k}d_k)}{\Bar{m}} & 0 & \frac{-\Bar{k}}{\Bar{m}} & \frac{-\Bar{c}}{\Bar{m}} & 0 & \frac{1}{\Bar{m}}\\
        0 & 0 & 0 & W_1(b_h-\Bar{h}d_h) & W_1\Bar{h} & 0 & W_1 & 0\\
        0 & 0 & 0 & -(b_h-\Bar{h}d_h) & -\Bar{h} & 0 & -1 & 0
    \end{bmatrix}
\end{align*}
\textbf{Note}: \textit{The above generalized plant matrix has been formed assuming $W_2$ is a part of the controller. If $W_2$ is considered a part of the plant, then it will populate the last column corresponding to row 1 and row 6 of the generalized plant matrix.}.

The number of uncertainty channels in the system will be 4, corresponding to the 4 real parametric perturbations introduced. The above uncertain system is put into the \emph{Robust Control Toolbox} in MATLAB, by inputting the following uncertain plant transfer function $$G_p(s) = \frac{h_p}{m_ps^2 + c_ps + k_p}$$
The uncertain state space model in MATLAB obtained has 2 states, and 4 uncertainty output channels.

\subsection{PI Controller Design}
First the lag compensator is designed. $\tau$ is taken as $10$, and $k_0$ is set such that the gain cross-over frequency is $1$ rad/s. For such a condition, we obtain $k_0$ as $5.3547$. A seperate lead compensator is not designed for this controller, instead the output weight $W_2$ can be considered as a lead compensator with $T = 0.7$ and $\alpha = 0.1$. Thus, the lead-lag compensated PI controller is given by,
$$K(s) = 0.3k_0\left(\frac{0.7s+1}{0.07s+1}\right)\left(\frac{10s+1}{10s}\right)$$
\subsection{Robust Stability}
We can obtain the unperturbed closed loop of the system with, $$\bm{N} = \mathcal{F}_l(\bm{P},K)$$
We also know that the $\mathbf{M}$ matrix which mapes $\bm{u}_\Delta$ to $\bm{y}_\Delta$ is, $$\bm{M} = \bm{N}_{11}$$
In order to compute the robust stability using \emph{Robust Control Toolbox}, the following uncertain transfer function mapping the disturbance signal to the exogenous output is used,
$$H(s) = \frac{W_1}{1 + \frac{W_1h_pK}{m_ps^2+c_ps+k_p}}$$
\begin{figure}[h!]
    \centering
    \scalebox{0.7}{
    \begin{tikzpicture}
        % This file was created by matlab2tikz.
%
%The latest updates can be retrieved from
%  http://www.mathworks.com/matlabcentral/fileexchange/22022-matlab2tikz-matlab2tikz
%where you can also make suggestions and rate matlab2tikz.
%
\begin{tikzpicture}

\begin{axis}[%
width=4.521in,
height=3.563in,
at={(0.758in,0.484in)},
scale only axis,
xmode=log,
xmin=0.001,
xmax=1000,
xminorticks=true,
xlabel style={font=\color{white!15!black}},
xlabel={Frequency (rad/s)},
ymin=0,
ymax=140,
ylabel style={font=\color{white!15!black}},
ylabel={Upper Singular Value},
axis background/.style={fill=white},
xmajorgrids,
xminorgrids,
ymajorgrids,
legend style={legend cell align=left, align=left, draw=white!15!black}
]
\addplot [color=black, dashed, line width=2.0pt]
  table[row sep=crcr]{%
0.001	128.525522485934\\
0.00114975699539774	111.788507122876\\
0.00132194114846603	97.2320379837464\\
0.00151991108295293	84.5721765198955\\
0.00174752840000768	73.5619793612115\\
0.00200923300256505	63.9866814148349\\
0.00231012970008316	55.659506620645\\
0.00265608778294669	48.4180246415969\\
0.00305385550883342	42.1209824126411\\
0.00351119173421513	36.6455487295078\\
0.00403701725859655	31.8849181078137\\
0.00464158883361278	27.7462271395396\\
0.00533669923120631	24.1487426526272\\
0.00613590727341317	21.0222862562941\\
0.00705480231071865	18.3058644290822\\
0.00811130830789687	15.9464772631793\\
0.0093260334688322	13.8980823884421\\
0.0107226722201032	12.1206935226076\\
0.0123284673944207	10.5795955813302\\
0.014174741629268	9.24466037937507\\
0.0162975083462064	8.08974871210874\\
0.0187381742286038	7.09218608971857\\
0.0215443469003188	6.23230070484834\\
0.0247707635599171	5.49301350211747\\
0.028480358684358	4.85947170622135\\
0.0327454916287773	4.31871911590095\\
0.0376493580679247	3.85939908696552\\
0.0432876128108306	3.47148935835022\\
0.0497702356433211	3.14607119066897\\
0.0572236765935022	2.87513761558439\\
0.0657933224657568	2.65144567772731\\
0.0756463327554629	2.46841471701818\\
0.0869749002617783	2.32006760406135\\
0.1	2.20100623595422\\
0.114975699539774	2.10640862329452\\
0.132194114846603	2.03203385995206\\
0.151991108295293	1.97422328365865\\
0.174752840000768	1.92989036721042\\
0.200923300256505	1.89649695764104\\
0.231012970008316	1.87201797397264\\
0.265608778294669	1.8548995811679\\
0.305385550883342	1.84401690544844\\
0.351119173421513	1.83863691474224\\
0.403701725859655	1.83839080626327\\
0.464158883361278	1.84325868446015\\
0.533669923120631	1.85356777577136\\
0.613590727341317	1.87000395175936\\
0.705480231071864	1.89363480729013\\
0.811130830789687	1.92594084247842\\
0.93260334688322	1.96884943380528\\
1.07226722201032	2.02476449980215\\
1.23284673944207	2.09658357164474\\
1.41747416292681	2.18769395124685\\
1.62975083462064	2.30194107607475\\
1.87381742286038	2.44356473717966\\
2.15443469003188	2.61710126395948\\
2.47707635599171	2.82725059676704\\
2.8480358684358	3.07870489955311\\
3.27454916287773	3.37592953691489\\
3.76493580679247	3.72287915399845\\
4.32876128108306	4.1226250895038\\
4.97702356433211	4.57687188710502\\
5.72236765935022	5.0853582982438\\
6.57933224657568	5.64517904898552\\
7.56463327554629	6.25012936177565\\
8.69749002617784	6.89025306030412\\
10	7.55183377695489\\
11.4975699539774	8.21805384392097\\
13.2194114846603	8.87040853402828\\
15.1991108295293	9.4907109125183\\
17.4752840000768	10.0632534183056\\
20.0923300256505	10.5765637681108\\
23.1012970008316	11.0243082431881\\
26.5608778294669	11.4052061157657\\
30.5385550883342	11.7221506632419\\
35.1119173421513	11.9809201152414\\
40.3701725859656	12.188859547003\\
46.4158883361278	12.3537881583448\\
53.3669923120631	12.4832339882697\\
61.3590727341317	12.5839849431104\\
70.5480231071865	12.6618882139252\\
81.1130830789687	12.7218173107534\\
93.260334688322	12.7677369261332\\
107.226722201032	12.8028148252046\\
123.284673944207	12.8295481538903\\
141.74741629268	12.849885590581\\
162.975083462065	12.8653362713332\\
187.381742286039	12.8770622404805\\
215.443469003188	12.8859544102102\\
247.707635599171	12.8926935902511\\
284.80358684358	12.8977987511453\\
327.454916287773	12.9016647573171\\
376.493580679247	12.9045916215201\\
432.876128108306	12.9068070460919\\
497.702356433211	12.9084837119838\\
572.236765935022	12.9097524940036\\
657.933224657568	12.910712536284\\
756.463327554629	12.9114389191605\\
869.749002617783	12.9119884847959\\
1000	12.9124042589106\\
};
\addlegendentry{Manually Computed M}

\addplot [color=red, line width=2.0pt]
  table[row sep=crcr]{%
0.001	57.2209293073404\\
0.00114975699539774	49.7688397503392\\
0.00132194114846603	43.2875410557534\\
0.00151991108295293	37.6506090100798\\
0.00174752840000768	32.7480896067038\\
0.00200923300256505	28.4843542794775\\
0.00231012970008316	24.7762345663473\\
0.00265608778294669	21.5513998167646\\
0.00305385550883342	18.7469462968998\\
0.00351119173421513	16.3081701694127\\
0.00403701725859655	14.1875004099854\\
0.00464158883361278	12.3435708407443\\
0.00533669923120631	10.7404131715915\\
0.00613590727341317	9.34675529677416\\
0.00705480231071865	8.13541114106464\\
0.00811130830789687	7.08275012670899\\
0.0093260334688322	6.16823587221005\\
0.0107226722201032	5.37402506535579\\
0.0123284673944207	4.68461859945449\\
0.014174741629268	4.08655804332532\\
0.0162975083462064	3.56816134896495\\
0.0187381742286038	3.11929240117757\\
0.0215443469003188	2.73115959726924\\
0.0247707635599171	2.39613913465942\\
0.028480358684358	2.10761911587577\\
0.0327454916287773	1.85986101251337\\
0.0376493580679247	1.64787555069787\\
0.0432876128108306	1.46731080232503\\
0.0497702356433211	1.31435128688943\\
0.0572236765935022	1.18562820712437\\
0.0657933224657568	1.0781423407318\\
0.0756463327554629	0.98920209385023\\
0.0869749002617783	0.916379152917601\\
0.1	0.857482676486124\\
0.114975699539774	0.810550373694092\\
0.132194114846603	0.773852165496002\\
0.151991108295293	0.745900645397006\\
0.174752840000768	0.725463133526845\\
0.200923300256505	0.711573285417078\\
0.231012970008316	0.703545709546235\\
0.265608778294669	0.701001255532824\\
0.305385550883342	0.703904935025556\\
0.351119173421513	0.712595856986144\\
0.403701725859655	0.727752168097404\\
0.464158883361278	0.750194419657966\\
0.533669923120631	0.780420526591222\\
0.613590727341317	0.817871928252621\\
0.705480231071864	0.860214481067787\\
0.811130830789687	0.903197043718718\\
0.93260334688322	0.941548813207732\\
1.07226722201032	0.970737581428567\\
1.23284673944207	0.988608725130639\\
1.41747416292681	0.995817521543079\\
1.62975083462064	0.994923650272566\\
1.87381742286038	0.98901246106728\\
2.15443469003188	0.980706646924304\\
2.47707635599171	0.971802440767655\\
2.8480358684358	0.963324735143389\\
3.27454916287773	0.955737989326906\\
3.76493580679247	0.949157853216804\\
4.32876128108306	0.943509367493387\\
4.97702356433211	0.938629296637759\\
5.72236765935022	0.934326974738647\\
6.57933224657568	0.930419058804694\\
7.56463327554629	0.926749613707482\\
8.69749002617784	0.923202321454486\\
10	0.919707623666063\\
11.4975699539774	0.916244738436012\\
13.2194114846603	0.912837229145607\\
15.1991108295293	0.909541516237799\\
17.4752840000768	0.906430012845773\\
20.0923300256505	0.903572956433649\\
23.1012970008316	0.901023744976031\\
26.5608778294669	0.898811033293963\\
30.5385550883342	0.89693802453159\\
35.1119173421513	0.89538701488238\\
40.3701725859656	0.894126339145316\\
46.4158883361278	0.89311729636141\\
53.3669923120631	0.892319666351564\\
61.3590727341317	0.891695392827426\\
70.5480231071865	0.891210611575137\\
81.1130830789687	0.890836448218463\\
93.260334688322	0.890549026577609\\
107.226722201032	0.890329041654987\\
123.284673944207	0.890161141592593\\
141.74741629268	0.890033268374971\\
162.975083462065	0.889936038486594\\
187.381742286039	0.889862200347831\\
215.443469003188	0.889806179212858\\
247.707635599171	0.889763706305145\\
284.80358684358	0.889731522591897\\
327.454916287773	0.8897071455236\\
376.493580679247	0.889688687240019\\
432.876128108306	0.889674713954028\\
497.702356433211	0.889664137791966\\
572.236765935022	0.889656133943788\\
657.933224657568	0.889650077396925\\
756.463327554629	0.889645494736863\\
869.749002617783	0.889642027490356\\
1000	0.889639404283651\\
};
\addlegendentry{M from Robust Control Toolbox}

\end{axis}
\end{tikzpicture}%
    \end{tikzpicture}}
    \caption{Comparison of the upper singular values of $\mathbf{M}$ computed manually with the one computed from \emph{Robust Control Toolbox}.}
    \label{fig:ssv}
\end{figure}
\begin{figure}[h!]
    \centering
    \scalebox{0.7}{
    \begin{tikzpicture}
        % This file was created by matlab2tikz.
%
%The latest updates can be retrieved from
%  http://www.mathworks.com/matlabcentral/fileexchange/22022-matlab2tikz-matlab2tikz
%where you can also make suggestions and rate matlab2tikz.
%
\begin{tikzpicture}

\begin{axis}[%
width=4.474in,
height=3.563in,
at={(0.806in,0.484in)},
scale only axis,
xmode=log,
xmin=0.001,
xmax=1000,
xminorticks=true,
xlabel style={font=\color{white!15!black}},
xlabel={Frequency (rad/s)},
ymin=0.6,
ymax=0.6002,
ylabel style={font=\color{white!15!black}},
ylabel={Structured Singular Value},
axis background/.style={fill=white},
xmajorgrids,
xminorgrids,
ymajorgrids,
legend style={legend cell align=left, align=left, draw=white!15!black}
]
\addplot [color=red]
  table[row sep=crcr]{%
0.001	0.600000340275155\\
0.00114975699539774	0.600000303446415\\
0.00132194114846603	0.600000369653702\\
0.00151991108295293	0.600000391616797\\
0.00174752840000768	0.600000351274006\\
0.00200923300256505	0.600000458558298\\
0.00231012970008316	0.600000508626772\\
0.00265608778294669	0.600000515945317\\
0.00305385550883342	0.600000359890072\\
0.00351119173421513	0.600000755526859\\
0.00403701725859655	0.600000617014528\\
0.00464158883361278	0.600000358842016\\
0.00533669923120631	0.600000478114046\\
0.00613590727341317	0.600000349740781\\
0.00705480231071865	0.600000386708984\\
0.00811130830789687	0.600000331922276\\
0.0093260334688322	0.600000506908223\\
0.0107226722201032	0.600000302257317\\
0.0123284673944207	0.600000316015132\\
0.014174741629268	0.600000300071634\\
0.0162975083462064	0.60000030882217\\
0.0187381742286038	0.600000312691648\\
0.0215443469003188	0.600000300221671\\
0.0247707635599171	0.600000348849607\\
0.028480358684358	0.600000302119978\\
0.0327454916287773	0.600000301104874\\
0.0376493580679247	0.600000342560524\\
0.0432876128108306	0.600000301915112\\
0.0497702356433211	0.600000340494793\\
0.0572236765935022	0.600000300065691\\
0.0657933224657568	0.600000314993302\\
0.0756463327554629	0.6000003006312\\
0.0869749002617783	0.600000468364682\\
0.1	0.600000382277408\\
0.114975699539774	0.600000312917969\\
0.132194114846603	0.600000305740325\\
0.151991108295293	0.600000335556097\\
0.174752840000768	0.600000300139694\\
0.200923300256505	0.600000356649738\\
0.231012970008316	0.600000333852732\\
0.265608778294669	0.600000352710968\\
0.305385550883342	0.600000301040145\\
0.351119173421513	0.600000319335999\\
0.403701725859655	0.600000303008251\\
0.464158883361278	0.600000314789264\\
0.533669923120631	0.600000327879648\\
0.613590727341317	0.60000031632902\\
0.705480231071864	0.600000311532099\\
0.811130830789687	0.600000968236792\\
0.93260334688322	0.600059415056767\\
1.07226722201032	0.600001900649574\\
1.23284673944207	0.600045344708669\\
1.41747416292681	0.600001115385202\\
1.62975083462064	0.600001886819797\\
1.87381742286038	0.600027979604195\\
2.15443469003188	0.600133049781685\\
2.47707635599171	0.600035156821348\\
2.8480358684358	0.600137782666474\\
3.27454916287773	0.600115754972028\\
3.76493580679247	0.600072858648583\\
4.32876128108306	0.600059058349708\\
4.97702356433211	0.600175876908626\\
5.72236765935022	0.600149003283884\\
6.57933224657568	0.600037844047125\\
7.56463327554629	0.600129796456446\\
8.69749002617784	0.600093481843494\\
10	0.600193722885227\\
11.4975699539774	0.600067402227291\\
13.2194114846603	0.600144553767107\\
15.1991108295293	0.600081300360477\\
17.4752840000768	0.600034543437946\\
20.0923300256505	0.600120622273841\\
23.1012970008316	0.600036027581542\\
26.5608778294669	0.600026943758038\\
30.5385550883342	0.600015773314596\\
35.1119173421513	0.600032402060817\\
40.3701725859656	0.600028498659195\\
46.4158883361278	0.600008062701707\\
53.3669923120631	0.600018307338888\\
61.3590727341317	0.600025246690404\\
70.5480231071865	0.600005347247073\\
81.1130830789687	0.600007270691961\\
93.260334688322	0.600007954284663\\
107.226722201032	0.60000462015688\\
123.284673944207	0.600006811189973\\
141.74741629268	0.600005814240362\\
162.975083462065	0.600002225286458\\
187.381742286039	0.600003434699088\\
215.443469003188	0.600002738626805\\
247.707635599171	0.600002103520323\\
284.80358684358	0.600001809938279\\
327.454916287773	0.600001431494853\\
376.493580679247	0.60000118230587\\
432.876128108306	0.600001017250571\\
497.702356433211	0.6000009653417\\
572.236765935022	0.600000821684035\\
657.933224657568	0.600000675451635\\
756.463327554629	0.600000708276678\\
869.749002617783	0.600000545187693\\
1000	0.60000051486879\\
};
\addlegendentry{M from Robust Control Toolbox}

\end{axis}
\end{tikzpicture}%
    \end{tikzpicture}}
    \caption{Structured singular values of $\mathbf{M}$ computed from \emph{Robust Control Toolbox}.}
    \label{fig:usv}
\end{figure}
The command \emph{lftdata(.)} is used to obtain the unperturbed closed loop system, and the $\mathbf{M}$ is obtained by taking the first 4 rows and columns of the closed loop system matrix.



Figure \ref{fig:ssv} shows us that the manually computed $\mathbf{M}$ has more conservatism than the one computed from \emph{Robust Control Toolbox}. Additionally, since the perturbation matrix is structured in the form of a block diagonal matrix, it makes more sense to compute the \emph{structured singular value} of $\mathbf{M}$, since the structured singular value consitutes the necessary and sufficient condition for RS when $\mathbf{\Delta}$ is structured.
Figure \ref{fig:usv} shows us that $\mu_\Delta(\mathbf{M})$ is roughly equal to $0.6$, which implies RS.

\subsection{Nominal Performance}
\begin{figure}[h!]
    \centering
    \scalebox{0.7}{
    \begin{tikzpicture}
        % This file was created by matlab2tikz.
%
%The latest updates can be retrieved from
%  http://www.mathworks.com/matlabcentral/fileexchange/22022-matlab2tikz-matlab2tikz
%where you can also make suggestions and rate matlab2tikz.
%
\begin{tikzpicture}

\begin{axis}[%
width=4.521in,
height=3.563in,
at={(0.758in,0.484in)},
scale only axis,
xmode=log,
xmin=0.001,
xmax=1000,
xminorticks=true,
xlabel style={font=\color{white!15!black}},
xlabel={Frequency (rad/s)},
ymin=0.5,
ymax=4.5,
ylabel style={font=\color{white!15!black}},
ylabel={Upper Singular Value},
axis background/.style={fill=white},
xmajorgrids,
xminorgrids,
ymajorgrids,
legend style={legend cell align=left, align=left, draw=white!15!black}
]
\addplot [color=red]
  table[row sep=crcr]{%
0.001	0.595195103883239\\
0.00114975699539774	0.595200829739262\\
0.00132194114846603	0.595208427505368\\
0.00151991108295293	0.595218521470749\\
0.00174752840000768	0.595231953520382\\
0.00200923300256505	0.595249866186923\\
0.00231012970008316	0.595273822952567\\
0.00265608778294669	0.595305986765299\\
0.00305385550883342	0.595349392808403\\
0.00351119173421513	0.595408380195091\\
0.00403701725859655	0.595489304438244\\
0.00464158883361278	0.595601773610852\\
0.00533669923120631	0.595760926020644\\
0.00613590727341317	0.595991947090471\\
0.00705480231071865	0.596339893722345\\
0.00811130830789687	0.596893797677413\\
0.0093260334688322	0.597856391561125\\
0.0107226722201032	0.599800254342413\\
0.0123284673944207	0.605014960159333\\
0.014174741629268	0.626447991834858\\
0.0162975083462064	0.68667513180581\\
0.0187381742286038	0.771122908311933\\
0.0215443469003188	0.871533561032883\\
0.0247707635599171	0.987386666684402\\
0.028480358684358	1.11941052753217\\
0.0327454916287773	1.26845273570405\\
0.0376493580679247	1.43511249000193\\
0.0432876128108306	1.61947369456022\\
0.0497702356433211	1.82083911872878\\
0.0572236765935022	2.03747039324224\\
0.0657933224657568	2.26638280026341\\
0.0756463327554629	2.50327225982153\\
0.0869749002617783	2.74265767253452\\
0.1	2.9782880699895\\
0.114975699539774	3.20378564799623\\
0.132194114846603	3.41339711042127\\
0.151991108295293	3.60265886176641\\
0.174752840000768	3.7687932487047\\
0.200923300256505	3.91074372205327\\
0.231012970008316	4.02887434063569\\
0.265608778294669	4.1244387525134\\
0.305385550883342	4.19893542918487\\
0.351119173421513	4.25342627167176\\
0.403701725859655	4.28784902096503\\
0.464158883361278	4.30035165739872\\
0.533669923120631	4.28677242835656\\
0.613590727341317	4.2406208424237\\
0.705480231071864	4.15421119331531\\
0.811130830789687	4.02157745030738\\
0.93260334688322	3.84277699676158\\
1.07226722201032	3.62720125801902\\
1.23284673944207	3.39270402785529\\
1.41747416292681	3.1600970188236\\
1.62975083462064	2.94660210309679\\
1.87381742286038	2.76225654354182\\
2.15443469003188	2.61009176082889\\
2.47707635599171	2.48835667614681\\
2.8480358684358	2.39291969366311\\
3.27454916287773	2.31897525467076\\
3.76493580679247	2.2619759783324\\
4.32876128108306	2.21801775081647\\
4.97702356433211	2.18392042270406\\
5.72236765935022	2.15717123642601\\
6.57933224657568	2.13582428765666\\
7.56463327554629	2.11840005729265\\
8.69749002617784	2.10380101115439\\
10	2.09124448993673\\
11.4975699539774	2.08020699584565\\
13.2194114846603	2.07037207504923\\
15.1991108295293	2.06157645457194\\
17.4752840000768	2.05375452058587\\
20.0923300256505	2.0468866061631\\
23.1012970008316	2.0409585418431\\
26.5608778294669	2.03593738434245\\
30.5385550883342	2.03176320131985\\
35.1119173421513	2.02835254321636\\
40.3701725859656	2.02560771770745\\
46.4158883361278	2.02342695415928\\
53.3669923120631	2.02171260079756\\
61.3590727341317	2.02037638732998\\
70.5480231071865	2.01934196700474\\
81.1130830789687	2.01854544546908\\
93.260334688322	2.01793465612301\\
107.226722201032	2.01746779385484\\
123.284673944207	2.01711182495109\\
141.74741629268	2.01684092266787\\
162.975083462065	2.01663505647904\\
187.381742286039	2.0164787853603\\
215.443469003188	2.01636026079883\\
247.707635599171	2.01627042251215\\
284.80358684358	2.016202360509\\
327.454916287773	2.01615081523125\\
376.493580679247	2.0161117893726\\
432.876128108306	2.01608224840491\\
497.702356433211	2.01605989066976\\
572.236765935022	2.01604297151981\\
657.933224657568	2.01603016917141\\
756.463327554629	2.01602048258511\\
869.749002617783	2.01601315384759\\
1000	2.01600760924517\\
};
\addlegendentry{Upper Singular Value (N22)}

\end{axis}
\end{tikzpicture}%
    \end{tikzpicture}}
    \caption{Upper singular values of $\mathbf{N}_{22}$.}
    \label{fig:np}
\end{figure}
Figure \ref{fig:np} shows us that $\Bar{\sigma}(\mathbf{N}_{22}) > 1$, for higher frequencies which implies NP is not satisfied.

\subsection{Uncertain Time-Delay}
Since the uncertain time-delay has uniform bounds, multiplicative uncertainty can be used to model the perturbation. Using first-order Pade´ approximation for the time delay, we have a standard form of multiplicative uncertainty weight to model Gain and Time-Delay Uncertainty, as proposed by Lundstörm \cite{Sko05},
$$w_M(s) = \frac{\left(1 + \frac{r_k}{2}\right)\tau_{max}s + r_k}{\frac{\tau_{max}}{2}s + 1}$$
Since there is only time-delay uncertainty, $r_k = 0$ and
\begin{align*}
    w_\tau(s) = \frac{\tau_{max}s}{\frac{\tau_{max}}{2}s + 1}, \|\Delta_\tau(j\omega)\| \le 1\\
    G_p(s) = \left(\frac{h_p}{m_ps^2 + c_ps + k_p}\right)e^{-\tau_ps}
\end{align*}

Figure \ref{fig:blockDiagramC} shows the generalised representation of the system with the additional time delay uncertainty. The generalized plant is obtained as,
\begin{align*}
    \bm{P} = 
    \begin{bmatrix}
        \frac{-b_m}{\Bar{m}} & -1 & -(b_k-\Bar{k}d_k) & 0 & 1 & -\Bar{k} & -\Bar{c} & 0 & 1\\
        0 & 0 & 0 & 0 & 0 & 0 & r_c\Bar{c} & 0 & 0\\
        0 & 0 & -d_k & 0 & 0 & 1 & 0 & 0 & 0\\
        0 & 0 & 0 & -d_h & 0 & 1 & 0 & 0 & 0\\
        0 & 0 & 0 & 0 & 0 & 1 & 0 & 0 & w_\tau\\
        0 & 0 & 0 & 0 & 0 & 0 & 1 & 0 & 0\\
        \frac{d_m}{\Bar{m}} - \frac{b_m}{\Bar{m}^2} & \frac{-1}{\Bar{m}} & \frac{-(b_k-\Bar{k}d_k)}{\Bar{m}} & 0 & \frac{-1}{\Bar{m}} & \frac{-\Bar{k}}{\Bar{m}} & \frac{-\Bar{c}}{\Bar{m}} & 0 & \frac{1}{\Bar{m}}\\
        0 & 0 & 0 & W_1(b_h-\Bar{h}d_h) & 0 & W_1\Bar{h} & 0 & W_1 & 0\\
        0 & 0 & 0 & -(b_h-\Bar{h}d_h) & 0 & -\Bar{h} & 0 & -1 & 0
    \end{bmatrix}
\end{align*}
\begin{figure}[htb]
  \centering
  \tikzstyle{block}     = [draw, rectangle, minimum height=0.5cm, minimum width=0.5cm]
    \tikzstyle{branch}    = [circle, inner sep=0pt, minimum size=1mm, fill=black, draw=black]
    \tikzstyle{connector} = [->, thin]
    \tikzstyle{dummy}     = [inner sep=0pt, minimum size=0pt]
    \tikzstyle{inout}     = []
    \tikzstyle{sum}       = [circle, inner sep=0pt, minimum size=2mm, draw=black, thick]
    \begin{tikzpicture}[auto, node distance=2cm, >=stealth']
      %%%%%%%%%%%%%%%%%%%%%%% BLOCKS %%%%%%%%%%%%%%%%%%%%%%%%%%%%%%%%%%
      % Mass
      \node[block] (m) {$\begin{matrix}
          \frac{1}{\Bar{m}}&\frac{d_m}{\Bar{m}} - \frac{b_m}{\Bar{m}^2}\\
        1&\frac{-b_m}{\Bar{m}}
      \end{matrix}$};
      % Differentiator 1
      \node[block, right of = m] (diff1) {$\frac{1}{s}$};
      % Differentiator 1
      \node[block, right of = diff1] (diff2) {$\frac{1}{s}$};
      % Output Gain
      \node[block, right of = diff2] (h) {$\begin{matrix}
        \Bar{h}&b_h-\Bar{h}d_h\\
        1&-d_h
    \end{matrix}$};
     % Damping
     \node[block, below of = m, node distance = 4cm] (c) {$\begin{matrix}
        \Bar{c}&1\\
        r_c\Bar{c}&0
     \end{matrix}$};
     % Stiffness
     \node[block, below of = c, node distance = 4cm] (k) {$\begin{matrix}
        \Bar{k}&b_k-\Bar{k}d_k\\
        1&-d_k
     \end{matrix}$};
     % Deltam
     \node[block,below of = m] (delm) {$\Delta_m$};
     % Deltah
     \node[block,below of = h] (delh) {$\Delta_h$};
     % Deltac
     \node[block,below of = c] (delc) {$\Delta_c$};
     % Deltak
     \node[block,below of = k] (delk) {$\Delta_k$};
     % W1 weight
     \node[block,right of = h, node distance = 3cm] (w1) {$W_1$};
     % Controller
     \node[block,left of = m, node distance = 5.5cm] (K) {$K$};
     % Time Delay uncertainty weight
     \node[block,above right = 0.5cm of K] (wtau) {$w_\tau$};
     % Deltatau
     \node[block,right of = wtau, node distance = 1.5cm] (deltau) {$\Delta_\tau$};

     %%%%%%%%%%%%%%%%%%%%%%% SUMMATIONS %%%%%%%%%%%%%%%%%%%%%%%%%%%%%%%%%%
     \node[sum,left of = K,node distance = 0.75cm] (s1) {};
     \node[sum,left of = m] (s2) {};
     \node[sum,right of = h] (s3) {};
     \node[dummy] (d1) [below right=0.1cm and 0.05cm of s1] {$-$};
     \node[dummy] (d2) [below right=0.1cm and 0.05cm of s2] {$-$};

     %%%%%%%%%%%%%%%%%%%%%%% I/O SIGNALS %%%%%%%%%%%%%%%%%%%%%%%%%%%%%%%%%%
     \node[inout,right of = w1, node distance = 0.75cm] (z) {$z$};
     \node[inout,above of = s3] (d) {$d$};

     %%%%%%%%%%%%%%%%%%%%%%% BRANCHES %%%%%%%%%%%%%%%%%%%%%%%%%%%%%%%%%%
     \node[branch,right of = diff1,node distance = 1cm] (b1) {};
     \node[branch,right of = diff2,node distance = 0.5cm] (b2) {};
     \node[branch,right of = s3,node distance = 0.25cm] (b3) {};
     \node[branch,below of = s2,node distance = 4cm] (b4) {};
     \node[branch,below of = b3,node distance = 11cm] (b5) {};
     \node[branch,right of = K,node distance = 0.5cm] (b6) {};
     

     %%%%%%%%%%%%%%%%%%%%%%% CONNECTIONS %%%%%%%%%%%%%%%%%%%%%%%%%%%%%%%%%%
     % Interconnection between m and delm
     \draw[->] (delm.west) -| ++(-1,1) node [yshift = -0.3cm, xshift = -0.3cm] {$u_m$} |- (m.200);
     \draw[->] (m.-20) -| ++(0.25,-1) node [yshift = 0.3cm, xshift = 0.3cm] {$y_m$} |- (delm.east);

     % Interconnection between c and delc
     \draw[->] (delc.east) -| ++(1,1) node [yshift = -0.3cm, xshift = -0.3cm] {$u_c$} |- (c.-20);
     \draw[->] (c.200) -| ++(-0.75,-1) node [yshift = 0.3cm, xshift = 0.3cm] {$y_c$} |- (delc.west);

     % Interconnection between k and delk
     \draw[->] (delk.east) -| ++(1,1) node [yshift = -0.3cm, xshift = 0.3cm] {$u_k$} |- (k.-20);
     \draw[->] (k.200) -| ++(-0.25,-1) node [yshift = 0.3cm, xshift = 0.3cm] {$y_k$} |- (delk.west);

     % Interconnection between h and delh
     \draw[->] (delh.west) -| ++(-1,1) node [yshift = -0.3cm, xshift = 0.3cm] {$u_h$} |- (h.200);
     \draw[->] (h.-20) -| ++(0.25,-1) node [yshift = 0.3cm, xshift = 0.3cm] {$y_h$} |- (delh.east);

     \draw[connector] (s1) -- (K);
     \draw[connector] (K) -- node[] {$F$} (s2);
     \draw[connector] (s2) -- (m.180);
     \draw[connector] (m.0) -- node[] {$\dot{x}_2$} (diff1);
     \draw[connector] (diff1) -- node[] {$x_2$} (diff2);
     \draw[connector] (diff2) -- node[] {$x_1$} (h.180);
     \draw[connector] (h.0) -- (s3);
     \draw[connector] (d) -- (s3);
     \draw[connector] (s3) -- (w1);
     \draw[connector] (w1) -- (z);
     \draw[thick] (b3) -- (b5);
     \draw[connector] (b5) -| (s1);
     \draw[connector] (c.180) -- (b4);
     \draw[connector] (k.180) -| (s2);
     \draw[connector] (b1) |- (c.0);
     \draw[connector] (b2) |- (k.0);
     \draw[connector] (b6) |- (wtau);
     \draw[connector] (wtau) -- node[] {$y_\tau$} (deltau);
     \draw[connector] (deltau) -| node[] {$u_\tau$} (s2);
     
    
    \end{tikzpicture}
    
	  \caption{Block Diagram for the uncertain dynamic system with additional time-delay uncertainty.}
    \label{fig:blockDiagramC}
\end{figure}
\begin{figure}[h!]
    \centering
    \scalebox{0.7}{
    \begin{tikzpicture}
        % This file was created by matlab2tikz.
%
%The latest updates can be retrieved from
%  http://www.mathworks.com/matlabcentral/fileexchange/22022-matlab2tikz-matlab2tikz
%where you can also make suggestions and rate matlab2tikz.
%
\begin{tikzpicture}

\begin{axis}[%
width=4.521in,
height=3.563in,
at={(0.758in,0.484in)},
scale only axis,
xmode=log,
xmin=0.001,
xmax=1000,
xminorticks=true,
xlabel style={font=\color{white!15!black}},
xlabel={Frequency (rad/s)},
ymin=0.6,
ymax=0.85,
ylabel style={font=\color{white!15!black}},
ylabel={Structured Singular Value},
axis background/.style={fill=white},
xmajorgrids,
xminorgrids,
ymajorgrids,
legend style={legend cell align=left, align=left, draw=white!15!black}
]
\addplot [color=red]
  table[row sep=crcr]{%
0.001	0.600012295349717\\
0.00114975699539774	0.600014080514533\\
0.00132194114846603	0.600016130821977\\
0.00151991108295293	0.60001848549136\\
0.00174752840000768	0.600001028763694\\
0.00200923300256505	0.600000881309738\\
0.00231012970008316	0.600027178242869\\
0.00265608778294669	0.600000547106235\\
0.00305385550883342	0.600001100465485\\
0.00351119173421513	0.600003331464845\\
0.00403701725859655	0.600002856688716\\
0.00464158883361278	0.60000058700308\\
0.00533669923120631	0.60000045899755\\
0.00613590727341317	0.600000843868508\\
0.00705480231071865	0.600000564838907\\
0.00811130830789687	0.600000534963396\\
0.0093260334688322	0.600001786981197\\
0.0107226722201032	0.600000789386863\\
0.0123284673944207	0.600001140182409\\
0.014174741629268	0.600000461268884\\
0.0162975083462064	0.600000458780218\\
0.0187381742286038	0.60000098924123\\
0.0215443469003188	0.600000590568849\\
0.0247707635599171	0.600000850846264\\
0.028480358684358	0.600001458091271\\
0.0327454916287773	0.600001554132689\\
0.0376493580679247	0.600001492815031\\
0.0432876128108306	0.600001788649039\\
0.0497702356433211	0.60000225545556\\
0.0572236765935022	0.600005183345166\\
0.0657933224657568	0.600002685978594\\
0.0756463327554629	0.600004282139741\\
0.0869749002617783	0.600005789589362\\
0.1	0.600007172595302\\
0.114975699539774	0.600009130613685\\
0.132194114846603	0.600011745880123\\
0.151991108295293	0.601348917217868\\
0.174752840000768	0.600019928311135\\
0.200923300256505	0.60003139219622\\
0.231012970008316	0.600039090376825\\
0.265608778294669	0.600055960613865\\
0.305385550883342	0.600075158005905\\
0.351119173421513	0.600118523276375\\
0.403701725859655	0.600196617626383\\
0.464158883361278	0.600392643171905\\
0.533669923120631	0.601220845353441\\
0.613590727341317	0.631501984443038\\
0.705480231071864	0.692995928008632\\
0.811130830789687	0.742759381809802\\
0.93260334688322	0.789848971339974\\
1.07226722201032	0.804871208693054\\
1.23284673944207	0.810421818437543\\
1.41747416292681	0.816293713019121\\
1.62975083462064	0.82264872901734\\
1.87381742286038	0.832428125748615\\
2.15443469003188	0.83453232739319\\
2.47707635599171	0.824617786263821\\
2.8480358684358	0.810611214701455\\
3.27454916287773	0.79543646935122\\
3.76493580679247	0.780322752731953\\
4.32876128108306	0.765182056928367\\
4.97702356433211	0.749687404012351\\
5.72236765935022	0.733633891670498\\
6.57933224657568	0.716658123564811\\
7.56463327554629	0.698788542957639\\
8.69749002617784	0.682902087475284\\
10	0.668837187004054\\
11.4975699539774	0.658468720882833\\
13.2194114846603	0.64993230566115\\
15.1991108295293	0.64343656143371\\
17.4752840000768	0.637769996230789\\
20.0923300256505	0.633413538216953\\
23.1012970008316	0.629700218121817\\
26.5608778294669	0.6264985872872\\
30.5385550883342	0.623687116721074\\
35.1119173421513	0.621013196994132\\
40.3701725859656	0.617956398310318\\
46.4158883361278	0.614900710768887\\
53.3669923120631	0.6120970915932\\
61.3590727341317	0.609655511838786\\
70.5480231071865	0.607606015286135\\
81.1130830789687	0.60593626480871\\
93.260334688322	0.604598460914084\\
107.226722201032	0.603543423478362\\
123.284673944207	0.60272188588727\\
141.74741629268	0.60207775461926\\
162.975083462065	0.601590884837475\\
187.381742286039	0.601207899418726\\
215.443469003188	0.600918064991939\\
247.707635599171	0.600700485041552\\
284.80358684358	0.600528231799238\\
327.454916287773	0.600400925771931\\
376.493580679247	0.600306654725615\\
432.876128108306	0.600231333137576\\
497.702356433211	0.600175237982539\\
572.236765935022	0.600132954628363\\
657.933224657568	0.600854245291005\\
756.463327554629	0.600707484071244\\
869.749002617783	0.600588177730218\\
1000	0.600487380736219\\
};
\addlegendentry{M from Robust Control Toolbox}

\end{axis}

\begin{axis}[%
width=5.833in,
height=4.375in,
at={(0in,0in)},
scale only axis,
xmin=0,
xmax=1,
ymin=0,
ymax=1,
axis line style={draw=none},
ticks=none,
axis x line*=bottom,
axis y line*=left
]
\end{axis}
\end{tikzpicture}%
    \end{tikzpicture}}
    \caption{Structured singular values of $\mathbf{M}$ with added time delay uncertainty computed from \emph{Robust Control Toolbox}.}
    \label{fig:ssv2}
\end{figure}

It is found that the upper singular values of $\mathbf{M}$ and $\mathbf{N}_{22}$ stay roughly the same as compared to when there is no time delay. Hence, the time delay has no effect on the nominal performance. But as figure \ref{fig:ssv2} shows, the robustness slightly decreases between the frequencies $0.61$ rad/s and $100$ rad/s.