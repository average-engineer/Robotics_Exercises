\section{Exercise 6}
For the given scenario, it is assumed that the manipulator model from Angela makes use of the base coordinates $\left(x_B,y_B,z_B\right)$ to compute the end-effector coordinates $\left(x_E,y_E,z_E\right)$ and the orientation quaternion $\mathbf{q}$ which are then fed back to Donald's crane model. \figref{fig:workflow} illustrates this assumed workflow.
\begin{figure}[h!]
	\centering
\tikzstyle{block}     = [draw, rectangle, minimum height=1.5cm, minimum width=1.6cm]
\tikzstyle{branch}    = [circle, inner sep=0pt, minimum size=0.01mm, fill=black, draw=black]
\tikzstyle{connector} = [->, thick]
\tikzstyle{dummy}     = [inner sep=0pt, minimum size=0pt]
\tikzstyle{inout}     = []
\begin{tikzpicture}[auto, node distance=3cm, >=stealth']
	\node[block](crane) {\begin{tabular}{c}
			3-DOF\\
			Gantry Crane
	\end{tabular}};

	\node[block, right of = crane, node distance=6cm](man){\begin{tabular}{c}
			6-DOF\\
			Robot Manipulator
	\end{tabular}};
	
	\node[branch,right of = man](bur){};
	\node[branch,left of = crane](bul){};
	\node[branch,below of = bul](bll){};

	\draw[connector](crane) --node[yshift=0.25cm]{$\left(x_B,y_B,z_B\right)\trans$}(man);
	\draw[thick](man) -- (bur);
	\draw[thick](bur) |- node[xshift = -3cm, yshift=-0.25cm]{$\left(x_E,y_E,z_E,\mathbf{q}\right)\trans$}(bll);
	\draw[connector](bll) |- (crane);
\end{tikzpicture}
	\caption{Workflow between Donald and Angela}
	\label{fig:workflow}
\end{figure}
\begin{enumerate}
	\item It will be possible for Angela to make use of just a single linear scaling factor which is $\frac{1}{39.3701}$, that converts the base coordinates from inches into meters.
	\item While Donald can use a single linear scaling factor for converting end-effector coordinates from meters to inches $\left(39.3701\right)$ and the euler angles from radians to degrees $\left(\frac{180}{\pi}\right)$, he will first have to convert the output quaternions from Angela's model into the equivalent euler angles which is a non-linear tranformation and is given by
	\begin{equation*}
		\left[\begin{array}{l}
			\phi \\
			\theta \\
			\psi
		\end{array}\right]=\left[\begin{array}{c}
			a \tan 2\left(2 q_2 q_3+2 q_0 q_1 , q_3^2-q_2^2-q_1^2+q_0^2\right) \\
			-a \sin \left(2 q_1 q_3-2 q_0 q_2\right) \\
			a \tan 2\left(2 q_1 q_2+2 q_0 q_3, q_1^2+q_0^2-q_3^2-q_2^2\right)
		\end{array}\right] = \mathcal{F}\left(\mathbf{q}\right)
	\end{equation*}
	assuming $\mathbf{q} = \left(q_1, q_2, q_3, q_4\right)\trans$. Hence, there exists no single linear scaling factor that Donald can use to convert the 6D pose to imperial system.
\end{enumerate}

\begin{figure}[h!]
	\centering
\tikzstyle{block}     = [draw, rectangle, minimum height=1.5cm, minimum width=1.6cm]
\tikzstyle{blockM}     = [draw, rectangle, minimum height=1.25cm, minimum width=1.25cm]
\tikzstyle{branch}    = [circle, inner sep=0pt, minimum size=0.01mm, fill=black, draw=black]
\tikzstyle{connector} = [->, thick]
\tikzstyle{dummy}     = [inner sep=0pt, minimum size=0pt]
\tikzstyle{inout}     = []
\begin{tikzpicture}[auto, node distance=3cm, >=stealth']
	\node[block](crane) {\begin{tabular}{c}
			3-DOF\\
			Gantry Crane
	\end{tabular}};
	\node[block, right of = crane, node distance = 5cm](t1) {$\frac{1}{m}\bm{I}_{3\times3}$};
	
	\node[block, right of = t1, node distance=5cm](man){\begin{tabular}{c}
			6-DOF\\
			Robot Manipulator
	\end{tabular}};
	
	\node[branch,right of = man](bur){};
	\node[branch,left of = crane](bul){};
	\node[branch,below of = bul](bll){};
	\node[branch,below of = bur](blr){};
	
	\node[branch, left of = blr](b1){};
	\node[branch, above of = b1, node distance = 1cm](b2){};
	\node[branch, below of = b1, node distance = 1cm](b3){};
	
	\node[block, left of = b2, node distance = 2cm](t2) {$m\bm{I}_{3\times3}$};
	
	\node[block, left of = b3, node distance = 2cm](t3){};
	\node[blockM, left of = b3, node distance = 2cm](t31){$\mathcal{F}$};
	\node[block, left of = t3, node distance = 2cm](t4){$\frac{180}{\pi}\bm{I}_{3\times3}$};
	
	\node[branch, left of = t4, node distance = 1.5cm](b4){};
	\node[branch, above of = b4, node distance = 2cm](b5){};
	\node[branch, above of = b4, node distance = 1cm](b6){};
	
	\draw[thick](b2)--(b3);
	\draw[connector](b2) -- node[xshift = 0.5cm,yshift = 0.8cm]{$\left(x_E,y_E,z_E\right)_{\text{m}}$}(t2);
	\draw[connector](b3) -- node{$\mathbf{q}$}(t3);
	\draw[connector](t3) -- (t4);
	\draw[thick](t4)--(b4);
	\draw[thick](t2)--(b5);
	\draw[thick](b5)--(b4);
	
	\draw[connector](crane) --node[yshift=0.25cm]{$\left(x_B,y_B,z_B\right)_{\text{in}}\trans$}(t1);
	\draw[connector](t1) --node[yshift=0.25cm]{$\left(x_B,y_B,z_B\right)_{\text{m}}\trans$}(man);
	\draw[thick](man) -- (bur);
	\draw[thick](b6) -- (bll);
	\draw[thick](bur) |- (b1);
	\draw[connector](bll) |- node[xshift = 1.5cm, yshift=-2cm]{\begin{tabular}{c}
			$\left(x_E,y_E,z_E\right)_{\text{in}}$\\
			$\left(\phi,\theta,\psi\right)$
		\end{tabular}}(crane);
	%\draw[thick](bur) |- node[xshift = -3cm, yshift=-0.25cm]{$\left(x_E,y_E,z_E,\mathbf{q}\right)\trans$}(bll);
	%\draw[connector](bll) |- (crane);
\end{tikzpicture}
	\caption{Modified Workflow between Donald and Angela}
	\label{fig:modworkflow}
\end{figure}
\figref{fig:modworkflow} illustrates the way Donald and Angela can use each other's models. In the figure, $\mathcal{F}$ refers to the non-linear transformation between quaternions and euler angles, and $m = 39.3701$