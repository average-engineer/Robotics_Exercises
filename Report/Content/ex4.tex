\section{Exercise 4}
The geometric object in which the end-effector lives is called the \emph{Workspace}.
\subsection{Forward Kinematics}
The formulation of the forward kinematics is based on \cite{Craig}, which makes use of the \emph{Denavit-Hartenberg} parameters.

For the given formulation, it is assumed that $q_1 \in \left[0,\pi\right]$ and $q_2 \in \left[\frac{-\pi}{2},\frac{\pi}{2}\right]$.

For brevity, only the Z and X axes of the joint frames are drawn in 2D, axis coming out of the plane is denoted by $\oplus$ and axis going into the plane is denoted by $\otimes$.

\begin{figure}[h!]
	\centering
\tikzstyle{mod}    = [circle, inner sep=0pt, minimum size=1.0cm, draw=black]
\tikzstyle{branch}    = [circle, inner sep=0pt, minimum size=0.1mm, fill=black, draw=black]
\tikzstyle{dummy}     = [inner sep=0pt, minimum size=0pt]
\begin{tikzpicture}[auto, node distance=2cm, >=stealth']
	% Frames 1 and 2
	\coordinate (base) at (0,0);
	\node[] at (base){$\otimes$};
	\node[yshift = -0.25cm] at (base){$z_0,z_1$};
	\coordinate (x1) at ($(base) + (0,1)$);
	\node[yshift = 0.25cm] at (x1){$x_0,x_1$};
	\draw[thick,->](base)--(x1);
	\draw[->, bend left = 45] ($(base) + (-0.5,0.2)$)node[yshift = 0.3cm]{$q_1$} to ($(base) + (0.5,0.2)$);
	
	% Frame 2
	\coordinate (x2) at ($(base) + (2,0)$);
	\node[] at (x2){$\oplus$};
	\node[yshift = -0.25cm] at (x2){$x_2$};
	\coordinate (z2) at ($(x2) + (1,0)$);
	\node[yshift = 0.25cm] at (z2){$z_2$};
	\draw[thick,->](x2)--(z2);
	\draw[->, bend right = 45] ($(x2) + (0.5,-0.2)$)node[xshift = 0.3cm]{$q_2$} to ($(x2) + (0.5,0.2)$);
	
\end{tikzpicture}
	\caption{Joint Frame Assignment}
	\label{fig:frames}
\end{figure}

\figref{fig:frames} illustrates the joint frame allocation along with the base frame $\left\{0\right\}$. 


\begin{table}[h!]
	\centering
	\begin{tabular}{|c|c|c|c|}
		\hline
		\textbf{Frame $\left(i\right)$} &  $\alpha_{i-1} \left[\text{deg}\right]$ & $a_{i-1} \left[\text{m}\right]$ $d_i \left[\text{m}\right]$ & $\theta_i \left[\text{deg}\right]$ \\ \hline
		1                                  & 0                         & 0                 &       $q_1$                              \\ \hline
		2                                 & -90                         & 0                 &       $q_2$                              \\ \hline
		3                                 & 90                         & $q_3$                    & 0                                   \\ \hline
	\end{tabular}
	\caption{Denavit-Hartenberg parameters}
	\label{tab:dh}
\end{table}
Table \ref{tab:dh} summarizes the DH parameters according to the frame allocation of the manipulator. The homogenous transformation matrix is used to represent the pose. The transformation matrix between two arbritary frames $\left\{A\right\}$ and $\left\{B\right\}$ is given by
\begin{equation*}
	\bm{T}_{B}^{A} = \left[\begin{array}{c|c}
		\bm{R}_{B}^{A} & \bm{r}_{B}^{A}\\
		\hline
		\bm{0}_{1\times3} & 1
	\end{array}\right]
\end{equation*}
where $\bm{R}_{B}^{A}$ describes the orientation of $\left\{B\right\}$ with respect to $\left\{A\right\}$ while $\bm{r}_{B}^{A}$ represents the position of origin of $\left\{B\right\}$ with respect to the origin of $\left\{A\right\}$. Assuming $E\left(x,y,z\right)$ directly lies on the origin of $\left\{3\right\}$, the forward kinematics are given by
\begin{equation*}
	\bm{T}_{3}^{0} = \left[\begin{array}{c|c}
		\bm{R}_{3}^{0} & \bm{r}_{3}^{0}\\
		\hline
		\bm{0}_{1\times3} & 1
	\end{array}\right]
\end{equation*}
where $\bm{r}_{3}^{0} = \left(x \text{ } y \text{ } z\right)\trans$ and 
\begin{equation*}
	\bm{T}_{3}^{0} = \bm{T}_{1}^{0}\bm{T}_{2}^{1}\bm{T}_{3}^{2}
\end{equation*}
where 
\begin{equation*}
	\bm{T}_{i}^{i-1}=\left[\begin{array}{cccc}
		\cos \theta_i & -\sin \theta_i & 0 & a_{i-1} \\
		-\sin \theta_i \cos \alpha_{i-1} & \cos \theta_i \cos \alpha_{i-1} & -\sin \alpha_{i-1} & -\sin \alpha_{i-1} d_i \\
		\sin \theta_i \sin \alpha_{i-1} & \cos \theta_i \sin \alpha_{i-1} & \cos \alpha_{i-1} & \cos \alpha_{i-1} d_i \\
		0 & 0 & 0 & 1
	\end{array}\right]
\end{equation*}
Hence, 
\begin{equation}
	\begin{bmatrix}
		x\\
		y\\
		z
	\end{bmatrix} = \begin{bmatrix}
	q_3\cos\left(q_1\right)\sin\left(q_2\right)\\
	q_3\sin\left(q_1\right)\sin\left(q_2\right)\\
	q_3\cos\left(q_2\right)
\end{bmatrix}
\label{eq:fwdKin}
\end{equation}

\subsection{Inverse Kinematics}
Inverse kinematics can be computed with the help of the obtained forward kinematics expression. We know from \eqref{eq:fwdKin}
\begin{align*}
	&x^2 + y^2 + z^2 = q_3^2 \\
	\implies&q_3 = \sqrt{x^2 + y^2 + z^2}
\end{align*}
Moreover,
\begin{align*}
	\cos\left(q_2\right) &= \frac{z}{q_3} = \frac{z}{\sqrt{x^2 + y^2 + z^2}}\\
	\sin\left(q_2\right) &= \sqrt{1 - \cos\left(q_2\right)^2} = \frac{\sqrt{x^2 + y^2}}{\sqrt{x^2 + y^2 + z^2}} \\
	\implies \tan\left(q_2\right) &= \frac{\sqrt{x^2 + y^2}}{z} \\
	\implies q_2 &= \arctan2\left(\sqrt{x^2 + y^2},z\right)
\end{align*}
The negative version of $\sin\left(q_2\right)$ is not considered since it is assumed that $q_2 \in \left[\frac{-\pi}{2},\frac{\pi}{2}\right]$ and hence $\arctan2$ is used. Finally the expression for $q_1$ is obtained trivially
\begin{equation*}
	q_1 = \arctan\left(\frac{y}{x}\right)
\end{equation*}
Since $q_1 \in \left[0,\pi\right]$, hence $\arctan$ is used and zero values of $x$ represents arbritary values of $q_1$.
Hence, the inverse kinematics can be summarized as 
\begin{equation}
	\begin{bmatrix}
		q_1\\
		q_2\\
		q_3
	\end{bmatrix} = \begin{bmatrix}
	\arctan\left(\frac{y}{x}\right)\\
	\arctan2\left(\sqrt{x^2 + y^2},z\right)\\
	\sqrt{x^2 + y^2 + z^2}
\end{bmatrix}
\label{eq:invKin}
\end{equation}
\subsection{Program Verification}
The verification of kinematics is done in the MATLAB\textsuperscript{\textregistered} script \emph{Kinematics.m}. The input DH parameters are read from the file \emph{dhParam.txt}. The script first computes the forward kinematics according to \eqref{eq:fwdKin} for the input DH parameters and then the inverse kinematics according to \eqref{eq:invKin}. The output of the inverse kinematics matches the inputs in the the file \emph{dhParam.txt}.

\subsection{Workspace}
\begin{figure}[h!]
	\centering
	\scalebox{1}{% This file was created by matlab2tikz.
%
%The latest updates can be retrieved from
%  http://www.mathworks.com/matlabcentral/fileexchange/22022-matlab2tikz-matlab2tikz
%where you can also make suggestions and rate matlab2tikz.
%
\definecolor{mycolor1}{rgb}{1.00000,0.00000,1.00000}%
%
\begin{tikzpicture}

\begin{axis}[%
width=2.533in,
height=3.566in,
at={(1.752in,0.481in)},
scale only axis,
plot box ratio=1 2.001 2.002,
xmin=0,
xmax=0.998972696375168,
tick align=outside,
xlabel style={font=\color{white!15!black}},
xlabel={X},
ymin=-0.999486216200688,
ymax=0.999486216200688,
ylabel style={font=\color{white!15!black}},
ylabel={Y},
zmin=-1,
zmax=1,
zlabel style={font=\color{white!15!black}},
zlabel={Z},
view={-37.5}{30},
axis background/.style={fill=white},
axis x line*=bottom,
axis y line*=left,
axis z line*=left,
xmajorgrids,
ymajorgrids,
zmajorgrids
]
\addplot3[only marks, mark=o, mark options={}, mark size=1.5000pt, draw=mycolor1, forget plot] table[row sep=crcr]{%
x	y	z\\
0	-0	0\\
0	-0	1\\
0	-0	0\\
3.92316949100234e-18	-0.0640702199807129	0.997945392750336\\
0	-0	0\\
7.83021783704894e-18	-0.127877161684506	0.991790013823246\\
0	-0	0\\
1.17050901384266e-17	-0.191158628701372	0.981559156991065\\
0	-0	0\\
1.55318637136916e-17	-0.253654583909507	0.967294863039029\\
0	-0	0\\
1.92948135293827e-17	-0.315108218023621	0.949055747010669\\
0	-0	0\\
2.2978476817557e-17	-0.375267004879374	0.926916757346022\\
0	-0	0\\
2.65677166156201e-17	-0.433883739117558	0.900968867902419\\
0	-0	0\\
3.00477839673524e-17	-0.490717552003938	0.871318704123389\\
0	-0	0\\
3.34043785295333e-17	-0.545534901210549	0.838088104891841\\
0	-0	0\\
3.66237073351197e-17	-0.598110530491216	0.801413621867957\\
0	-0	0\\
3.96925414715054e-17	-0.648228395307788	0.761445958369135\\
0	-0	0\\
4.25982704409613e-17	-0.695682550603486	0.718349350097728\\
0	-0	0\\
4.53289539798749e-17	-0.740277997075315	0.672300890261317\\
0	-0	0\\
4.78733711238551e-17	-0.78183148246803	0.623489801858734\\
0	-0	0\\
5.02210663170815e-17	-0.820172254596956	0.57211666012217\\
0	-0	0\\
5.2362392376426e-17	-0.855142763005346	0.518392568310525\\
0	-0	0\\
5.42885501337979e-17	-0.886599306373	0.462538290240835\\
0	-0	0\\
5.59916245938125e-17	-0.914412623015812	0.404783343122394\\
0	-0	0\\
5.74646174582053e-17	-0.93846842204976	0.345365054421308\\
0	-0	0\\
5.87014758833406e-17	-0.958667853036661	0.284527586631032\\
0	-0	0\\
5.96971173526441e-17	-0.974927912181824	0.222520933956314\\
0	-0	0\\
6.04474505617541e-17	-0.98718178341445	0.15959989503338\\
0	-0	0\\
6.09493922305684e-17	-0.995379112949198	0.0960230259076819\\
0	-0	0\\
6.12008797731036e-17	-0.999486216200688	0.0320515775716553\\
0	-0	-0\\
6.12008797731036e-17	-0.999486216200688	-0.0320515775716552\\
0	-0	-0\\
6.09493922305684e-17	-0.995379112949198	-0.0960230259076818\\
0	-0	-0\\
6.04474505617541e-17	-0.98718178341445	-0.159599895033379\\
0	-0	-0\\
5.96971173526441e-17	-0.974927912181824	-0.222520933956314\\
0	-0	-0\\
5.87014758833406e-17	-0.958667853036661	-0.284527586631032\\
0	-0	-0\\
5.74646174582053e-17	-0.93846842204976	-0.345365054421307\\
0	-0	-0\\
5.59916245938125e-17	-0.914412623015812	-0.404783343122394\\
0	-0	-0\\
5.42885501337979e-17	-0.886599306373	-0.462538290240835\\
0	-0	-0\\
5.2362392376426e-17	-0.855142763005346	-0.518392568310525\\
0	-0	-0\\
5.02210663170815e-17	-0.820172254596956	-0.57211666012217\\
0	-0	-0\\
4.78733711238551e-17	-0.78183148246803	-0.623489801858733\\
0	-0	-0\\
4.53289539798749e-17	-0.740277997075315	-0.672300890261317\\
0	-0	-0\\
4.25982704409613e-17	-0.695682550603486	-0.718349350097727\\
0	-0	-0\\
3.96925414715054e-17	-0.648228395307789	-0.761445958369134\\
0	-0	-0\\
3.66237073351197e-17	-0.598110530491216	-0.801413621867957\\
0	-0	-0\\
3.34043785295333e-17	-0.545534901210549	-0.838088104891841\\
0	-0	-0\\
3.00477839673524e-17	-0.490717552003938	-0.871318704123389\\
0	-0	-0\\
2.65677166156201e-17	-0.433883739117558	-0.900968867902419\\
0	-0	-0\\
2.2978476817557e-17	-0.375267004879374	-0.926916757346022\\
0	-0	-0\\
1.92948135293827e-17	-0.315108218023621	-0.949055747010669\\
0	-0	-0\\
1.55318637136916e-17	-0.253654583909507	-0.967294863039029\\
0	-0	-0\\
1.17050901384267e-17	-0.191158628701373	-0.981559156991065\\
0	-0	-0\\
7.83021783704894e-18	-0.127877161684506	-0.991790013823246\\
0	-0	-0\\
3.92316949100236e-18	-0.0640702199807132	-0.997945392750336\\
0	-0	-0\\
7.49879891330929e-33	-1.22464679914735e-16	-1\\
0	-0	0\\
0	-0	1\\
0	-0	0\\
0.00410499308837696	-0.063938580842253	0.997945392750336\\
0	-0	0\\
0.00819311787963553	-0.127614424341043	0.991790013823246\\
0	-0	0\\
0.0122475753921084	-0.190765872797007	0.981559156991065\\
0	-0	0\\
0.0162517049901984	-0.253133423362496	0.967294863039029\\
0	-0	0\\
0.0201890528465039	-0.314460794394441	0.949055747010669\\
0	-0	0\\
0.0240434395541249	-0.374495978570589	0.926916757346022\\
0	-0	0\\
0.0277990266113163	-0.432992278441656	0.900968867902419\\
0	-0	0\\
0.0314403815052894	-0.489709320164053	0.871318704123389\\
0	-0	0\\
0.0349525411277165	-0.544414041247577	0.838088104891841\\
0	-0	0\\
0.0383210732613533	-0.596881648259169	0.801413621867957\\
0	-0	0\\
0.0415321358851147	-0.646896540547351	0.761445958369135\\
0	-0	0\\
0.044572534053909	-0.694253196191552	0.718349350097728\\
0	-0	0\\
0.0474297741194972	-0.738757016535758	0.672300890261317\\
0	-0	0\\
0.0500921150695738	-0.780225125836136	0.623489801858734\\
0	-0	0\\
0.0525486167741045	-0.818487122736688	0.57211666012217\\
0	-0	0\\
0.0547891849406674	-0.853385780484978	0.518392568310525\\
0	-0	0\\
0.0568046125940658	-0.884777693010579	0.462538290240835\\
0	-0	0\\
0.0585866179097641	-0.91253386421138	0.404783343122394\\
0	-0	0\\
0.0601278782456809	-0.936540238026236	0.345365054421308\\
0	-0	0\\
0.0614220602324969	-0.956698167115792	0.284527586631032\\
0	-0	0\\
0.0624638457988269	-0.972924818225555	0.222520933956314\\
0	-0	0\\
0.0632489540243166	-0.985153512565511	0.15959989503338\\
0	-0	0\\
0.0637741587308623	-0.993333999807569	0.0960230259076819\\
0	-0	0\\
0.0640373017396687	-0.997432664574943	0.0320515775716553\\
0	-0	-0\\
0.0640373017396687	-0.997432664574943	-0.0320515775716552\\
0	-0	-0\\
0.0637741587308623	-0.993333999807569	-0.0960230259076818\\
0	-0	-0\\
0.0632489540243166	-0.985153512565511	-0.159599895033379\\
0	-0	-0\\
0.0624638457988269	-0.972924818225555	-0.222520933956314\\
0	-0	-0\\
0.0614220602324969	-0.956698167115792	-0.284527586631032\\
0	-0	-0\\
0.0601278782456809	-0.936540238026237	-0.345365054421307\\
0	-0	-0\\
0.0585866179097641	-0.91253386421138	-0.404783343122394\\
0	-0	-0\\
0.0568046125940658	-0.884777693010579	-0.462538290240835\\
0	-0	-0\\
0.0547891849406674	-0.853385780484978	-0.518392568310525\\
0	-0	-0\\
0.0525486167741044	-0.818487122736688	-0.57211666012217\\
0	-0	-0\\
0.0500921150695738	-0.780225125836136	-0.623489801858733\\
0	-0	-0\\
0.0474297741194972	-0.738757016535758	-0.672300890261317\\
0	-0	-0\\
0.044572534053909	-0.694253196191552	-0.718349350097727\\
0	-0	-0\\
0.0415321358851147	-0.646896540547352	-0.761445958369134\\
0	-0	-0\\
0.0383210732613533	-0.596881648259169	-0.801413621867957\\
0	-0	-0\\
0.0349525411277165	-0.544414041247577	-0.838088104891841\\
0	-0	-0\\
0.0314403815052894	-0.489709320164054	-0.871318704123389\\
0	-0	-0\\
0.0277990266113163	-0.432992278441656	-0.900968867902419\\
0	-0	-0\\
0.0240434395541249	-0.37449597857059	-0.926916757346022\\
0	-0	-0\\
0.0201890528465039	-0.314460794394441	-0.949055747010669\\
0	-0	-0\\
0.0162517049901984	-0.253133423362496	-0.967294863039029\\
0	-0	-0\\
0.0122475753921084	-0.190765872797007	-0.981559156991065\\
0	-0	-0\\
0.00819311787963553	-0.127614424341043	-0.991790013823246\\
0	-0	-0\\
0.00410499308837698	-0.0639385808422533	-0.997945392750336\\
0	-0	-0\\
7.84633898200472e-18	-1.22213063095555e-16	-1\\
0	-0	0\\
0	-0	1\\
0	-0	0\\
0.00819311787963551	-0.0635442043603297	0.997945392750336\\
0	-0	0\\
0.0163525684804853	-0.126827291954754	0.991790013823246\\
0	-0	0\\
0.0244448228698339	-0.189589219002167	0.981559156991065\\
0	-0	0\\
0.0324366282386122	-0.25157208328194	0.967294863039029\\
0	-0	0\\
0.0402951445443232	-0.312521183909465	0.949055747010669\\
0	-0	0\\
0.0479880794578201	-0.372186067956723	0.926916757346022\\
0	-0	0\\
0.0554838210594141	-0.430321559617085	0.900968867902419\\
0	-0	0\\
0.0627515677390326	-0.486688767685295	0.871318704123389\\
0	-0	0\\
0.0697614547666425	-0.541056067212673	0.838088104891841\\
0	-0	0\\
0.076484677012831	-0.593200051303712	0.801413621867957\\
0	-0	0\\
0.0828936073152621	-0.642906449142932	0.761445958369135\\
0	-0	0\\
0.0889619100046117	-0.689971006479623	0.718349350097728\\
0	-0	0\\
0.0946646491234825	-0.734200324952372	0.672300890261317\\
0	-0	0\\
0.0999783908936014	-0.775412656804416	0.623489801858734\\
0	-0	0\\
0.104881300010241	-0.813438651724158	0.57211666012217\\
0	-0	0\\
0.10935322936817	-0.848122052741921	0.518392568310525\\
0	-0	0\\
0.113375802850431	-0.879320338323358	0.462538290240835\\
0	-0	0\\
0.116932490839746	-0.906905308021003	0.404783343122394\\
0	-0	0\\
0.120008678142261	-0.930763609277412	0.345365054421308\\
0	-0	0\\
0.122591724044507	-0.950797203215131	0.284527586631032\\
0	-0	0\\
0.124671014256813	-0.966923767499479	0.222520933956314\\
0	-0	0\\
0.126238004529689	-0.979077034618674	0.15959989503338\\
0	-0	0\\
0.127286255763985	-0.987207064191256	0.0960230259076819\\
0	-0	0\\
0.127811460470531	-0.991280448181824	0.0320515775716553\\
0	-0	-0\\
0.127811460470531	-0.991280448181824	-0.0320515775716552\\
0	-0	-0\\
0.127286255763985	-0.987207064191256	-0.0960230259076818\\
0	-0	-0\\
0.126238004529689	-0.979077034618674	-0.159599895033379\\
0	-0	-0\\
0.124671014256813	-0.966923767499479	-0.222520933956314\\
0	-0	-0\\
0.122591724044508	-0.950797203215131	-0.284527586631032\\
0	-0	-0\\
0.120008678142261	-0.930763609277412	-0.345365054421307\\
0	-0	-0\\
0.116932490839746	-0.906905308021003	-0.404783343122394\\
0	-0	-0\\
0.113375802850431	-0.879320338323358	-0.462538290240835\\
0	-0	-0\\
0.10935322936817	-0.848122052741921	-0.518392568310525\\
0	-0	-0\\
0.104881300010241	-0.813438651724158	-0.57211666012217\\
0	-0	-0\\
0.0999783908936015	-0.775412656804416	-0.623489801858733\\
0	-0	-0\\
0.0946646491234825	-0.734200324952372	-0.672300890261317\\
0	-0	-0\\
0.0889619100046117	-0.689971006479623	-0.718349350097727\\
0	-0	-0\\
0.0828936073152621	-0.642906449142932	-0.761445958369134\\
0	-0	-0\\
0.076484677012831	-0.593200051303712	-0.801413621867957\\
0	-0	-0\\
0.0697614547666425	-0.541056067212673	-0.838088104891841\\
0	-0	-0\\
0.0627515677390327	-0.486688767685295	-0.871318704123389\\
0	-0	-0\\
0.0554838210594141	-0.430321559617085	-0.900968867902419\\
0	-0	-0\\
0.0479880794578202	-0.372186067956723	-0.926916757346022\\
0	-0	-0\\
0.0402951445443233	-0.312521183909466	-0.949055747010669\\
0	-0	-0\\
0.0324366282386122	-0.25157208328194	-0.967294863039029\\
0	-0	-0\\
0.0244448228698339	-0.189589219002167	-0.981559156991065\\
0	-0	-0\\
0.0163525684804853	-0.126827291954754	-0.991790013823246\\
0	-0	-0\\
0.00819311787963555	-0.06354420436033	-0.997945392750336\\
0	-0	-0\\
1.56604356740979e-17	-1.21459246585495e-16	-1\\
0	-0	0\\
0	-0	1\\
0	-0	0\\
0.0122475753921084	-0.0628887111125007	0.997945392750336\\
0	-0	0\\
0.0244448228698339	-0.125518999021454	0.991790013823246\\
0	-0	0\\
0.0365416213269892	-0.187633502439687	0.981559156991065\\
0	-0	0\\
0.0484882624239586	-0.248976979549135	0.967294863039029\\
0	-0	0\\
0.0602356548499284	-0.309297356844222	0.949055747010669\\
0	-0	0\\
0.0717355260496124	-0.36834676495596	0.926916757346022\\
0	-0	0\\
0.0829406205855365	-0.425882557200362	0.900968867902419\\
0	-0	0\\
0.0938048943207672	-0.481668306665704	0.871318704123389\\
0	-0	0\\
0.104283703624147	-0.53547477774143	0.838088104891841\\
0	-0	0\\
0.114333988820551	-0.587080868096437	0.801413621867957\\
0	-0	0\\
0.123914451132328	-0.636274517235984	0.761445958369135\\
0	-0	0\\
0.132985722384836	-0.682853577903752	0.718349350097728\\
0	-0	0\\
0.141510526778716	-0.726626646748281	0.672300890261317\\
0	-0	0\\
0.14945383406415	-0.767413850840394	0.623489801858734\\
0	-0	0\\
0.156783003487667	-0.805047586809649	0.57211666012217\\
0	-0	0\\
0.163467917920005	-0.839373209562538	0.518392568310525\\
0	-0	0\\
0.169481107613851	-0.870249667752345	0.462538290240835\\
0	-0	0\\
0.174797863082928	-0.89755008338939	0.404783343122394\\
0	-0	0\\
0.179396336638573	-0.921162273209898	0.345365054421308\\
0	-0	0\\
0.183257632166577	-0.940989209661099	0.284527586631032\\
0	-0	0\\
0.18636588277537	-0.95694941960825	0.222520933956314\\
0	-0	0\\
0.188708315996482	-0.968977319125224	0.15959989503338\\
0	-0	0\\
0.190275306269357	-0.977023482992929	0.0960230259076819\\
0	-0	0\\
0.191060414494847	-0.981054847798137	0.0320515775716553\\
0	-0	-0\\
0.191060414494847	-0.981054847798137	-0.0320515775716552\\
0	-0	-0\\
0.190275306269357	-0.977023482992929	-0.0960230259076818\\
0	-0	-0\\
0.188708315996482	-0.968977319125224	-0.159599895033379\\
0	-0	-0\\
0.18636588277537	-0.95694941960825	-0.222520933956314\\
0	-0	-0\\
0.183257632166577	-0.940989209661099	-0.284527586631032\\
0	-0	-0\\
0.179396336638573	-0.921162273209898	-0.345365054421307\\
0	-0	-0\\
0.174797863082928	-0.89755008338939	-0.404783343122394\\
0	-0	-0\\
0.169481107613851	-0.870249667752345	-0.462538290240835\\
0	-0	-0\\
0.163467917920005	-0.839373209562538	-0.518392568310525\\
0	-0	-0\\
0.156783003487667	-0.805047586809649	-0.57211666012217\\
0	-0	-0\\
0.14945383406415	-0.767413850840394	-0.623489801858733\\
0	-0	-0\\
0.141510526778716	-0.726626646748281	-0.672300890261317\\
0	-0	-0\\
0.132985722384836	-0.682853577903752	-0.718349350097727\\
0	-0	-0\\
0.123914451132328	-0.636274517235984	-0.761445958369134\\
0	-0	-0\\
0.114333988820551	-0.587080868096437	-0.801413621867957\\
0	-0	-0\\
0.104283703624147	-0.53547477774143	-0.838088104891841\\
0	-0	-0\\
0.0938048943207672	-0.481668306665705	-0.871318704123389\\
0	-0	-0\\
0.0829406205855365	-0.425882557200362	-0.900968867902419\\
0	-0	-0\\
0.0717355260496125	-0.368346764955961	-0.926916757346022\\
0	-0	-0\\
0.0602356548499285	-0.309297356844222	-0.949055747010669\\
0	-0	-0\\
0.0484882624239586	-0.248976979549135	-0.967294863039029\\
0	-0	-0\\
0.0365416213269892	-0.187633502439687	-0.981559156991065\\
0	-0	-0\\
0.0244448228698339	-0.125518999021454	-0.991790013823246\\
0	-0	-0\\
0.0122475753921084	-0.062888711112501	-0.997945392750336\\
0	-0	-0\\
2.34101802768533e-17	-1.20206327978288e-16	-1\\
0	-0	0\\
0	-0	1\\
0	-0	0\\
0.0162517049901983	-0.0619747946611242	0.997945392750336\\
0	-0	0\\
0.0324366282386122	-0.123694921597434	0.991790013823246\\
0	-0	0\\
0.0484882624239586	-0.184906759568423	0.981559156991065\\
0	-0	0\\
0.0643406479383054	-0.245358776001969	0.967294863039029\\
0	-0	0\\
0.0799286439292479	-0.304802560595631	0.949055747010669\\
0	-0	0\\
0.0951881959776448	-0.362993846087861	0.926916757346022\\
0	-0	0\\
0.110056599310966	-0.41969351200458	0.900968867902419\\
0	-0	0\\
0.124472756470651	-0.474668567256497	0.871318704123389\\
0	-0	0\\
0.138377428374676	-0.527693107549468	0.838088104891841\\
0	-0	0\\
0.151713477743644	-0.578549243673702	0.801413621867957\\
0	-0	0\\
0.164426103890125	-0.627027996857257	0.761445958369135\\
0	-0	0\\
0.176463067906432	-0.672930157504642	0.718349350097728\\
0	-0	0\\
0.187774907325503	-0.716067103791774	0.672300890261317\\
0	-0	0\\
0.198315139372782	-0.756261576753514	0.623489801858734\\
0	-0	0\\
0.208040451973913	-0.793348408678774	0.57211666012217\\
0	-0	0\\
0.216910881733348	-0.827175201820073	0.518392568310525\\
0	-0	0\\
0.224889978152501	-0.85760295462857	0.462538290240835\\
0	-0	0\\
0.231944953412677	-0.88450663294124	0.404783343122394\\
0	-0	0\\
0.238046817107244	-0.907775683773077	0.345365054421308\\
0	-0	0\\
0.243170495369435	-0.927314489603017	0.284527586631032\\
0	-0	0\\
0.247294933906245	-0.943042761286844	0.222520933956314\\
0	-0	0\\
0.250403184515038	-0.954895867982505	0.15959989503338\\
0	-0	0\\
0.252482474727344	-0.962825102732105	0.0960230259076819\\
0	-0	0\\
0.253524260293674	-0.966797882609242	0.0320515775716553\\
0	-0	-0\\
0.253524260293674	-0.966797882609242	-0.0320515775716552\\
0	-0	-0\\
0.252482474727344	-0.962825102732105	-0.0960230259076818\\
0	-0	-0\\
0.250403184515038	-0.954895867982505	-0.159599895033379\\
0	-0	-0\\
0.247294933906245	-0.943042761286844	-0.222520933956314\\
0	-0	-0\\
0.243170495369435	-0.927314489603017	-0.284527586631032\\
0	-0	-0\\
0.238046817107244	-0.907775683773077	-0.345365054421307\\
0	-0	-0\\
0.231944953412677	-0.88450663294124	-0.404783343122394\\
0	-0	-0\\
0.224889978152501	-0.85760295462857	-0.462538290240835\\
0	-0	-0\\
0.216910881733348	-0.827175201820073	-0.518392568310525\\
0	-0	-0\\
0.208040451973913	-0.793348408678774	-0.57211666012217\\
0	-0	-0\\
0.198315139372782	-0.756261576753514	-0.623489801858733\\
0	-0	-0\\
0.187774907325503	-0.716067103791774	-0.672300890261317\\
0	-0	-0\\
0.176463067906432	-0.672930157504642	-0.718349350097727\\
0	-0	-0\\
0.164426103890125	-0.627027996857257	-0.761445958369134\\
0	-0	-0\\
0.151713477743644	-0.578549243673702	-0.801413621867957\\
0	-0	-0\\
0.138377428374676	-0.527693107549468	-0.838088104891841\\
0	-0	-0\\
0.124472756470651	-0.474668567256497	-0.871318704123389\\
0	-0	-0\\
0.110056599310966	-0.41969351200458	-0.900968867902419\\
0	-0	-0\\
0.0951881959776449	-0.362993846087861	-0.926916757346022\\
0	-0	-0\\
0.079928643929248	-0.304802560595631	-0.949055747010669\\
0	-0	-0\\
0.0643406479383054	-0.245358776001969	-0.967294863039029\\
0	-0	-0\\
0.0484882624239587	-0.184906759568423	-0.981559156991065\\
0	-0	-0\\
0.0324366282386122	-0.123694921597434	-0.991790013823246\\
0	-0	-0\\
0.0162517049901984	-0.0619747946611245	-0.997945392750336\\
0	-0	-0\\
3.10637274273832e-17	-1.18459455785242e-16	-1\\
0	-0	0\\
0	-0	1\\
0	-0	0\\
0.0201890528465038	-0.0608062104849334	0.997945392750336\\
0	-0	0\\
0.0402951445443231	-0.121362555208093	0.991790013823246\\
0	-0	0\\
0.0602356548499284	-0.181420195159716	0.981559156991065\\
0	-0	0\\
0.0799286439292479	-0.240732340614918	0.967294863039029\\
0	-0	0\\
0.0992931890660217	-0.299055265245608	0.949055747010669\\
0	-0	0\\
0.118249717190601	-0.356149307644251	0.926916757346022\\
0	-0	0\\
0.136720331862759	-0.411779856143996	0.900968867902419\\
0	-0	0\\
0.154629133364874	-0.465718312888344	0.871318704123389\\
0	-0	0\\
0.171902530590148	-0.517743033188769	0.838088104891841\\
0	-0	0\\
0.18846954344425	-0.567640236310288	0.801413621867957\\
0	-0	0\\
0.204262094517748	-0.61520488394236	0.761445958369135\\
0	-0	0\\
0.219215288830792	-0.660241522745279	0.718349350097728\\
0	-0	0\\
0.233267680500498	-0.702565087509875	0.672300890261317\\
0	-0	0\\
0.246361525235267	-0.742001661630154	0.623489801858734\\
0	-0	0\\
0.258443017618462	-0.778389191763938	0.57211666012217\\
0	-0	0\\
0.26946251220641	-0.811578153744806	0.518392568310525\\
0	-0	0\\
0.279374727532174	-0.841432167008968	0.462538290240835\\
0	-0	0\\
0.288138932176818	-0.867828555012257	0.404783343122394\\
0	-0	0\\
0.295719112143539	-0.890658849334359	0.345365054421308\\
0	-0	0\\
0.302084118846912	-0.909829235398822	0.284527586631032\\
0	-0	0\\
0.307207797109103	-0.925260937977272	0.222520933956314\\
0	-0	0\\
0.311069092637107	-0.936890544893725	0.15959989503338\\
0	-0	0\\
0.313652138539354	-0.944670267598818	0.0960230259076819\\
0	-0	0\\
0.31494632052617	-0.94856813754321	0.0320515775716553\\
0	-0	-0\\
0.31494632052617	-0.94856813754321	-0.0320515775716552\\
0	-0	-0\\
0.313652138539354	-0.944670267598818	-0.0960230259076818\\
0	-0	-0\\
0.311069092637107	-0.936890544893725	-0.159599895033379\\
0	-0	-0\\
0.307207797109103	-0.925260937977272	-0.222520933956314\\
0	-0	-0\\
0.302084118846912	-0.909829235398822	-0.284527586631032\\
0	-0	-0\\
0.295719112143539	-0.890658849334359	-0.345365054421307\\
0	-0	-0\\
0.288138932176818	-0.867828555012257	-0.404783343122394\\
0	-0	-0\\
0.279374727532174	-0.841432167008968	-0.462538290240835\\
0	-0	-0\\
0.26946251220641	-0.811578153744806	-0.518392568310525\\
0	-0	-0\\
0.258443017618462	-0.778389191763938	-0.57211666012217\\
0	-0	-0\\
0.246361525235267	-0.742001661630155	-0.623489801858733\\
0	-0	-0\\
0.233267680500498	-0.702565087509875	-0.672300890261317\\
0	-0	-0\\
0.219215288830792	-0.660241522745279	-0.718349350097727\\
0	-0	-0\\
0.204262094517749	-0.615204883942361	-0.761445958369134\\
0	-0	-0\\
0.18846954344425	-0.567640236310288	-0.801413621867957\\
0	-0	-0\\
0.171902530590148	-0.517743033188769	-0.838088104891841\\
0	-0	-0\\
0.154629133364874	-0.465718312888344	-0.871318704123389\\
0	-0	-0\\
0.136720331862759	-0.411779856143996	-0.900968867902419\\
0	-0	-0\\
0.118249717190601	-0.356149307644251	-0.926916757346022\\
0	-0	-0\\
0.0992931890660219	-0.299055265245609	-0.949055747010669\\
0	-0	-0\\
0.0799286439292479	-0.240732340614918	-0.967294863039029\\
0	-0	-0\\
0.0602356548499285	-0.181420195159716	-0.981559156991065\\
0	-0	-0\\
0.0402951445443231	-0.121362555208093	-0.991790013823246\\
0	-0	-0\\
0.0201890528465039	-0.0608062104849337	-0.997945392750336\\
0	-0	-0\\
3.85896270587653e-17	-1.16225808278902e-16	-1\\
0	-0	0\\
0	-0	1\\
0	-0	0\\
0.0240434395541248	-0.0593877605469687	0.997945392750336\\
0	-0	0\\
0.04798807945782	-0.118531484047215	0.991790013823246\\
0	-0	0\\
0.0717355260496123	-0.177188136254588	0.981559156991065\\
0	-0	0\\
0.0951881959776447	-0.235116684403355	0.967294863039029\\
0	-0	0\\
0.118249717190601	-0.292079087663538	0.949055747010669\\
0	-0	0\\
0.140825324951136	-0.347841275301743	0.926916757346022\\
0	-0	0\\
0.16282225124451	-0.402174108528014	0.900968867902419\\
0	-0	0\\
0.184150105982256	-0.454854322076268	0.871318704123389\\
0	-0	0\\
0.204721248434448	-0.505665441649164	0.838088104891841\\
0	-0	0\\
0.224451147364252	-0.554398673457427	0.801413621867957\\
0	-0	0\\
0.243258728384917	-0.60085376219831	0.761445958369135\\
0	-0	0\\
0.261066707111814	-0.644839813947593	0.718349350097728\\
0	-0	0\\
0.277801906740556	-0.686176080583659	0.672300890261317\\
0	-0	0\\
0.293395558746179	-0.724692702520299	0.623489801858734\\
0	-0	0\\
0.307783585467763	-0.760231406696186	0.57211666012217\\
0	-0	0\\
0.320906863417289	-0.792646156952833	0.518392568310525\\
0	-0	0\\
0.332711466230726	-0.821803754128493	0.462538290240835\\
0	-0	0\\
0.343148886263036	-0.847584383402087	0.404783343122394\\
0	-0	0\\
0.352176233916486	-0.869882106638002	0.345365054421308\\
0	-0	0\\
0.359756413883208	-0.888605297708614	0.284527586631032\\
0	-0	0\\
0.365858277577775	-0.903677019005703	0.222520933956314\\
0	-0	0\\
0.37045675113342	-0.915035337593585	0.15959989503338\\
0	-0	0\\
0.373532938435934	-0.92263357970483	0.0960230259076819\\
0	-0	0\\
0.375074198771851	-0.926440522532786	0.0320515775716553\\
0	-0	-0\\
0.375074198771851	-0.926440522532786	-0.0320515775716552\\
0	-0	-0\\
0.373532938435934	-0.92263357970483	-0.0960230259076818\\
0	-0	-0\\
0.37045675113342	-0.915035337593585	-0.159599895033379\\
0	-0	-0\\
0.365858277577775	-0.903677019005703	-0.222520933956314\\
0	-0	-0\\
0.359756413883208	-0.888605297708614	-0.284527586631032\\
0	-0	-0\\
0.352176233916486	-0.869882106638002	-0.345365054421307\\
0	-0	-0\\
0.343148886263036	-0.847584383402087	-0.404783343122394\\
0	-0	-0\\
0.332711466230726	-0.821803754128493	-0.462538290240835\\
0	-0	-0\\
0.320906863417289	-0.792646156952833	-0.518392568310525\\
0	-0	-0\\
0.307783585467763	-0.760231406696186	-0.57211666012217\\
0	-0	-0\\
0.293395558746179	-0.724692702520299	-0.623489801858733\\
0	-0	-0\\
0.277801906740556	-0.686176080583659	-0.672300890261317\\
0	-0	-0\\
0.261066707111814	-0.644839813947593	-0.718349350097727\\
0	-0	-0\\
0.243258728384917	-0.600853762198311	-0.761445958369134\\
0	-0	-0\\
0.224451147364252	-0.554398673457427	-0.801413621867957\\
0	-0	-0\\
0.204721248434448	-0.505665441649164	-0.838088104891841\\
0	-0	-0\\
0.184150105982256	-0.454854322076268	-0.871318704123389\\
0	-0	-0\\
0.16282225124451	-0.402174108528014	-0.900968867902419\\
0	-0	-0\\
0.140825324951136	-0.347841275301744	-0.926916757346022\\
0	-0	-0\\
0.118249717190601	-0.292079087663538	-0.949055747010669\\
0	-0	-0\\
0.0951881959776447	-0.235116684403355	-0.967294863039029\\
0	-0	-0\\
0.0717355260496124	-0.177188136254588	-0.981559156991065\\
0	-0	-0\\
0.0479880794578201	-0.118531484047215	-0.991790013823246\\
0	-0	-0\\
0.0240434395541249	-0.059387760546969	-0.997945392750336\\
0	-0	-0\\
4.5956953635114e-17	-1.13514563995985e-16	-1\\
0	-0	0\\
0	-0	1\\
0	-0	0\\
0.0277990266113162	-0.0577252735622819	0.997945392750336\\
0	-0	0\\
0.055483821059414	-0.115213341593464	0.991790013823246\\
0	-0	0\\
0.0829406205855364	-0.172227973290854	0.981559156991065\\
0	-0	0\\
0.110056599310965	-0.228534883303208	0.967294863039029\\
0	-0	0\\
0.136720331862759	-0.28390269445949	0.949055747010669\\
0	-0	0\\
0.16282225124451	-0.338103888547301	0.926916757346022\\
0	-0	0\\
0.188255099070633	-0.390915741234015	0.900968867902419\\
0	-0	0\\
0.212914366314083	-0.442121237288834	0.871318704123389\\
0	-0	0\\
0.236698722756361	-0.491509962344926	0.838088104891841\\
0	-0	0\\
0.259510433375115	-0.538878967537186	0.801413621867957\\
0	-0	0\\
0.281255759958318	-0.58403360346266	0.761445958369135\\
0	-0	0\\
0.301845346294681	-0.62678832003669	0.718349350097728\\
0	-0	0\\
0.321194585357495	-0.666967428958017	0.672300890261317\\
0	-0	0\\
0.339223966973052	-0.704405825649691	0.623489801858734\\
0	-0	0\\
0.355859404545005	-0.738949667709194	0.57211666012217\\
0	-0	0\\
0.371032539492079	-0.770457007079873	0.518392568310525\\
0	-0	0\\
0.384681022148151	-0.798798373345952	0.462538290240835\\
0	-0	0\\
0.396748767970395	-0.823857305754238	0.404783343122394\\
0	-0	0\\
0.407186188002705	-0.845530831776342	0.345365054421308\\
0	-0	0\\
0.415950392647348	-0.863729890244883	0.284527586631032\\
0	-0	0\\
0.423005367907524	-0.878379697324927	0.222520933956314\\
0	-0	0\\
0.428322123376601	-0.889420053816808	0.15959989503338\\
0	-0	0\\
0.431878811365916	-0.896805592527553	0.0960230259076819\\
0	-0	0\\
0.433660816681615	-0.900505964694406	0.0320515775716553\\
0	-0	-0\\
0.433660816681615	-0.900505964694406	-0.0320515775716552\\
0	-0	-0\\
0.431878811365916	-0.896805592527553	-0.0960230259076818\\
0	-0	-0\\
0.428322123376601	-0.889420053816808	-0.159599895033379\\
0	-0	-0\\
0.423005367907524	-0.878379697324927	-0.222520933956314\\
0	-0	-0\\
0.415950392647348	-0.863729890244883	-0.284527586631032\\
0	-0	-0\\
0.407186188002705	-0.845530831776342	-0.345365054421307\\
0	-0	-0\\
0.396748767970395	-0.823857305754238	-0.404783343122394\\
0	-0	-0\\
0.384681022148151	-0.798798373345952	-0.462538290240835\\
0	-0	-0\\
0.371032539492079	-0.770457007079874	-0.518392568310525\\
0	-0	-0\\
0.355859404545005	-0.738949667709194	-0.57211666012217\\
0	-0	-0\\
0.339223966973052	-0.704405825649691	-0.623489801858733\\
0	-0	-0\\
0.321194585357495	-0.666967428958017	-0.672300890261317\\
0	-0	-0\\
0.301845346294681	-0.626788320036691	-0.718349350097727\\
0	-0	-0\\
0.281255759958318	-0.58403360346266	-0.761445958369134\\
0	-0	-0\\
0.259510433375115	-0.538878967537186	-0.801413621867957\\
0	-0	-0\\
0.236698722756361	-0.491509962344926	-0.838088104891841\\
0	-0	-0\\
0.212914366314083	-0.442121237288835	-0.871318704123389\\
0	-0	-0\\
0.188255099070633	-0.390915741234015	-0.900968867902419\\
0	-0	-0\\
0.16282225124451	-0.338103888547302	-0.926916757346022\\
0	-0	-0\\
0.13672033186276	-0.283902694459491	-0.949055747010669\\
0	-0	-0\\
0.110056599310965	-0.228534883303208	-0.967294863039029\\
0	-0	-0\\
0.0829406205855365	-0.172227973290855	-0.981559156991065\\
0	-0	-0\\
0.055483821059414	-0.115213341593464	-0.991790013823246\\
0	-0	-0\\
0.0277990266113163	-0.0577252735622822	-0.997945392750336\\
0	-0	-0\\
5.31354332312403e-17	-1.10336864020811e-16	-1\\
0	-0	0\\
0	-0	1\\
0	-0	0\\
0.0314403815052892	-0.0558255810464953	0.997945392750336\\
0	-0	0\\
0.0627515677390326	-0.111421762805921	0.991790013823246\\
0	-0	0\\
0.0938048943207671	-0.166560088642084	0.981559156991065\\
0	-0	0\\
0.124472756470651	-0.221013983346989	0.967294863039029\\
0	-0	0\\
0.154629133364874	-0.274559684186972	0.949055747010669\\
0	-0	0\\
0.184150105982256	-0.326977160391762	0.926916757346022\\
0	-0	0\\
0.212914366314083	-0.378051017308122	0.900968867902419\\
0	-0	0\\
0.240803715844737	-0.427571381502673	0.871318704123389\\
0	-0	0\\
0.267703551254751	-0.475334763176857	0.838088104891841\\
0	-0	0\\
0.293503335350426	-0.521144892350159	0.801413621867957\\
0	-0	0\\
0.318097051284879	-0.564813525375566	0.761445958369135\\
0	-0	0\\
0.341383638203998	-0.606161218473084	0.718349350097728\\
0	-0	0\\
0.363267406527177	-0.645018065102722	0.672300890261317\\
0	-0	0\\
0.383658431156321	-0.681224394146912	0.623489801858734\\
0	-0	0\\
0.402472920997369	-0.714631426033378	0.57211666012217\\
0	-0	0\\
0.419633563275867	-0.745101884102313	0.518392568310525\\
0	-0	0\\
0.435069841231748	-0.772510558705618	0.462538290240835\\
0	-0	0\\
0.448718323887819	-0.796744821720207	0.404783343122394\\
0	-0	0\\
0.460522926701257	-0.817705089361119	0.345365054421308\\
0	-0	0\\
0.470435142027021	-0.835305231392655	0.284527586631032\\
0	-0	0\\
0.478414238446175	-0.849472925055988	0.222520933956314\\
0	-0	0\\
0.484427428140021	-0.860149952258895	0.15959989503338\\
0	-0	0\\
0.488450001622282	-0.867292438806384	0.0960230259076819\\
0	-0	0\\
0.49046542927568	-0.870871034689173	0.0320515775716553\\
0	-0	-0\\
0.49046542927568	-0.870871034689173	-0.0320515775716552\\
0	-0	-0\\
0.488450001622282	-0.867292438806384	-0.0960230259076818\\
0	-0	-0\\
0.484427428140021	-0.860149952258895	-0.159599895033379\\
0	-0	-0\\
0.478414238446175	-0.849472925055988	-0.222520933956314\\
0	-0	-0\\
0.470435142027021	-0.835305231392655	-0.284527586631032\\
0	-0	-0\\
0.460522926701257	-0.817705089361119	-0.345365054421307\\
0	-0	-0\\
0.448718323887819	-0.796744821720207	-0.404783343122394\\
0	-0	-0\\
0.435069841231748	-0.772510558705618	-0.462538290240835\\
0	-0	-0\\
0.419633563275867	-0.745101884102313	-0.518392568310525\\
0	-0	-0\\
0.402472920997369	-0.714631426033378	-0.57211666012217\\
0	-0	-0\\
0.383658431156321	-0.681224394146912	-0.623489801858733\\
0	-0	-0\\
0.363267406527177	-0.645018065102722	-0.672300890261317\\
0	-0	-0\\
0.341383638203999	-0.606161218473084	-0.718349350097727\\
0	-0	-0\\
0.318097051284879	-0.564813525375567	-0.761445958369134\\
0	-0	-0\\
0.293503335350426	-0.521144892350159	-0.801413621867957\\
0	-0	-0\\
0.267703551254751	-0.475334763176857	-0.838088104891841\\
0	-0	-0\\
0.240803715844738	-0.427571381502673	-0.871318704123389\\
0	-0	-0\\
0.212914366314083	-0.378051017308122	-0.900968867902419\\
0	-0	-0\\
0.184150105982256	-0.326977160391762	-0.926916757346022\\
0	-0	-0\\
0.154629133364875	-0.274559684186972	-0.949055747010669\\
0	-0	-0\\
0.124472756470651	-0.221013983346989	-0.967294863039029\\
0	-0	-0\\
0.0938048943207672	-0.166560088642084	-0.981559156991065\\
0	-0	-0\\
0.0627515677390326	-0.111421762805921	-0.991790013823246\\
0	-0	-0\\
0.0314403815052894	-0.0558255810464955	-0.997945392750336\\
0	-0	-0\\
6.00955679347047e-17	-1.06705766204193e-16	-1\\
0	-0	0\\
0	-0	1\\
0	-0	0\\
0.0349525411277163	-0.053696489243639	0.997945392750336\\
0	-0	0\\
0.0697614547666423	-0.107172328095115	0.991790013823246\\
0	-0	0\\
0.104283703624147	-0.160207772862056	0.981559156991065\\
0	-0	0\\
0.138377428374676	-0.212584889525847	0.967294863039029\\
0	-0	0\\
0.171902530590148	-0.264088449279261	0.949055747010669\\
0	-0	0\\
0.204721248434448	-0.314506812947792	0.926916757346022\\
0	-0	0\\
0.236698722756361	-0.36363280066042	0.900968867902419\\
0	-0	0\\
0.267703551254751	-0.411264543196144	0.871318704123389\\
0	-0	0\\
0.297608328438803	-0.457206311507906	0.838088104891841\\
0	-0	0\\
0.326290169164514	-0.501269321015237	0.801413621867957\\
0	-0	0\\
0.353631213596107	-0.543272507360583	0.761445958369135\\
0	-0	0\\
0.379519111517375	-0.583043270441598	0.718349350097728\\
0	-0	0\\
0.403847484002825	-0.620418183661979	0.672300890261317\\
0	-0	0\\
0.426516360551494	-0.655243665486409	0.623489801858734\\
0	-0	0\\
0.447432589887183	-0.687376610540031	0.57211666012217\\
0	-0	0\\
0.466510222737037	-0.716684977659123	0.518392568310525\\
0	-0	0\\
0.483670865015536	-0.743048332476568	0.462538290240835\\
0	-0	0\\
0.49884399996261	-0.766358342312499	0.404783343122394\\
0	-0	0\\
0.511967277912135	-0.78651922133652	0.345365054421308\\
0	-0	0\\
0.522986772500083	-0.803448124172224	0.284527586631032\\
0	-0	0\\
0.531857202259518	-0.817075486326623	0.222520933956314\\
0	-0	0\\
0.538542116691855	-0.827345310045564	0.15959989503338\\
0	-0	0\\
0.543014046049784	-0.834215394420515	0.0960230259076819\\
0	-0	0\\
0.545254614216347	-0.837657508801151	0.0320515775716553\\
0	-0	-0\\
0.545254614216347	-0.837657508801151	-0.0320515775716552\\
0	-0	-0\\
0.543014046049784	-0.834215394420515	-0.0960230259076818\\
0	-0	-0\\
0.538542116691855	-0.827345310045564	-0.159599895033379\\
0	-0	-0\\
0.531857202259518	-0.817075486326623	-0.222520933956314\\
0	-0	-0\\
0.522986772500084	-0.803448124172225	-0.284527586631032\\
0	-0	-0\\
0.511967277912136	-0.78651922133652	-0.345365054421307\\
0	-0	-0\\
0.49884399996261	-0.766358342312499	-0.404783343122394\\
0	-0	-0\\
0.483670865015536	-0.743048332476568	-0.462538290240835\\
0	-0	-0\\
0.466510222737037	-0.716684977659123	-0.518392568310525\\
0	-0	-0\\
0.447432589887183	-0.687376610540031	-0.57211666012217\\
0	-0	-0\\
0.426516360551494	-0.655243665486409	-0.623489801858733\\
0	-0	-0\\
0.403847484002825	-0.620418183661979	-0.672300890261317\\
0	-0	-0\\
0.379519111517376	-0.583043270441598	-0.718349350097727\\
0	-0	-0\\
0.353631213596107	-0.543272507360584	-0.761445958369134\\
0	-0	-0\\
0.326290169164514	-0.501269321015237	-0.801413621867957\\
0	-0	-0\\
0.297608328438803	-0.457206311507906	-0.838088104891841\\
0	-0	-0\\
0.267703551254751	-0.411264543196144	-0.871318704123389\\
0	-0	-0\\
0.236698722756361	-0.36363280066042	-0.900968867902419\\
0	-0	-0\\
0.204721248434448	-0.314506812947792	-0.926916757346022\\
0	-0	-0\\
0.171902530590148	-0.264088449279262	-0.949055747010669\\
0	-0	-0\\
0.138377428374676	-0.212584889525847	-0.967294863039029\\
0	-0	-0\\
0.104283703624147	-0.160207772862056	-0.981559156991065\\
0	-0	-0\\
0.0697614547666424	-0.107172328095115	-0.991790013823246\\
0	-0	-0\\
0.0349525411277165	-0.0536964892436393	-0.997945392750336\\
0	-0	-0\\
6.68087570590666e-17	-1.02636191505926e-16	-1\\
0	-0	0\\
0	-0	1\\
0	-0	0\\
0.0383210732613531	-0.0513467470486199	0.997945392750336\\
0	-0	0\\
0.0764846770128309	-0.102482499299774	0.991790013823246\\
0	-0	0\\
0.114333988820551	-0.153197128978879	0.981559156991065\\
0	-0	0\\
0.151713477743644	-0.203282238794328	0.967294863039029\\
0	-0	0\\
0.188469543444249	-0.252532018286668	0.949055747010669\\
0	-0	0\\
0.224451147364252	-0.300744089547919	0.926916757346022\\
0	-0	0\\
0.259510433375115	-0.347720338835814	0.900968867902419\\
0	-0	0\\
0.293503335350426	-0.393267730665653	0.871318704123389\\
0	-0	0\\
0.326290169164514	-0.437199101034524	0.838088104891841\\
0	-0	0\\
0.357736206684484	-0.47933392651833	0.801413621867957\\
0	-0	0\\
0.387712229397011	-0.519499066081268	0.761445958369135\\
0	-0	0\\
0.416095059394933	-0.557529472549478	0.718349350097728\\
0	-0	0\\
0.442768065541692	-0.593268870825285	0.672300890261317\\
0	-0	0\\
0.467621642733687	-0.626570400055098	0.623489801858734\\
0	-0	0\\
0.490553662291162	-0.657297217112154	0.57211666012217\\
0	-0	0\\
0.511469891626852	-0.685323058914286	0.518392568310525\\
0	-0	0\\
0.530284381467899	-0.710532761266004	0.462538290240835\\
0	-0	0\\
0.546919819039852	-0.732822732092881	0.404783343122394\\
0	-0	0\\
0.561307845761437	-0.752101377123605	0.345365054421308\\
0	-0	0\\
0.573389338144632	-0.768289476270488	0.284527586631032\\
0	-0	0\\
0.583114650745764	-0.7813205091618	0.222520933956314\\
0	-0	0\\
0.590443820169281	-0.791140928488243	0.15959989503338\\
0	-0	0\\
0.595346729285921	-0.797710380040331	0.0960230259076819\\
0	-0	0\\
0.597803230990452	-0.801001868532493	0.0320515775716553\\
0	-0	-0\\
0.597803230990452	-0.801001868532493	-0.0320515775716552\\
0	-0	-0\\
0.595346729285921	-0.797710380040331	-0.0960230259076818\\
0	-0	-0\\
0.590443820169281	-0.791140928488243	-0.159599895033379\\
0	-0	-0\\
0.583114650745764	-0.7813205091618	-0.222520933956314\\
0	-0	-0\\
0.573389338144632	-0.768289476270488	-0.284527586631032\\
0	-0	-0\\
0.561307845761437	-0.752101377123605	-0.345365054421307\\
0	-0	-0\\
0.546919819039852	-0.732822732092881	-0.404783343122394\\
0	-0	-0\\
0.530284381467899	-0.710532761266004	-0.462538290240835\\
0	-0	-0\\
0.511469891626852	-0.685323058914286	-0.518392568310525\\
0	-0	-0\\
0.490553662291162	-0.657297217112154	-0.57211666012217\\
0	-0	-0\\
0.467621642733687	-0.626570400055098	-0.623489801858733\\
0	-0	-0\\
0.442768065541692	-0.593268870825285	-0.672300890261317\\
0	-0	-0\\
0.416095059394934	-0.557529472549478	-0.718349350097727\\
0	-0	-0\\
0.387712229397011	-0.519499066081269	-0.761445958369134\\
0	-0	-0\\
0.357736206684484	-0.47933392651833	-0.801413621867957\\
0	-0	-0\\
0.326290169164514	-0.437199101034524	-0.838088104891841\\
0	-0	-0\\
0.293503335350426	-0.393267730665653	-0.871318704123389\\
0	-0	-0\\
0.259510433375115	-0.347720338835814	-0.900968867902419\\
0	-0	-0\\
0.224451147364252	-0.30074408954792	-0.926916757346022\\
0	-0	-0\\
0.18846954344425	-0.252532018286668	-0.949055747010669\\
0	-0	-0\\
0.151713477743644	-0.203282238794328	-0.967294863039029\\
0	-0	-0\\
0.114333988820551	-0.153197128978879	-0.981559156991065\\
0	-0	-0\\
0.0764846770128309	-0.102482499299774	-0.991790013823246\\
0	-0	-0\\
0.0383210732613533	-0.0513467470486201	-0.997945392750336\\
0	-0	-0\\
7.32474146702393e-17	-9.8144862681368e-17	-1\\
0	-0	0\\
0	-0	1\\
0	-0	0\\
0.0415321358851145	-0.0487860100561352	0.997945392750336\\
0	-0	0\\
0.0828936073152619	-0.0973715479323834	0.991790013823246\\
0	-0	0\\
0.123914451132328	-0.145556965232046	0.981559156991065\\
0	-0	0\\
0.164426103890125	-0.193144257739699	0.967294863039029\\
0	-0	0\\
0.204262094517748	-0.239937879062986	0.949055747010669\\
0	-0	0\\
0.243258728384917	-0.28574554417469	0.926916757346022\\
0	-0	0\\
0.281255759958318	-0.330379019553153	0.900968867902419\\
0	-0	0\\
0.318097051284879	-0.373654896674194	0.871318704123389\\
0	-0	0\\
0.353631213596107	-0.415395345676077	0.838088104891841\\
0	-0	0\\
0.387712229397011	-0.455428846100555	0.801413621867957\\
0	-0	0\\
0.42020005248331	-0.493590891707225	0.761445958369135\\
0	-0	0\\
0.450961183421327	-0.529724666464955	0.718349350097728\\
0	-0	0\\
0.479869218125796	-0.563681688942597	0.672300890261317\\
0	-0	0\\
0.50680536728136	-0.59532242245103	0.623489801858734\\
0	-0	0\\
0.531658944473356	-0.624516848429353	0.57211666012217\\
0	-0	0\\
0.554327821022024	-0.651145000719035	0.518392568310525\\
0	-0	0\\
0.574718845651168	-0.675097458530599	0.462538290240835\\
0	-0	0\\
0.592748227266726	-0.696275796077109	0.404783343122394\\
0	-0	0\\
0.608341879272349	-0.714592987026849	0.345365054421308\\
0	-0	0\\
0.621435724007117	-0.72997376211318	0.284527586631032\\
0	-0	0\\
0.631975956054396	-0.742354918432108	0.222520933956314\\
0	-0	0\\
0.63991926333983	-0.751685579156567	0.15959989503338\\
0	-0	0\\
0.645233005109949	-0.757927402600221	0.0960230259076819\\
0	-0	0\\
0.647895346060025	-0.761054739771673	0.0320515775716553\\
0	-0	-0\\
0.647895346060025	-0.761054739771673	-0.0320515775716552\\
0	-0	-0\\
0.645233005109949	-0.757927402600221	-0.0960230259076818\\
0	-0	-0\\
0.63991926333983	-0.751685579156567	-0.159599895033379\\
0	-0	-0\\
0.631975956054396	-0.742354918432108	-0.222520933956314\\
0	-0	-0\\
0.621435724007117	-0.729973762113181	-0.284527586631032\\
0	-0	-0\\
0.608341879272349	-0.714592987026849	-0.345365054421307\\
0	-0	-0\\
0.592748227266726	-0.696275796077109	-0.404783343122394\\
0	-0	-0\\
0.574718845651168	-0.675097458530599	-0.462538290240835\\
0	-0	-0\\
0.554327821022024	-0.651145000719035	-0.518392568310525\\
0	-0	-0\\
0.531658944473356	-0.624516848429353	-0.57211666012217\\
0	-0	-0\\
0.50680536728136	-0.59532242245103	-0.623489801858733\\
0	-0	-0\\
0.479869218125796	-0.563681688942597	-0.672300890261317\\
0	-0	-0\\
0.450961183421327	-0.529724666464956	-0.718349350097727\\
0	-0	-0\\
0.420200052483311	-0.493590891707225	-0.761445958369134\\
0	-0	-0\\
0.387712229397011	-0.455428846100555	-0.801413621867957\\
0	-0	-0\\
0.353631213596107	-0.415395345676077	-0.838088104891841\\
0	-0	-0\\
0.318097051284879	-0.373654896674194	-0.871318704123389\\
0	-0	-0\\
0.281255759958318	-0.330379019553153	-0.900968867902419\\
0	-0	-0\\
0.243258728384917	-0.28574554417469	-0.926916757346022\\
0	-0	-0\\
0.204262094517749	-0.239937879062986	-0.949055747010669\\
0	-0	-0\\
0.164426103890125	-0.193144257739699	-0.967294863039029\\
0	-0	-0\\
0.123914451132328	-0.145556965232046	-0.981559156991065\\
0	-0	-0\\
0.082893607315262	-0.0973715479323835	-0.991790013823246\\
0	-0	-0\\
0.0415321358851148	-0.0487860100561354	-0.997945392750336\\
0	-0	-0\\
7.93850829430108e-17	-9.32502355640449e-17	-1\\
0	-0	0\\
0	-0	1\\
0	-0	0\\
0.0445725340539088	-0.0460248008837636	0.997945392750336\\
0	-0	0\\
0.0889619100046116	-0.0918604759884069	0.991790013823246\\
0	-0	0\\
0.132985722384836	-0.137318676693204	0.981559156991065\\
0	-0	0\\
0.176463067906432	-0.182212605500704	0.967294863039029\\
0	-0	0\\
0.219215288830792	-0.226357783627721	0.949055747010669\\
0	-0	0\\
0.261066707111814	-0.269572809068219	0.926916757346022\\
0	-0	0\\
0.301845346294681	-0.31168010201307	0.900968867902419\\
0	-0	0\\
0.341383638203999	-0.352506634563577	0.871318704123389\\
0	-0	0\\
0.379519111517376	-0.391884641740226	0.838088104891841\\
0	-0	0\\
0.416095059394934	-0.429652310864972	0.801413621867957\\
0	-0	0\\
0.450961183421327	-0.465654446484243	0.761445958369135\\
0	-0	0\\
0.483974211214172	-0.499743108100344	0.718349350097728\\
0	-0	0\\
0.514998485160996	-0.5317782180907	0.672300890261317\\
0	-0	0\\
0.543906519865464	-0.561628137316852	0.623489801858734\\
0	-0	0\\
0.570579526012222	-0.589170206057911	0.57211666012217\\
0	-0	0\\
0.594907898497672	-0.614291248045665	0.518392568310525\\
0	-0	0\\
0.616791666820851	-0.636888035530141	0.462538290240835\\
0	-0	0\\
0.636140905883665	-0.656867713464567	0.404783343122394\\
0	-0	0\\
0.652876105512406	-0.674148181066685	0.345365054421308\\
0	-0	0\\
0.666928497182112	-0.688658429188469	0.284527586631032\\
0	-0	0\\
0.678240336601183	-0.700338832107947	0.222520933956314\\
0	-0	0\\
0.686765140995063	-0.709141392544086	0.15959989503338\\
0	-0	0\\
0.692467880113934	-0.715029938887909	0.0960230259076819\\
0	-0	0\\
0.695325120179522	-0.717980273839401	0.0320515775716553\\
0	-0	-0\\
0.695325120179522	-0.717980273839401	-0.0320515775716552\\
0	-0	-0\\
0.692467880113934	-0.715029938887909	-0.0960230259076818\\
0	-0	-0\\
0.686765140995063	-0.709141392544086	-0.159599895033379\\
0	-0	-0\\
0.678240336601183	-0.700338832107947	-0.222520933956314\\
0	-0	-0\\
0.666928497182112	-0.688658429188469	-0.284527586631032\\
0	-0	-0\\
0.652876105512407	-0.674148181066685	-0.345365054421307\\
0	-0	-0\\
0.636140905883665	-0.656867713464567	-0.404783343122394\\
0	-0	-0\\
0.616791666820851	-0.636888035530141	-0.462538290240835\\
0	-0	-0\\
0.594907898497672	-0.614291248045665	-0.518392568310525\\
0	-0	-0\\
0.570579526012222	-0.589170206057911	-0.57211666012217\\
0	-0	-0\\
0.543906519865464	-0.561628137316852	-0.623489801858733\\
0	-0	-0\\
0.514998485160996	-0.5317782180907	-0.672300890261317\\
0	-0	-0\\
0.483974211214173	-0.499743108100344	-0.718349350097727\\
0	-0	-0\\
0.450961183421328	-0.465654446484243	-0.761445958369134\\
0	-0	-0\\
0.416095059394934	-0.429652310864972	-0.801413621867957\\
0	-0	-0\\
0.379519111517376	-0.391884641740226	-0.838088104891841\\
0	-0	-0\\
0.341383638203999	-0.352506634563577	-0.871318704123389\\
0	-0	-0\\
0.301845346294681	-0.31168010201307	-0.900968867902419\\
0	-0	-0\\
0.261066707111814	-0.269572809068219	-0.926916757346022\\
0	-0	-0\\
0.219215288830792	-0.226357783627721	-0.949055747010669\\
0	-0	-0\\
0.176463067906432	-0.182212605500704	-0.967294863039029\\
0	-0	-0\\
0.132985722384836	-0.137318676693204	-0.981559156991065\\
0	-0	-0\\
0.0889619100046116	-0.0918604759884069	-0.991790013823246\\
0	-0	-0\\
0.044572534053909	-0.0460248008837638	-0.997945392750336\\
0	-0	-0\\
8.51965408819226e-17	-8.79724232266763e-17	-1\\
0	-0	0\\
0	-0	1\\
0	-0	0\\
0.047429774119497	-0.0430744659322717	0.997945392750336\\
0	-0	0\\
0.0946646491234824	-0.0859719296445837	0.991790013823246\\
0	-0	0\\
0.141510526778716	-0.128516116257065	0.981559156991065\\
0	-0	0\\
0.187774907325503	-0.170532202581226	0.967294863039029\\
0	-0	0\\
0.233267680500498	-0.211847535505937	0.949055747010669\\
0	-0	0\\
0.277801906740556	-0.252292341466101	0.926916757346022\\
0	-0	0\\
0.321194585357495	-0.291700424078643	0.900968867902419\\
0	-0	0\\
0.363267406527177	-0.329909847079101	0.871318704123389\\
0	-0	0\\
0.403847484002825	-0.366763599752471	0.838088104891841\\
0	-0	0\\
0.442768065541692	-0.402110242123913	0.801413621867957\\
0	-0	0\\
0.479869218125795	-0.435804527258091	0.761445958369135\\
0	-0	0\\
0.514998485160996	-0.467707998109987	0.718349350097728\\
0	-0	0\\
0.548011512953841	-0.497689556474599	0.672300890261317\\
0	-0	0\\
0.578772643891858	-0.525626001697582	0.623489801858734\\
0	-0	0\\
0.60715547388978	-0.551402536933165	0.57211666012217\\
0	-0	0\\
0.633043371811049	-0.574913240869016	0.518392568310525\\
0	-0	0\\
0.656329958730169	-0.596061502979634	0.462538290240835\\
0	-0	0\\
0.676919545066531	-0.614760420519717	0.404783343122394\\
0	-0	0\\
0.694727523793429	-0.630933155626187	0.345365054421308\\
0	-0	0\\
0.709680718106472	-0.644513251061452	0.284527586631032\\
0	-0	0\\
0.72171768212278	-0.655444903300447	0.222520933956314\\
0	-0	0\\
0.730788953375287	-0.663683191839289	0.15959989503338\\
0	-0	0\\
0.736857256064637	-0.669194263783266	0.0960230259076819\\
0	-0	0\\
0.739897654233431	-0.671955472955637	0.0320515775716553\\
0	-0	-0\\
0.739897654233431	-0.671955472955637	-0.0320515775716552\\
0	-0	-0\\
0.736857256064637	-0.669194263783266	-0.0960230259076818\\
0	-0	-0\\
0.730788953375287	-0.663683191839289	-0.159599895033379\\
0	-0	-0\\
0.72171768212278	-0.655444903300447	-0.222520933956314\\
0	-0	-0\\
0.709680718106472	-0.644513251061452	-0.284527586631032\\
0	-0	-0\\
0.694727523793429	-0.630933155626187	-0.345365054421307\\
0	-0	-0\\
0.676919545066531	-0.614760420519717	-0.404783343122394\\
0	-0	-0\\
0.656329958730169	-0.596061502979634	-0.462538290240835\\
0	-0	-0\\
0.633043371811049	-0.574913240869017	-0.518392568310525\\
0	-0	-0\\
0.60715547388978	-0.551402536933165	-0.57211666012217\\
0	-0	-0\\
0.578772643891858	-0.525626001697582	-0.623489801858733\\
0	-0	-0\\
0.548011512953841	-0.497689556474599	-0.672300890261317\\
0	-0	-0\\
0.514998485160996	-0.467707998109988	-0.718349350097727\\
0	-0	-0\\
0.479869218125796	-0.435804527258091	-0.761445958369134\\
0	-0	-0\\
0.442768065541692	-0.402110242123913	-0.801413621867957\\
0	-0	-0\\
0.403847484002825	-0.366763599752471	-0.838088104891841\\
0	-0	-0\\
0.363267406527177	-0.329909847079102	-0.871318704123389\\
0	-0	-0\\
0.321194585357495	-0.291700424078643	-0.900968867902419\\
0	-0	-0\\
0.277801906740556	-0.252292341466101	-0.926916757346022\\
0	-0	-0\\
0.233267680500498	-0.211847535505938	-0.949055747010669\\
0	-0	-0\\
0.187774907325503	-0.170532202581226	-0.967294863039029\\
0	-0	-0\\
0.141510526778716	-0.128516116257065	-0.981559156991065\\
0	-0	-0\\
0.0946646491234824	-0.0859719296445838	-0.991790013823246\\
0	-0	-0\\
0.0474297741194973	-0.0430744659322719	-0.997945392750336\\
0	-0	-0\\
9.06579079597499e-17	-8.23331133322438e-17	-1\\
0	-0	0\\
0	-0	1\\
0	-0	0\\
0.0500921150695736	-0.0399471287608202	0.997945392750336\\
0	-0	0\\
0.0999783908936013	-0.0797301062009299	0.991790013823246\\
0	-0	0\\
0.14945383406415	-0.119185455532606	0.981559156991065\\
0	-0	0\\
0.198315139372781	-0.158151046262298	0.967294863039029\\
0	-0	0\\
0.246361525235266	-0.196466760419606	0.949055747010669\\
0	-0	0\\
0.293395558746178	-0.233975150516361	0.926916757346022\\
0	-0	0\\
0.339223966973052	-0.270522086532133	0.900968867902419\\
0	-0	0\\
0.383658431156321	-0.305957389267538	0.871318704123389\\
0	-0	0\\
0.426516360551494	-0.340135447462789	0.838088104891841\\
0	-0	0\\
0.467621642733687	-0.37291581614559	0.801413621867957\\
0	-0	0\\
0.50680536728136	-0.404163793749658	0.761445958369135\\
0	-0	0\\
0.543906519865464	-0.433750975632346	0.718349350097728\\
0	-0	0\\
0.578772643891858	-0.461555781716869	0.672300890261317\\
0	-0	0\\
0.611260466978157	-0.487463956090912	0.623489801858734\\
0	-0	0\\
0.641236489690684	-0.511369036508687	0.57211666012217\\
0	-0	0\\
0.668577534122277	-0.533172791867133	0.518392568310525\\
0	-0	0\\
0.69317125005673	-0.552785625858592	0.462538290240835\\
0	-0	0\\
0.714916576639932	-0.570126945141254	0.404783343122394\\
0	-0	0\\
0.733724157660597	-0.585125490514483	0.345365054421308\\
0	-0	0\\
0.749516708734096	-0.597719629738165	0.284527586631032\\
0	-0	0\\
0.762229334880576	-0.607857610792794	0.222520933956314\\
0	-0	0\\
0.771809797192353	-0.615497774539627	0.15959989503338\\
0	-0	0\\
0.778218727494784	-0.620608725907018	0.0960230259076819\\
0	-0	0\\
0.781429790118546	-0.623169462899502	0.0320515775716553\\
0	-0	-0\\
0.781429790118546	-0.623169462899502	-0.0320515775716552\\
0	-0	-0\\
0.778218727494784	-0.620608725907018	-0.0960230259076818\\
0	-0	-0\\
0.771809797192353	-0.615497774539627	-0.159599895033379\\
0	-0	-0\\
0.762229334880576	-0.607857610792794	-0.222520933956314\\
0	-0	-0\\
0.749516708734096	-0.597719629738165	-0.284527586631032\\
0	-0	-0\\
0.733724157660597	-0.585125490514483	-0.345365054421307\\
0	-0	-0\\
0.714916576639932	-0.570126945141254	-0.404783343122394\\
0	-0	-0\\
0.69317125005673	-0.552785625858592	-0.462538290240835\\
0	-0	-0\\
0.668577534122277	-0.533172791867133	-0.518392568310525\\
0	-0	-0\\
0.641236489690684	-0.511369036508687	-0.57211666012217\\
0	-0	-0\\
0.611260466978157	-0.487463956090912	-0.623489801858733\\
0	-0	-0\\
0.578772643891858	-0.461555781716869	-0.672300890261317\\
0	-0	-0\\
0.543906519865464	-0.433750975632346	-0.718349350097727\\
0	-0	-0\\
0.50680536728136	-0.404163793749658	-0.761445958369134\\
0	-0	-0\\
0.467621642733687	-0.37291581614559	-0.801413621867957\\
0	-0	-0\\
0.426516360551494	-0.340135447462789	-0.838088104891841\\
0	-0	-0\\
0.383658431156321	-0.305957389267538	-0.871318704123389\\
0	-0	-0\\
0.339223966973052	-0.270522086532133	-0.900968867902419\\
0	-0	-0\\
0.293395558746179	-0.233975150516362	-0.926916757346022\\
0	-0	-0\\
0.246361525235267	-0.196466760419606	-0.949055747010669\\
0	-0	-0\\
0.198315139372781	-0.158151046262298	-0.967294863039029\\
0	-0	-0\\
0.14945383406415	-0.119185455532606	-0.981559156991065\\
0	-0	-0\\
0.0999783908936013	-0.0797301062009299	-0.991790013823246\\
0	-0	-0\\
0.0500921150695738	-0.0399471287608204	-0.997945392750336\\
0	-0	-0\\
9.57467422477103e-17	-7.63554790147315e-17	-1\\
0	-0	0\\
0	-0	1\\
0	-0	0\\
0.0525486167741042	-0.0366556402686582	0.997945392750336\\
0	-0	0\\
0.104881300010241	-0.0731606546488422	0.991790013823246\\
0	-0	0\\
0.156783003487667	-0.109365036206163	0.981559156991065\\
0	-0	0\\
0.208040451973913	-0.145120013370986	0.967294863039029\\
0	-0	0\\
0.258443017618462	-0.180278661272722	0.949055747010669\\
0	-0	0\\
0.307783585467763	-0.214696505485637	0.926916757346022\\
0	-0	0\\
0.355859404545005	-0.248232115705256	0.900968867902419\\
0	-0	0\\
0.402472920997369	-0.28074768691582	0.871318704123389\\
0	-0	0\\
0.447432589887183	-0.312109605660657	0.838088104891841\\
0	-0	0\\
0.490553662291162	-0.342188999088534	0.801413621867957\\
0	-0	0\\
0.531658944473356	-0.370862264519845	0.761445958369135\\
0	-0	0\\
0.570579526012222	-0.398011577356539	0.718349350097728\\
0	-0	0\\
0.60715547388978	-0.423525375248659	0.672300890261317\\
0	-0	0\\
0.641236489690684	-0.447298816527974	0.623489801858734\\
0	-0	0\\
0.672682527210654	-0.46923421102488	0.57211666012217\\
0	-0	0\\
0.701364367936365	-0.489241421498263	0.518392568310525\\
0	-0	0\\
0.727164152032041	-0.507238234028753	0.462538290240835\\
0	-0	0\\
0.749975862650795	-0.523150695853359	0.404783343122394\\
0	-0	0\\
0.7697057615806	-0.536913419253232	0.345365054421308\\
0	-0	0\\
0.786272774434701	-0.548469850245825	0.284527586631032\\
0	-0	0\\
0.799608823803669	-0.557772500977345	0.222520933956314\\
0	-0	0\\
0.809659109000073	-0.564783144860522	0.15959989503338\\
0	-0	0\\
0.816382331246262	-0.569472973655863	0.0960230259076819\\
0	-0	0\\
0.819750863379899	-0.571822715850882	0.0320515775716553\\
0	-0	-0\\
0.819750863379899	-0.571822715850882	-0.0320515775716552\\
0	-0	-0\\
0.816382331246262	-0.569472973655863	-0.0960230259076818\\
0	-0	-0\\
0.809659109000073	-0.564783144860522	-0.159599895033379\\
0	-0	-0\\
0.799608823803669	-0.557772500977345	-0.222520933956314\\
0	-0	-0\\
0.786272774434701	-0.548469850245825	-0.284527586631032\\
0	-0	-0\\
0.7697057615806	-0.536913419253232	-0.345365054421307\\
0	-0	-0\\
0.749975862650795	-0.523150695853359	-0.404783343122394\\
0	-0	-0\\
0.727164152032041	-0.507238234028753	-0.462538290240835\\
0	-0	-0\\
0.701364367936365	-0.489241421498263	-0.518392568310525\\
0	-0	-0\\
0.672682527210654	-0.46923421102488	-0.57211666012217\\
0	-0	-0\\
0.641236489690685	-0.447298816527974	-0.623489801858733\\
0	-0	-0\\
0.60715547388978	-0.423525375248659	-0.672300890261317\\
0	-0	-0\\
0.570579526012222	-0.398011577356539	-0.718349350097727\\
0	-0	-0\\
0.531658944473356	-0.370862264519846	-0.761445958369134\\
0	-0	-0\\
0.490553662291162	-0.342188999088534	-0.801413621867957\\
0	-0	-0\\
0.447432589887183	-0.312109605660657	-0.838088104891841\\
0	-0	-0\\
0.402472920997369	-0.28074768691582	-0.871318704123389\\
0	-0	-0\\
0.355859404545005	-0.248232115705256	-0.900968867902419\\
0	-0	-0\\
0.307783585467763	-0.214696505485638	-0.926916757346022\\
0	-0	-0\\
0.258443017618463	-0.180278661272723	-0.949055747010669\\
0	-0	-0\\
0.208040451973913	-0.145120013370986	-0.967294863039029\\
0	-0	-0\\
0.156783003487667	-0.109365036206163	-0.981559156991065\\
0	-0	-0\\
0.104881300010241	-0.0731606546488423	-0.991790013823246\\
0	-0	-0\\
0.0525486167741045	-0.0366556402686583	-0.997945392750336\\
0	-0	-0\\
1.00442132634163e-16	-7.00640836557489e-17	-1\\
0	-0	0\\
0	-0	1\\
0	-0	0\\
0.0547891849406672	-0.0332135258880221	0.997945392750336\\
0	-0	0\\
0.10935322936817	-0.0662905702738913	0.991790013823246\\
0	-0	0\\
0.163467917920005	-0.0990952124872224	0.981559156991065\\
0	-0	0\\
0.216910881733348	-0.131492651216587	0.967294863039029\\
0	-0	0\\
0.26946251220641	-0.163349758437018	0.949055747010669\\
0	-0	0\\
0.320906863417289	-0.194535626461617	0.926916757346022\\
0	-0	0\\
0.371032539492079	-0.224922105869325	0.900968867902419\\
0	-0	0\\
0.419633563275867	-0.254384332098375	0.871318704123389\\
0	-0	0\\
0.466510222737037	-0.282801238541565	0.838088104891841\\
0	-0	0\\
0.511469891626852	-0.310056054034912	0.801413621867957\\
0	-0	0\\
0.554327821022024	-0.336036782695415	0.761445958369135\\
0	-0	0\\
0.594907898497672	-0.360636664136158	0.718349350097728\\
0	-0	0\\
0.633043371811049	-0.383754612167644	0.672300890261317\\
0	-0	0\\
0.668577534122277	-0.405295630182627	0.623489801858734\\
0	-0	0\\
0.701364367936365	-0.42517120151755	0.57211666012217\\
0	-0	0\\
0.731269145120418	-0.4432996531865	0.518392568310525\\
0	-0	0\\
0.758168980530431	-0.45960649149303	0.462538290240835\\
0	-0	0\\
0.781953336972708	-0.474024708140731	0.404783343122394\\
0	-0	0\\
0.802524479424899	-0.486495055584701	0.345365054421308\\
0	-0	0\\
0.819797876650173	-0.496966290492411	0.284527586631032\\
0	-0	0\\
0.833702548554198	-0.505395384313553	0.222520933956314\\
0	-0	0\\
0.844181357857578	-0.511747700093581	0.15959989503338\\
0	-0	0\\
0.851191244885188	-0.515997134804387	0.0960230259076819\\
0	-0	0\\
0.854703404507615	-0.518126226607243	0.0320515775716553\\
0	-0	-0\\
0.854703404507615	-0.518126226607243	-0.0320515775716552\\
0	-0	-0\\
0.851191244885188	-0.515997134804387	-0.0960230259076818\\
0	-0	-0\\
0.844181357857578	-0.511747700093581	-0.159599895033379\\
0	-0	-0\\
0.833702548554198	-0.505395384313553	-0.222520933956314\\
0	-0	-0\\
0.819797876650173	-0.496966290492411	-0.284527586631032\\
0	-0	-0\\
0.802524479424899	-0.486495055584701	-0.345365054421307\\
0	-0	-0\\
0.781953336972708	-0.474024708140731	-0.404783343122394\\
0	-0	-0\\
0.758168980530431	-0.45960649149303	-0.462538290240835\\
0	-0	-0\\
0.731269145120418	-0.4432996531865	-0.518392568310525\\
0	-0	-0\\
0.701364367936365	-0.42517120151755	-0.57211666012217\\
0	-0	-0\\
0.668577534122277	-0.405295630182627	-0.623489801858733\\
0	-0	-0\\
0.633043371811049	-0.383754612167644	-0.672300890261317\\
0	-0	-0\\
0.594907898497672	-0.360636664136158	-0.718349350097727\\
0	-0	-0\\
0.554327821022024	-0.336036782695415	-0.761445958369134\\
0	-0	-0\\
0.511469891626852	-0.310056054034912	-0.801413621867957\\
0	-0	-0\\
0.466510222737037	-0.282801238541565	-0.838088104891841\\
0	-0	-0\\
0.419633563275867	-0.254384332098375	-0.871318704123389\\
0	-0	-0\\
0.37103253949208	-0.224922105869325	-0.900968867902419\\
0	-0	-0\\
0.320906863417289	-0.194535626461617	-0.926916757346022\\
0	-0	-0\\
0.269462512206411	-0.163349758437018	-0.949055747010669\\
0	-0	-0\\
0.216910881733348	-0.131492651216587	-0.967294863039029\\
0	-0	-0\\
0.163467917920005	-0.0990952124872225	-0.981559156991065\\
0	-0	-0\\
0.10935322936817	-0.0662905702738913	-0.991790013823246\\
0	-0	-0\\
0.0547891849406675	-0.0332135258880222	-0.997945392750336\\
0	-0	-0\\
1.04724784752852e-16	-6.3484779948326e-17	-1\\
0	-0	0\\
0	-0	1\\
0	-0	0\\
0.0568046125940656	-0.0296349300052332	0.997945392750336\\
0	-0	0\\
0.113375802850431	-0.0591480837264023	0.991790013823246\\
0	-0	0\\
0.169481107613851	-0.0884181852843154	0.981559156991065\\
0	-0	0\\
0.224889978152501	-0.117324957553254	0.967294863039029\\
0	-0	0\\
0.279374727532174	-0.145749616405482	0.949055747010669\\
0	-0	0\\
0.332711466230726	-0.173575358820705	0.926916757346022\\
0	-0	0\\
0.384681022148151	-0.200687842854736	0.900968867902419\\
0	-0	0\\
0.435069841231748	-0.22697565749507	0.871318704123389\\
0	-0	0\\
0.483670865015536	-0.25233078047263	0.838088104891841\\
0	-0	0\\
0.530284381467899	-0.276649022148446	0.801413621867957\\
0	-0	0\\
0.574718845651168	-0.299830453651225	0.761445958369135\\
0	-0	0\\
0.616791666820851	-0.32177981750652	0.718349350097728\\
0	-0	0\\
0.656329958730169	-0.342406919070126	0.672300890261317\\
0	-0	0\\
0.69317125005673	-0.36162699715722	0.623489801858734\\
0	-0	0\\
0.727164152032041	-0.379361072344247	0.57211666012217\\
0	-0	0\\
0.758168980530431	-0.395536271512317	0.518392568310525\\
0	-0	0\\
0.786058330061085	-0.410086127298478	0.462538290240835\\
0	-0	0\\
0.810717597304535	-0.422950851224371	0.404783343122394\\
0	-0	0\\
0.832045452042281	-0.434077579379911	0.345365054421308\\
0	-0	0\\
0.849954253544396	-0.443420589652429	0.284527586631032\\
0	-0	0\\
0.864370410704082	-0.450941489608648	0.222520933956314\\
0	-0	0\\
0.875234684439313	-0.456609374257418	0.15959989503338\\
0	-0	0\\
0.882502431118931	-0.460400953044961	0.0960230259076819\\
0	-0	0\\
0.886143786012904	-0.462300645560748	0.0320515775716553\\
0	-0	-0\\
0.886143786012904	-0.462300645560748	-0.0320515775716552\\
0	-0	-0\\
0.882502431118931	-0.460400953044961	-0.0960230259076818\\
0	-0	-0\\
0.875234684439313	-0.456609374257418	-0.159599895033379\\
0	-0	-0\\
0.864370410704082	-0.450941489608648	-0.222520933956314\\
0	-0	-0\\
0.849954253544397	-0.443420589652429	-0.284527586631032\\
0	-0	-0\\
0.832045452042282	-0.434077579379911	-0.345365054421307\\
0	-0	-0\\
0.810717597304535	-0.422950851224371	-0.404783343122394\\
0	-0	-0\\
0.786058330061085	-0.410086127298478	-0.462538290240835\\
0	-0	-0\\
0.758168980530431	-0.395536271512317	-0.518392568310525\\
0	-0	-0\\
0.727164152032041	-0.379361072344247	-0.57211666012217\\
0	-0	-0\\
0.69317125005673	-0.36162699715722	-0.623489801858733\\
0	-0	-0\\
0.656329958730169	-0.342406919070126	-0.672300890261317\\
0	-0	-0\\
0.616791666820851	-0.32177981750652	-0.718349350097727\\
0	-0	-0\\
0.574718845651168	-0.299830453651225	-0.761445958369134\\
0	-0	-0\\
0.530284381467899	-0.276649022148446	-0.801413621867957\\
0	-0	-0\\
0.483670865015536	-0.25233078047263	-0.838088104891841\\
0	-0	-0\\
0.435069841231748	-0.22697565749507	-0.871318704123389\\
0	-0	-0\\
0.384681022148151	-0.200687842854736	-0.900968867902419\\
0	-0	-0\\
0.332711466230727	-0.173575358820705	-0.926916757346022\\
0	-0	-0\\
0.279374727532175	-0.145749616405482	-0.949055747010669\\
0	-0	-0\\
0.224889978152501	-0.117324957553254	-0.967294863039029\\
0	-0	-0\\
0.169481107613851	-0.0884181852843155	-0.981559156991065\\
0	-0	-0\\
0.113375802850431	-0.0591480837264023	-0.991790013823246\\
0	-0	-0\\
0.0568046125940659	-0.0296349300052333	-0.997945392750336\\
0	-0	-0\\
1.08577100267596e-16	-5.66446036626528e-17	-1\\
0	-0	0\\
0	-0	1\\
0	-0	0\\
0.0585866179097638	-0.0259345578383802	0.997945392750336\\
0	-0	0\\
0.116932490839746	-0.0517625450156572	0.991790013823246\\
0	-0	0\\
0.174797863082928	-0.0773778287924339	0.981559156991065\\
0	-0	0\\
0.231944953412677	-0.10267515047321	0.967294863039029\\
0	-0	0\\
0.288138932176817	-0.127550557936941	0.949055747010669\\
0	-0	0\\
0.343148886263036	-0.151901832798601	0.926916757346022\\
0	-0	0\\
0.396748767970395	-0.17562891044645	0.900968867902419\\
0	-0	0\\
0.448718323887819	-0.198634291228991	0.871318704123389\\
0	-0	0\\
0.49884399996261	-0.220823441101951	0.838088104891841\\
0	-0	0\\
0.546919819039852	-0.242105180088943	0.801413621867957\\
0	-0	0\\
0.592748227266726	-0.262392056959551	0.761445958369135\\
0	-0	0\\
0.636140905883665	-0.281600708585193	0.718349350097728\\
0	-0	0\\
0.676919545066531	-0.299652202496096	0.672300890261317\\
0	-0	0\\
0.714916576639932	-0.316472361231746	0.623489801858734\\
0	-0	0\\
0.749975862650795	-0.331992067151987	0.57211666012217\\
0	-0	0\\
0.781953336972708	-0.346147546456225	0.518392568310525\\
0	-0	0\\
0.810717597304535	-0.358880631243658	0.462538290240835\\
0	-0	0\\
0.836150445130658	-0.370138998537658	0.404783343122394\\
0	-0	0\\
0.858147371424032	-0.3798763852921	0.345365054421308\\
0	-0	0\\
0.87661798609619	-0.388052778496147	0.284527586631032\\
0	-0	0\\
0.891486389429511	-0.394634579596294	0.222520933956314\\
0	-0	0\\
0.902691483965435	-0.399594742560028	0.15959989503338\\
0	-0	0\\
0.910187225567029	-0.402912885013779	0.0960230259076819\\
0	-0	0\\
0.913942812624221	-0.404575371998466	0.0320515775716553\\
0	-0	-0\\
0.913942812624221	-0.404575371998466	-0.0320515775716552\\
0	-0	-0\\
0.910187225567029	-0.402912885013779	-0.0960230259076818\\
0	-0	-0\\
0.902691483965435	-0.399594742560028	-0.159599895033379\\
0	-0	-0\\
0.891486389429511	-0.394634579596294	-0.222520933956314\\
0	-0	-0\\
0.87661798609619	-0.388052778496147	-0.284527586631032\\
0	-0	-0\\
0.858147371424032	-0.3798763852921	-0.345365054421307\\
0	-0	-0\\
0.836150445130658	-0.370138998537658	-0.404783343122394\\
0	-0	-0\\
0.810717597304535	-0.358880631243658	-0.462538290240835\\
0	-0	-0\\
0.781953336972708	-0.346147546456225	-0.518392568310525\\
0	-0	-0\\
0.749975862650795	-0.331992067151987	-0.57211666012217\\
0	-0	-0\\
0.714916576639932	-0.316472361231746	-0.623489801858733\\
0	-0	-0\\
0.676919545066531	-0.299652202496096	-0.672300890261317\\
0	-0	-0\\
0.636140905883665	-0.281600708585193	-0.718349350097727\\
0	-0	-0\\
0.592748227266726	-0.262392056959551	-0.761445958369134\\
0	-0	-0\\
0.546919819039852	-0.242105180088943	-0.801413621867957\\
0	-0	-0\\
0.49884399996261	-0.220823441101951	-0.838088104891841\\
0	-0	-0\\
0.448718323887819	-0.198634291228991	-0.871318704123389\\
0	-0	-0\\
0.396748767970395	-0.17562891044645	-0.900968867902419\\
0	-0	-0\\
0.343148886263037	-0.151901832798601	-0.926916757346022\\
0	-0	-0\\
0.288138932176818	-0.127550557936942	-0.949055747010669\\
0	-0	-0\\
0.231944953412677	-0.10267515047321	-0.967294863039029\\
0	-0	-0\\
0.174797863082928	-0.0773778287924339	-0.981559156991065\\
0	-0	-0\\
0.116932490839746	-0.0517625450156572	-0.991790013823246\\
0	-0	-0\\
0.0585866179097641	-0.0259345578383803	-0.997945392750336\\
0	-0	-0\\
1.11983249187625e-16	-4.95716625503004e-17	-1\\
0	-0	0\\
0	-0	1\\
0	-0	0\\
0.0601278782456807	-0.0221276150104241	0.997945392750336\\
0	-0	0\\
0.12000867814226	-0.0441643029044118	0.991790013823246\\
0	-0	0\\
0.179396336638573	-0.066019510204552	0.981559156991065\\
0	-0	0\\
0.238046817107244	-0.0876034291761211	0.967294863039029\\
0	-0	0\\
0.295719112143539	-0.108827366866329	0.949055747010669\\
0	-0	0\\
0.352176233916486	-0.129604109562686	0.926916757346022\\
0	-0	0\\
0.407186188002705	-0.149848281172856	0.900968867902419\\
0	-0	0\\
0.460522926701257	-0.169476694053331	0.871318704123389\\
0	-0	0\\
0.511967277912136	-0.188408690845304	0.838088104891841\\
0	-0	0\\
0.561307845761437	-0.206566475913056	0.801413621867957\\
0	-0	0\\
0.608341879272348	-0.223875435022911	0.761445958369135\\
0	-0	0\\
0.652876105512406	-0.240264441949127	0.718349350097728\\
0	-0	0\\
0.694727523793429	-0.255666150746813	0.672300890261317\\
0	-0	0\\
0.733724157660597	-0.270017272490863	0.623489801858734\\
0	-0	0\\
0.7697057615806	-0.283258835343724	0.57211666012217\\
0	-0	0\\
0.802524479424899	-0.295336426883329	0.518392568310525\\
0	-0	0\\
0.832045452042282	-0.306200417695405	0.462538290240835\\
0	-0	0\\
0.858147371424032	-0.315806165311387	0.404783343122394\\
0	-0	0\\
0.880722979184567	-0.324114197653894	0.345365054421308\\
0	-0	0\\
0.899679507309146	-0.331090375235964	0.284527586631032\\
0	-0	0\\
0.914939059357543	-0.336706031447527	0.222520933956314\\
0	-0	0\\
0.926438930557227	-0.340938090352655	0.15959989503338\\
0	-0	0\\
0.934131865470724	-0.343769161513533	0.0960230259076819\\
0	-0	0\\
0.937986252178345	-0.345187611451497	0.0320515775716553\\
0	-0	-0\\
0.937986252178345	-0.345187611451497	-0.0320515775716552\\
0	-0	-0\\
0.934131865470724	-0.343769161513533	-0.0960230259076818\\
0	-0	-0\\
0.926438930557227	-0.340938090352655	-0.159599895033379\\
0	-0	-0\\
0.914939059357543	-0.336706031447527	-0.222520933956314\\
0	-0	-0\\
0.899679507309147	-0.331090375235964	-0.284527586631032\\
0	-0	-0\\
0.880722979184567	-0.324114197653894	-0.345365054421307\\
0	-0	-0\\
0.858147371424032	-0.315806165311387	-0.404783343122394\\
0	-0	-0\\
0.832045452042282	-0.306200417695405	-0.462538290240835\\
0	-0	-0\\
0.802524479424899	-0.295336426883329	-0.518392568310525\\
0	-0	-0\\
0.769705761580599	-0.283258835343724	-0.57211666012217\\
0	-0	-0\\
0.733724157660597	-0.270017272490863	-0.623489801858733\\
0	-0	-0\\
0.694727523793429	-0.255666150746813	-0.672300890261317\\
0	-0	-0\\
0.652876105512407	-0.240264441949127	-0.718349350097727\\
0	-0	-0\\
0.608341879272349	-0.223875435022911	-0.761445958369134\\
0	-0	-0\\
0.561307845761437	-0.206566475913056	-0.801413621867957\\
0	-0	-0\\
0.511967277912136	-0.188408690845304	-0.838088104891841\\
0	-0	-0\\
0.460522926701257	-0.169476694053331	-0.871318704123389\\
0	-0	-0\\
0.407186188002705	-0.149848281172856	-0.900968867902419\\
0	-0	-0\\
0.352176233916486	-0.129604109562686	-0.926916757346022\\
0	-0	-0\\
0.29571911214354	-0.108827366866329	-0.949055747010669\\
0	-0	-0\\
0.238046817107244	-0.0876034291761211	-0.967294863039029\\
0	-0	-0\\
0.179396336638573	-0.0660195102045521	-0.981559156991065\\
0	-0	-0\\
0.12000867814226	-0.0441643029044118	-0.991790013823246\\
0	-0	-0\\
0.060127878245681	-0.0221276150104242	-0.997945392750336\\
0	-0	-0\\
1.14929234916411e-16	-4.22950208434406e-17	-1\\
0	-0	0\\
0	-0	1\\
0	-0	0\\
0.0614220602324966	-0.0182297450660316	0.997945392750336\\
0	-0	0\\
0.122591724044507	-0.0363845801993188	0.991790013823246\\
0	-0	0\\
0.183257632166577	-0.0543899032880991	0.981559156991065\\
0	-0	0\\
0.243170495369435	-0.0721717265976708	0.967294863039029\\
0	-0	0\\
0.302084118846912	-0.089656980801866	0.949055747010669\\
0	-0	0\\
0.359756413883208	-0.106773815240584	0.926916757346022\\
0	-0	0\\
0.415950392647348	-0.123451893169567	0.900968867902419\\
0	-0	0\\
0.470435142027021	-0.139622680789169	0.871318704123389\\
0	-0	0\\
0.522986772500083	-0.155219728864436	0.838088104891841\\
0	-0	0\\
0.573389338144632	-0.170178945779272	0.801413621867957\\
0	-0	0\\
0.621435724007117	-0.184438860902632	0.761445958369135\\
0	-0	0\\
0.666928497182112	-0.197940877184531	0.718349350097728\\
0	-0	0\\
0.709680718106472	-0.210629511943894	0.672300890261317\\
0	-0	0\\
0.749516708734096	-0.222452624858791	0.623489801858734\\
0	-0	0\\
0.786272774434701	-0.233361632222205	0.57211666012217\\
0	-0	0\\
0.819797876650173	-0.243311706582904	0.518392568310525\\
0	-0	0\\
0.849954253544396	-0.252261960951057	0.462538290240835\\
0	-0	0\\
0.87661798609619	-0.260175616811641	0.404783343122394\\
0	-0	0\\
0.899679507309146	-0.267020155255251	0.345365054421308\\
0	-0	0\\
0.91904405244592	-0.272767450605274	0.284527586631032\\
0	-0	0\\
0.934632048436863	-0.277393885992325	0.222520933956314\\
0	-0	0\\
0.946379440862833	-0.280880450401032	0.15959989503338\\
0	-0	0\\
0.954237957168544	-0.283212816790373	0.0960230259076819\\
0	-0	0\\
0.958175305024849	-0.284381400966564	0.0320515775716553\\
0	-0	-0\\
0.958175305024849	-0.284381400966564	-0.0320515775716552\\
0	-0	-0\\
0.954237957168544	-0.283212816790373	-0.0960230259076818\\
0	-0	-0\\
0.946379440862833	-0.280880450401032	-0.159599895033379\\
0	-0	-0\\
0.934632048436863	-0.277393885992325	-0.222520933956314\\
0	-0	-0\\
0.91904405244592	-0.272767450605274	-0.284527586631032\\
0	-0	-0\\
0.899679507309146	-0.267020155255251	-0.345365054421307\\
0	-0	-0\\
0.87661798609619	-0.260175616811641	-0.404783343122394\\
0	-0	-0\\
0.849954253544396	-0.252261960951057	-0.462538290240835\\
0	-0	-0\\
0.819797876650173	-0.243311706582904	-0.518392568310525\\
0	-0	-0\\
0.786272774434701	-0.233361632222204	-0.57211666012217\\
0	-0	-0\\
0.749516708734096	-0.222452624858791	-0.623489801858733\\
0	-0	-0\\
0.709680718106472	-0.210629511943894	-0.672300890261317\\
0	-0	-0\\
0.666928497182112	-0.197940877184531	-0.718349350097727\\
0	-0	-0\\
0.621435724007118	-0.184438860902632	-0.761445958369134\\
0	-0	-0\\
0.573389338144632	-0.170178945779272	-0.801413621867957\\
0	-0	-0\\
0.522986772500083	-0.155219728864436	-0.838088104891841\\
0	-0	-0\\
0.470435142027021	-0.139622680789169	-0.871318704123389\\
0	-0	-0\\
0.415950392647348	-0.123451893169567	-0.900968867902419\\
0	-0	-0\\
0.359756413883208	-0.106773815240584	-0.926916757346022\\
0	-0	-0\\
0.302084118846913	-0.0896569808018661	-0.949055747010669\\
0	-0	-0\\
0.243170495369435	-0.0721717265976708	-0.967294863039029\\
0	-0	-0\\
0.183257632166577	-0.0543899032880991	-0.981559156991065\\
0	-0	-0\\
0.122591724044507	-0.0363845801993188	-0.991790013823246\\
0	-0	-0\\
0.0614220602324969	-0.0182297450660317	-0.997945392750336\\
0	-0	-0\\
1.17402951766681e-16	-3.48445798236815e-17	-1\\
0	-0	0\\
0	-0	1\\
0	-0	0\\
0.0624638457988266	-0.0142569651888948	0.997945392750336\\
0	-0	0\\
0.124671014256813	-0.0284553454497189	0.991790013823246\\
0	-0	0\\
0.186365882775369	-0.0425367965924377	0.981559156991065\\
0	-0	0\\
0.247294933906245	-0.0564434549138439	0.967294863039029\\
0	-0	0\\
0.307207797109103	-0.070118174971926	0.949055747010669\\
0	-0	0\\
0.365858277577774	-0.0835047644087471	0.926916757346022\\
0	-0	0\\
0.423005367907524	-0.0965482148568969	0.900968867902419\\
0	-0	0\\
0.478414238446175	-0.109194927980673	0.871318704123389\\
0	-0	0\\
0.531857202259518	-0.121392935723137	0.838088104891841\\
0	-0	0\\
0.583114650745764	-0.133092113854012	0.801413621867957\\
0	-0	0\\
0.631975956054396	-0.144244387940892	0.761445958369135\\
0	-0	0\\
0.678240336601183	-0.154803930897399	0.718349350097728\\
0	-0	0\\
0.721717682122779	-0.164727351296509	0.672300890261317\\
0	-0	0\\
0.762229334880576	-0.173973871675236	0.623489801858734\\
0	-0	0\\
0.799608823803669	-0.182505496097971	0.57211666012217\\
0	-0	0\\
0.833702548554198	-0.190287166289933	0.518392568310525\\
0	-0	0\\
0.864370410704082	-0.197286905699141	0.462538290240835\\
0	-0	0\\
0.891486389429511	-0.203475950894922	0.404783343122394\\
0	-0	0\\
0.914939059357543	-0.208828869763021	0.345365054421308\\
0	-0	0\\
0.934632048436863	-0.213323666011612	0.284527586631032\\
0	-0	0\\
0.95048443395121	-0.216941869558779	0.222520933956314\\
0	-0	0\\
0.962431075048179	-0.219668612430044	0.15959989503338\\
0	-0	0\\
0.970422880416957	-0.221492689854063	0.0960230259076819\\
0	-0	0\\
0.974427010015047	-0.22240660630544	0.0320515775716553\\
0	-0	-0\\
0.974427010015047	-0.22240660630544	-0.0320515775716552\\
0	-0	-0\\
0.970422880416957	-0.221492689854063	-0.0960230259076818\\
0	-0	-0\\
0.962431075048179	-0.219668612430044	-0.159599895033379\\
0	-0	-0\\
0.95048443395121	-0.216941869558779	-0.222520933956314\\
0	-0	-0\\
0.934632048436863	-0.213323666011612	-0.284527586631032\\
0	-0	-0\\
0.914939059357543	-0.208828869763021	-0.345365054421307\\
0	-0	-0\\
0.891486389429511	-0.203475950894922	-0.404783343122394\\
0	-0	-0\\
0.864370410704082	-0.197286905699141	-0.462538290240835\\
0	-0	-0\\
0.833702548554198	-0.190287166289933	-0.518392568310525\\
0	-0	-0\\
0.799608823803669	-0.182505496097971	-0.57211666012217\\
0	-0	-0\\
0.762229334880576	-0.173973871675236	-0.623489801858733\\
0	-0	-0\\
0.721717682122779	-0.164727351296509	-0.672300890261317\\
0	-0	-0\\
0.678240336601183	-0.154803930897399	-0.718349350097727\\
0	-0	-0\\
0.631975956054396	-0.144244387940892	-0.761445958369134\\
0	-0	-0\\
0.583114650745764	-0.133092113854012	-0.801413621867957\\
0	-0	-0\\
0.531857202259518	-0.121392935723137	-0.838088104891841\\
0	-0	-0\\
0.478414238446175	-0.109194927980673	-0.871318704123389\\
0	-0	-0\\
0.423005367907524	-0.0965482148568969	-0.900968867902419\\
0	-0	-0\\
0.365858277577775	-0.0835047644087472	-0.926916757346022\\
0	-0	-0\\
0.307207797109104	-0.0701181749719261	-0.949055747010669\\
0	-0	-0\\
0.247294933906245	-0.0564434549138439	-0.967294863039029\\
0	-0	-0\\
0.18636588277537	-0.0425367965924377	-0.981559156991065\\
0	-0	-0\\
0.124671014256813	-0.0284553454497189	-0.991790013823246\\
0	-0	-0\\
0.0624638457988269	-0.0142569651888948	-0.997945392750336\\
0	-0	-0\\
1.19394234705288e-16	-2.7250954951288e-17	-1\\
0	-0	0\\
0	-0	1\\
0	-0	0\\
0.0632489540243163	-0.0102256003836873	0.997945392750336\\
0	-0	0\\
0.126238004529689	-0.0204091815820136	0.991790013823246\\
0	-0	0\\
0.188708315996481	-0.0305088970754637	0.981559156991065\\
0	-0	0\\
0.250403184515038	-0.0404832449666929	0.967294863039029\\
0	-0	0\\
0.311069092637107	-0.050291238520725	0.949055747010669\\
0	-0	0\\
0.37045675113342	-0.0598925745882387	0.926916757346022\\
0	-0	0\\
0.428322123376601	-0.0692477992198524	0.900968867902419\\
0	-0	0\\
0.484427428140021	-0.0783184697908653	0.871318704123389\\
0	-0	0\\
0.538542116691855	-0.0870673129702485	0.838088104891841\\
0	-0	0\\
0.590443820169281	-0.0954583778847568	0.801413621867957\\
0	-0	0\\
0.63991926333983	-0.103457183848779	0.761445958369135\\
0	-0	0\\
0.686765140995063	-0.11103086205287	0.718349350097728\\
0	-0	0\\
0.730788953375287	-0.118148290628741	0.672300890261317\\
0	-0	0\\
0.771809797192353	-0.124780222535689	0.623489801858734\\
0	-0	0\\
0.809659109000073	-0.130899405742964	0.57211666012217\\
0	-0	0\\
0.844181357857578	-0.136480695214207	0.518392568310525\\
0	-0	0\\
0.875234684439313	-0.141501156233798	0.462538290240835\\
0	-0	0\\
0.902691483965435	-0.145940158650521	0.404783343122394\\
0	-0	0\\
0.926438930557227	-0.149779461651283	0.345365054421308\\
0	-0	0\\
0.946379440862833	-0.153003288716526	0.284527586631032\\
0	-0	0\\
0.962431075048179	-0.155598392449331	0.222520933956314\\
0	-0	0\\
0.974527873505334	-0.15755410901181	0.15959989503338\\
0	-0	0\\
0.982620127894683	-0.15886240194511	0.0960230259076819\\
0	-0	0\\
0.986674585407156	-0.159517895192939	0.0320515775716553\\
0	-0	-0\\
0.986674585407156	-0.159517895192939	-0.0320515775716552\\
0	-0	-0\\
0.982620127894683	-0.15886240194511	-0.0960230259076818\\
0	-0	-0\\
0.974527873505334	-0.15755410901181	-0.159599895033379\\
0	-0	-0\\
0.962431075048179	-0.155598392449331	-0.222520933956314\\
0	-0	-0\\
0.946379440862833	-0.153003288716526	-0.284527586631032\\
0	-0	-0\\
0.926438930557227	-0.149779461651283	-0.345365054421307\\
0	-0	-0\\
0.902691483965435	-0.145940158650521	-0.404783343122394\\
0	-0	-0\\
0.875234684439313	-0.141501156233798	-0.462538290240835\\
0	-0	-0\\
0.844181357857578	-0.136480695214207	-0.518392568310525\\
0	-0	-0\\
0.809659109000073	-0.130899405742964	-0.57211666012217\\
0	-0	-0\\
0.771809797192353	-0.124780222535689	-0.623489801858733\\
0	-0	-0\\
0.730788953375287	-0.118148290628741	-0.672300890261317\\
0	-0	-0\\
0.686765140995063	-0.11103086205287	-0.718349350097727\\
0	-0	-0\\
0.63991926333983	-0.103457183848779	-0.761445958369134\\
0	-0	-0\\
0.590443820169282	-0.0954583778847568	-0.801413621867957\\
0	-0	-0\\
0.538542116691855	-0.0870673129702485	-0.838088104891841\\
0	-0	-0\\
0.484427428140021	-0.0783184697908653	-0.871318704123389\\
0	-0	-0\\
0.428322123376601	-0.0692477992198524	-0.900968867902419\\
0	-0	-0\\
0.37045675113342	-0.0598925745882388	-0.926916757346022\\
0	-0	-0\\
0.311069092637108	-0.0502912385207251	-0.949055747010669\\
0	-0	-0\\
0.250403184515038	-0.0404832449666929	-0.967294863039029\\
0	-0	-0\\
0.188708315996482	-0.0305088970754638	-0.981559156991065\\
0	-0	-0\\
0.126238004529689	-0.0204091815820136	-0.991790013823246\\
0	-0	-0\\
0.0632489540243166	-0.0102256003836873	-0.997945392750336\\
0	-0	-0\\
1.20894901123508e-16	-1.95453500596881e-17	-1\\
0	-0	0\\
0	-0	1\\
0	-0	0\\
0.063774158730862	-0.00615221639311886	0.997945392750336\\
0	-0	0\\
0.127286255763985	-0.0122791520094321	0.991790013823246\\
0	-0	0\\
0.190275306269357	-0.0183556299562688	0.981559156991065\\
0	-0	0\\
0.252482474727343	-0.0243566806823449	0.967294863039029\\
0	-0	0\\
0.313652138539354	-0.0302576445830056	0.949055747010669\\
0	-0	0\\
0.373532938435934	-0.0360342733318303	0.926916757346022\\
0	-0	0\\
0.431878811365916	-0.0416628295222071	0.900968867902419\\
0	-0	0\\
0.488450001622282	-0.0471201842094283	0.871318704123389\\
0	-0	0\\
0.543014046049784	-0.0523839119524851	0.838088104891841\\
0	-0	0\\
0.595346729285921	-0.0574323829650153	0.801413621867957\\
0	-0	0\\
0.645233005109949	-0.0622448519967347	0.761445958369135\\
0	-0	0\\
0.692467880113934	-0.0668015435801207	0.718349350097728\\
0	-0	0\\
0.736857256064637	-0.0710837332920498	0.672300890261317\\
0	-0	0\\
0.778218727494784	-0.0750738246964689	0.623489801858734\\
0	-0	0\\
0.816382331246262	-0.0787554216519252	0.57211666012217\\
0	-0	0\\
0.851191244885188	-0.0821133956868289	0.518392568310525\\
0	-0	0\\
0.882502431118931	-0.0851339481655873	0.462538290240835\\
0	-0	0\\
0.910187225567029	-0.0878046669901586	0.404783343122394\\
0	-0	0\\
0.934131865470724	-0.0901145776040253	0.345365054421308\\
0	-0	0\\
0.954237957168544	-0.0920541880890009	0.284527586631032\\
0	-0	0\\
0.970422880416957	-0.0936155281695573	0.222520933956314\\
0	-0	0\\
0.982620127894683	-0.0947921819643972	0.15959989503338\\
0	-0	0\\
0.990779578495533	-0.0955793143506861	0.0960230259076819\\
0	-0	0\\
0.994867703286791	-0.0959736908326095	0.0320515775716553\\
0	-0	-0\\
0.994867703286791	-0.0959736908326095	-0.0320515775716552\\
0	-0	-0\\
0.990779578495533	-0.0955793143506861	-0.0960230259076818\\
0	-0	-0\\
0.982620127894683	-0.0947921819643972	-0.159599895033379\\
0	-0	-0\\
0.970422880416957	-0.0936155281695573	-0.222520933956314\\
0	-0	-0\\
0.954237957168544	-0.0920541880890009	-0.284527586631032\\
0	-0	-0\\
0.934131865470724	-0.0901145776040254	-0.345365054421307\\
0	-0	-0\\
0.910187225567029	-0.0878046669901586	-0.404783343122394\\
0	-0	-0\\
0.882502431118931	-0.0851339481655873	-0.462538290240835\\
0	-0	-0\\
0.851191244885188	-0.0821133956868289	-0.518392568310525\\
0	-0	-0\\
0.816382331246262	-0.0787554216519252	-0.57211666012217\\
0	-0	-0\\
0.778218727494784	-0.0750738246964689	-0.623489801858733\\
0	-0	-0\\
0.736857256064637	-0.0710837332920498	-0.672300890261317\\
0	-0	-0\\
0.692467880113934	-0.0668015435801207	-0.718349350097727\\
0	-0	-0\\
0.645233005109949	-0.0622448519967348	-0.761445958369134\\
0	-0	-0\\
0.595346729285921	-0.0574323829650153	-0.801413621867957\\
0	-0	-0\\
0.543014046049784	-0.0523839119524851	-0.838088104891841\\
0	-0	-0\\
0.488450001622282	-0.0471201842094283	-0.871318704123389\\
0	-0	-0\\
0.431878811365916	-0.0416628295222071	-0.900968867902419\\
0	-0	-0\\
0.373532938435934	-0.0360342733318303	-0.926916757346022\\
0	-0	-0\\
0.313652138539355	-0.0302576445830056	-0.949055747010669\\
0	-0	-0\\
0.252482474727343	-0.0243566806823449	-0.967294863039029\\
0	-0	-0\\
0.190275306269357	-0.0183556299562688	-0.981559156991065\\
0	-0	-0\\
0.127286255763985	-0.0122791520094321	-0.991790013823246\\
0	-0	-0\\
0.0637741587308623	-0.0061522163931189	-0.997945392750336\\
0	-0	-0\\
1.21898784461137e-16	-1.17594291322286e-17	-1\\
0	-0	0\\
0	-0	1\\
0	-0	0\\
0.0640373017396685	-0.00205355162574483	0.997945392750336\\
0	-0	0\\
0.12781146047053	-0.00409866476737403	0.991790013823246\\
0	-0	0\\
0.191060414494847	-0.00612693561631326	0.981559156991065\\
0	-0	0\\
0.253524260293673	-0.00813002957258149	0.967294863039029\\
0	-0	0\\
0.31494632052617	-0.0100997154934501	0.949055747010669\\
0	-0	0\\
0.375074198771851	-0.012027899516974	0.926916757346022\\
0	-0	0\\
0.433660816681615	-0.0139066583214062	0.900968867902419\\
0	-0	0\\
0.49046542927568	-0.0157282716838269	0.871318704123389\\
0	-0	0\\
0.545254614216347	-0.0174852542041951	0.838088104891841\\
0	-0	0\\
0.597803230990452	-0.019170386064463	0.801413621867957\\
0	-0	0\\
0.647895346060025	-0.0207767426963571	0.761445958369135\\
0	-0	0\\
0.695325120179522	-0.0222977232359146	0.718349350097728\\
0	-0	0\\
0.739897654233431	-0.023727077647849	0.672300890261317\\
0	-0	0\\
0.781429790118546	-0.0250589324082862	0.623489801858734\\
0	-0	0\\
0.819750863379899	-0.0262878146403336	0.57211666012217\\
0	-0	0\\
0.854703404507615	-0.0274086746033054	0.518392568310525\\
0	-0	0\\
0.886143786012904	-0.0284169064431899	0.462538290240835\\
0	-0	0\\
0.913942812624221	-0.029308367119092	0.404783343122394\\
0	-0	0\\
0.937986252178345	-0.0300793934278767	0.345365054421308\\
0	-0	0\\
0.958175305024849	-0.0307268170570566	0.284527586631032\\
0	-0	0\\
0.974427010015047	-0.0312479776040675	0.222520933956314\\
0	-0	0\\
0.986674585407156	-0.0316407335084331	0.15959989503338\\
0	-0	0\\
0.994867703286791	-0.0319034708518965	0.0960230259076819\\
0	-0	0\\
0.998972696375168	-0.0320351099903565	0.0320515775716553\\
0	-0	-0\\
0.998972696375168	-0.0320351099903565	-0.0320515775716552\\
0	-0	-0\\
0.994867703286791	-0.0319034708518965	-0.0960230259076818\\
0	-0	-0\\
0.986674585407156	-0.0316407335084331	-0.159599895033379\\
0	-0	-0\\
0.974427010015047	-0.0312479776040675	-0.222520933956314\\
0	-0	-0\\
0.958175305024849	-0.0307268170570567	-0.284527586631032\\
0	-0	-0\\
0.937986252178345	-0.0300793934278767	-0.345365054421307\\
0	-0	-0\\
0.913942812624221	-0.029308367119092	-0.404783343122394\\
0	-0	-0\\
0.886143786012904	-0.0284169064431899	-0.462538290240835\\
0	-0	-0\\
0.854703404507615	-0.0274086746033054	-0.518392568310525\\
0	-0	-0\\
0.819750863379899	-0.0262878146403336	-0.57211666012217\\
0	-0	-0\\
0.781429790118546	-0.0250589324082862	-0.623489801858733\\
0	-0	-0\\
0.739897654233431	-0.023727077647849	-0.672300890261317\\
0	-0	-0\\
0.695325120179522	-0.0222977232359146	-0.718349350097727\\
0	-0	-0\\
0.647895346060026	-0.0207767426963571	-0.761445958369134\\
0	-0	-0\\
0.597803230990452	-0.019170386064463	-0.801413621867957\\
0	-0	-0\\
0.545254614216347	-0.0174852542041951	-0.838088104891841\\
0	-0	-0\\
0.49046542927568	-0.015728271683827	-0.871318704123389\\
0	-0	-0\\
0.433660816681615	-0.0139066583214062	-0.900968867902419\\
0	-0	-0\\
0.375074198771851	-0.012027899516974	-0.926916757346022\\
0	-0	-0\\
0.314946320526171	-0.0100997154934501	-0.949055747010669\\
0	-0	-0\\
0.253524260293673	-0.00813002957258149	-0.967294863039029\\
0	-0	-0\\
0.191060414494847	-0.00612693561631327	-0.981559156991065\\
0	-0	-0\\
0.127811460470531	-0.00409866476737404	-0.991790013823246\\
0	-0	-0\\
0.0640373017396688	-0.00205355162574484	-0.997945392750336\\
0	-0	-0\\
1.22401759546207e-16	-3.92518618807506e-18	-1\\
0	0	0\\
0	0	1\\
0	0	0\\
0.0640373017396685	0.00205355162574483	0.997945392750336\\
0	0	0\\
0.12781146047053	0.00409866476737403	0.991790013823246\\
0	0	0\\
0.191060414494847	0.00612693561631326	0.981559156991065\\
0	0	0\\
0.253524260293673	0.00813002957258149	0.967294863039029\\
0	0	0\\
0.31494632052617	0.0100997154934501	0.949055747010669\\
0	0	0\\
0.375074198771851	0.012027899516974	0.926916757346022\\
0	0	0\\
0.433660816681615	0.0139066583214062	0.900968867902419\\
0	0	0\\
0.49046542927568	0.0157282716838269	0.871318704123389\\
0	0	0\\
0.545254614216347	0.0174852542041951	0.838088104891841\\
0	0	0\\
0.597803230990452	0.019170386064463	0.801413621867957\\
0	0	0\\
0.647895346060025	0.0207767426963571	0.761445958369135\\
0	0	0\\
0.695325120179522	0.0222977232359146	0.718349350097728\\
0	0	0\\
0.739897654233431	0.023727077647849	0.672300890261317\\
0	0	0\\
0.781429790118546	0.0250589324082862	0.623489801858734\\
0	0	0\\
0.819750863379899	0.0262878146403336	0.57211666012217\\
0	0	0\\
0.854703404507615	0.0274086746033054	0.518392568310525\\
0	0	0\\
0.886143786012904	0.0284169064431899	0.462538290240835\\
0	0	0\\
0.913942812624221	0.029308367119092	0.404783343122394\\
0	0	0\\
0.937986252178345	0.0300793934278767	0.345365054421308\\
0	0	0\\
0.958175305024849	0.0307268170570566	0.284527586631032\\
0	0	0\\
0.974427010015047	0.0312479776040675	0.222520933956314\\
0	0	0\\
0.986674585407156	0.0316407335084331	0.15959989503338\\
0	0	0\\
0.994867703286791	0.0319034708518965	0.0960230259076819\\
0	0	0\\
0.998972696375168	0.0320351099903565	0.0320515775716553\\
0	0	-0\\
0.998972696375168	0.0320351099903565	-0.0320515775716552\\
0	0	-0\\
0.994867703286791	0.0319034708518965	-0.0960230259076818\\
0	0	-0\\
0.986674585407156	0.0316407335084331	-0.159599895033379\\
0	0	-0\\
0.974427010015047	0.0312479776040675	-0.222520933956314\\
0	0	-0\\
0.958175305024849	0.0307268170570567	-0.284527586631032\\
0	0	-0\\
0.937986252178345	0.0300793934278767	-0.345365054421307\\
0	0	-0\\
0.913942812624221	0.029308367119092	-0.404783343122394\\
0	0	-0\\
0.886143786012904	0.0284169064431899	-0.462538290240835\\
0	0	-0\\
0.854703404507615	0.0274086746033054	-0.518392568310525\\
0	0	-0\\
0.819750863379899	0.0262878146403336	-0.57211666012217\\
0	0	-0\\
0.781429790118546	0.0250589324082862	-0.623489801858733\\
0	0	-0\\
0.739897654233431	0.023727077647849	-0.672300890261317\\
0	0	-0\\
0.695325120179522	0.0222977232359146	-0.718349350097727\\
0	0	-0\\
0.647895346060026	0.0207767426963571	-0.761445958369134\\
0	0	-0\\
0.597803230990452	0.019170386064463	-0.801413621867957\\
0	0	-0\\
0.545254614216347	0.0174852542041951	-0.838088104891841\\
0	0	-0\\
0.49046542927568	0.015728271683827	-0.871318704123389\\
0	0	-0\\
0.433660816681615	0.0139066583214062	-0.900968867902419\\
0	0	-0\\
0.375074198771851	0.012027899516974	-0.926916757346022\\
0	0	-0\\
0.314946320526171	0.0100997154934501	-0.949055747010669\\
0	0	-0\\
0.253524260293673	0.00813002957258149	-0.967294863039029\\
0	0	-0\\
0.191060414494847	0.00612693561631327	-0.981559156991065\\
0	0	-0\\
0.127811460470531	0.00409866476737404	-0.991790013823246\\
0	0	-0\\
0.0640373017396688	0.00205355162574484	-0.997945392750336\\
0	0	-0\\
1.22401759546207e-16	3.92518618807506e-18	-1\\
0	0	0\\
0	0	1\\
0	0	0\\
0.063774158730862	0.00615221639311886	0.997945392750336\\
0	0	0\\
0.127286255763985	0.0122791520094321	0.991790013823246\\
0	0	0\\
0.190275306269357	0.0183556299562688	0.981559156991065\\
0	0	0\\
0.252482474727343	0.0243566806823449	0.967294863039029\\
0	0	0\\
0.313652138539354	0.0302576445830056	0.949055747010669\\
0	0	0\\
0.373532938435934	0.0360342733318303	0.926916757346022\\
0	0	0\\
0.431878811365916	0.0416628295222071	0.900968867902419\\
0	0	0\\
0.488450001622282	0.0471201842094283	0.871318704123389\\
0	0	0\\
0.543014046049784	0.0523839119524851	0.838088104891841\\
0	0	0\\
0.595346729285921	0.0574323829650153	0.801413621867957\\
0	0	0\\
0.645233005109949	0.0622448519967347	0.761445958369135\\
0	0	0\\
0.692467880113934	0.0668015435801207	0.718349350097728\\
0	0	0\\
0.736857256064637	0.0710837332920498	0.672300890261317\\
0	0	0\\
0.778218727494784	0.0750738246964689	0.623489801858734\\
0	0	0\\
0.816382331246262	0.0787554216519252	0.57211666012217\\
0	0	0\\
0.851191244885188	0.0821133956868289	0.518392568310525\\
0	0	0\\
0.882502431118931	0.0851339481655873	0.462538290240835\\
0	0	0\\
0.910187225567029	0.0878046669901586	0.404783343122394\\
0	0	0\\
0.934131865470724	0.0901145776040253	0.345365054421308\\
0	0	0\\
0.954237957168544	0.0920541880890009	0.284527586631032\\
0	0	0\\
0.970422880416957	0.0936155281695573	0.222520933956314\\
0	0	0\\
0.982620127894683	0.0947921819643972	0.15959989503338\\
0	0	0\\
0.990779578495533	0.0955793143506861	0.0960230259076819\\
0	0	0\\
0.994867703286791	0.0959736908326095	0.0320515775716553\\
0	0	-0\\
0.994867703286791	0.0959736908326095	-0.0320515775716552\\
0	0	-0\\
0.990779578495533	0.0955793143506861	-0.0960230259076818\\
0	0	-0\\
0.982620127894683	0.0947921819643972	-0.159599895033379\\
0	0	-0\\
0.970422880416957	0.0936155281695573	-0.222520933956314\\
0	0	-0\\
0.954237957168544	0.0920541880890009	-0.284527586631032\\
0	0	-0\\
0.934131865470724	0.0901145776040254	-0.345365054421307\\
0	0	-0\\
0.910187225567029	0.0878046669901586	-0.404783343122394\\
0	0	-0\\
0.882502431118931	0.0851339481655873	-0.462538290240835\\
0	0	-0\\
0.851191244885188	0.0821133956868289	-0.518392568310525\\
0	0	-0\\
0.816382331246262	0.0787554216519252	-0.57211666012217\\
0	0	-0\\
0.778218727494784	0.0750738246964689	-0.623489801858733\\
0	0	-0\\
0.736857256064637	0.0710837332920498	-0.672300890261317\\
0	0	-0\\
0.692467880113934	0.0668015435801207	-0.718349350097727\\
0	0	-0\\
0.645233005109949	0.0622448519967348	-0.761445958369134\\
0	0	-0\\
0.595346729285921	0.0574323829650153	-0.801413621867957\\
0	0	-0\\
0.543014046049784	0.0523839119524851	-0.838088104891841\\
0	0	-0\\
0.488450001622282	0.0471201842094283	-0.871318704123389\\
0	0	-0\\
0.431878811365916	0.0416628295222071	-0.900968867902419\\
0	0	-0\\
0.373532938435934	0.0360342733318303	-0.926916757346022\\
0	0	-0\\
0.313652138539355	0.0302576445830056	-0.949055747010669\\
0	0	-0\\
0.252482474727343	0.0243566806823449	-0.967294863039029\\
0	0	-0\\
0.190275306269357	0.0183556299562688	-0.981559156991065\\
0	0	-0\\
0.127286255763985	0.0122791520094321	-0.991790013823246\\
0	0	-0\\
0.0637741587308623	0.0061522163931189	-0.997945392750336\\
0	0	-0\\
1.21898784461137e-16	1.17594291322286e-17	-1\\
0	0	0\\
0	0	1\\
0	0	0\\
0.0632489540243163	0.0102256003836873	0.997945392750336\\
0	0	0\\
0.126238004529689	0.0204091815820136	0.991790013823246\\
0	0	0\\
0.188708315996481	0.0305088970754637	0.981559156991065\\
0	0	0\\
0.250403184515038	0.0404832449666929	0.967294863039029\\
0	0	0\\
0.311069092637107	0.050291238520725	0.949055747010669\\
0	0	0\\
0.37045675113342	0.0598925745882387	0.926916757346022\\
0	0	0\\
0.428322123376601	0.0692477992198524	0.900968867902419\\
0	0	0\\
0.484427428140021	0.0783184697908653	0.871318704123389\\
0	0	0\\
0.538542116691855	0.0870673129702485	0.838088104891841\\
0	0	0\\
0.590443820169281	0.0954583778847568	0.801413621867957\\
0	0	0\\
0.63991926333983	0.103457183848779	0.761445958369135\\
0	0	0\\
0.686765140995063	0.11103086205287	0.718349350097728\\
0	0	0\\
0.730788953375287	0.118148290628741	0.672300890261317\\
0	0	0\\
0.771809797192353	0.124780222535689	0.623489801858734\\
0	0	0\\
0.809659109000073	0.130899405742964	0.57211666012217\\
0	0	0\\
0.844181357857578	0.136480695214207	0.518392568310525\\
0	0	0\\
0.875234684439313	0.141501156233798	0.462538290240835\\
0	0	0\\
0.902691483965435	0.145940158650521	0.404783343122394\\
0	0	0\\
0.926438930557227	0.149779461651283	0.345365054421308\\
0	0	0\\
0.946379440862833	0.153003288716526	0.284527586631032\\
0	0	0\\
0.962431075048179	0.155598392449331	0.222520933956314\\
0	0	0\\
0.974527873505334	0.15755410901181	0.15959989503338\\
0	0	0\\
0.982620127894683	0.15886240194511	0.0960230259076819\\
0	0	0\\
0.986674585407156	0.159517895192939	0.0320515775716553\\
0	0	-0\\
0.986674585407156	0.159517895192939	-0.0320515775716552\\
0	0	-0\\
0.982620127894683	0.15886240194511	-0.0960230259076818\\
0	0	-0\\
0.974527873505334	0.15755410901181	-0.159599895033379\\
0	0	-0\\
0.962431075048179	0.155598392449331	-0.222520933956314\\
0	0	-0\\
0.946379440862833	0.153003288716526	-0.284527586631032\\
0	0	-0\\
0.926438930557227	0.149779461651283	-0.345365054421307\\
0	0	-0\\
0.902691483965435	0.145940158650521	-0.404783343122394\\
0	0	-0\\
0.875234684439313	0.141501156233798	-0.462538290240835\\
0	0	-0\\
0.844181357857578	0.136480695214207	-0.518392568310525\\
0	0	-0\\
0.809659109000073	0.130899405742964	-0.57211666012217\\
0	0	-0\\
0.771809797192353	0.124780222535689	-0.623489801858733\\
0	0	-0\\
0.730788953375287	0.118148290628741	-0.672300890261317\\
0	0	-0\\
0.686765140995063	0.11103086205287	-0.718349350097727\\
0	0	-0\\
0.63991926333983	0.103457183848779	-0.761445958369134\\
0	0	-0\\
0.590443820169282	0.0954583778847568	-0.801413621867957\\
0	0	-0\\
0.538542116691855	0.0870673129702485	-0.838088104891841\\
0	0	-0\\
0.484427428140021	0.0783184697908653	-0.871318704123389\\
0	0	-0\\
0.428322123376601	0.0692477992198524	-0.900968867902419\\
0	0	-0\\
0.37045675113342	0.0598925745882388	-0.926916757346022\\
0	0	-0\\
0.311069092637108	0.0502912385207251	-0.949055747010669\\
0	0	-0\\
0.250403184515038	0.0404832449666929	-0.967294863039029\\
0	0	-0\\
0.188708315996482	0.0305088970754638	-0.981559156991065\\
0	0	-0\\
0.126238004529689	0.0204091815820136	-0.991790013823246\\
0	0	-0\\
0.0632489540243166	0.0102256003836873	-0.997945392750336\\
0	0	-0\\
1.20894901123508e-16	1.95453500596881e-17	-1\\
0	0	0\\
0	0	1\\
0	0	0\\
0.0624638457988266	0.0142569651888948	0.997945392750336\\
0	0	0\\
0.124671014256813	0.0284553454497189	0.991790013823246\\
0	0	0\\
0.186365882775369	0.0425367965924377	0.981559156991065\\
0	0	0\\
0.247294933906245	0.0564434549138439	0.967294863039029\\
0	0	0\\
0.307207797109103	0.070118174971926	0.949055747010669\\
0	0	0\\
0.365858277577774	0.0835047644087471	0.926916757346022\\
0	0	0\\
0.423005367907524	0.0965482148568969	0.900968867902419\\
0	0	0\\
0.478414238446175	0.109194927980673	0.871318704123389\\
0	0	0\\
0.531857202259518	0.121392935723137	0.838088104891841\\
0	0	0\\
0.583114650745764	0.133092113854012	0.801413621867957\\
0	0	0\\
0.631975956054396	0.144244387940892	0.761445958369135\\
0	0	0\\
0.678240336601183	0.154803930897399	0.718349350097728\\
0	0	0\\
0.721717682122779	0.164727351296509	0.672300890261317\\
0	0	0\\
0.762229334880576	0.173973871675236	0.623489801858734\\
0	0	0\\
0.799608823803669	0.182505496097971	0.57211666012217\\
0	0	0\\
0.833702548554198	0.190287166289933	0.518392568310525\\
0	0	0\\
0.864370410704082	0.197286905699141	0.462538290240835\\
0	0	0\\
0.891486389429511	0.203475950894922	0.404783343122394\\
0	0	0\\
0.914939059357543	0.208828869763021	0.345365054421308\\
0	0	0\\
0.934632048436863	0.213323666011612	0.284527586631032\\
0	0	0\\
0.95048443395121	0.216941869558779	0.222520933956314\\
0	0	0\\
0.962431075048179	0.219668612430044	0.15959989503338\\
0	0	0\\
0.970422880416957	0.221492689854063	0.0960230259076819\\
0	0	0\\
0.974427010015047	0.22240660630544	0.0320515775716553\\
0	0	-0\\
0.974427010015047	0.22240660630544	-0.0320515775716552\\
0	0	-0\\
0.970422880416957	0.221492689854063	-0.0960230259076818\\
0	0	-0\\
0.962431075048179	0.219668612430044	-0.159599895033379\\
0	0	-0\\
0.95048443395121	0.216941869558779	-0.222520933956314\\
0	0	-0\\
0.934632048436863	0.213323666011612	-0.284527586631032\\
0	0	-0\\
0.914939059357543	0.208828869763021	-0.345365054421307\\
0	0	-0\\
0.891486389429511	0.203475950894922	-0.404783343122394\\
0	0	-0\\
0.864370410704082	0.197286905699141	-0.462538290240835\\
0	0	-0\\
0.833702548554198	0.190287166289933	-0.518392568310525\\
0	0	-0\\
0.799608823803669	0.182505496097971	-0.57211666012217\\
0	0	-0\\
0.762229334880576	0.173973871675236	-0.623489801858733\\
0	0	-0\\
0.721717682122779	0.164727351296509	-0.672300890261317\\
0	0	-0\\
0.678240336601183	0.154803930897399	-0.718349350097727\\
0	0	-0\\
0.631975956054396	0.144244387940892	-0.761445958369134\\
0	0	-0\\
0.583114650745764	0.133092113854012	-0.801413621867957\\
0	0	-0\\
0.531857202259518	0.121392935723137	-0.838088104891841\\
0	0	-0\\
0.478414238446175	0.109194927980673	-0.871318704123389\\
0	0	-0\\
0.423005367907524	0.0965482148568969	-0.900968867902419\\
0	0	-0\\
0.365858277577775	0.0835047644087472	-0.926916757346022\\
0	0	-0\\
0.307207797109104	0.0701181749719261	-0.949055747010669\\
0	0	-0\\
0.247294933906245	0.0564434549138439	-0.967294863039029\\
0	0	-0\\
0.18636588277537	0.0425367965924377	-0.981559156991065\\
0	0	-0\\
0.124671014256813	0.0284553454497189	-0.991790013823246\\
0	0	-0\\
0.0624638457988269	0.0142569651888948	-0.997945392750336\\
0	0	-0\\
1.19394234705288e-16	2.7250954951288e-17	-1\\
0	0	0\\
0	0	1\\
0	0	0\\
0.0614220602324966	0.0182297450660316	0.997945392750336\\
0	0	0\\
0.122591724044507	0.0363845801993188	0.991790013823246\\
0	0	0\\
0.183257632166577	0.0543899032880991	0.981559156991065\\
0	0	0\\
0.243170495369435	0.0721717265976708	0.967294863039029\\
0	0	0\\
0.302084118846912	0.089656980801866	0.949055747010669\\
0	0	0\\
0.359756413883208	0.106773815240584	0.926916757346022\\
0	0	0\\
0.415950392647348	0.123451893169567	0.900968867902419\\
0	0	0\\
0.470435142027021	0.139622680789169	0.871318704123389\\
0	0	0\\
0.522986772500083	0.155219728864436	0.838088104891841\\
0	0	0\\
0.573389338144632	0.170178945779272	0.801413621867957\\
0	0	0\\
0.621435724007117	0.184438860902632	0.761445958369135\\
0	0	0\\
0.666928497182112	0.197940877184531	0.718349350097728\\
0	0	0\\
0.709680718106472	0.210629511943894	0.672300890261317\\
0	0	0\\
0.749516708734096	0.222452624858791	0.623489801858734\\
0	0	0\\
0.786272774434701	0.233361632222205	0.57211666012217\\
0	0	0\\
0.819797876650173	0.243311706582904	0.518392568310525\\
0	0	0\\
0.849954253544396	0.252261960951057	0.462538290240835\\
0	0	0\\
0.87661798609619	0.260175616811641	0.404783343122394\\
0	0	0\\
0.899679507309146	0.267020155255251	0.345365054421308\\
0	0	0\\
0.91904405244592	0.272767450605274	0.284527586631032\\
0	0	0\\
0.934632048436863	0.277393885992325	0.222520933956314\\
0	0	0\\
0.946379440862833	0.280880450401032	0.15959989503338\\
0	0	0\\
0.954237957168544	0.283212816790373	0.0960230259076819\\
0	0	0\\
0.958175305024849	0.284381400966564	0.0320515775716553\\
0	0	-0\\
0.958175305024849	0.284381400966564	-0.0320515775716552\\
0	0	-0\\
0.954237957168544	0.283212816790373	-0.0960230259076818\\
0	0	-0\\
0.946379440862833	0.280880450401032	-0.159599895033379\\
0	0	-0\\
0.934632048436863	0.277393885992325	-0.222520933956314\\
0	0	-0\\
0.91904405244592	0.272767450605274	-0.284527586631032\\
0	0	-0\\
0.899679507309146	0.267020155255251	-0.345365054421307\\
0	0	-0\\
0.87661798609619	0.260175616811641	-0.404783343122394\\
0	0	-0\\
0.849954253544396	0.252261960951057	-0.462538290240835\\
0	0	-0\\
0.819797876650173	0.243311706582904	-0.518392568310525\\
0	0	-0\\
0.786272774434701	0.233361632222204	-0.57211666012217\\
0	0	-0\\
0.749516708734096	0.222452624858791	-0.623489801858733\\
0	0	-0\\
0.709680718106472	0.210629511943894	-0.672300890261317\\
0	0	-0\\
0.666928497182112	0.197940877184531	-0.718349350097727\\
0	0	-0\\
0.621435724007118	0.184438860902632	-0.761445958369134\\
0	0	-0\\
0.573389338144632	0.170178945779272	-0.801413621867957\\
0	0	-0\\
0.522986772500083	0.155219728864436	-0.838088104891841\\
0	0	-0\\
0.470435142027021	0.139622680789169	-0.871318704123389\\
0	0	-0\\
0.415950392647348	0.123451893169567	-0.900968867902419\\
0	0	-0\\
0.359756413883208	0.106773815240584	-0.926916757346022\\
0	0	-0\\
0.302084118846913	0.0896569808018661	-0.949055747010669\\
0	0	-0\\
0.243170495369435	0.0721717265976708	-0.967294863039029\\
0	0	-0\\
0.183257632166577	0.0543899032880991	-0.981559156991065\\
0	0	-0\\
0.122591724044507	0.0363845801993188	-0.991790013823246\\
0	0	-0\\
0.0614220602324969	0.0182297450660317	-0.997945392750336\\
0	0	-0\\
1.17402951766681e-16	3.48445798236815e-17	-1\\
0	0	0\\
0	0	1\\
0	0	0\\
0.0601278782456807	0.0221276150104241	0.997945392750336\\
0	0	0\\
0.12000867814226	0.0441643029044118	0.991790013823246\\
0	0	0\\
0.179396336638573	0.066019510204552	0.981559156991065\\
0	0	0\\
0.238046817107244	0.0876034291761211	0.967294863039029\\
0	0	0\\
0.295719112143539	0.108827366866329	0.949055747010669\\
0	0	0\\
0.352176233916486	0.129604109562686	0.926916757346022\\
0	0	0\\
0.407186188002705	0.149848281172856	0.900968867902419\\
0	0	0\\
0.460522926701257	0.169476694053331	0.871318704123389\\
0	0	0\\
0.511967277912136	0.188408690845304	0.838088104891841\\
0	0	0\\
0.561307845761437	0.206566475913056	0.801413621867957\\
0	0	0\\
0.608341879272348	0.223875435022911	0.761445958369135\\
0	0	0\\
0.652876105512406	0.240264441949127	0.718349350097728\\
0	0	0\\
0.694727523793429	0.255666150746813	0.672300890261317\\
0	0	0\\
0.733724157660597	0.270017272490863	0.623489801858734\\
0	0	0\\
0.7697057615806	0.283258835343724	0.57211666012217\\
0	0	0\\
0.802524479424899	0.295336426883329	0.518392568310525\\
0	0	0\\
0.832045452042282	0.306200417695405	0.462538290240835\\
0	0	0\\
0.858147371424032	0.315806165311387	0.404783343122394\\
0	0	0\\
0.880722979184567	0.324114197653894	0.345365054421308\\
0	0	0\\
0.899679507309146	0.331090375235964	0.284527586631032\\
0	0	0\\
0.914939059357543	0.336706031447527	0.222520933956314\\
0	0	0\\
0.926438930557227	0.340938090352655	0.15959989503338\\
0	0	0\\
0.934131865470724	0.343769161513533	0.0960230259076819\\
0	0	0\\
0.937986252178345	0.345187611451497	0.0320515775716553\\
0	0	-0\\
0.937986252178345	0.345187611451497	-0.0320515775716552\\
0	0	-0\\
0.934131865470724	0.343769161513533	-0.0960230259076818\\
0	0	-0\\
0.926438930557227	0.340938090352655	-0.159599895033379\\
0	0	-0\\
0.914939059357543	0.336706031447527	-0.222520933956314\\
0	0	-0\\
0.899679507309147	0.331090375235964	-0.284527586631032\\
0	0	-0\\
0.880722979184567	0.324114197653894	-0.345365054421307\\
0	0	-0\\
0.858147371424032	0.315806165311387	-0.404783343122394\\
0	0	-0\\
0.832045452042282	0.306200417695405	-0.462538290240835\\
0	0	-0\\
0.802524479424899	0.295336426883329	-0.518392568310525\\
0	0	-0\\
0.769705761580599	0.283258835343724	-0.57211666012217\\
0	0	-0\\
0.733724157660597	0.270017272490863	-0.623489801858733\\
0	0	-0\\
0.694727523793429	0.255666150746813	-0.672300890261317\\
0	0	-0\\
0.652876105512407	0.240264441949127	-0.718349350097727\\
0	0	-0\\
0.608341879272349	0.223875435022911	-0.761445958369134\\
0	0	-0\\
0.561307845761437	0.206566475913056	-0.801413621867957\\
0	0	-0\\
0.511967277912136	0.188408690845304	-0.838088104891841\\
0	0	-0\\
0.460522926701257	0.169476694053331	-0.871318704123389\\
0	0	-0\\
0.407186188002705	0.149848281172856	-0.900968867902419\\
0	0	-0\\
0.352176233916486	0.129604109562686	-0.926916757346022\\
0	0	-0\\
0.29571911214354	0.108827366866329	-0.949055747010669\\
0	0	-0\\
0.238046817107244	0.0876034291761211	-0.967294863039029\\
0	0	-0\\
0.179396336638573	0.0660195102045521	-0.981559156991065\\
0	0	-0\\
0.12000867814226	0.0441643029044118	-0.991790013823246\\
0	0	-0\\
0.060127878245681	0.0221276150104242	-0.997945392750336\\
0	0	-0\\
1.14929234916411e-16	4.22950208434406e-17	-1\\
0	0	0\\
0	0	1\\
0	0	0\\
0.0585866179097638	0.0259345578383802	0.997945392750336\\
0	0	0\\
0.116932490839746	0.0517625450156572	0.991790013823246\\
0	0	0\\
0.174797863082928	0.0773778287924339	0.981559156991065\\
0	0	0\\
0.231944953412677	0.10267515047321	0.967294863039029\\
0	0	0\\
0.288138932176817	0.127550557936941	0.949055747010669\\
0	0	0\\
0.343148886263036	0.151901832798601	0.926916757346022\\
0	0	0\\
0.396748767970395	0.17562891044645	0.900968867902419\\
0	0	0\\
0.448718323887819	0.198634291228991	0.871318704123389\\
0	0	0\\
0.49884399996261	0.220823441101951	0.838088104891841\\
0	0	0\\
0.546919819039852	0.242105180088943	0.801413621867957\\
0	0	0\\
0.592748227266726	0.262392056959551	0.761445958369135\\
0	0	0\\
0.636140905883665	0.281600708585193	0.718349350097728\\
0	0	0\\
0.676919545066531	0.299652202496096	0.672300890261317\\
0	0	0\\
0.714916576639932	0.316472361231746	0.623489801858734\\
0	0	0\\
0.749975862650795	0.331992067151987	0.57211666012217\\
0	0	0\\
0.781953336972708	0.346147546456225	0.518392568310525\\
0	0	0\\
0.810717597304535	0.358880631243658	0.462538290240835\\
0	0	0\\
0.836150445130658	0.370138998537658	0.404783343122394\\
0	0	0\\
0.858147371424032	0.3798763852921	0.345365054421308\\
0	0	0\\
0.87661798609619	0.388052778496147	0.284527586631032\\
0	0	0\\
0.891486389429511	0.394634579596294	0.222520933956314\\
0	0	0\\
0.902691483965435	0.399594742560028	0.15959989503338\\
0	0	0\\
0.910187225567029	0.402912885013779	0.0960230259076819\\
0	0	0\\
0.913942812624221	0.404575371998466	0.0320515775716553\\
0	0	-0\\
0.913942812624221	0.404575371998466	-0.0320515775716552\\
0	0	-0\\
0.910187225567029	0.402912885013779	-0.0960230259076818\\
0	0	-0\\
0.902691483965435	0.399594742560028	-0.159599895033379\\
0	0	-0\\
0.891486389429511	0.394634579596294	-0.222520933956314\\
0	0	-0\\
0.87661798609619	0.388052778496147	-0.284527586631032\\
0	0	-0\\
0.858147371424032	0.3798763852921	-0.345365054421307\\
0	0	-0\\
0.836150445130658	0.370138998537658	-0.404783343122394\\
0	0	-0\\
0.810717597304535	0.358880631243658	-0.462538290240835\\
0	0	-0\\
0.781953336972708	0.346147546456225	-0.518392568310525\\
0	0	-0\\
0.749975862650795	0.331992067151987	-0.57211666012217\\
0	0	-0\\
0.714916576639932	0.316472361231746	-0.623489801858733\\
0	0	-0\\
0.676919545066531	0.299652202496096	-0.672300890261317\\
0	0	-0\\
0.636140905883665	0.281600708585193	-0.718349350097727\\
0	0	-0\\
0.592748227266726	0.262392056959551	-0.761445958369134\\
0	0	-0\\
0.546919819039852	0.242105180088943	-0.801413621867957\\
0	0	-0\\
0.49884399996261	0.220823441101951	-0.838088104891841\\
0	0	-0\\
0.448718323887819	0.198634291228991	-0.871318704123389\\
0	0	-0\\
0.396748767970395	0.17562891044645	-0.900968867902419\\
0	0	-0\\
0.343148886263037	0.151901832798601	-0.926916757346022\\
0	0	-0\\
0.288138932176818	0.127550557936942	-0.949055747010669\\
0	0	-0\\
0.231944953412677	0.10267515047321	-0.967294863039029\\
0	0	-0\\
0.174797863082928	0.0773778287924339	-0.981559156991065\\
0	0	-0\\
0.116932490839746	0.0517625450156572	-0.991790013823246\\
0	0	-0\\
0.0585866179097641	0.0259345578383803	-0.997945392750336\\
0	0	-0\\
1.11983249187625e-16	4.95716625503004e-17	-1\\
0	0	0\\
0	0	1\\
0	0	0\\
0.0568046125940656	0.0296349300052332	0.997945392750336\\
0	0	0\\
0.113375802850431	0.0591480837264023	0.991790013823246\\
0	0	0\\
0.169481107613851	0.0884181852843154	0.981559156991065\\
0	0	0\\
0.224889978152501	0.117324957553254	0.967294863039029\\
0	0	0\\
0.279374727532174	0.145749616405482	0.949055747010669\\
0	0	0\\
0.332711466230726	0.173575358820705	0.926916757346022\\
0	0	0\\
0.384681022148151	0.200687842854736	0.900968867902419\\
0	0	0\\
0.435069841231748	0.22697565749507	0.871318704123389\\
0	0	0\\
0.483670865015536	0.25233078047263	0.838088104891841\\
0	0	0\\
0.530284381467899	0.276649022148446	0.801413621867957\\
0	0	0\\
0.574718845651168	0.299830453651225	0.761445958369135\\
0	0	0\\
0.616791666820851	0.32177981750652	0.718349350097728\\
0	0	0\\
0.656329958730169	0.342406919070126	0.672300890261317\\
0	0	0\\
0.69317125005673	0.36162699715722	0.623489801858734\\
0	0	0\\
0.727164152032041	0.379361072344247	0.57211666012217\\
0	0	0\\
0.758168980530431	0.395536271512317	0.518392568310525\\
0	0	0\\
0.786058330061085	0.410086127298478	0.462538290240835\\
0	0	0\\
0.810717597304535	0.422950851224371	0.404783343122394\\
0	0	0\\
0.832045452042281	0.434077579379911	0.345365054421308\\
0	0	0\\
0.849954253544396	0.443420589652429	0.284527586631032\\
0	0	0\\
0.864370410704082	0.450941489608648	0.222520933956314\\
0	0	0\\
0.875234684439313	0.456609374257418	0.15959989503338\\
0	0	0\\
0.882502431118931	0.460400953044961	0.0960230259076819\\
0	0	0\\
0.886143786012904	0.462300645560748	0.0320515775716553\\
0	0	-0\\
0.886143786012904	0.462300645560748	-0.0320515775716552\\
0	0	-0\\
0.882502431118931	0.460400953044961	-0.0960230259076818\\
0	0	-0\\
0.875234684439313	0.456609374257418	-0.159599895033379\\
0	0	-0\\
0.864370410704082	0.450941489608648	-0.222520933956314\\
0	0	-0\\
0.849954253544397	0.443420589652429	-0.284527586631032\\
0	0	-0\\
0.832045452042282	0.434077579379911	-0.345365054421307\\
0	0	-0\\
0.810717597304535	0.422950851224371	-0.404783343122394\\
0	0	-0\\
0.786058330061085	0.410086127298478	-0.462538290240835\\
0	0	-0\\
0.758168980530431	0.395536271512317	-0.518392568310525\\
0	0	-0\\
0.727164152032041	0.379361072344247	-0.57211666012217\\
0	0	-0\\
0.69317125005673	0.36162699715722	-0.623489801858733\\
0	0	-0\\
0.656329958730169	0.342406919070126	-0.672300890261317\\
0	0	-0\\
0.616791666820851	0.32177981750652	-0.718349350097727\\
0	0	-0\\
0.574718845651168	0.299830453651225	-0.761445958369134\\
0	0	-0\\
0.530284381467899	0.276649022148446	-0.801413621867957\\
0	0	-0\\
0.483670865015536	0.25233078047263	-0.838088104891841\\
0	0	-0\\
0.435069841231748	0.22697565749507	-0.871318704123389\\
0	0	-0\\
0.384681022148151	0.200687842854736	-0.900968867902419\\
0	0	-0\\
0.332711466230727	0.173575358820705	-0.926916757346022\\
0	0	-0\\
0.279374727532175	0.145749616405482	-0.949055747010669\\
0	0	-0\\
0.224889978152501	0.117324957553254	-0.967294863039029\\
0	0	-0\\
0.169481107613851	0.0884181852843155	-0.981559156991065\\
0	0	-0\\
0.113375802850431	0.0591480837264023	-0.991790013823246\\
0	0	-0\\
0.0568046125940659	0.0296349300052333	-0.997945392750336\\
0	0	-0\\
1.08577100267596e-16	5.66446036626528e-17	-1\\
0	0	0\\
0	0	1\\
0	0	0\\
0.0547891849406672	0.0332135258880221	0.997945392750336\\
0	0	0\\
0.10935322936817	0.0662905702738913	0.991790013823246\\
0	0	0\\
0.163467917920005	0.0990952124872224	0.981559156991065\\
0	0	0\\
0.216910881733348	0.131492651216587	0.967294863039029\\
0	0	0\\
0.26946251220641	0.163349758437018	0.949055747010669\\
0	0	0\\
0.320906863417289	0.194535626461617	0.926916757346022\\
0	0	0\\
0.371032539492079	0.224922105869325	0.900968867902419\\
0	0	0\\
0.419633563275867	0.254384332098375	0.871318704123389\\
0	0	0\\
0.466510222737037	0.282801238541565	0.838088104891841\\
0	0	0\\
0.511469891626852	0.310056054034912	0.801413621867957\\
0	0	0\\
0.554327821022024	0.336036782695415	0.761445958369135\\
0	0	0\\
0.594907898497672	0.360636664136158	0.718349350097728\\
0	0	0\\
0.633043371811049	0.383754612167644	0.672300890261317\\
0	0	0\\
0.668577534122277	0.405295630182627	0.623489801858734\\
0	0	0\\
0.701364367936365	0.42517120151755	0.57211666012217\\
0	0	0\\
0.731269145120418	0.4432996531865	0.518392568310525\\
0	0	0\\
0.758168980530431	0.45960649149303	0.462538290240835\\
0	0	0\\
0.781953336972708	0.474024708140731	0.404783343122394\\
0	0	0\\
0.802524479424899	0.486495055584701	0.345365054421308\\
0	0	0\\
0.819797876650173	0.496966290492411	0.284527586631032\\
0	0	0\\
0.833702548554198	0.505395384313553	0.222520933956314\\
0	0	0\\
0.844181357857578	0.511747700093581	0.15959989503338\\
0	0	0\\
0.851191244885188	0.515997134804387	0.0960230259076819\\
0	0	0\\
0.854703404507615	0.518126226607243	0.0320515775716553\\
0	0	-0\\
0.854703404507615	0.518126226607243	-0.0320515775716552\\
0	0	-0\\
0.851191244885188	0.515997134804387	-0.0960230259076818\\
0	0	-0\\
0.844181357857578	0.511747700093581	-0.159599895033379\\
0	0	-0\\
0.833702548554198	0.505395384313553	-0.222520933956314\\
0	0	-0\\
0.819797876650173	0.496966290492411	-0.284527586631032\\
0	0	-0\\
0.802524479424899	0.486495055584701	-0.345365054421307\\
0	0	-0\\
0.781953336972708	0.474024708140731	-0.404783343122394\\
0	0	-0\\
0.758168980530431	0.45960649149303	-0.462538290240835\\
0	0	-0\\
0.731269145120418	0.4432996531865	-0.518392568310525\\
0	0	-0\\
0.701364367936365	0.42517120151755	-0.57211666012217\\
0	0	-0\\
0.668577534122277	0.405295630182627	-0.623489801858733\\
0	0	-0\\
0.633043371811049	0.383754612167644	-0.672300890261317\\
0	0	-0\\
0.594907898497672	0.360636664136158	-0.718349350097727\\
0	0	-0\\
0.554327821022024	0.336036782695415	-0.761445958369134\\
0	0	-0\\
0.511469891626852	0.310056054034912	-0.801413621867957\\
0	0	-0\\
0.466510222737037	0.282801238541565	-0.838088104891841\\
0	0	-0\\
0.419633563275867	0.254384332098375	-0.871318704123389\\
0	0	-0\\
0.37103253949208	0.224922105869325	-0.900968867902419\\
0	0	-0\\
0.320906863417289	0.194535626461617	-0.926916757346022\\
0	0	-0\\
0.269462512206411	0.163349758437018	-0.949055747010669\\
0	0	-0\\
0.216910881733348	0.131492651216587	-0.967294863039029\\
0	0	-0\\
0.163467917920005	0.0990952124872225	-0.981559156991065\\
0	0	-0\\
0.10935322936817	0.0662905702738913	-0.991790013823246\\
0	0	-0\\
0.0547891849406675	0.0332135258880222	-0.997945392750336\\
0	0	-0\\
1.04724784752852e-16	6.3484779948326e-17	-1\\
0	0	0\\
0	0	1\\
0	0	0\\
0.0525486167741042	0.0366556402686582	0.997945392750336\\
0	0	0\\
0.104881300010241	0.0731606546488422	0.991790013823246\\
0	0	0\\
0.156783003487667	0.109365036206163	0.981559156991065\\
0	0	0\\
0.208040451973913	0.145120013370986	0.967294863039029\\
0	0	0\\
0.258443017618462	0.180278661272722	0.949055747010669\\
0	0	0\\
0.307783585467763	0.214696505485637	0.926916757346022\\
0	0	0\\
0.355859404545005	0.248232115705256	0.900968867902419\\
0	0	0\\
0.402472920997369	0.28074768691582	0.871318704123389\\
0	0	0\\
0.447432589887183	0.312109605660657	0.838088104891841\\
0	0	0\\
0.490553662291162	0.342188999088534	0.801413621867957\\
0	0	0\\
0.531658944473356	0.370862264519845	0.761445958369135\\
0	0	0\\
0.570579526012222	0.398011577356539	0.718349350097728\\
0	0	0\\
0.60715547388978	0.423525375248659	0.672300890261317\\
0	0	0\\
0.641236489690684	0.447298816527974	0.623489801858734\\
0	0	0\\
0.672682527210654	0.46923421102488	0.57211666012217\\
0	0	0\\
0.701364367936365	0.489241421498263	0.518392568310525\\
0	0	0\\
0.727164152032041	0.507238234028753	0.462538290240835\\
0	0	0\\
0.749975862650795	0.523150695853359	0.404783343122394\\
0	0	0\\
0.7697057615806	0.536913419253232	0.345365054421308\\
0	0	0\\
0.786272774434701	0.548469850245825	0.284527586631032\\
0	0	0\\
0.799608823803669	0.557772500977345	0.222520933956314\\
0	0	0\\
0.809659109000073	0.564783144860522	0.15959989503338\\
0	0	0\\
0.816382331246262	0.569472973655863	0.0960230259076819\\
0	0	0\\
0.819750863379899	0.571822715850882	0.0320515775716553\\
0	0	-0\\
0.819750863379899	0.571822715850882	-0.0320515775716552\\
0	0	-0\\
0.816382331246262	0.569472973655863	-0.0960230259076818\\
0	0	-0\\
0.809659109000073	0.564783144860522	-0.159599895033379\\
0	0	-0\\
0.799608823803669	0.557772500977345	-0.222520933956314\\
0	0	-0\\
0.786272774434701	0.548469850245825	-0.284527586631032\\
0	0	-0\\
0.7697057615806	0.536913419253232	-0.345365054421307\\
0	0	-0\\
0.749975862650795	0.523150695853359	-0.404783343122394\\
0	0	-0\\
0.727164152032041	0.507238234028753	-0.462538290240835\\
0	0	-0\\
0.701364367936365	0.489241421498263	-0.518392568310525\\
0	0	-0\\
0.672682527210654	0.46923421102488	-0.57211666012217\\
0	0	-0\\
0.641236489690685	0.447298816527974	-0.623489801858733\\
0	0	-0\\
0.60715547388978	0.423525375248659	-0.672300890261317\\
0	0	-0\\
0.570579526012222	0.398011577356539	-0.718349350097727\\
0	0	-0\\
0.531658944473356	0.370862264519846	-0.761445958369134\\
0	0	-0\\
0.490553662291162	0.342188999088534	-0.801413621867957\\
0	0	-0\\
0.447432589887183	0.312109605660657	-0.838088104891841\\
0	0	-0\\
0.402472920997369	0.28074768691582	-0.871318704123389\\
0	0	-0\\
0.355859404545005	0.248232115705256	-0.900968867902419\\
0	0	-0\\
0.307783585467763	0.214696505485638	-0.926916757346022\\
0	0	-0\\
0.258443017618463	0.180278661272723	-0.949055747010669\\
0	0	-0\\
0.208040451973913	0.145120013370986	-0.967294863039029\\
0	0	-0\\
0.156783003487667	0.109365036206163	-0.981559156991065\\
0	0	-0\\
0.104881300010241	0.0731606546488423	-0.991790013823246\\
0	0	-0\\
0.0525486167741045	0.0366556402686583	-0.997945392750336\\
0	0	-0\\
1.00442132634163e-16	7.00640836557489e-17	-1\\
0	0	0\\
0	0	1\\
0	0	0\\
0.0500921150695736	0.0399471287608202	0.997945392750336\\
0	0	0\\
0.0999783908936013	0.0797301062009299	0.991790013823246\\
0	0	0\\
0.14945383406415	0.119185455532606	0.981559156991065\\
0	0	0\\
0.198315139372781	0.158151046262298	0.967294863039029\\
0	0	0\\
0.246361525235266	0.196466760419606	0.949055747010669\\
0	0	0\\
0.293395558746178	0.233975150516361	0.926916757346022\\
0	0	0\\
0.339223966973052	0.270522086532133	0.900968867902419\\
0	0	0\\
0.383658431156321	0.305957389267538	0.871318704123389\\
0	0	0\\
0.426516360551494	0.340135447462789	0.838088104891841\\
0	0	0\\
0.467621642733687	0.37291581614559	0.801413621867957\\
0	0	0\\
0.50680536728136	0.404163793749658	0.761445958369135\\
0	0	0\\
0.543906519865464	0.433750975632346	0.718349350097728\\
0	0	0\\
0.578772643891858	0.461555781716869	0.672300890261317\\
0	0	0\\
0.611260466978157	0.487463956090912	0.623489801858734\\
0	0	0\\
0.641236489690684	0.511369036508687	0.57211666012217\\
0	0	0\\
0.668577534122277	0.533172791867133	0.518392568310525\\
0	0	0\\
0.69317125005673	0.552785625858592	0.462538290240835\\
0	0	0\\
0.714916576639932	0.570126945141254	0.404783343122394\\
0	0	0\\
0.733724157660597	0.585125490514483	0.345365054421308\\
0	0	0\\
0.749516708734096	0.597719629738165	0.284527586631032\\
0	0	0\\
0.762229334880576	0.607857610792794	0.222520933956314\\
0	0	0\\
0.771809797192353	0.615497774539627	0.15959989503338\\
0	0	0\\
0.778218727494784	0.620608725907018	0.0960230259076819\\
0	0	0\\
0.781429790118546	0.623169462899502	0.0320515775716553\\
0	0	-0\\
0.781429790118546	0.623169462899502	-0.0320515775716552\\
0	0	-0\\
0.778218727494784	0.620608725907018	-0.0960230259076818\\
0	0	-0\\
0.771809797192353	0.615497774539627	-0.159599895033379\\
0	0	-0\\
0.762229334880576	0.607857610792794	-0.222520933956314\\
0	0	-0\\
0.749516708734096	0.597719629738165	-0.284527586631032\\
0	0	-0\\
0.733724157660597	0.585125490514483	-0.345365054421307\\
0	0	-0\\
0.714916576639932	0.570126945141254	-0.404783343122394\\
0	0	-0\\
0.69317125005673	0.552785625858592	-0.462538290240835\\
0	0	-0\\
0.668577534122277	0.533172791867133	-0.518392568310525\\
0	0	-0\\
0.641236489690684	0.511369036508687	-0.57211666012217\\
0	0	-0\\
0.611260466978157	0.487463956090912	-0.623489801858733\\
0	0	-0\\
0.578772643891858	0.461555781716869	-0.672300890261317\\
0	0	-0\\
0.543906519865464	0.433750975632346	-0.718349350097727\\
0	0	-0\\
0.50680536728136	0.404163793749658	-0.761445958369134\\
0	0	-0\\
0.467621642733687	0.37291581614559	-0.801413621867957\\
0	0	-0\\
0.426516360551494	0.340135447462789	-0.838088104891841\\
0	0	-0\\
0.383658431156321	0.305957389267538	-0.871318704123389\\
0	0	-0\\
0.339223966973052	0.270522086532133	-0.900968867902419\\
0	0	-0\\
0.293395558746179	0.233975150516362	-0.926916757346022\\
0	0	-0\\
0.246361525235267	0.196466760419606	-0.949055747010669\\
0	0	-0\\
0.198315139372781	0.158151046262298	-0.967294863039029\\
0	0	-0\\
0.14945383406415	0.119185455532606	-0.981559156991065\\
0	0	-0\\
0.0999783908936013	0.0797301062009299	-0.991790013823246\\
0	0	-0\\
0.0500921150695738	0.0399471287608204	-0.997945392750336\\
0	0	-0\\
9.57467422477103e-17	7.63554790147315e-17	-1\\
0	0	0\\
0	0	1\\
0	0	0\\
0.047429774119497	0.0430744659322717	0.997945392750336\\
0	0	0\\
0.0946646491234824	0.0859719296445837	0.991790013823246\\
0	0	0\\
0.141510526778716	0.128516116257065	0.981559156991065\\
0	0	0\\
0.187774907325503	0.170532202581226	0.967294863039029\\
0	0	0\\
0.233267680500498	0.211847535505937	0.949055747010669\\
0	0	0\\
0.277801906740556	0.252292341466101	0.926916757346022\\
0	0	0\\
0.321194585357495	0.291700424078643	0.900968867902419\\
0	0	0\\
0.363267406527177	0.329909847079101	0.871318704123389\\
0	0	0\\
0.403847484002825	0.366763599752471	0.838088104891841\\
0	0	0\\
0.442768065541692	0.402110242123913	0.801413621867957\\
0	0	0\\
0.479869218125795	0.435804527258091	0.761445958369135\\
0	0	0\\
0.514998485160996	0.467707998109987	0.718349350097728\\
0	0	0\\
0.548011512953841	0.497689556474599	0.672300890261317\\
0	0	0\\
0.578772643891858	0.525626001697582	0.623489801858734\\
0	0	0\\
0.60715547388978	0.551402536933165	0.57211666012217\\
0	0	0\\
0.633043371811049	0.574913240869016	0.518392568310525\\
0	0	0\\
0.656329958730169	0.596061502979634	0.462538290240835\\
0	0	0\\
0.676919545066531	0.614760420519717	0.404783343122394\\
0	0	0\\
0.694727523793429	0.630933155626187	0.345365054421308\\
0	0	0\\
0.709680718106472	0.644513251061452	0.284527586631032\\
0	0	0\\
0.72171768212278	0.655444903300447	0.222520933956314\\
0	0	0\\
0.730788953375287	0.663683191839289	0.15959989503338\\
0	0	0\\
0.736857256064637	0.669194263783266	0.0960230259076819\\
0	0	0\\
0.739897654233431	0.671955472955637	0.0320515775716553\\
0	0	-0\\
0.739897654233431	0.671955472955637	-0.0320515775716552\\
0	0	-0\\
0.736857256064637	0.669194263783266	-0.0960230259076818\\
0	0	-0\\
0.730788953375287	0.663683191839289	-0.159599895033379\\
0	0	-0\\
0.72171768212278	0.655444903300447	-0.222520933956314\\
0	0	-0\\
0.709680718106472	0.644513251061452	-0.284527586631032\\
0	0	-0\\
0.694727523793429	0.630933155626187	-0.345365054421307\\
0	0	-0\\
0.676919545066531	0.614760420519717	-0.404783343122394\\
0	0	-0\\
0.656329958730169	0.596061502979634	-0.462538290240835\\
0	0	-0\\
0.633043371811049	0.574913240869017	-0.518392568310525\\
0	0	-0\\
0.60715547388978	0.551402536933165	-0.57211666012217\\
0	0	-0\\
0.578772643891858	0.525626001697582	-0.623489801858733\\
0	0	-0\\
0.548011512953841	0.497689556474599	-0.672300890261317\\
0	0	-0\\
0.514998485160996	0.467707998109988	-0.718349350097727\\
0	0	-0\\
0.479869218125796	0.435804527258091	-0.761445958369134\\
0	0	-0\\
0.442768065541692	0.402110242123913	-0.801413621867957\\
0	0	-0\\
0.403847484002825	0.366763599752471	-0.838088104891841\\
0	0	-0\\
0.363267406527177	0.329909847079102	-0.871318704123389\\
0	0	-0\\
0.321194585357495	0.291700424078643	-0.900968867902419\\
0	0	-0\\
0.277801906740556	0.252292341466101	-0.926916757346022\\
0	0	-0\\
0.233267680500498	0.211847535505938	-0.949055747010669\\
0	0	-0\\
0.187774907325503	0.170532202581226	-0.967294863039029\\
0	0	-0\\
0.141510526778716	0.128516116257065	-0.981559156991065\\
0	0	-0\\
0.0946646491234824	0.0859719296445838	-0.991790013823246\\
0	0	-0\\
0.0474297741194973	0.0430744659322719	-0.997945392750336\\
0	0	-0\\
9.06579079597499e-17	8.23331133322438e-17	-1\\
0	0	0\\
0	0	1\\
0	0	0\\
0.0445725340539088	0.0460248008837636	0.997945392750336\\
0	0	0\\
0.0889619100046116	0.0918604759884069	0.991790013823246\\
0	0	0\\
0.132985722384836	0.137318676693204	0.981559156991065\\
0	0	0\\
0.176463067906432	0.182212605500704	0.967294863039029\\
0	0	0\\
0.219215288830792	0.226357783627721	0.949055747010669\\
0	0	0\\
0.261066707111814	0.269572809068219	0.926916757346022\\
0	0	0\\
0.301845346294681	0.31168010201307	0.900968867902419\\
0	0	0\\
0.341383638203999	0.352506634563577	0.871318704123389\\
0	0	0\\
0.379519111517376	0.391884641740226	0.838088104891841\\
0	0	0\\
0.416095059394934	0.429652310864972	0.801413621867957\\
0	0	0\\
0.450961183421327	0.465654446484243	0.761445958369135\\
0	0	0\\
0.483974211214172	0.499743108100344	0.718349350097728\\
0	0	0\\
0.514998485160996	0.5317782180907	0.672300890261317\\
0	0	0\\
0.543906519865464	0.561628137316852	0.623489801858734\\
0	0	0\\
0.570579526012222	0.589170206057911	0.57211666012217\\
0	0	0\\
0.594907898497672	0.614291248045665	0.518392568310525\\
0	0	0\\
0.616791666820851	0.636888035530141	0.462538290240835\\
0	0	0\\
0.636140905883665	0.656867713464567	0.404783343122394\\
0	0	0\\
0.652876105512406	0.674148181066685	0.345365054421308\\
0	0	0\\
0.666928497182112	0.688658429188469	0.284527586631032\\
0	0	0\\
0.678240336601183	0.700338832107947	0.222520933956314\\
0	0	0\\
0.686765140995063	0.709141392544086	0.15959989503338\\
0	0	0\\
0.692467880113934	0.715029938887909	0.0960230259076819\\
0	0	0\\
0.695325120179522	0.717980273839401	0.0320515775716553\\
0	0	-0\\
0.695325120179522	0.717980273839401	-0.0320515775716552\\
0	0	-0\\
0.692467880113934	0.715029938887909	-0.0960230259076818\\
0	0	-0\\
0.686765140995063	0.709141392544086	-0.159599895033379\\
0	0	-0\\
0.678240336601183	0.700338832107947	-0.222520933956314\\
0	0	-0\\
0.666928497182112	0.688658429188469	-0.284527586631032\\
0	0	-0\\
0.652876105512407	0.674148181066685	-0.345365054421307\\
0	0	-0\\
0.636140905883665	0.656867713464567	-0.404783343122394\\
0	0	-0\\
0.616791666820851	0.636888035530141	-0.462538290240835\\
0	0	-0\\
0.594907898497672	0.614291248045665	-0.518392568310525\\
0	0	-0\\
0.570579526012222	0.589170206057911	-0.57211666012217\\
0	0	-0\\
0.543906519865464	0.561628137316852	-0.623489801858733\\
0	0	-0\\
0.514998485160996	0.5317782180907	-0.672300890261317\\
0	0	-0\\
0.483974211214173	0.499743108100344	-0.718349350097727\\
0	0	-0\\
0.450961183421328	0.465654446484243	-0.761445958369134\\
0	0	-0\\
0.416095059394934	0.429652310864972	-0.801413621867957\\
0	0	-0\\
0.379519111517376	0.391884641740226	-0.838088104891841\\
0	0	-0\\
0.341383638203999	0.352506634563577	-0.871318704123389\\
0	0	-0\\
0.301845346294681	0.31168010201307	-0.900968867902419\\
0	0	-0\\
0.261066707111814	0.269572809068219	-0.926916757346022\\
0	0	-0\\
0.219215288830792	0.226357783627721	-0.949055747010669\\
0	0	-0\\
0.176463067906432	0.182212605500704	-0.967294863039029\\
0	0	-0\\
0.132985722384836	0.137318676693204	-0.981559156991065\\
0	0	-0\\
0.0889619100046116	0.0918604759884069	-0.991790013823246\\
0	0	-0\\
0.044572534053909	0.0460248008837638	-0.997945392750336\\
0	0	-0\\
8.51965408819226e-17	8.79724232266763e-17	-1\\
0	0	0\\
0	0	1\\
0	0	0\\
0.0415321358851145	0.0487860100561352	0.997945392750336\\
0	0	0\\
0.0828936073152619	0.0973715479323834	0.991790013823246\\
0	0	0\\
0.123914451132328	0.145556965232046	0.981559156991065\\
0	0	0\\
0.164426103890125	0.193144257739699	0.967294863039029\\
0	0	0\\
0.204262094517748	0.239937879062986	0.949055747010669\\
0	0	0\\
0.243258728384917	0.28574554417469	0.926916757346022\\
0	0	0\\
0.281255759958318	0.330379019553153	0.900968867902419\\
0	0	0\\
0.318097051284879	0.373654896674194	0.871318704123389\\
0	0	0\\
0.353631213596107	0.415395345676077	0.838088104891841\\
0	0	0\\
0.387712229397011	0.455428846100555	0.801413621867957\\
0	0	0\\
0.42020005248331	0.493590891707225	0.761445958369135\\
0	0	0\\
0.450961183421327	0.529724666464955	0.718349350097728\\
0	0	0\\
0.479869218125796	0.563681688942597	0.672300890261317\\
0	0	0\\
0.50680536728136	0.59532242245103	0.623489801858734\\
0	0	0\\
0.531658944473356	0.624516848429353	0.57211666012217\\
0	0	0\\
0.554327821022024	0.651145000719035	0.518392568310525\\
0	0	0\\
0.574718845651168	0.675097458530599	0.462538290240835\\
0	0	0\\
0.592748227266726	0.696275796077109	0.404783343122394\\
0	0	0\\
0.608341879272349	0.714592987026849	0.345365054421308\\
0	0	0\\
0.621435724007117	0.72997376211318	0.284527586631032\\
0	0	0\\
0.631975956054396	0.742354918432108	0.222520933956314\\
0	0	0\\
0.63991926333983	0.751685579156567	0.15959989503338\\
0	0	0\\
0.645233005109949	0.757927402600221	0.0960230259076819\\
0	0	0\\
0.647895346060025	0.761054739771673	0.0320515775716553\\
0	0	-0\\
0.647895346060025	0.761054739771673	-0.0320515775716552\\
0	0	-0\\
0.645233005109949	0.757927402600221	-0.0960230259076818\\
0	0	-0\\
0.63991926333983	0.751685579156567	-0.159599895033379\\
0	0	-0\\
0.631975956054396	0.742354918432108	-0.222520933956314\\
0	0	-0\\
0.621435724007117	0.729973762113181	-0.284527586631032\\
0	0	-0\\
0.608341879272349	0.714592987026849	-0.345365054421307\\
0	0	-0\\
0.592748227266726	0.696275796077109	-0.404783343122394\\
0	0	-0\\
0.574718845651168	0.675097458530599	-0.462538290240835\\
0	0	-0\\
0.554327821022024	0.651145000719035	-0.518392568310525\\
0	0	-0\\
0.531658944473356	0.624516848429353	-0.57211666012217\\
0	0	-0\\
0.50680536728136	0.59532242245103	-0.623489801858733\\
0	0	-0\\
0.479869218125796	0.563681688942597	-0.672300890261317\\
0	0	-0\\
0.450961183421327	0.529724666464956	-0.718349350097727\\
0	0	-0\\
0.420200052483311	0.493590891707225	-0.761445958369134\\
0	0	-0\\
0.387712229397011	0.455428846100555	-0.801413621867957\\
0	0	-0\\
0.353631213596107	0.415395345676077	-0.838088104891841\\
0	0	-0\\
0.318097051284879	0.373654896674194	-0.871318704123389\\
0	0	-0\\
0.281255759958318	0.330379019553153	-0.900968867902419\\
0	0	-0\\
0.243258728384917	0.28574554417469	-0.926916757346022\\
0	0	-0\\
0.204262094517749	0.239937879062986	-0.949055747010669\\
0	0	-0\\
0.164426103890125	0.193144257739699	-0.967294863039029\\
0	0	-0\\
0.123914451132328	0.145556965232046	-0.981559156991065\\
0	0	-0\\
0.082893607315262	0.0973715479323835	-0.991790013823246\\
0	0	-0\\
0.0415321358851148	0.0487860100561354	-0.997945392750336\\
0	0	-0\\
7.93850829430108e-17	9.32502355640449e-17	-1\\
0	0	0\\
0	0	1\\
0	0	0\\
0.0383210732613531	0.0513467470486199	0.997945392750336\\
0	0	0\\
0.0764846770128309	0.102482499299774	0.991790013823246\\
0	0	0\\
0.114333988820551	0.153197128978879	0.981559156991065\\
0	0	0\\
0.151713477743644	0.203282238794328	0.967294863039029\\
0	0	0\\
0.188469543444249	0.252532018286668	0.949055747010669\\
0	0	0\\
0.224451147364252	0.300744089547919	0.926916757346022\\
0	0	0\\
0.259510433375115	0.347720338835814	0.900968867902419\\
0	0	0\\
0.293503335350426	0.393267730665653	0.871318704123389\\
0	0	0\\
0.326290169164514	0.437199101034524	0.838088104891841\\
0	0	0\\
0.357736206684484	0.47933392651833	0.801413621867957\\
0	0	0\\
0.387712229397011	0.519499066081268	0.761445958369135\\
0	0	0\\
0.416095059394933	0.557529472549478	0.718349350097728\\
0	0	0\\
0.442768065541692	0.593268870825285	0.672300890261317\\
0	0	0\\
0.467621642733687	0.626570400055098	0.623489801858734\\
0	0	0\\
0.490553662291162	0.657297217112154	0.57211666012217\\
0	0	0\\
0.511469891626852	0.685323058914286	0.518392568310525\\
0	0	0\\
0.530284381467899	0.710532761266004	0.462538290240835\\
0	0	0\\
0.546919819039852	0.732822732092881	0.404783343122394\\
0	0	0\\
0.561307845761437	0.752101377123605	0.345365054421308\\
0	0	0\\
0.573389338144632	0.768289476270488	0.284527586631032\\
0	0	0\\
0.583114650745764	0.7813205091618	0.222520933956314\\
0	0	0\\
0.590443820169281	0.791140928488243	0.15959989503338\\
0	0	0\\
0.595346729285921	0.797710380040331	0.0960230259076819\\
0	0	0\\
0.597803230990452	0.801001868532493	0.0320515775716553\\
0	0	-0\\
0.597803230990452	0.801001868532493	-0.0320515775716552\\
0	0	-0\\
0.595346729285921	0.797710380040331	-0.0960230259076818\\
0	0	-0\\
0.590443820169281	0.791140928488243	-0.159599895033379\\
0	0	-0\\
0.583114650745764	0.7813205091618	-0.222520933956314\\
0	0	-0\\
0.573389338144632	0.768289476270488	-0.284527586631032\\
0	0	-0\\
0.561307845761437	0.752101377123605	-0.345365054421307\\
0	0	-0\\
0.546919819039852	0.732822732092881	-0.404783343122394\\
0	0	-0\\
0.530284381467899	0.710532761266004	-0.462538290240835\\
0	0	-0\\
0.511469891626852	0.685323058914286	-0.518392568310525\\
0	0	-0\\
0.490553662291162	0.657297217112154	-0.57211666012217\\
0	0	-0\\
0.467621642733687	0.626570400055098	-0.623489801858733\\
0	0	-0\\
0.442768065541692	0.593268870825285	-0.672300890261317\\
0	0	-0\\
0.416095059394934	0.557529472549478	-0.718349350097727\\
0	0	-0\\
0.387712229397011	0.519499066081269	-0.761445958369134\\
0	0	-0\\
0.357736206684484	0.47933392651833	-0.801413621867957\\
0	0	-0\\
0.326290169164514	0.437199101034524	-0.838088104891841\\
0	0	-0\\
0.293503335350426	0.393267730665653	-0.871318704123389\\
0	0	-0\\
0.259510433375115	0.347720338835814	-0.900968867902419\\
0	0	-0\\
0.224451147364252	0.30074408954792	-0.926916757346022\\
0	0	-0\\
0.18846954344425	0.252532018286668	-0.949055747010669\\
0	0	-0\\
0.151713477743644	0.203282238794328	-0.967294863039029\\
0	0	-0\\
0.114333988820551	0.153197128978879	-0.981559156991065\\
0	0	-0\\
0.0764846770128309	0.102482499299774	-0.991790013823246\\
0	0	-0\\
0.0383210732613533	0.0513467470486201	-0.997945392750336\\
0	0	-0\\
7.32474146702393e-17	9.8144862681368e-17	-1\\
0	0	0\\
0	0	1\\
0	0	0\\
0.0349525411277163	0.053696489243639	0.997945392750336\\
0	0	0\\
0.0697614547666423	0.107172328095115	0.991790013823246\\
0	0	0\\
0.104283703624147	0.160207772862056	0.981559156991065\\
0	0	0\\
0.138377428374676	0.212584889525847	0.967294863039029\\
0	0	0\\
0.171902530590148	0.264088449279261	0.949055747010669\\
0	0	0\\
0.204721248434448	0.314506812947792	0.926916757346022\\
0	0	0\\
0.236698722756361	0.36363280066042	0.900968867902419\\
0	0	0\\
0.267703551254751	0.411264543196144	0.871318704123389\\
0	0	0\\
0.297608328438803	0.457206311507906	0.838088104891841\\
0	0	0\\
0.326290169164514	0.501269321015237	0.801413621867957\\
0	0	0\\
0.353631213596107	0.543272507360583	0.761445958369135\\
0	0	0\\
0.379519111517375	0.583043270441598	0.718349350097728\\
0	0	0\\
0.403847484002825	0.620418183661979	0.672300890261317\\
0	0	0\\
0.426516360551494	0.655243665486409	0.623489801858734\\
0	0	0\\
0.447432589887183	0.687376610540031	0.57211666012217\\
0	0	0\\
0.466510222737037	0.716684977659123	0.518392568310525\\
0	0	0\\
0.483670865015536	0.743048332476568	0.462538290240835\\
0	0	0\\
0.49884399996261	0.766358342312499	0.404783343122394\\
0	0	0\\
0.511967277912135	0.78651922133652	0.345365054421308\\
0	0	0\\
0.522986772500083	0.803448124172224	0.284527586631032\\
0	0	0\\
0.531857202259518	0.817075486326623	0.222520933956314\\
0	0	0\\
0.538542116691855	0.827345310045564	0.15959989503338\\
0	0	0\\
0.543014046049784	0.834215394420515	0.0960230259076819\\
0	0	0\\
0.545254614216347	0.837657508801151	0.0320515775716553\\
0	0	-0\\
0.545254614216347	0.837657508801151	-0.0320515775716552\\
0	0	-0\\
0.543014046049784	0.834215394420515	-0.0960230259076818\\
0	0	-0\\
0.538542116691855	0.827345310045564	-0.159599895033379\\
0	0	-0\\
0.531857202259518	0.817075486326623	-0.222520933956314\\
0	0	-0\\
0.522986772500084	0.803448124172225	-0.284527586631032\\
0	0	-0\\
0.511967277912136	0.78651922133652	-0.345365054421307\\
0	0	-0\\
0.49884399996261	0.766358342312499	-0.404783343122394\\
0	0	-0\\
0.483670865015536	0.743048332476568	-0.462538290240835\\
0	0	-0\\
0.466510222737037	0.716684977659123	-0.518392568310525\\
0	0	-0\\
0.447432589887183	0.687376610540031	-0.57211666012217\\
0	0	-0\\
0.426516360551494	0.655243665486409	-0.623489801858733\\
0	0	-0\\
0.403847484002825	0.620418183661979	-0.672300890261317\\
0	0	-0\\
0.379519111517376	0.583043270441598	-0.718349350097727\\
0	0	-0\\
0.353631213596107	0.543272507360584	-0.761445958369134\\
0	0	-0\\
0.326290169164514	0.501269321015237	-0.801413621867957\\
0	0	-0\\
0.297608328438803	0.457206311507906	-0.838088104891841\\
0	0	-0\\
0.267703551254751	0.411264543196144	-0.871318704123389\\
0	0	-0\\
0.236698722756361	0.36363280066042	-0.900968867902419\\
0	0	-0\\
0.204721248434448	0.314506812947792	-0.926916757346022\\
0	0	-0\\
0.171902530590148	0.264088449279262	-0.949055747010669\\
0	0	-0\\
0.138377428374676	0.212584889525847	-0.967294863039029\\
0	0	-0\\
0.104283703624147	0.160207772862056	-0.981559156991065\\
0	0	-0\\
0.0697614547666424	0.107172328095115	-0.991790013823246\\
0	0	-0\\
0.0349525411277165	0.0536964892436393	-0.997945392750336\\
0	0	-0\\
6.68087570590666e-17	1.02636191505926e-16	-1\\
0	0	0\\
0	0	1\\
0	0	0\\
0.0314403815052892	0.0558255810464953	0.997945392750336\\
0	0	0\\
0.0627515677390326	0.111421762805921	0.991790013823246\\
0	0	0\\
0.0938048943207671	0.166560088642084	0.981559156991065\\
0	0	0\\
0.124472756470651	0.221013983346989	0.967294863039029\\
0	0	0\\
0.154629133364874	0.274559684186972	0.949055747010669\\
0	0	0\\
0.184150105982256	0.326977160391762	0.926916757346022\\
0	0	0\\
0.212914366314083	0.378051017308122	0.900968867902419\\
0	0	0\\
0.240803715844737	0.427571381502673	0.871318704123389\\
0	0	0\\
0.267703551254751	0.475334763176857	0.838088104891841\\
0	0	0\\
0.293503335350426	0.521144892350159	0.801413621867957\\
0	0	0\\
0.318097051284879	0.564813525375566	0.761445958369135\\
0	0	0\\
0.341383638203998	0.606161218473084	0.718349350097728\\
0	0	0\\
0.363267406527177	0.645018065102722	0.672300890261317\\
0	0	0\\
0.383658431156321	0.681224394146912	0.623489801858734\\
0	0	0\\
0.402472920997369	0.714631426033378	0.57211666012217\\
0	0	0\\
0.419633563275867	0.745101884102313	0.518392568310525\\
0	0	0\\
0.435069841231748	0.772510558705618	0.462538290240835\\
0	0	0\\
0.448718323887819	0.796744821720207	0.404783343122394\\
0	0	0\\
0.460522926701257	0.817705089361119	0.345365054421308\\
0	0	0\\
0.470435142027021	0.835305231392655	0.284527586631032\\
0	0	0\\
0.478414238446175	0.849472925055988	0.222520933956314\\
0	0	0\\
0.484427428140021	0.860149952258895	0.15959989503338\\
0	0	0\\
0.488450001622282	0.867292438806384	0.0960230259076819\\
0	0	0\\
0.49046542927568	0.870871034689173	0.0320515775716553\\
0	0	-0\\
0.49046542927568	0.870871034689173	-0.0320515775716552\\
0	0	-0\\
0.488450001622282	0.867292438806384	-0.0960230259076818\\
0	0	-0\\
0.484427428140021	0.860149952258895	-0.159599895033379\\
0	0	-0\\
0.478414238446175	0.849472925055988	-0.222520933956314\\
0	0	-0\\
0.470435142027021	0.835305231392655	-0.284527586631032\\
0	0	-0\\
0.460522926701257	0.817705089361119	-0.345365054421307\\
0	0	-0\\
0.448718323887819	0.796744821720207	-0.404783343122394\\
0	0	-0\\
0.435069841231748	0.772510558705618	-0.462538290240835\\
0	0	-0\\
0.419633563275867	0.745101884102313	-0.518392568310525\\
0	0	-0\\
0.402472920997369	0.714631426033378	-0.57211666012217\\
0	0	-0\\
0.383658431156321	0.681224394146912	-0.623489801858733\\
0	0	-0\\
0.363267406527177	0.645018065102722	-0.672300890261317\\
0	0	-0\\
0.341383638203999	0.606161218473084	-0.718349350097727\\
0	0	-0\\
0.318097051284879	0.564813525375567	-0.761445958369134\\
0	0	-0\\
0.293503335350426	0.521144892350159	-0.801413621867957\\
0	0	-0\\
0.267703551254751	0.475334763176857	-0.838088104891841\\
0	0	-0\\
0.240803715844738	0.427571381502673	-0.871318704123389\\
0	0	-0\\
0.212914366314083	0.378051017308122	-0.900968867902419\\
0	0	-0\\
0.184150105982256	0.326977160391762	-0.926916757346022\\
0	0	-0\\
0.154629133364875	0.274559684186972	-0.949055747010669\\
0	0	-0\\
0.124472756470651	0.221013983346989	-0.967294863039029\\
0	0	-0\\
0.0938048943207672	0.166560088642084	-0.981559156991065\\
0	0	-0\\
0.0627515677390326	0.111421762805921	-0.991790013823246\\
0	0	-0\\
0.0314403815052894	0.0558255810464955	-0.997945392750336\\
0	0	-0\\
6.00955679347047e-17	1.06705766204193e-16	-1\\
0	0	0\\
0	0	1\\
0	0	0\\
0.0277990266113162	0.0577252735622819	0.997945392750336\\
0	0	0\\
0.055483821059414	0.115213341593464	0.991790013823246\\
0	0	0\\
0.0829406205855364	0.172227973290854	0.981559156991065\\
0	0	0\\
0.110056599310965	0.228534883303208	0.967294863039029\\
0	0	0\\
0.136720331862759	0.28390269445949	0.949055747010669\\
0	0	0\\
0.16282225124451	0.338103888547301	0.926916757346022\\
0	0	0\\
0.188255099070633	0.390915741234015	0.900968867902419\\
0	0	0\\
0.212914366314083	0.442121237288834	0.871318704123389\\
0	0	0\\
0.236698722756361	0.491509962344926	0.838088104891841\\
0	0	0\\
0.259510433375115	0.538878967537186	0.801413621867957\\
0	0	0\\
0.281255759958318	0.58403360346266	0.761445958369135\\
0	0	0\\
0.301845346294681	0.62678832003669	0.718349350097728\\
0	0	0\\
0.321194585357495	0.666967428958017	0.672300890261317\\
0	0	0\\
0.339223966973052	0.704405825649691	0.623489801858734\\
0	0	0\\
0.355859404545005	0.738949667709194	0.57211666012217\\
0	0	0\\
0.371032539492079	0.770457007079873	0.518392568310525\\
0	0	0\\
0.384681022148151	0.798798373345952	0.462538290240835\\
0	0	0\\
0.396748767970395	0.823857305754238	0.404783343122394\\
0	0	0\\
0.407186188002705	0.845530831776342	0.345365054421308\\
0	0	0\\
0.415950392647348	0.863729890244883	0.284527586631032\\
0	0	0\\
0.423005367907524	0.878379697324927	0.222520933956314\\
0	0	0\\
0.428322123376601	0.889420053816808	0.15959989503338\\
0	0	0\\
0.431878811365916	0.896805592527553	0.0960230259076819\\
0	0	0\\
0.433660816681615	0.900505964694406	0.0320515775716553\\
0	0	-0\\
0.433660816681615	0.900505964694406	-0.0320515775716552\\
0	0	-0\\
0.431878811365916	0.896805592527553	-0.0960230259076818\\
0	0	-0\\
0.428322123376601	0.889420053816808	-0.159599895033379\\
0	0	-0\\
0.423005367907524	0.878379697324927	-0.222520933956314\\
0	0	-0\\
0.415950392647348	0.863729890244883	-0.284527586631032\\
0	0	-0\\
0.407186188002705	0.845530831776342	-0.345365054421307\\
0	0	-0\\
0.396748767970395	0.823857305754238	-0.404783343122394\\
0	0	-0\\
0.384681022148151	0.798798373345952	-0.462538290240835\\
0	0	-0\\
0.371032539492079	0.770457007079874	-0.518392568310525\\
0	0	-0\\
0.355859404545005	0.738949667709194	-0.57211666012217\\
0	0	-0\\
0.339223966973052	0.704405825649691	-0.623489801858733\\
0	0	-0\\
0.321194585357495	0.666967428958017	-0.672300890261317\\
0	0	-0\\
0.301845346294681	0.626788320036691	-0.718349350097727\\
0	0	-0\\
0.281255759958318	0.58403360346266	-0.761445958369134\\
0	0	-0\\
0.259510433375115	0.538878967537186	-0.801413621867957\\
0	0	-0\\
0.236698722756361	0.491509962344926	-0.838088104891841\\
0	0	-0\\
0.212914366314083	0.442121237288835	-0.871318704123389\\
0	0	-0\\
0.188255099070633	0.390915741234015	-0.900968867902419\\
0	0	-0\\
0.16282225124451	0.338103888547302	-0.926916757346022\\
0	0	-0\\
0.13672033186276	0.283902694459491	-0.949055747010669\\
0	0	-0\\
0.110056599310965	0.228534883303208	-0.967294863039029\\
0	0	-0\\
0.0829406205855365	0.172227973290855	-0.981559156991065\\
0	0	-0\\
0.055483821059414	0.115213341593464	-0.991790013823246\\
0	0	-0\\
0.0277990266113163	0.0577252735622822	-0.997945392750336\\
0	0	-0\\
5.31354332312403e-17	1.10336864020811e-16	-1\\
0	0	0\\
0	0	1\\
0	0	0\\
0.0240434395541248	0.0593877605469687	0.997945392750336\\
0	0	0\\
0.04798807945782	0.118531484047215	0.991790013823246\\
0	0	0\\
0.0717355260496123	0.177188136254588	0.981559156991065\\
0	0	0\\
0.0951881959776447	0.235116684403355	0.967294863039029\\
0	0	0\\
0.118249717190601	0.292079087663538	0.949055747010669\\
0	0	0\\
0.140825324951136	0.347841275301743	0.926916757346022\\
0	0	0\\
0.16282225124451	0.402174108528014	0.900968867902419\\
0	0	0\\
0.184150105982256	0.454854322076268	0.871318704123389\\
0	0	0\\
0.204721248434448	0.505665441649164	0.838088104891841\\
0	0	0\\
0.224451147364252	0.554398673457427	0.801413621867957\\
0	0	0\\
0.243258728384917	0.60085376219831	0.761445958369135\\
0	0	0\\
0.261066707111814	0.644839813947593	0.718349350097728\\
0	0	0\\
0.277801906740556	0.686176080583659	0.672300890261317\\
0	0	0\\
0.293395558746179	0.724692702520299	0.623489801858734\\
0	0	0\\
0.307783585467763	0.760231406696186	0.57211666012217\\
0	0	0\\
0.320906863417289	0.792646156952833	0.518392568310525\\
0	0	0\\
0.332711466230726	0.821803754128493	0.462538290240835\\
0	0	0\\
0.343148886263036	0.847584383402087	0.404783343122394\\
0	0	0\\
0.352176233916486	0.869882106638002	0.345365054421308\\
0	0	0\\
0.359756413883208	0.888605297708614	0.284527586631032\\
0	0	0\\
0.365858277577775	0.903677019005703	0.222520933956314\\
0	0	0\\
0.37045675113342	0.915035337593585	0.15959989503338\\
0	0	0\\
0.373532938435934	0.92263357970483	0.0960230259076819\\
0	0	0\\
0.375074198771851	0.926440522532786	0.0320515775716553\\
0	0	-0\\
0.375074198771851	0.926440522532786	-0.0320515775716552\\
0	0	-0\\
0.373532938435934	0.92263357970483	-0.0960230259076818\\
0	0	-0\\
0.37045675113342	0.915035337593585	-0.159599895033379\\
0	0	-0\\
0.365858277577775	0.903677019005703	-0.222520933956314\\
0	0	-0\\
0.359756413883208	0.888605297708614	-0.284527586631032\\
0	0	-0\\
0.352176233916486	0.869882106638002	-0.345365054421307\\
0	0	-0\\
0.343148886263036	0.847584383402087	-0.404783343122394\\
0	0	-0\\
0.332711466230726	0.821803754128493	-0.462538290240835\\
0	0	-0\\
0.320906863417289	0.792646156952833	-0.518392568310525\\
0	0	-0\\
0.307783585467763	0.760231406696186	-0.57211666012217\\
0	0	-0\\
0.293395558746179	0.724692702520299	-0.623489801858733\\
0	0	-0\\
0.277801906740556	0.686176080583659	-0.672300890261317\\
0	0	-0\\
0.261066707111814	0.644839813947593	-0.718349350097727\\
0	0	-0\\
0.243258728384917	0.600853762198311	-0.761445958369134\\
0	0	-0\\
0.224451147364252	0.554398673457427	-0.801413621867957\\
0	0	-0\\
0.204721248434448	0.505665441649164	-0.838088104891841\\
0	0	-0\\
0.184150105982256	0.454854322076268	-0.871318704123389\\
0	0	-0\\
0.16282225124451	0.402174108528014	-0.900968867902419\\
0	0	-0\\
0.140825324951136	0.347841275301744	-0.926916757346022\\
0	0	-0\\
0.118249717190601	0.292079087663538	-0.949055747010669\\
0	0	-0\\
0.0951881959776447	0.235116684403355	-0.967294863039029\\
0	0	-0\\
0.0717355260496124	0.177188136254588	-0.981559156991065\\
0	0	-0\\
0.0479880794578201	0.118531484047215	-0.991790013823246\\
0	0	-0\\
0.0240434395541249	0.059387760546969	-0.997945392750336\\
0	0	-0\\
4.5956953635114e-17	1.13514563995985e-16	-1\\
0	0	0\\
0	0	1\\
0	0	0\\
0.0201890528465038	0.0608062104849334	0.997945392750336\\
0	0	0\\
0.0402951445443231	0.121362555208093	0.991790013823246\\
0	0	0\\
0.0602356548499284	0.181420195159716	0.981559156991065\\
0	0	0\\
0.0799286439292479	0.240732340614918	0.967294863039029\\
0	0	0\\
0.0992931890660217	0.299055265245608	0.949055747010669\\
0	0	0\\
0.118249717190601	0.356149307644251	0.926916757346022\\
0	0	0\\
0.136720331862759	0.411779856143996	0.900968867902419\\
0	0	0\\
0.154629133364874	0.465718312888344	0.871318704123389\\
0	0	0\\
0.171902530590148	0.517743033188769	0.838088104891841\\
0	0	0\\
0.18846954344425	0.567640236310288	0.801413621867957\\
0	0	0\\
0.204262094517748	0.61520488394236	0.761445958369135\\
0	0	0\\
0.219215288830792	0.660241522745279	0.718349350097728\\
0	0	0\\
0.233267680500498	0.702565087509875	0.672300890261317\\
0	0	0\\
0.246361525235267	0.742001661630154	0.623489801858734\\
0	0	0\\
0.258443017618462	0.778389191763938	0.57211666012217\\
0	0	0\\
0.26946251220641	0.811578153744806	0.518392568310525\\
0	0	0\\
0.279374727532174	0.841432167008968	0.462538290240835\\
0	0	0\\
0.288138932176818	0.867828555012257	0.404783343122394\\
0	0	0\\
0.295719112143539	0.890658849334359	0.345365054421308\\
0	0	0\\
0.302084118846912	0.909829235398822	0.284527586631032\\
0	0	0\\
0.307207797109103	0.925260937977272	0.222520933956314\\
0	0	0\\
0.311069092637107	0.936890544893725	0.15959989503338\\
0	0	0\\
0.313652138539354	0.944670267598818	0.0960230259076819\\
0	0	0\\
0.31494632052617	0.94856813754321	0.0320515775716553\\
0	0	-0\\
0.31494632052617	0.94856813754321	-0.0320515775716552\\
0	0	-0\\
0.313652138539354	0.944670267598818	-0.0960230259076818\\
0	0	-0\\
0.311069092637107	0.936890544893725	-0.159599895033379\\
0	0	-0\\
0.307207797109103	0.925260937977272	-0.222520933956314\\
0	0	-0\\
0.302084118846912	0.909829235398822	-0.284527586631032\\
0	0	-0\\
0.295719112143539	0.890658849334359	-0.345365054421307\\
0	0	-0\\
0.288138932176818	0.867828555012257	-0.404783343122394\\
0	0	-0\\
0.279374727532174	0.841432167008968	-0.462538290240835\\
0	0	-0\\
0.26946251220641	0.811578153744806	-0.518392568310525\\
0	0	-0\\
0.258443017618462	0.778389191763938	-0.57211666012217\\
0	0	-0\\
0.246361525235267	0.742001661630155	-0.623489801858733\\
0	0	-0\\
0.233267680500498	0.702565087509875	-0.672300890261317\\
0	0	-0\\
0.219215288830792	0.660241522745279	-0.718349350097727\\
0	0	-0\\
0.204262094517749	0.615204883942361	-0.761445958369134\\
0	0	-0\\
0.18846954344425	0.567640236310288	-0.801413621867957\\
0	0	-0\\
0.171902530590148	0.517743033188769	-0.838088104891841\\
0	0	-0\\
0.154629133364874	0.465718312888344	-0.871318704123389\\
0	0	-0\\
0.136720331862759	0.411779856143996	-0.900968867902419\\
0	0	-0\\
0.118249717190601	0.356149307644251	-0.926916757346022\\
0	0	-0\\
0.0992931890660219	0.299055265245609	-0.949055747010669\\
0	0	-0\\
0.0799286439292479	0.240732340614918	-0.967294863039029\\
0	0	-0\\
0.0602356548499285	0.181420195159716	-0.981559156991065\\
0	0	-0\\
0.0402951445443231	0.121362555208093	-0.991790013823246\\
0	0	-0\\
0.0201890528465039	0.0608062104849337	-0.997945392750336\\
0	0	-0\\
3.85896270587653e-17	1.16225808278902e-16	-1\\
0	0	0\\
0	0	1\\
0	0	0\\
0.0162517049901983	0.0619747946611242	0.997945392750336\\
0	0	0\\
0.0324366282386122	0.123694921597434	0.991790013823246\\
0	0	0\\
0.0484882624239586	0.184906759568423	0.981559156991065\\
0	0	0\\
0.0643406479383054	0.245358776001969	0.967294863039029\\
0	0	0\\
0.0799286439292479	0.304802560595631	0.949055747010669\\
0	0	0\\
0.0951881959776448	0.362993846087861	0.926916757346022\\
0	0	0\\
0.110056599310966	0.41969351200458	0.900968867902419\\
0	0	0\\
0.124472756470651	0.474668567256497	0.871318704123389\\
0	0	0\\
0.138377428374676	0.527693107549468	0.838088104891841\\
0	0	0\\
0.151713477743644	0.578549243673702	0.801413621867957\\
0	0	0\\
0.164426103890125	0.627027996857257	0.761445958369135\\
0	0	0\\
0.176463067906432	0.672930157504642	0.718349350097728\\
0	0	0\\
0.187774907325503	0.716067103791774	0.672300890261317\\
0	0	0\\
0.198315139372782	0.756261576753514	0.623489801858734\\
0	0	0\\
0.208040451973913	0.793348408678774	0.57211666012217\\
0	0	0\\
0.216910881733348	0.827175201820073	0.518392568310525\\
0	0	0\\
0.224889978152501	0.85760295462857	0.462538290240835\\
0	0	0\\
0.231944953412677	0.88450663294124	0.404783343122394\\
0	0	0\\
0.238046817107244	0.907775683773077	0.345365054421308\\
0	0	0\\
0.243170495369435	0.927314489603017	0.284527586631032\\
0	0	0\\
0.247294933906245	0.943042761286844	0.222520933956314\\
0	0	0\\
0.250403184515038	0.954895867982505	0.15959989503338\\
0	0	0\\
0.252482474727344	0.962825102732105	0.0960230259076819\\
0	0	0\\
0.253524260293674	0.966797882609242	0.0320515775716553\\
0	0	-0\\
0.253524260293674	0.966797882609242	-0.0320515775716552\\
0	0	-0\\
0.252482474727344	0.962825102732105	-0.0960230259076818\\
0	0	-0\\
0.250403184515038	0.954895867982505	-0.159599895033379\\
0	0	-0\\
0.247294933906245	0.943042761286844	-0.222520933956314\\
0	0	-0\\
0.243170495369435	0.927314489603017	-0.284527586631032\\
0	0	-0\\
0.238046817107244	0.907775683773077	-0.345365054421307\\
0	0	-0\\
0.231944953412677	0.88450663294124	-0.404783343122394\\
0	0	-0\\
0.224889978152501	0.85760295462857	-0.462538290240835\\
0	0	-0\\
0.216910881733348	0.827175201820073	-0.518392568310525\\
0	0	-0\\
0.208040451973913	0.793348408678774	-0.57211666012217\\
0	0	-0\\
0.198315139372782	0.756261576753514	-0.623489801858733\\
0	0	-0\\
0.187774907325503	0.716067103791774	-0.672300890261317\\
0	0	-0\\
0.176463067906432	0.672930157504642	-0.718349350097727\\
0	0	-0\\
0.164426103890125	0.627027996857257	-0.761445958369134\\
0	0	-0\\
0.151713477743644	0.578549243673702	-0.801413621867957\\
0	0	-0\\
0.138377428374676	0.527693107549468	-0.838088104891841\\
0	0	-0\\
0.124472756470651	0.474668567256497	-0.871318704123389\\
0	0	-0\\
0.110056599310966	0.41969351200458	-0.900968867902419\\
0	0	-0\\
0.0951881959776449	0.362993846087861	-0.926916757346022\\
0	0	-0\\
0.079928643929248	0.304802560595631	-0.949055747010669\\
0	0	-0\\
0.0643406479383054	0.245358776001969	-0.967294863039029\\
0	0	-0\\
0.0484882624239587	0.184906759568423	-0.981559156991065\\
0	0	-0\\
0.0324366282386122	0.123694921597434	-0.991790013823246\\
0	0	-0\\
0.0162517049901984	0.0619747946611245	-0.997945392750336\\
0	0	-0\\
3.10637274273832e-17	1.18459455785242e-16	-1\\
0	0	0\\
0	0	1\\
0	0	0\\
0.0122475753921084	0.0628887111125007	0.997945392750336\\
0	0	0\\
0.0244448228698339	0.125518999021454	0.991790013823246\\
0	0	0\\
0.0365416213269892	0.187633502439687	0.981559156991065\\
0	0	0\\
0.0484882624239586	0.248976979549135	0.967294863039029\\
0	0	0\\
0.0602356548499284	0.309297356844222	0.949055747010669\\
0	0	0\\
0.0717355260496124	0.36834676495596	0.926916757346022\\
0	0	0\\
0.0829406205855365	0.425882557200362	0.900968867902419\\
0	0	0\\
0.0938048943207672	0.481668306665704	0.871318704123389\\
0	0	0\\
0.104283703624147	0.53547477774143	0.838088104891841\\
0	0	0\\
0.114333988820551	0.587080868096437	0.801413621867957\\
0	0	0\\
0.123914451132328	0.636274517235984	0.761445958369135\\
0	0	0\\
0.132985722384836	0.682853577903752	0.718349350097728\\
0	0	0\\
0.141510526778716	0.726626646748281	0.672300890261317\\
0	0	0\\
0.14945383406415	0.767413850840394	0.623489801858734\\
0	0	0\\
0.156783003487667	0.805047586809649	0.57211666012217\\
0	0	0\\
0.163467917920005	0.839373209562538	0.518392568310525\\
0	0	0\\
0.169481107613851	0.870249667752345	0.462538290240835\\
0	0	0\\
0.174797863082928	0.89755008338939	0.404783343122394\\
0	0	0\\
0.179396336638573	0.921162273209898	0.345365054421308\\
0	0	0\\
0.183257632166577	0.940989209661099	0.284527586631032\\
0	0	0\\
0.18636588277537	0.95694941960825	0.222520933956314\\
0	0	0\\
0.188708315996482	0.968977319125224	0.15959989503338\\
0	0	0\\
0.190275306269357	0.977023482992929	0.0960230259076819\\
0	0	0\\
0.191060414494847	0.981054847798137	0.0320515775716553\\
0	0	-0\\
0.191060414494847	0.981054847798137	-0.0320515775716552\\
0	0	-0\\
0.190275306269357	0.977023482992929	-0.0960230259076818\\
0	0	-0\\
0.188708315996482	0.968977319125224	-0.159599895033379\\
0	0	-0\\
0.18636588277537	0.95694941960825	-0.222520933956314\\
0	0	-0\\
0.183257632166577	0.940989209661099	-0.284527586631032\\
0	0	-0\\
0.179396336638573	0.921162273209898	-0.345365054421307\\
0	0	-0\\
0.174797863082928	0.89755008338939	-0.404783343122394\\
0	0	-0\\
0.169481107613851	0.870249667752345	-0.462538290240835\\
0	0	-0\\
0.163467917920005	0.839373209562538	-0.518392568310525\\
0	0	-0\\
0.156783003487667	0.805047586809649	-0.57211666012217\\
0	0	-0\\
0.14945383406415	0.767413850840394	-0.623489801858733\\
0	0	-0\\
0.141510526778716	0.726626646748281	-0.672300890261317\\
0	0	-0\\
0.132985722384836	0.682853577903752	-0.718349350097727\\
0	0	-0\\
0.123914451132328	0.636274517235984	-0.761445958369134\\
0	0	-0\\
0.114333988820551	0.587080868096437	-0.801413621867957\\
0	0	-0\\
0.104283703624147	0.53547477774143	-0.838088104891841\\
0	0	-0\\
0.0938048943207672	0.481668306665705	-0.871318704123389\\
0	0	-0\\
0.0829406205855365	0.425882557200362	-0.900968867902419\\
0	0	-0\\
0.0717355260496125	0.368346764955961	-0.926916757346022\\
0	0	-0\\
0.0602356548499285	0.309297356844222	-0.949055747010669\\
0	0	-0\\
0.0484882624239586	0.248976979549135	-0.967294863039029\\
0	0	-0\\
0.0365416213269892	0.187633502439687	-0.981559156991065\\
0	0	-0\\
0.0244448228698339	0.125518999021454	-0.991790013823246\\
0	0	-0\\
0.0122475753921084	0.062888711112501	-0.997945392750336\\
0	0	-0\\
2.34101802768533e-17	1.20206327978288e-16	-1\\
0	0	0\\
0	0	1\\
0	0	0\\
0.00819311787963551	0.0635442043603297	0.997945392750336\\
0	0	0\\
0.0163525684804853	0.126827291954754	0.991790013823246\\
0	0	0\\
0.0244448228698339	0.189589219002167	0.981559156991065\\
0	0	0\\
0.0324366282386122	0.25157208328194	0.967294863039029\\
0	0	0\\
0.0402951445443232	0.312521183909465	0.949055747010669\\
0	0	0\\
0.0479880794578201	0.372186067956723	0.926916757346022\\
0	0	0\\
0.0554838210594141	0.430321559617085	0.900968867902419\\
0	0	0\\
0.0627515677390326	0.486688767685295	0.871318704123389\\
0	0	0\\
0.0697614547666425	0.541056067212673	0.838088104891841\\
0	0	0\\
0.076484677012831	0.593200051303712	0.801413621867957\\
0	0	0\\
0.0828936073152621	0.642906449142932	0.761445958369135\\
0	0	0\\
0.0889619100046117	0.689971006479623	0.718349350097728\\
0	0	0\\
0.0946646491234825	0.734200324952372	0.672300890261317\\
0	0	0\\
0.0999783908936014	0.775412656804416	0.623489801858734\\
0	0	0\\
0.104881300010241	0.813438651724158	0.57211666012217\\
0	0	0\\
0.10935322936817	0.848122052741921	0.518392568310525\\
0	0	0\\
0.113375802850431	0.879320338323358	0.462538290240835\\
0	0	0\\
0.116932490839746	0.906905308021003	0.404783343122394\\
0	0	0\\
0.120008678142261	0.930763609277412	0.345365054421308\\
0	0	0\\
0.122591724044507	0.950797203215131	0.284527586631032\\
0	0	0\\
0.124671014256813	0.966923767499479	0.222520933956314\\
0	0	0\\
0.126238004529689	0.979077034618674	0.15959989503338\\
0	0	0\\
0.127286255763985	0.987207064191256	0.0960230259076819\\
0	0	0\\
0.127811460470531	0.991280448181824	0.0320515775716553\\
0	0	-0\\
0.127811460470531	0.991280448181824	-0.0320515775716552\\
0	0	-0\\
0.127286255763985	0.987207064191256	-0.0960230259076818\\
0	0	-0\\
0.126238004529689	0.979077034618674	-0.159599895033379\\
0	0	-0\\
0.124671014256813	0.966923767499479	-0.222520933956314\\
0	0	-0\\
0.122591724044508	0.950797203215131	-0.284527586631032\\
0	0	-0\\
0.120008678142261	0.930763609277412	-0.345365054421307\\
0	0	-0\\
0.116932490839746	0.906905308021003	-0.404783343122394\\
0	0	-0\\
0.113375802850431	0.879320338323358	-0.462538290240835\\
0	0	-0\\
0.10935322936817	0.848122052741921	-0.518392568310525\\
0	0	-0\\
0.104881300010241	0.813438651724158	-0.57211666012217\\
0	0	-0\\
0.0999783908936015	0.775412656804416	-0.623489801858733\\
0	0	-0\\
0.0946646491234825	0.734200324952372	-0.672300890261317\\
0	0	-0\\
0.0889619100046117	0.689971006479623	-0.718349350097727\\
0	0	-0\\
0.0828936073152621	0.642906449142932	-0.761445958369134\\
0	0	-0\\
0.076484677012831	0.593200051303712	-0.801413621867957\\
0	0	-0\\
0.0697614547666425	0.541056067212673	-0.838088104891841\\
0	0	-0\\
0.0627515677390327	0.486688767685295	-0.871318704123389\\
0	0	-0\\
0.0554838210594141	0.430321559617085	-0.900968867902419\\
0	0	-0\\
0.0479880794578202	0.372186067956723	-0.926916757346022\\
0	0	-0\\
0.0402951445443233	0.312521183909466	-0.949055747010669\\
0	0	-0\\
0.0324366282386122	0.25157208328194	-0.967294863039029\\
0	0	-0\\
0.0244448228698339	0.189589219002167	-0.981559156991065\\
0	0	-0\\
0.0163525684804853	0.126827291954754	-0.991790013823246\\
0	0	-0\\
0.00819311787963555	0.06354420436033	-0.997945392750336\\
0	0	-0\\
1.56604356740979e-17	1.21459246585495e-16	-1\\
0	0	0\\
0	0	1\\
0	0	0\\
0.00410499308837696	0.063938580842253	0.997945392750336\\
0	0	0\\
0.00819311787963553	0.127614424341043	0.991790013823246\\
0	0	0\\
0.0122475753921084	0.190765872797007	0.981559156991065\\
0	0	0\\
0.0162517049901984	0.253133423362496	0.967294863039029\\
0	0	0\\
0.0201890528465039	0.314460794394441	0.949055747010669\\
0	0	0\\
0.0240434395541249	0.374495978570589	0.926916757346022\\
0	0	0\\
0.0277990266113163	0.432992278441656	0.900968867902419\\
0	0	0\\
0.0314403815052894	0.489709320164053	0.871318704123389\\
0	0	0\\
0.0349525411277165	0.544414041247577	0.838088104891841\\
0	0	0\\
0.0383210732613533	0.596881648259169	0.801413621867957\\
0	0	0\\
0.0415321358851147	0.646896540547351	0.761445958369135\\
0	0	0\\
0.044572534053909	0.694253196191552	0.718349350097728\\
0	0	0\\
0.0474297741194972	0.738757016535758	0.672300890261317\\
0	0	0\\
0.0500921150695738	0.780225125836136	0.623489801858734\\
0	0	0\\
0.0525486167741045	0.818487122736688	0.57211666012217\\
0	0	0\\
0.0547891849406674	0.853385780484978	0.518392568310525\\
0	0	0\\
0.0568046125940658	0.884777693010579	0.462538290240835\\
0	0	0\\
0.0585866179097641	0.91253386421138	0.404783343122394\\
0	0	0\\
0.0601278782456809	0.936540238026236	0.345365054421308\\
0	0	0\\
0.0614220602324969	0.956698167115792	0.284527586631032\\
0	0	0\\
0.0624638457988269	0.972924818225555	0.222520933956314\\
0	0	0\\
0.0632489540243166	0.985153512565511	0.15959989503338\\
0	0	0\\
0.0637741587308623	0.993333999807569	0.0960230259076819\\
0	0	0\\
0.0640373017396687	0.997432664574943	0.0320515775716553\\
0	0	-0\\
0.0640373017396687	0.997432664574943	-0.0320515775716552\\
0	0	-0\\
0.0637741587308623	0.993333999807569	-0.0960230259076818\\
0	0	-0\\
0.0632489540243166	0.985153512565511	-0.159599895033379\\
0	0	-0\\
0.0624638457988269	0.972924818225555	-0.222520933956314\\
0	0	-0\\
0.0614220602324969	0.956698167115792	-0.284527586631032\\
0	0	-0\\
0.0601278782456809	0.936540238026237	-0.345365054421307\\
0	0	-0\\
0.0585866179097641	0.91253386421138	-0.404783343122394\\
0	0	-0\\
0.0568046125940658	0.884777693010579	-0.462538290240835\\
0	0	-0\\
0.0547891849406674	0.853385780484978	-0.518392568310525\\
0	0	-0\\
0.0525486167741044	0.818487122736688	-0.57211666012217\\
0	0	-0\\
0.0500921150695738	0.780225125836136	-0.623489801858733\\
0	0	-0\\
0.0474297741194972	0.738757016535758	-0.672300890261317\\
0	0	-0\\
0.044572534053909	0.694253196191552	-0.718349350097727\\
0	0	-0\\
0.0415321358851147	0.646896540547352	-0.761445958369134\\
0	0	-0\\
0.0383210732613533	0.596881648259169	-0.801413621867957\\
0	0	-0\\
0.0349525411277165	0.544414041247577	-0.838088104891841\\
0	0	-0\\
0.0314403815052894	0.489709320164054	-0.871318704123389\\
0	0	-0\\
0.0277990266113163	0.432992278441656	-0.900968867902419\\
0	0	-0\\
0.0240434395541249	0.37449597857059	-0.926916757346022\\
0	0	-0\\
0.0201890528465039	0.314460794394441	-0.949055747010669\\
0	0	-0\\
0.0162517049901984	0.253133423362496	-0.967294863039029\\
0	0	-0\\
0.0122475753921084	0.190765872797007	-0.981559156991065\\
0	0	-0\\
0.00819311787963553	0.127614424341043	-0.991790013823246\\
0	0	-0\\
0.00410499308837698	0.0639385808422533	-0.997945392750336\\
0	0	-0\\
7.84633898200472e-18	1.22213063095555e-16	-1\\
0	0	0\\
0	0	1\\
0	0	0\\
3.92316949100234e-18	0.0640702199807129	0.997945392750336\\
0	0	0\\
7.83021783704894e-18	0.127877161684506	0.991790013823246\\
0	0	0\\
1.17050901384266e-17	0.191158628701372	0.981559156991065\\
0	0	0\\
1.55318637136916e-17	0.253654583909507	0.967294863039029\\
0	0	0\\
1.92948135293827e-17	0.315108218023621	0.949055747010669\\
0	0	0\\
2.2978476817557e-17	0.375267004879374	0.926916757346022\\
0	0	0\\
2.65677166156201e-17	0.433883739117558	0.900968867902419\\
0	0	0\\
3.00477839673524e-17	0.490717552003938	0.871318704123389\\
0	0	0\\
3.34043785295333e-17	0.545534901210549	0.838088104891841\\
0	0	0\\
3.66237073351197e-17	0.598110530491216	0.801413621867957\\
0	0	0\\
3.96925414715054e-17	0.648228395307788	0.761445958369135\\
0	0	0\\
4.25982704409613e-17	0.695682550603486	0.718349350097728\\
0	0	0\\
4.53289539798749e-17	0.740277997075315	0.672300890261317\\
0	0	0\\
4.78733711238551e-17	0.78183148246803	0.623489801858734\\
0	0	0\\
5.02210663170815e-17	0.820172254596956	0.57211666012217\\
0	0	0\\
5.2362392376426e-17	0.855142763005346	0.518392568310525\\
0	0	0\\
5.42885501337979e-17	0.886599306373	0.462538290240835\\
0	0	0\\
5.59916245938125e-17	0.914412623015812	0.404783343122394\\
0	0	0\\
5.74646174582053e-17	0.93846842204976	0.345365054421308\\
0	0	0\\
5.87014758833406e-17	0.958667853036661	0.284527586631032\\
0	0	0\\
5.96971173526441e-17	0.974927912181824	0.222520933956314\\
0	0	0\\
6.04474505617541e-17	0.98718178341445	0.15959989503338\\
0	0	0\\
6.09493922305684e-17	0.995379112949198	0.0960230259076819\\
0	0	0\\
6.12008797731036e-17	0.999486216200688	0.0320515775716553\\
0	0	-0\\
6.12008797731036e-17	0.999486216200688	-0.0320515775716552\\
0	0	-0\\
6.09493922305684e-17	0.995379112949198	-0.0960230259076818\\
0	0	-0\\
6.04474505617541e-17	0.98718178341445	-0.159599895033379\\
0	0	-0\\
5.96971173526441e-17	0.974927912181824	-0.222520933956314\\
0	0	-0\\
5.87014758833406e-17	0.958667853036661	-0.284527586631032\\
0	0	-0\\
5.74646174582053e-17	0.93846842204976	-0.345365054421307\\
0	0	-0\\
5.59916245938125e-17	0.914412623015812	-0.404783343122394\\
0	0	-0\\
5.42885501337979e-17	0.886599306373	-0.462538290240835\\
0	0	-0\\
5.2362392376426e-17	0.855142763005346	-0.518392568310525\\
0	0	-0\\
5.02210663170815e-17	0.820172254596956	-0.57211666012217\\
0	0	-0\\
4.78733711238551e-17	0.78183148246803	-0.623489801858733\\
0	0	-0\\
4.53289539798749e-17	0.740277997075315	-0.672300890261317\\
0	0	-0\\
4.25982704409613e-17	0.695682550603486	-0.718349350097727\\
0	0	-0\\
3.96925414715054e-17	0.648228395307789	-0.761445958369134\\
0	0	-0\\
3.66237073351197e-17	0.598110530491216	-0.801413621867957\\
0	0	-0\\
3.34043785295333e-17	0.545534901210549	-0.838088104891841\\
0	0	-0\\
3.00477839673524e-17	0.490717552003938	-0.871318704123389\\
0	0	-0\\
2.65677166156201e-17	0.433883739117558	-0.900968867902419\\
0	0	-0\\
2.2978476817557e-17	0.375267004879374	-0.926916757346022\\
0	0	-0\\
1.92948135293827e-17	0.315108218023621	-0.949055747010669\\
0	0	-0\\
1.55318637136916e-17	0.253654583909507	-0.967294863039029\\
0	0	-0\\
1.17050901384267e-17	0.191158628701373	-0.981559156991065\\
0	0	-0\\
7.83021783704894e-18	0.127877161684506	-0.991790013823246\\
0	0	-0\\
3.92316949100236e-18	0.0640702199807132	-0.997945392750336\\
0	0	-0\\
7.49879891330929e-33	1.22464679914735e-16	-1\\
};
\end{axis}
\end{tikzpicture}%}
	\caption{3D Workspace of the manipulator}
	\label{fig:workspace}
\end{figure}
\figref{fig:workspace} illustrates the three-dimensional workspace of the manipulator for the given joint constraints. It should be noted that the workspace is plotted with respect to the assumed base frame orientation and the present exposition of the workspace figure may not be optimal for illustration.