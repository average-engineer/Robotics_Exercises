\section{Exercise 4}
The geometric object in which the end-effector lives is called the \emph{Workspace}.
\subsection{Forward Kinematics}
The formulation of the forward kinematics is based on \cite{Craig}, which makes use of the \emph{Denavit-Hartenberg} parameters.

For the given formulation, it is assumed that $q_1 \in \left[0,\pi\right]$ and $q_2 \in \left[\frac{-\pi}{2},\frac{\pi}{2}\right]$.

For brevity, only the Z and X axes of the joint frames are drawn in 2D, axis coming out of the plane is denoted by $\oplus$ and axis going into the plane is denoted by $\otimes$.

\begin{figure}[h!]
	\centering
\tikzstyle{mod}    = [circle, inner sep=0pt, minimum size=1.0cm, draw=black]
\tikzstyle{branch}    = [circle, inner sep=0pt, minimum size=0.1mm, fill=black, draw=black]
\tikzstyle{dummy}     = [inner sep=0pt, minimum size=0pt]
\begin{tikzpicture}[auto, node distance=2cm, >=stealth']
	% Frames 1 and 2
	\coordinate (base) at (0,0);
	\node[] at (base){$\otimes$};
	\node[yshift = -0.25cm] at (base){$z_0,z_1$};
	\coordinate (x1) at ($(base) + (0,1)$);
	\node[yshift = 0.25cm] at (x1){$x_0,x_1$};
	\draw[thick,->](base)--(x1);
	\draw[->, bend left = 45] ($(base) + (-0.5,0.2)$)node[yshift = 0.3cm]{$q_1$} to ($(base) + (0.5,0.2)$);
	
	% Frame 2
	\coordinate (x2) at ($(base) + (2,0)$);
	\node[] at (x2){$\oplus$};
	\node[yshift = -0.25cm] at (x2){$x_2$};
	\coordinate (z2) at ($(x2) + (1,0)$);
	\node[yshift = 0.25cm] at (z2){$z_2$};
	\draw[thick,->](x2)--(z2);
	\draw[->, bend right = 45] ($(x2) + (0.5,-0.2)$)node[xshift = 0.3cm]{$q_2$} to ($(x2) + (0.5,0.2)$);
	
\end{tikzpicture}
	\caption{Joint Frame Assignment}
	\label{fig:frames}
\end{figure}

\figref{fig:frames} illustrates the joint frame allocation along with the base frame $\left\{0\right\}$. 


\begin{table}[h!]
	\centering
	\begin{tabular}{|c|c|c|c|}
		\hline
		\textbf{Frame $\left(i\right)$} &  $\alpha_{i-1} \left[\text{deg}\right]$ & $a_{i-1} \left[\text{m}\right]$ $d_i \left[\text{m}\right]$ & $\theta_i \left[\text{deg}\right]$ \\ \hline
		1                                  & 0                         & 0                 &       $q_1$                              \\ \hline
		2                                 & -90                         & 0                 &       $q_2$                              \\ \hline
		3                                 & 90                         & $q_3$                    & 0                                   \\ \hline
	\end{tabular}
	\caption{Denavit-Hartenberg parameters}
	\label{tab:dh}
\end{table}
Table \ref{tab:dh} summarizes the DH parameters according to the frame allocation of the manipulator. The homogenous transformation matrix is used to represent the pose. The transformation matrix between two arbritary frames $\left\{A\right\}$ and $\left\{B\right\}$ is given by
\begin{equation*}
	\bm{T}_{B}^{A} = \left[\begin{array}{c|c}
		\bm{R}_{B}^{A} & \bm{r}_{B}^{A}\\
		\hline
		\bm{0}_{1\times3} & 1
	\end{array}\right]
\end{equation*}
where $\bm{R}_{B}^{A}$ describes the orientation of $\left\{B\right\}$ with respect to $\left\{A\right\}$ while $\bm{r}_{B}^{A}$ represents the position of origin of $\left\{B\right\}$ with respect to the origin of $\left\{A\right\}$. Assuming $E\left(x,y,z\right)$ directly lies on the origin of $\left\{3\right\}$, the forward kinematics are given by
\begin{equation*}
	\bm{T}_{3}^{0} = \left[\begin{array}{c|c}
		\bm{R}_{3}^{0} & \bm{r}_{3}^{0}\\
		\hline
		\bm{0}_{1\times3} & 1
	\end{array}\right]
\end{equation*}
where $\bm{r}_{3}^{0} = \left(x \text{ } y \text{ } z\right)\trans$ and 
\begin{equation*}
	\bm{T}_{3}^{0} = \bm{T}_{1}^{0}\bm{T}_{2}^{1}\bm{T}_{3}^{2}
\end{equation*}
where 
\begin{equation*}
	\bm{T}_{i}^{i-1}=\left[\begin{array}{cccc}
		\cos \theta_i & -\sin \theta_i & 0 & a_{i-1} \\
		-\sin \theta_i \cos \alpha_{i-1} & \cos \theta_i \cos \alpha_{i-1} & -\sin \alpha_{i-1} & -\sin \alpha_{i-1} d_i \\
		\sin \theta_i \sin \alpha_{i-1} & \cos \theta_i \sin \alpha_{i-1} & \cos \alpha_{i-1} & \cos \alpha_{i-1} d_i \\
		0 & 0 & 0 & 1
	\end{array}\right]
\end{equation*}
Hence, 
\begin{equation}
	\begin{bmatrix}
		x\\
		y\\
		z
	\end{bmatrix} = \begin{bmatrix}
	q_3\cos\left(q_1\right)\sin\left(q_2\right)\\
	q_3\sin\left(q_1\right)\sin\left(q_2\right)\\
	q_3\cos\left(q_2\right)
\end{bmatrix}
\label{eq:fwdKin}
\end{equation}

\subsection{Inverse Kinematics}
Inverse kinematics can be computed with the help of the obtained forward kinematics expression. We know from \eqref{eq:fwdKin}
\begin{align*}
	&x^2 + y^2 + z^2 = q_3^2 \\
	\implies&q_3 = \sqrt{x^2 + y^2 + z^2}
\end{align*}
Moreover,
\begin{align*}
	\cos\left(q_2\right) &= \frac{z}{q_3} = \frac{z}{\sqrt{x^2 + y^2 + z^2}}\\
	\sin\left(q_2\right) &= \sqrt{1 - \cos\left(q_2\right)^2} = \frac{\sqrt{x^2 + y^2}}{\sqrt{x^2 + y^2 + z^2}} \\
	\implies \tan\left(q_2\right) &= \frac{\sqrt{x^2 + y^2}}{z} \\
	\implies q_2 &= \arctan2\left(\sqrt{x^2 + y^2},z\right)
\end{align*}
The negative version of $\sin\left(q_2\right)$ is not considered since it is assumed that $q_2 \in \left[\frac{-\pi}{2},\frac{\pi}{2}\right]$ and hence $\arctan2$ is used. Finally the expression for $q_1$ is obtained trivially
\begin{equation*}
	q_1 = \arctan\left(\frac{y}{x}\right)
\end{equation*}
Since $q_1 \in \left[0,\pi\right]$, hence $\arctan$ is used and zero values of $x$ represents arbritary values of $q_1$.
Hence, the inverse kinematics can be summarized as 
\begin{equation}
	\begin{bmatrix}
		q_1\\
		q_2\\
		q_3
	\end{bmatrix} = \begin{bmatrix}
	\arctan\left(\frac{y}{x}\right)\\
	\arctan2\left(\sqrt{x^2 + y^2},z\right)\\
	\sqrt{x^2 + y^2 + z^2}
\end{bmatrix}
\label{eq:invKin}
\end{equation}
\subsection{Program Verification}
The verification of kinematics is done in the MATLAB\textsuperscript{\textregistered} script \emph{Kinematics.m}. The input DH parameters are read from the file \emph{dhParam.txt}. The script first computes the forward kinematics according to \eqref{eq:fwdKin} for the input DH parameters and then the inverse kinematics according to \eqref{eq:invKin}. The output of the inverse kinematics matches the inputs in the the file \emph{dhParam.txt}.

\subsection{Workspace}
